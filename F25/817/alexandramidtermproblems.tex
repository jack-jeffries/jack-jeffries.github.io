\documentclass[12pt]{article}
\usepackage{comment}
\usepackage{amscd}
\usepackage{amssymb, amsthm,amsmath}
\usepackage{enumitem}
\usepackage[all, knot]{xy}
%\usepackage[top=1.2in, bottom=1.2in, left=1.2in, right=1.2in]{geometry}
\xyoption{all}
\xyoption{arc}
\usepackage{hyperref}
%\usepackage{fullpage}

\theoremstyle{theorem}
\newtheorem{thm}{Theorem}

\theoremstyle{definition}
\newtheorem{pr}{Problem}
\newtheorem{defn}{Definition}
\newtheorem{eg}{Example}
\newtheorem{ex}{Exercise}
\newtheorem{rem}{Remark}

\newcommand{\ul}{\underline}
\newcommand{\RR}{\mathbb{R}}

\setlength{\oddsidemargin}{0cm}
\setlength{\evensidemargin}{0cm}
\setlength{\topmargin}{-.5in}
\setlength{\textheight}{9in}
\setlength{\textwidth}{6.5in}

\setlist[itemize]{leftmargin=*}


%%%% === mathcal letters ========

\def\cP{\mathcal P}
\def\cE{\mathcal E}
\def\cL{\mathcal L}
\def\cJ{\mathcal J}
\def\cJmor{\cJ^\mor}
\def\ctJ{\tilde{\mathcal J}}
\def\tPhi{\tilde{\Phi}}
\def\cA{\mathcal A}
\def\cB{\mathcal B}
\def\cC{\mathcal C}
\def\cZ{\mathcal Z}
\def\cD{\mathcal D}
\def\cF{\mathcal F}
\def\cG{\mathcal G}
\def\cO{\mathcal O}
\def\cI{\mathcal I}
\def\cS{\mathcal S}
\def\cT{\mathcal T}
\def\cM{\mathcal M}
\def\cN{\mathcal N}
\def\cK{\mathcal K}
\def\cKH{\mathcal KH}


%%% ========= mathbb letters ============


\newcommand{\Q}{\mathbb{Q}}
\newcommand{\bP}{\mathbb{P}}
\newcommand{\bM}{\mathbb{M}}
\newcommand{\A}{\mathbb{A}}
\newcommand{\bH}{{\mathbb{H}}}
\newcommand{\G}{\mathbb{G}}
\newcommand{\bR}{{\mathbb{R}}}
\newcommand{\bL}{{\mathbb{L}}}
\newcommand{\R}{{\mathbb{R}}}
\newcommand{\F}{\mathbb{F}}
\newcommand{\E}{\mathbb{E}}
\newcommand{\bF}{\mathbb{F}}
\newcommand{\bE}{\mathbb{E}}
\newcommand{\bK}{\mathbb{K}}
\newcommand{\bD}{\mathbb{D}}
\newcommand{\bS}{\mathbb{S}}
\newcommand{\bN}{\mathbb{N}}
\newcommand{\bG}{\mathbb{G}}
\newcommand{\C}{\mathbb{C}}
\newcommand{\Z}{\mathbb{Z}}
\newcommand{\N}{\mathbb{N}}
\renewcommand{\H}{\mathbb{H}}
\newcommand{\M}{\mathcal{M}}
\newcommand{\W}{\mathcal{W}}

%%%% ======== fraktur letters =============

\newcommand{\fc}{{\mathfrak c}}
\newcommand{\fp}{{\mathfrak p}}
\newcommand{\fm}{{\mathfrak m}}
\newcommand{\fq}{{\mathfrak q}}


%%%%% ===== greek letters ==============

\def\a{\alpha}
\def\b{\beta}
\def\d{\delta}
\def\td{\tilde{\delta}}
\def\e{\epsilon}
\def\oo{\overline{\omega}}


%%%% ======== operators =============
\def\Ann{\operatorname{Ann}}
\def\Aut{\operatorname{Aut}}
\def\Hom{\operatorname{Hom}}
\def\uHom{\operatorname{\underline{Hom}}}
\def\uSpec{\operatorname{\underline{Spec}}}
\def\Tor{\operatorname{Tor}}
 \def\Ext{\operatorname{Ext}}
\def\Mat{\operatorname{Mat}}
\def\End{\operatorname{End}}
\def\Fun{\operatorname{Fun}}
\def\Inn{\operatorname{Inn}}
\def\O{\operatorname{Orbit}}
\newcommand{\GL}{\operatorname{GL}}
\def\im{\operatorname{im}}
\def\ker{\operatorname{ker}}
\def\coker{\operatorname{coker}}
\def\dm{\operatorname{dim}}
\def\rank{\operatorname{rank}}
\def\trace{\operatorname{trace}}
\def\charpoly{\operatorname{char poly}}
\def\Gal{\operatorname{Gal}}


%%%%% =========== arrows =====================

\def\lra{\longrightarrow}
\def\into{\hookrightarrow}
\def\onto{\twoheadrightarrow}
\newcommand{\xra}[1]{\xrightarrow{#1}}
\newcommand{\xla}[1]{\xleftarrow{#1}}
\newcommand{\xroa}[1]{\overset{#1}{\twoheadrightarrow}}
\def\ov#1{\overline{#1}}


%%%%%%% ======= symbols ====================

\newcommand{\tensor}{\otimes}
\newcommand{\homotopic}{\simeq}
\newcommand{\homeq}{\cong}
\newcommand{\iso}{\approx}
\newcommand{\dual}{\vee}
\def\a{\alpha}
\def\b{\beta}
\def\id{{\mathrm id}}
\def\GL{\operatorname{GL}}
\def\norm{\mathrel{\unlhd}}


\pagenumbering{gobble}

\begin{document}
\title{ \vspace {-2em}MATH 817 -- Midterm review problems }
%\date{\vspace {-3em}(selected from previous qualifying exams)}
\date{}
\maketitle


\begin{enumerate}
\phantom{a}
 \vspace {-5em}
\item  Let $G$ be a finite group and $m$ a positive integer which is relatively prime to $|G|$. If $b \in G$ 
and $a^mb=ba^m$ for all $a\in G$, show that $b$ is in the center of $G$.

\medskip

%\item Let $G$ be a group (not necessarily finite). A subgroup $S$ of $G$ is said to be characteristic provided $\sigma(S) \subseteq S$ for every $\sigma \in \rm{Aut}(G)$. 
%\begin{enumerate}
%\item Prove that every characteristic subgroup of $G$ is a normal subgroup of $G$. 
%\item Prove that the center of a group $G$ is a characteristic subgroup of $G$. 
%\item  Prove that if $S$ is a characteristic subgroup of $G$ then $\sigma(S) = S$ for every $\sigma \in \rm{Aut}(G)$. 
%\item Let $p$ be a prime and let $P$ be the subgroup of $G$ generated by all elements of $G$ whose order is a power of $p$. Prove that $P$ is a characteristic subgroup of $G$. 
%\end{enumerate}
%
%\medskip

\item Let $G$ be a group. A subgroup $H$ of G is called 
a {\it characteristic} subgroup of G if $\alpha(H) = H$
for every automorphism $\alpha$ of $G$. \footnote{{\em Tip:} Consider in particular for each $g\in G$ the automorphism $a_g:G\to G$ givend by $a_g(g')=gg'g^{-1}$.}
Show that if $H$ is a characteristic subgroup of $N$ and $N$ is
a normal subgroup of $G$, then $H$ is a normal subgroup of $G$.

\medskip


\item 
Let $G$ be a finite group and let $H$ be a proper subgroup of $G$ with $[G:H]=h$. 
\begin{enumerate}
\item Prove that $H$ has at most $h$ distinct conjugate sets  $gHg^{-1}$ for $g\in G$.
\item
Prove that $G\neq \bigcup_{g\in G}  gHg^{-1}$.
\end{enumerate}

\medskip

\item Let $H$ be a subgroup of a group $G$.  
\begin{itemize}
\item[(a)] Show that the centralizer $C_G(H)$ of $H$ in $G$ is a normal subgroup of the normalizer $N_G(H)$ of $H$ in $G$.
\item[(b)] Show that the quotient $N_G(H)/C_G(H)$ is isomorphic to a subgroup of the automorphism group ${\rm Aut}(H)$ of $H$. 
\end{itemize}

\medskip
%\item   
%\begin{enumerate}
%\item Let $A$ and $B$ be two groups. Prove that the cartesian product $A\times B$ is a group with respect to the binary operation defined by $(a,b)\cdot(a',b')=(a\cdot_Aa',b\cdot_B b')$.
%\item
%Let $M$ and $N$ be normal subgroups of a group $G$ such that $G = MN$. Prove that
%$$G/(M \cap N) \cong (G/M) \times (G/N).$$
%\end{enumerate}
%
%\medskip

%\item Let $\F_3$ denote the field with 3 elements and let $V=\F_3^2$. Let $\alpha,\beta,\gamma$ and $\delta$ be the four one-dimensional subspaces of $V$ spanned by $\begin{bmatrix}1 \\0 \end{bmatrix} , \begin{bmatrix}0 \\1 \end{bmatrix} ,\begin{bmatrix}1 \\1 \end{bmatrix}$ and $\begin{bmatrix}1 \\-1 \end{bmatrix}$ respectively. Let $G := {\rm GL}_2(\F_3)$ act on $\{\alpha, \beta, \gamma, \delta\}$ by matrix multiplication. 
%\begin{enumerate}
%\item Prove that the kernel of the homomorphism $\rho : G \to S_4$ corresponding to this action is $\{\pm I_2\}$. (Note: $I_2$ denotes the $2 \times 2$ identity matrix.) 
%\item Prove that  $G/\{\pm I_2\} \cong S_4$. 
%\end{enumerate}
%
%\medskip

%\item 
%Let $G$ be a group and $K$ a finite cyclic normal subgroup of $G$. 
%\begin{enumerate}
%\item Prove that ${\rm Aut}(K)$ is an abelian group.
%\item Prove that $G' \subseteq C_G(K)$, where $G'=\{gg'g^{-1}g'^{-1} \mid g,g'\in G\}$ is the commutator subgroup of $G$ and $C_G(K) = \{g \in G \mid gk = kg \ \forall k \in K\}$. 
%\footnote{{\em Tip:} Consider an appropriate action of $G$ on $K$; show that the image of the corresponding permutation representation is contained in ${\rm Aut}(K)$.}
%\end{enumerate}
%\medskip


\item 
Let $G$ be a finite group  such that $|G|=nm$. 
\begin{enumerate}
\item Suppose $x \in G$ has order $n$ and let $\sigma_x \in \rm{Perm}(G)$ be the permutation such that $\sigma_x(g) = xg$ for every $g \in G$. Show that $\sigma_x$ is a product of $m$ disjoint $n$-cycles. \item If $n = 2$ and $m$ is odd, show that there is a homomorphism $f : G \to S_{nm}$ such that $f(G)\not\subseteq A_{nm}$. 
\item If $n = 2$ and $m$ is odd, conclude that $G$ has a subgroup of index 2.
\end{enumerate}

\medskip


\item Let $G$ be a group acting on a set $A$, and
let $H$ be a subgroup of $G$ satisfying that the induced
action of $H$ on $A$ is transitive (that is, for all
$a,b \in A$ there is an $h \in H$ with $h \cdot a = b$).
Let $t \in A$, and let ${\rm Stab}_G(t)$ be the stabilizer of $t$ in $G$.
Show that $G = H {\rm Stab}_G(t)$.


\medskip

\item
\begin{itemize}
\item[(a)] Let $G$ be a simple group of order 60.
Determine the number of elements of $G$ of order 5.
\item[(b)] Show that there is no simple group of order 30.
\end{itemize}




\end{enumerate}


\end{document}


\item 
\begin{enumerate}
\item Let $G$ be a group of order 15 acting on a set $S$ with 7 elements. Prove that there exists at least one fixed point, i.e., there exists $s\in S$ such that $g\cdot s=s$ for all $g\in G$.
\item Give an example of an action of $C_{15}$ on a set with 8 elements with no fixed points. Justify. 
\end{enumerate}




Let $A$ be a set, let $F(A)$ be the free group on $A$ and let $R$ be a subset of $F(A)$. Let $H$ be a group, and let $f:A \to H$ be a function satisfying the property that 
\begin{quote}
for any  $m \geq 0$ and any choices of $r_j \in R, g_j \in F(A), i_j \in \{1,-1\}$ for $1\leq j\leq m$ we have  that $f(g_1)f(r_1)^{i_1}f(g_1)^{-1} \cdots f(g_m)f(r_m)^{i_m}f(g_m)^{-1} =e_H$. 
\end{quote}
Prove that there is a unique homomorphism $F: \langle A \mid R \rangle \to H$ satisfying $F(a\langle R\rangle ^N) = f(a)$ for all $a \in A$.

