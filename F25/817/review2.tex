\documentclass[12pt]{article}
\usepackage{amssymb,amsfonts, amsmath, amscd}
\usepackage{fancyhdr}


\setlength{\oddsidemargin}{0cm}
\setlength{\evensidemargin}{0cm}
\setlength{\topmargin}{-.25in}
\setlength{\textheight}{9.2in}
\setlength{\textwidth}{6.5in}


\begin{document}


\pagenumbering{gobble}

\centerline{{\bf Math 817 Review Sheet \#2}}
\smallskip
\centerline{{A (not necessarily complete) list of important things to know for the Final Exam}}

\begin{description}

\item[Group Automorphisms:]  Definition, automorphism group of cyclic group, automorphism group of $\mathbb{Z}/n^\times$.

\item[Direct Products:] Definition of direct products, recognition theorem for direct products, internal direct products.

\item[Semidirect Products:] Definition of semidirect products, recognition theorem for semidirect products, internal semidirect products, uniqueness theorem.


\item[Free Groups:] Definition, universal mapping property.

\item[Presentations:] Definition, universal mapping property, how to find a presentation, how to prove a presentation is correct.

\item[Sylow's Theorem:] Statements, applications to simple groups, applications of classifying groups of a given order.


\item[Conjugacy classes:] Class equation, applications to simple groups and $p$-groups, conjugacy classes in symmetric and alternating groups.


\item[Fundamental Theorem of Finitely Generated Abelian Groups:] Invariant factor form, elementary divisor form, rank.


\item[Examples of Rings:]  Commutative, noncommutative, matrix rings, fields, domains, polynomial rings.

\item[Special elements:]  Unit, nilpotent, idempotent, zerodivisor, irreducible, prime.

\item[Ideals:]  Definition, sums and (finite) products of ideals, principal ideals, maximal ideals, prime ideals, kernels of ring of homomorphisms.

\item[Quotient rings:] Isomorphism theorems, determining when an ideal is maximal or prime.  

\item[Division theorem for polynomial rings:]   Statement and proof, applications to roots, polynomial ring over a field is a PID.

\item[PIDs:]  Examples and nonexamples.  Proof that every irreducible element is prime in a PID.

\item[UFDs:]  Definition, examples and nonexamples, theorem for when a Noetherian domain is a UFD.




\end{description}

\end{document}