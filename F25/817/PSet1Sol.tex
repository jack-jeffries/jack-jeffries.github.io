\documentclass[11pt]{article}
\usepackage[margin=0.9in]{geometry}
\usepackage{amsmath,amsfonts,amssymb,amsthm}
\usepackage{enumitem}
\usepackage[]{graphicx}
\usepackage{color,subfigure}
\definecolor{scarlet}{rgb}{0.81,0,0}
\usepackage{multicol}
\usepackage{float}
\usepackage[all]{xypic}
\usepackage[colorlinks=true,citecolor=scarlet,linkcolor=scarlet]{hyperref}
\usepackage{colonequals}

\usepackage{fancyhdr, lastpage}
\pagestyle{fancy}
%\fancyfoot[C]{{\thepage} of \pageref{LastPage}}




\makeatletter
\renewenvironment{proof}[1][\proofname]{\par
  \vspace{-\topsep}% remove the space after the theorem
  \pushQED{\qed}%
  \normalfont
  \topsep0pt \partopsep0pt % no space before
  \trivlist
  \item[\hskip\labelsep
        \itshape
    #1\@addpunct{.}]\ignorespaces
}{%
  \popQED\endtrivlist\@endpefalse
  \addvspace{6pt plus 6pt} % some space after
}
\makeatother


\DeclareMathOperator{\mSpec}{mSpec}
\DeclareMathOperator{\Spec}{Spec}
\DeclareMathOperator{\Ass}{Ass}
\DeclareMathOperator{\Supp}{Supp}
\DeclareMathOperator{\height}{height}
\DeclareMathOperator{\Hom}{Hom}
\DeclareMathOperator{\ann}{ann}
\DeclareMathOperator{\End}{End}
\DeclareMathOperator{\coker}{coker}
%\DeclareMathOperator{\ker}{ker}
\DeclareMathOperator{\rank}{rank}
\DeclareMathOperator{\im}{im}
\DeclareMathOperator{\M}{M}
\DeclareMathOperator{\Tor}{Tor}
\DeclareMathOperator{\id}{id}
\DeclareMathOperator{\ch}{char}
\DeclareMathOperator{\Aut}{Aut}
%\DeclareMathOperator{\dim}{dim}


\def\ra{\rightarrow}
\newcommand{\m}{\mathfrak{m}}
\newcommand{\C}{\mathbb{C}}
\newcommand{\Q}{\mathbb{Q}}
\newcommand{\Z}{\mathbb{Z}}
\newcommand{\R}{\mathbb{R}}
\newcommand{\N}{\mathbb{N}}
\newcommand{\ov}[1]{\overline{#1}}


\DeclareMathOperator{\lcm}{lcm}

\title{}
\date{\vspace{-0.5in}}

\makeatletter
\g@addto@macro\@floatboxreset\centering
\makeatother

\theoremstyle{definition}
\newtheorem{problem}{Problem}
\newtheorem*{solution}{Solution}


\begin{document}

\thispagestyle{fancy}
\pagestyle{fancy}
\rhead{UNL $\mid$ Fall 2025}
\lhead{Introduction to Modern Algebra I}

\vspace{3em}

\begin{center}
	{\LARGE Problem Set 1 solutions}
\end{center}

\



\begin{problem}
Let $G$ be a group and $x \in G$ any element. 
Recall that $|x|$ denotes the {\em order} of $x$, defined to be the least integer $n \geqslant 1$ such that $x^n = e$; if no such integer exists, we say $|x| = \infty$.
Also, let $|G|$ denote the cardinality of $G$; note that $|G|$ is an element of $\{1, 2, 3, \cdots \} \cup \{\infty\}$.

\begin{enumerate}[label=(\alph*),itemsep=-0.2em]

\item Prove that if $|x| = n$, then $e, x, \dots, x^{n-1}$ are all distinct elements of $G$. 

\begin{proof}
	If $e=x^0,x,x^2,\dots,x^{n-1}$ are not all distinct, then $x^i=x^j$ for some $0 \leqslant i<j \leqslant n-1$, and thus $x^{j-i} = e$. Since $0<j-i<n$, this contradicts the minimality of $n$.
\end{proof}


\item Prove that if $|x| = \infty$, then $x^i \neq x^j$ for all positive integers $i \neq j$. 


\begin{proof}
	Suppose $x^i=x^j$ for some $i < j$. Multiplying by the inverse of $x$ on the right gives $x^{j-i}=e$ and $j-i>0$, contradicting the assumption that $|x|=\infty$.
\end{proof} 

\item Conclude that $|x| \leqslant |G|$ in all cases.

\begin{proof}
	If $|x|=n$, then part (a) shows that $G$ contains $n$ distinct elements, and thus $|G| \geqslant n$.  If $|x|=\infty$ then part (b) shows that $G$ has infinitely many distinct elements, and thus $|G|$ is infinite.  In either case, we have $|x|\leqslant |G|$.
\end{proof}  
\end{enumerate}
\end{problem} 


\begin{problem}
	A group $G$ is called {\it cyclic} if it is generated by a single element. 
	
	\begin{enumerate}[label=(\alph*)]
		\item Prove that any cyclic group is abelian.
		
		\noindent Note: your proof will be very short, as you can use the fact that $x^ix^j = x^{i+j}$ without proof.
		
		\begin{proof}
			Let $G$ be a cyclic group.  Then there is some element $x$ of $G$ such that $G=\{x^i\mid i\in \mathbb Z\}$.  To show $G$ is abelian, it suffices to show that $x^ix^j=x^jx^i$ for all integers $i$ and $j$.  But this holds because $x^ix^j = x^{i+j} = x^{j+i} = x^j x^i$, which is known as the law of exponents. % (I won't bother proving the law of exponents here, but it is straightforward.)  	
			\end{proof}
		
%		\item Give an example of an abelian group which is not cyclic, with justification.
%		
%		\begin{solution}
%		An example of a non-cyclic abelian group is $\mathbb Z_8^{\times}=\{1, 3, 5, 7\}$.  This group is abelian, but no element generates the entire group, since all elements have order 1 or 2.	
%		\end{solution}
%		
		\item Prove that $(\mathbb Q, +)$ is not a cyclic group.

\begin{proof}
	If $\mathbb Q$ is cyclic, let $\frac{a}{b}$ be a generator, so that in additive notation $\mathbb Q=\{\frac{ma}{b}\mid m\in \mathbb Z\}$. Note that $a, b \neq 0$ are integers. Now $\frac{a}{2b} \in \mathbb Q$, so $\frac{a}{2b} = \frac{ma}{b}$ for some $m\in \mathbb Z$. But in $\Q$ we can now divide by $\frac{a}{b}$, concluding that $m = \frac{1}{2}$, which is a contradiction since $\frac{1}{2} \notin \mathbb Z$.  Thus $\mathbb Q$ is not cyclic.
\end{proof}

\item Prove that $\operatorname{GL}_2(\mathbb Z_2)$ is not cyclic.

\begin{proof}
	By (a), it suffices to prove $\operatorname{GL}_2(\mathbb Z_2)$ is not abelian.  Let 
	$$A=\begin{pmatrix} 1 & 1\\ 0 & 1 \end{pmatrix} \quad \textrm{and} \quad B=\begin{pmatrix} 1 & 1\\ 1 & 0 \end{pmatrix}.$$
	Since $\det(A)=\det(B)=1$, both matrices are in $\operatorname{GL}_2(\mathbb Z_2)$.  But $AB\neq BA$.
\end{proof}
	\end{enumerate}
\end{problem}



\begin{problem} Let $n\geq 2$, and consider\footnote{Note: If you are unsure which formulas about permutations require proof, please ask.} the symmetric group $S_n$.
	\begin{enumerate}[label=(\alph*)]
\item Let $\tau\in S_n$ be a permutation, and $(i_1 \, i_2 \, \cdots \, i_k)$ be a $k$-cycle. Show that 
\[ \tau (i_1 \, i_2 \, \cdots \, i_k)  \tau^{-1} = (\tau(i_1) \, \tau(i_2) \, \cdots \, \tau(i_k)).\]

\begin{proof}
Observe that the left-hand side sends an arbitrary $j$ to $j$ if $\alpha^{-1}(j) \notin \{i_1, \dots, i_k\}$
    and to $\alpha(i_{t+1 \pmod k})$ if $\alpha^{-1}(j)  = i_t$ for some $t$. Equivelently, it sends $\alpha(i_t)$ to $\alpha(i_{t \pmod k})$ and fixes all other elements. This is
      what the right-hand side does too.
      \end{proof}
\item Show that $S_n$ is generated by $(12)$ and the $n$-cycle $(12 \cdots n)$.
\begin{proof}
Note: In all calculations below, everything should be read modulo $n$. 

Let $H=\langle (12), (12\cdots n)\rangle$ be the group generated by $(12)$ and $(12\cdots n)$.  Since every permutation can be written as a product of transpositions, it suffices to show that every transposition is in $H$. We will use two useful formulas about permutations:

$$\begin{aligned}
\textrm{F}1: & \qquad (12\cdots n)(i\,\,\, i+1)(12\cdots n)^{-1}=(i+1\,\,\, i+2).\\
\textrm{F}2: & \qquad (ij) = (1j)(1i)(1j).
\end{aligned}$$
Both of these are special cases of (a).

 Now let us prove that $H = S_n$ using F$1$ and F$2$. Since $(12)$ and $(12\cdots n)$ are both in $H$, using F$1$ repeatedly gives us $(i \,\,\, i+1) \in H$ for all $i$. Now take $j = i+1$ in F$2$, which gives us
$$\begin{aligned} \textrm{F}3: & (i\, i+1)(1i)(i\, i+1)=(1\, i+1). \end{aligned}$$
Since $(1 \,\, 2) \in H$ and $(i \, \, i+1) \in H$ for all $i$, repeated applications of F$3$ give us $(1 \,\, j) \in H$ for all $j$. Finally, since $(1 \,\, i), (1 \, \, j) \in H$ for all $i, j$, then by F$2$ we conclude that $(i \,\, j) \in H$. This shows all transpositions are in $H$, and thus $H = S_n$.
\end{proof}


\item Show that, if $n\geq 3$, then $Z(S_n) = \{e\}$.
\begin{proof}
We again apply part (a) in a special case:
 $$
  \tau (i \, j) = (\tau(i) \, \tau(j)) \tau
  $$
  for any $\tau \in S_n$ and any $2$-cycle $(i \, j)$. Assume that $\tau$ is in the center. Then the above equation gives that  
  $(i \, j) = (\tau(i) \, \tau(j))$  and hence either ($\tau(i) = i$ and $\tau(j) = j$) or ($\tau(i) = j$ and $\tau(j) = i$) for all $i \ne j$.
We will show that $\tau(i) = i$ for all $i$. Pick any $i$. If $\tau(i) \ne i$, then by what we just proved, $\tau(j)  = i$ for all $j \ne i$. 
Since $n \geq 3$, 
we can find $1 \leq j, k \leq n$ so that $i,j,k$ are distinct, and hence $\tau(j) = i = \tau(k)$, which is not possible.
\end{proof}

\end{enumerate}
\end{problem}


\begin{problem}
	\begin{enumerate}[label=(\alph*)]
\item Suppose the cycle type of $\sigma \in S_n$ is $m_1, m_2, \ldots, m_k$. Recall this means that $\sigma$ a product of disjoint cycles of lengths $m_1, m_2, \ldots, m_k$. Prove that $|\sigma| = \lcm(m_1, \ldots, m_k)$. 
\item Given an example of two permutations $\sigma, \tau$ such that $|\sigma\tau| > \lcm(|\sigma|,|\tau|)$.
\end{enumerate}
\end{problem}

\vspace{0.5em}


\begin{proof}
\begin{enumerate}[label=(\alph*)]
\item
We first consider the case when $k = 1$; that is, we will first show the order of an $m$-cycle is $m$. Given an $m$-cycle $\alpha = (i_1 \, i_2 \, \cdots, i_m)$, note that for any $k$, we have $\alpha^k(i_j) = i_{j+k \pmod{m}}$.
It follows that $\alpha^m = e$ and, for each $1 \leqslant k < m$, $\alpha^k \neq e$; hence $|\alpha| =m$.

Now we consider the general case. Assume $g_1, \ldots, g_k$ are pairwise disjoint cycles, with $g_i$ a cycle of length $m_i$, and let $g \colonequals g_1 \cdots g_m$. Since these elements $g_1, \ldots, g_j$ are disjoint cycles, and disjoint cycles commute, we have $(g_1 \ldots g_k)^m = g_1^m \cdots g_k^m$ for all $m$. It follows that if $m$ is a multiple of $|g_i| = m_i$ for each $i$, then $g_i^m = (g_i^{m_i})^{\frac{m}{m_i}} = e$, and thus $g^m = e$. In particular, $g^{\lcm(m_1, \ldots, m_k)} = e$.

Now suppose $1 \leqslant m < \lcm(m_1, \dots, m_k)$. We need to prove that $g^m \neq e$. Note that $m$ is not a multiple of $m_i$ for at least one value of $i$; for notational simplicity and without loss of generality (since we can always renumber the list of cycles), let us assume $m_1$ does not divide $m$. Then 
$$g_1^m = g_1^{m \pmod{m_i}} \neq e.$$ 
Thus there is an integer $i$ with $1 \leqslant i \leqslant n$ such that $g_1^m(i) \neq i$. But since the cycles are disjoint, $g_j(i) = i$ for all $j \geqslant 2$ and hence also $g_j^m(i) = i$ for all such $j$.
This proves that $g^m = g_1^m \cdots g_k^m$ does not fix $i$ and thus cannot be the identity element.

\item One can take $\sigma=(1 \, 2)$ and $\tau=(2\, 3)$ in $S_3$. Both $\sigma$ and $\tau$ have order $2$, whereas $\sigma\tau=(1 \, 2 \, 3)$ has order $3$.
\end{enumerate}
\end{proof}

\end{document}