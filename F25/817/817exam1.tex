\documentclass[12pt]{article}
\usepackage{amssymb,amsfonts, amsmath, amscd}
\pagestyle{empty} \setlength{\oddsidemargin}{0in}
\setlength{\evensidemargin}{0in} \setlength{\textwidth}{6.5 in}
\setlength{\topmargin}{-.25in} \setlength{\headheight}{0in}
\setlength{\headsep}{0in} \setlength{\topskip}{0in}
\setlength{\textheight}{9.5 in} \font\bigbf = cmbx10 scaled
\magstep1\font\medbf = cmbx10 scaled \magstephalf
\medskip
\def\ans{\smallskip{\bf Answer: }}
\def\prob{\bigskip\noindent{\bf Problem: }}
\def\endprob{\medskip}
\def\defn{\bigskip\noindent{\bf Definition: }}
\def\enddefn{\medskip}
\def\thm{\bigskip\noindent{\bf Theorem: }}
\def\endthm{\medskip}
\def\pf{\smallskip{\bf Proof: }}
\def\endpf{\smallskip}
\def\conj{\bigskip\noindent{\bf Conjecture: }}
\def\endconj{\medskip}
\def\tf{\bigskip\noindent{\bf True or False? }}
\def\endtf{\medskip}
\def\cx{\smallskip{\bf False. Counterexample: }}
\def\endcx{\smallskip}
\def\eg{\medskip{\bf Example: }}
\def\endeg{\smallskip}
\def\hmwk{\bigskip\noindent{\bf Homework: }}
\def\endhmwk{\medskip}
\def\ques{\bigskip\noindent{\bf Question: }}
\def\endques{\medskip}
\def\cor{\bigskip\noindent{\bf Corollary: }}
\def\endcor{\medskip}
\def\com{\bigskip\noindent{\bf Comments. }}
\def\endcom{\medskip}
\def\tr{\operatorname{tr}}
\def\Hom{\operatorname{Hom}}



%\renewcommand{\labelenumi}{{\bf Problem \arabic{enumi}}}



\begin{document}

\centerline{\bigbf Math 817}
\medskip
\centerline{\bigbf Exam I}
\medskip
\centerline{\it Tuesday, October 12th }

\medskip

{\bf Instructions:}  Do {\bf four} of the five problems below. (Do not do more than four.)   All problems are equally weighted.  Justify your answers as much as you can.  You may use results we've proved in class (not just stated as an exercise) or on the homework.  (The exceptions to this would be \#2(a) and the finite case of \#4.  Those you should prove completely.)



\begin{enumerate}

\item Let $G$ be a group and $a,b\in G$ two elements of finite order.  
Suppose $ab=ba$ and $\langle a\rangle \cap \langle b\rangle =\{1\}$.  
\begin{enumerate}
\item Prove that $|ab|=\operatorname{lcm}(|a|,|b|)$.
\item Use (a) to prove that if $H$ and $K$ are finite cyclic groups of relatively prime orders then $H\times K$ is cyclic.
\end{enumerate}

\medskip

\item Let $G$ be a group acting on a set $X$.  For $a\in X$, let $\mathcal O(a)$ denote the orbit of $a$ and $G_a$ the stabilizer of $a$.  
\begin{enumerate} 
\item Prove the Orbit - Stabilizer theorem:  $|\mathcal O(a)|=[G:G_a]$.
\item Suppose $G$ is a group of prime order and there are at least two distinct orbits.  Prove that the action is trivial, i.e., that the kernel of the action is all of $G$.  \end{enumerate}

\medskip


\item Let $G$ be a group and $N\le H$ subgroups of $G$.
\begin{enumerate} 
\item Give an example such that $N\triangleleft H$ and $H\triangleleft G$ but $N$ is not normal in $G$.
\item Suppose $N\triangleleft G$ and $G/N$ is abelian.  Prove that $H\triangleleft G$ and $G/H$ is abelian.
\end{enumerate}


\medskip

\item Let $G$ be a group (possibly infinite) and $H\le K$ subgroups of $G$.  Suppose $[G:H]$ is finite.  Prove that $[G:H]=[G:K][K:H]$.  (Hint:  First prove this in the case $G$ is a finite group.  Then prove you can reduce to the finite case.)

\medskip

\item Let $G$ be a finite group and $K$ a subgroup of $G$ such that:
\begin{enumerate}
\item $[G:K]=2$:
\item $K$ is simple;
\item $\operatorname{Z}(G)=\{1\}$.
\end{enumerate}
Prove that the only normal subgroups of $G$ are $\{1\}$, $K$, and $G$.
(Hint: Let $H$ be a normal subgroup of $G$.  Then $H\cap K$ is a normal subgroup of $K$.)





\end{enumerate}





\end{document}
