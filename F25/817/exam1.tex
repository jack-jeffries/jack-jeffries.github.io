\documentclass[11pt]{article}
\usepackage[margin=1in]{geometry}
\usepackage{amsmath,amsfonts,amssymb,amsthm,enumerate}
\usepackage[]{graphicx}
\usepackage{color,subfigure}
\definecolor{scarlet}{rgb}{0.81,0,0}
\usepackage{multicol}
\usepackage{float}
\usepackage[all]{xypic}
\usepackage[colorlinks=true,citecolor=scarlet,linkcolor=scarlet]{hyperref}
\usepackage{colonequals}

\usepackage{fancyhdr, lastpage}
\pagestyle{fancy}
\fancyfoot[C]{{\thepage} of \pageref{LastPage}}



\DeclareMathOperator{\mSpec}{mSpec}
\DeclareMathOperator{\Spec}{Spec}
\DeclareMathOperator{\Ass}{Ass}
\DeclareMathOperator{\Supp}{Supp}
\DeclareMathOperator{\height}{height}
\DeclareMathOperator{\Hom}{Hom}
\DeclareMathOperator{\ann}{ann}
\DeclareMathOperator{\End}{End}
\DeclareMathOperator{\coker}{coker}
%\DeclareMathOperator{\ker}{ker}
\DeclareMathOperator{\rank}{rank}
\DeclareMathOperator{\im}{im}
\DeclareMathOperator{\M}{M}
\DeclareMathOperator{\Tor}{Tor}
\DeclareMathOperator{\id}{id}
\DeclareMathOperator{\ch}{char}
\DeclareMathOperator{\Aut}{Aut}
%\DeclareMathOperator{\dim}{dim}

\DeclareMathOperator{\lcm}{lcm}

\def\ra{\rightarrow}
\newcommand{\m}{\mathfrak{m}}
\newcommand{\C}{\mathbb{C}}
\newcommand{\Q}{\mathbb{Q}}
\newcommand{\Z}{\mathbb{Z}}
\newcommand{\R}{\mathbb{R}}
\newcommand{\N}{\mathbb{N}}
\newcommand{\ov}[1]{\overline{#1}}

\def\ov#1{\overline{#1}}


\title{}
\date{\vspace{-0.5in}}

\makeatletter
\g@addto@macro\@floatboxreset\centering
\makeatother

\theoremstyle{definition}
\newtheorem{problem}{Problem}


\begin{document}

\thispagestyle{fancy}
\pagestyle{fancy}
\rhead{UNL $\mid$ Fall 2025}
\lhead{Introduction to Modern Algebra I}

\vspace{3em}

\begin{center}
	{\LARGE Midterm Exam}
\end{center}

\

\noindent
{\bf Instructions:}
Solve \emph{two} problems from Part 1 and \emph{two} problems from Part 2. You may use any results proved in class or in the problem sets, except for the specific question being asked. You should clearly state any facts you are using. You are also allowed to use anything stated in
one problem to solve a different problem, even if you have not yet proved it. Remember to show
all your work, and to write clearly and using complete sentences. No calculators, notes, cellphones,
smartwatches, or other outside assistance allowed.

\section*{Part 1: Old problems}

Choose \emph{two} of the following problems.

\begin{enumerate}
 
 \item[(1)] Prove that, for any $n\geq 2$, there is no nontrivial\footnote{Recall that a homomorphism is \emph{trivial} if its image is $\{e\}$.} group homomorphism $\Z/n \to \Z$.
 
 
  \item[(2)] Let $G$ be a group of order $p^n$, for some prime $p$ and some $n\geq 1$, acting on a finite set $X$.
  \begin{enumerate}[(a)]
    \item Suppose $p$ does not divide $\# X$. Prove that there exists some element of $X$ that is fixed by all elements of $G$.
      \item Suppose $G$ acts faithfully\footnote{Recall this means that if $g \cdot x = x$ for all $x \in X$, then $g = e_G$.} on $X$.  Prove that $\# X \geq n \cdot p$.
            \end{enumerate}
            
            
 \item[(3)] Let $G$ be a group and $H$ a subgroup of $G$. The centralizer of $H$ in $G$ is the set of elements of $G$ that commute with each element of $H$:
  $$
  C_G(H) := \{g \in G \mid gh = hg \text{ for all $h \in H$} \} = \{g \in G \mid ghg^{-1} = h \text{ for all $h \in H$} \}.
  $$
  Prove that if $H$ is normal in $G$, then $G/C_G(H)$ is isomorphic to a subgroup of the automorphism group of $H$.
 
 


     
    


 \end{enumerate}



\section*{Part 2: New problems}

Choose \emph{two} of the following problems.

\begin{enumerate}
 
 \item[(4)] Let $G$ be a nontrivial finite group. Show\footnote{Note that you cannot use Cauchy's Theorem, since we have not shown it yet. Instead, for the forward direction, you might consider showing first that $G$ is cyclic.}  that $G$ has no nontrivial proper subgroups if and only if $|G|$ is prime. 



\item[(5)] Let $G$ be a group, and $H, H'$ be two subgroups of $G$. Prove that $H\cup H'$ is a subgroup of $G$ if and only if $H\subseteq H'$ or $H' \subseteq H$.


\item[(6)] Let $G$ be a group and $N\leq H$ two subgroups of $G$.
\begin{enumerate}
\item[(a)] Give an example such that $N\trianglelefteq H$ and $H\trianglelefteq G$ but $N$ is not normal in $G$.
\item[(b)] Suppose that $N\trianglelefteq G$ and $G/N$ is abelian. Prove that $H \trianglelefteq G$ and $G/H$ is abelian. 
\end{enumerate}

\end{enumerate}



\end{document}