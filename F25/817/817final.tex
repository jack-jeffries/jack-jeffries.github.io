\documentclass[11pt]{article}
\usepackage[margin=1in]{geometry}
\usepackage{amsmath,amsfonts,amssymb,amsthm,enumerate}
\usepackage[]{graphicx}
\usepackage{color,subfigure}
\definecolor{scarlet}{rgb}{0.81,0,0}
\usepackage{multicol}
\usepackage{float}
\usepackage[all]{xypic}
\usepackage[colorlinks=true,citecolor=scarlet,linkcolor=scarlet]{hyperref}
\usepackage{colonequals}

\usepackage{fancyhdr, lastpage}
\pagestyle{fancy}
\fancyfoot[C]{{\thepage} of \pageref{LastPage}}



\DeclareMathOperator{\mSpec}{mSpec}
\DeclareMathOperator{\Spec}{Spec}
\DeclareMathOperator{\Ass}{Ass}
\DeclareMathOperator{\Supp}{Supp}
\DeclareMathOperator{\height}{height}
\DeclareMathOperator{\Hom}{Hom}
\DeclareMathOperator{\ann}{ann}
\DeclareMathOperator{\End}{End}
\DeclareMathOperator{\coker}{coker}
%\DeclareMathOperator{\ker}{ker}
\DeclareMathOperator{\rank}{rank}
\DeclareMathOperator{\im}{im}
\DeclareMathOperator{\M}{M}
\DeclareMathOperator{\Tor}{Tor}
\DeclareMathOperator{\id}{id}
\DeclareMathOperator{\ch}{char}
\DeclareMathOperator{\Aut}{Aut}
%\DeclareMathOperator{\dim}{dim}

\DeclareMathOperator{\lcm}{lcm}

\def\ra{\rightarrow}
\newcommand{\m}{\mathfrak{m}}
\newcommand{\C}{\mathbb{C}}
\newcommand{\Q}{\mathbb{Q}}
\newcommand{\Z}{\mathbb{Z}}
\newcommand{\R}{\mathbb{R}}
\newcommand{\N}{\mathbb{N}}
\newcommand{\ov}[1]{\overline{#1}}

\def\ov#1{\overline{#1}}


\title{}
\date{\vspace{-0.5in}}

\makeatletter
\g@addto@macro\@floatboxreset\centering
\makeatother

\theoremstyle{definition}
\newtheorem{problem}{Problem}


\begin{document}

\thispagestyle{fancy}
\pagestyle{fancy}
\rhead{UNL $\mid$ Fall 2025}
\lhead{Introduction to Modern Algebra I}

\vspace{3em}

\begin{center}
	{\LARGE Midterm Exam}
\end{center}

\

\noindent
{\bf Instructions:}
Solve \emph{two} problems from Part 1 and \emph{two} problems from Part 2. You may use any results proved in class or in the problem sets, except for the specific question being asked. You should clearly state any facts you are using. You are also allowed to use anything stated in
one problem to solve a different problem, even if you have not yet proved it. Remember to show
all your work, and to write clearly and using complete sentences. No calculators, notes, cellphones,
smartwatches, or other outside assistance allowed.

\section*{Part 1: Groups}

Choose \emph{two} of the following problems.

\begin{enumerate}[(1)]
 
    
 \item  \begin{enumerate}[(a)]
 \item Show that there exists a nonabelian group of order $27$.
 \item Give, with justification, a presentation for the group you found in part (a).
 \end{enumerate}
 
\item Prove that no group of order $224= 2^5 \cdot 7$ is simple.
 
 
 \item Prove that $\mathbb{Q}/\mathbb{Z}$ is not a finitely generated group.
   
   

     
    


 \end{enumerate}



\section*{Part 2: Rings}

Choose \emph{two} of the following problems.

\begin{enumerate}[(1)]\setcounter{enumi}{3}
 
 \item Let $R$ be a ring. Let $I$ and $J$ be ideals of $R$, and recall that $I+J=\{ a + b \ | \ a\in I, \, b\in J\}$. 
 \begin{enumerate}[(a)]
 \item Show that if $I=(S)$ and $J=(T)$ for some subsets $S,T\subseteq R$, then $I+J = (S\cup T)$.
 \item Let $\pi:R \to R/I$ be the quotient homomorphism. Show that $\displaystyle \frac{R/I}{\pi(J)} \cong \frac{R}{I+J}$.
 \end{enumerate}
 
 \item     \begin{enumerate}[(a)]
    \item Prove that a finite integral domain must be a field.
      \item Prove\footnote{Hint: Consider the quotient ring $R/P$.} that if $R$ is a commutative ring and $P \subseteq R$ is a prime ideal such that $P$ has finite
index as a subgroup of $(R, +)$, then $P$ is a maximal ideal. %Give an example to show that this implication may fail if the finite index assumption is dropped.
\end{enumerate}
 
\item Consider the polynomial $f(x) = x^2 + x + [1]_5$ in $\mathbb{Z}/5[x]$. Show that $\displaystyle R= \frac{\mathbb{Z}/5[x]}{(f)}$ is a field, and determine the number of elements of $R$.




\end{enumerate}



\end{document}