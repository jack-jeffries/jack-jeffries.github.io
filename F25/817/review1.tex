\documentclass[12pt]{article}
\usepackage{amssymb,amsfonts, amsmath, amscd}
\usepackage{fancyhdr}


\setlength{\oddsidemargin}{0cm}
\setlength{\evensidemargin}{0cm}
\setlength{\topmargin}{-.25in}
\setlength{\textheight}{9.2in}
\setlength{\textwidth}{6.5in}


\begin{document}


\pagenumbering{gobble}

\centerline{{\bf Math 817 Review Sheet \#1}}
\smallskip
\centerline{{A (not necessarily complete) list of important things to know for Exam I}}

\begin{description}

\item[Examples of Groups:]  Cyclic groups (i.e., $\mathbf Z$ and $\mathbf Z_n$), matrix groups (e.g., $\operatorname{GL}_n(\mathbf R)$), the Klein 4-group, the dihedral groups,
the quaternions, permuations groups (i.e., $S_n$ and $A_n$).

\item[Orders of elements:] If $|x|=n$ then $x^s=1$ if and only if $n$ divides $s$.  The order of an element equals the order of the cyclic subgroup it generates.

\item[Subgroups of Cyclic Groups:]Let $G=\langle x\rangle$ be a cyclic group of finite order $n$.  Then every subgroup of $G$ is cyclic.  Moreover, for every positive divisor $d$ of $n$ there is a unique subgroup of $G$ order $d$, namely $\langle x^{\frac{n}{d}} \rangle$.

\item[Cosets:]  Let $G$ be a group, $H$ a subgroup, and $x\in G$.  The left coset $xH$ is defined to be $\{xh\in H\}$.   The set of left cosets of $H$ partitions the group $G$.  Also, $|xH|=|H|$ for every $x\in H$.  (This gives us Lagrange's Theorem below.)  We let $[G:H]$ (the index of $H$ in $G$) be the number of left cosets.  Another useful fact: $xH=yH$ if and only if $y^{-1}x\in H$.

\item[Cayley's Theorem:]  Let $G$ be a group and $H$ a subgroup such that $[G:H]=n$.  Then, by letting $G$ act on the set of left cosets of $H$ we obtain a homomorphism $\phi:G\to S_n$.  The $\ker \phi$ is the largest normal subgroup contained in $H$.  Also, $\ker \phi =\bigcap_{x\in G} xHx^{-1}$.

\item[Lagrange's Theorem:]  If $G$ is a finite group and $H$ is a subgroup then $|G|=|H|[G:H]$.  In particular, $|H|$ divides $|G|$.

\item[Products of Subgroups:] Let $H$ and $K$ be subgroups of $G$.  Then $HK:=\{hk\mid h\in H, k\in K\}$.   
\begin{enumerate}
\item $|HK|=\frac{|H||K|}{|H\cap K|}$.  
\item $HK$ is a subgroup if and only if $HK=KH$.
\item If either $H$ or $K$ is normal then $HK$ is a subgroup.
\end{enumerate}

\item[Group Actions]  Suppose $G$ acts on a set $X$.
\begin{enumerate}
\item For $a\in X$, the orbit of $a$ is $\mathcal{O}(a):=\{ga \mid g\in G\}$.
\item For $a\in X$, the stabilizer of $a$ is $G_a:=\{g\in G\mid ga=a\}$.
\item (Orbit-Stabilizer Theorem) For any $a\in X$, $|\mathcal{O}(a)|=[G:G_a]$.  In particular, the size of any orbit divides the order of $G$.
\item The action is called {\it faithful} if $\bigcap_{a\in X} G_a=\{1\}$; equivalently, for any $g\in G$ there exists an $a\in X$ such that $ga\neq a$.
\item The action is called {\it transitive} if there is only one orbit; equivalently, for any $a,b\in X$ there exists a $g\in G$ such that $ga=b$.
\end{enumerate}


\item[Class Formula:]  If $G$ is a finite group then 
$$|G|=|Z(G)|+\sum_i [G:C_G(g_i)]$$
where $g_i$ runs through the distinct conjugacy classes of $G$ with more than one element.

\item[Quotient Groups:] Let $K$ be a normal subgroup of $G$.
\begin{enumerate}
\item The set of left cosets of $K$ form a group under coset multiplication.  We denote this group by $G/K$.
\item $|G/K|=|G|/|K|$.
\item If $H$ is a subgroup of $G$ containing $K$ then $H/K$ is a subgroup of $G/K$.  Moreover, any subgroup of $G/K$ can be written uniquely in the form $H/K$ where $H$ is subgroup of $G$ containing $K$.
\item If $H$ is a subgroup of $G$ containing $K$ then $H/K \triangleleft G/K$ if and only if $H\triangleleft G$.  
\end{enumerate}

\item[Isomorphism Theorems:]
\begin{enumerate}  
\item Let $G$ be a group and $K\subseteq H$ normal subgroups of $G$.  Then $H/K \triangleleft G/K$ if and only if $H\triangleleft G$.
\item Let $\phi:G\to G'$ be a group homomorphism.  Then $G/\ker \phi \cong \operatorname{im} \phi$.
\item Let $H$ and $K$ be subgroups of $G$ with $K\triangleleft G$.  Then $H\cap K\triangleleft H$ and $HK/K \cong H/H\cap K$.
\end{enumerate}




\item[Cauchy's Theorem:]  If $G$ is a finite group and $p$ a prime divisor of $|G|$ then there exists an element of $G$ of order $p$.  

\item[Automorphisms of finite cyclic groups] Let $C_n=\langle x \rangle$ be a cyclic group of order $n$.  For $i\in \mathbb Z$, let $\phi_i:C_n\to C_n$ by the map defined by $\phi_i(x^j)=x^{ij}$ for all $j$.  Then $\phi_i$ is an automorphism if and only if $\gcd(i,n)=1$.  Hence, 
$$\operatorname{Aut}(C_n)=\{\phi_i\mid \gcd(i,n)=1\}\cong \mathbb Z_n^{\times}.$$



\end{description}

\end{document}