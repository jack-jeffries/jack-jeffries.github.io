\documentclass[12pt]{amsart}


\usepackage{times}
\usepackage[margin=0.7in]{geometry}
\usepackage{paralist,amsmath,amssymb,multicol,graphicx,framed,ifthen,color,xcolor,stmaryrd,enumitem,colonequals}
\usepackage[outline]{contour}
\contourlength{.4pt}
\contournumber{10}
\newcommand{\Bold}[1]{\contour{black}{#1}}

\definecolor{chianti}{rgb}{0.6,0,0}
\definecolor{meretale}{rgb}{0,0,.6}
\definecolor{leaf}{rgb}{0,.35,0}
\newcommand{\Q}{\mathbb{Q}}
\newcommand{\N}{\mathbb{N}}
\newcommand{\Z}{\mathbb{Z}}
\newcommand{\R}{\mathbb{R}}
\newcommand{\C}{\mathbb{C}}
\newcommand{\e}{\varepsilon}
\newcommand{\inv}{^{-1}}
\newcommand{\dabs}[1]{\left| #1 \right|}
\newcommand{\ds}{\displaystyle}
\newcommand{\solution}[1]{\ifthenelse {\equal{\displaysol}{1}} {\begin{framed}{\color{meretale}\noindent #1}\end{framed}} { \ }}
\newcommand{\solutione}[1]{\ifthenelse {\equal{\displaysol}{1}} {\begin{framed}{\color{leaf}This solution is embargoed.}\end{framed}} { \ }}
\newcommand{\showsol}[1]{\def\displaysol{#1}}

\newcommand{\rsa}{\rightsquigarrow}


\newcommand\itemA{\stepcounter{enumi}\item[{\Bold{(\theenumi)}}]}
\newcommand\itemB{\stepcounter{enumi}\item[(\theenumi)]}
\newcommand\itemC{\stepcounter{enumi}\item[{\it{(\theenumi)}}]}
\newcommand\itema{\stepcounter{enumii}\item[{\Bold{(\theenumii)}}]}
\newcommand\itemb{\stepcounter{enumii}\item[(\theenumii)]}
\newcommand\itemc{\stepcounter{enumii}\item[{\it{(\theenumii)}}]}
\newcommand\itemai{\stepcounter{enumiii}\item[{\Bold{(\theenumiii)}}]}
\newcommand\itembi{\stepcounter{enumiii}\item[(\theenumiii)]}
\newcommand\itemci{\stepcounter{enumiii}\item[{\it{(\theenumiii)}}]}
\newcommand\ceq{\colonequals}


\DeclareMathOperator{\ord}{ord}

\DeclareMathOperator{\res}{res}
\setlength\parindent{0pt}
%\usepackage{times}

%\addtolength{\textwidth}{100pt}
%\addtolength{\evensidemargin}{-45pt}
%\addtolength{\oddsidemargin}{-60pt}

\pagestyle{empty}
%\begin{document}\begin{itemize}

%\thispagestyle{empty}




\begin{document}
\showsol{0}
	
	\thispagestyle{empty}
	
	\section*{Permutation Groups}
	
	

\begin{framed}

\textsc{Definition:} Given a set $X$, the \textbf{permuatation group} on $X$ is the set $\mathrm{Perm}(X)$ of bijective functions on $X$. This is a group with composition of functions as the operation. The \textbf{symmetric group} $S_n$ is the permuation group on the set $[n]:= \{1,\dots,n\}$. % We typically write the composition of two permutations $\sigma, \tau$ simply as $\sigma \tau$ ($=$ first $\tau$ then $\sigma$).

\

 A \textbf{cycle} is a particular type of permutation. By way of example, in $S_7$:
 \begin{itemize}
 \item  $\alpha = (2 \ 4 \ 5)$ is a 3-cycle. It is the permutation given by $\alpha(2)=4$, $\alpha(4)=5$, $\alpha(5)=2$, and $\alpha(i)= i$ for $i\neq 2,4,5$.
 \item $\beta = (1 \ 6 \ 5 \ 4)$ is a 4-cycle. It is the permutation given by $\alpha(1)=6$, $\alpha(6)=5$, $\alpha(5)=4$, $\alpha(4)=1$, and $\alpha(i)= i$ for $i\neq 1,6,5,4$.
 \end{itemize}
 We will not consider 1-cycles. A 2-cycle is also called a \textbf{transposition}. 
\end{framed}

\begin{enumerate}
\itemA Warming up with cycles: Consider the symmetric group $S_5$.
\begin{enumerate}
\itema Write out the cycle $(1\, 4\, 3)$ explicitly as a function by listing the input and output values.
\itema Write out the product of cycles $(1 \, 3 \, 5) (2 \,5)$ explicitly as a function by listing the input and output values.
\itema Which of the following expressions yield the same permutation:
\begin{itemize}
\item $(1 \, 5 \, 3 \, 4)$
\item $(1 \, 4 \, 3 \, 5)$
\item $(3 \, 4 \, 1 \, 5)$
\end{itemize}
\itema What is the inverse of $(1 \, 5 \, 3 \, 4)$? How would you find the inverse of a cycle in general?
\itema What is the \emph{order}\footnote{Recall that the \textbf{order} of an element $g$ in a group $G$ is the smallest integer $n>0$ such that $g^n=e$.} of $(1 \, 5 \, 3 \, 4)$? How would you find the order of a cycle in general?

\end{enumerate}

\


\itemB Show the following \textsc{Lemma:} For any distinct $i_1,\dots,i_p\in [n]$,
\[  (i_1 \ i_2 \ \cdots \ i_p) = (i_1 \ i_2) (i_2 \ i_3) \cdots (i_{p-1} \ i_p).\]


\end{enumerate}






\begin{framed}
We say that two cycles $\sigma = (i_1 \, i_2 \, \cdots \, i_n)$ and $\tau = (j_1 \, j_2 \, \cdots \, j_m)$ are \textbf{disjoint} if $i_a \neq j_b$ for all $a,b$.

\


\textsc{Theorem 1:} Let $n\geq 1$ be an integer, and consider the symmetric group $S_n$.
\begin{enumerate}
\item Every permutation $\sigma \in S_n$ is equal to a product of disjoint cycles. 
\item Disjoint cycles commute: if $\sigma, \tau$ are disjoint cycles, then $\sigma\tau=\tau\sigma$.
\item The expression of a permutation $\sigma$ as a product of disjoint cycles is unique up to permuting factors.
\end{enumerate}


\

The \textbf{cycle type} of a permutation is the list of the lengths of the cycles in its expression as a product of disjoint cycles. 
\end{framed}

\begin{enumerate}
\setcounter{enumi}{2}
\itemA Theorem 1(1) in action: To write $\sigma\in S_n$ as a product of disjoint cycles,
\begin{itemize} 
\item Start with $1\in [n]$,
\item Look at $\sigma(1), \sigma^2(1), \dots$ until we get back to $1=\sigma^m(1)$. Make a cycle out of these: \\
\[ (1 \ \sigma(1)\ \sigma^2(1) \ \cdots \ \sigma^{m-1}(1) ).\]
\item Look at the smallest element of $[n]$ that hasn't appeared, and repeat.
\item Throw away the 1-cycles at the end.
\end{itemize}

\begin{enumerate}
\itema Write the following permutation in $S_7$ as a product of disjoint cycles:
\[\begin{array}{c||c|c|c|c|c|c|c}
 i & 1 & 2 & 3 & 4 & 5 & 6 & 7 \\ \hline
 \sigma(i) & 6 & 7 & 2 & 4 & 3 & 6 & 5
\end{array}
\]


\itema Write the following product of nondisjoint cycles in $S_7$ as a product of disjoint cycles: \[ (1 \ 3 \ 5 \ 7) (2 \ 3 \ 4 \ 5).\]

\itema What is the cycle type of $(1\ 2)( 3\ 4)$? What is the cycle type of $(1\ 2)(2 \ 3)$?
\end{enumerate}

\

\itemB Proof of Theorem 1: 
\begin{enumerate}
\itemb What is the key idea to prove part (1) of Theorem 1?
\itemb Prove part (2) of  Theorem 1.
\itemb Complete the proofs of parts (1) and (3) of Theorem 1.
\end{enumerate}


\end{enumerate}


\



\begin{framed}
\textsc{Theorem 2:} Let $n\geq 1$ be an integer, and consider the symmetric group $S_n$.
\begin{enumerate}
\item Every permutation $\sigma \in S_n$ is equal to a product of transpositions; thus, $S_n$ is \textbf{generated}\footnotemark by transpositions.
\item For a fixed $\sigma\in S_n$, either
\begin{itemize}
\item every expression of $\sigma$ as a product of transpositions involves an \textit{even} number of transpositions, or
\item every expression of $\sigma$ as a product of transpositions involves an \textit{odd} number of transpositions.
\end{itemize}
\end{enumerate}
In the first case, we say that $\sigma$ is an \textbf{even} permutation and define $\mathrm{sign}(\sigma) = 1$;
in the second case, we say that $\sigma$ is an \textbf{odd} permutation and define $\mathrm{sign}(\sigma) = -1$.
\end{framed}
\footnotetext{Recall that a group $G$ is \textbf{generated} by a set $S$ if every element of $G$ can be written as a product of elements of $S$ and their inverses.}

\


\begin{enumerate}
\setcounter{enumi}{4}
\itemA Signs of permutations:
\begin{enumerate}
\itema What is the sign of a transposition? Of a 3-cycle? Of a $p$-cycle? (Hint: Use the Lemma.)
\itema If the cycle type of $\sigma$ is $m_1,m_2,\dots,m_t$, then what is the sign of $\sigma$?
\end{enumerate}

\

\itemB Proving Theorem 2:
\begin{enumerate}
\itemb Prove the Lemma.
\itemb Explain how part (1) of Theorem 2 follows from the Lemma and Theorem 1.
\itemb Explain why part (2) of Theorem 2 reduces to the following claim: if $\tau_1,\dots,\tau_m$ are transpositions and $\tau_1\cdots \tau_m = e$, then $m$ is even.

\itemb Reconsider the claim above in the equivalent form: if  $\tau_1,\dots,\tau_m$ are transpositions and $m$ is odd, then $\tau_1\cdots \tau_m \neq e$. Proceed by induction on $m$ odd. Resolve the base case.
\itemb Show\footnote{Hint: You might find it useful to show that 
	$$(cd)(ab) = (ab)(cd) \qquad \textrm{and} \qquad (bc)(ab) = (ac)(bc)$$
	for all distinct $a, b, c, d$ in $[n]$.}
 the inductive step, and complete the proof.
\end{enumerate}


\end{enumerate}













\end{document}
