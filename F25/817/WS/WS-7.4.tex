\documentclass[12pt]{amsart}


\usepackage{times}
\usepackage[margin=.8in]{geometry}
\usepackage{paralist,amsmath,amssymb,multicol,graphicx,framed,ifthen,color,xcolor,stmaryrd,colonequals}
\usepackage[shortlabels]{enumitem}
\usepackage[all]{xy}
\usepackage[outline]{contour}
\contourlength{.4pt}
\contournumber{10}
\newcommand{\Bold}[1]{\contour{black}{#1}}

\definecolor{chianti}{rgb}{0.6,0,0}
\definecolor{meretale}{rgb}{0,0,.6}
\definecolor{leaf}{rgb}{0,.35,0}
\newcommand{\Q}{\mathbb{Q}}
\newcommand{\N}{\mathbb{N}}
\newcommand{\Z}{\mathbb{Z}}
\newcommand{\R}{\mathbb{R}}
\newcommand{\C}{\mathbb{C}}
\newcommand{\e}{\varepsilon}
\newcommand{\inv}{^{-1}}
\newcommand{\dabs}[1]{\left| #1 \right|}
\newcommand{\ds}{\displaystyle}
\newcommand{\solution}[1]{\ifthenelse {\equal{\displaysol}{1}} {\begin{framed}{\color{meretale}\noindent #1}\end{framed}} { \ }}
\newcommand{\solutione}[1]{\ifthenelse {\equal{\displaysol}{1}} {\begin{framed}{\color{leaf}This solution is embargoed.}\end{framed}} { \ }}
\newcommand{\showsol}[1]{\def\displaysol{#1}}

\newcommand{\rsa}{\rightsquigarrow}


\newcommand\itemA{\stepcounter{enumi}\item[{\Bold{(\theenumi)}}]}
\newcommand\itemB{\stepcounter{enumi}\item[(\theenumi)]}
\newcommand\itemC{\stepcounter{enumi}\item[{\it{(\theenumi)}}]}
\newcommand\itema{\stepcounter{enumii}\item[{\Bold{(\theenumii)}}]}
\newcommand\itemb{\stepcounter{enumii}\item[(\theenumii)]}
\newcommand\itemc{\stepcounter{enumii}\item[{\it{(\theenumii)}}]}
\newcommand\itemai{\stepcounter{enumiii}\item[{\Bold{(\theenumiii)}}]}
\newcommand\itembi{\stepcounter{enumiii}\item[(\theenumiii)]}
\newcommand\itemci{\stepcounter{enumiii}\item[{\it{(\theenumiii)}}]}
\newcommand\ceq{\colonequals}


\DeclareMathOperator{\ord}{ord}

\DeclareMathOperator{\res}{res}
\setlength\parindent{0pt}
%\usepackage{times}

%\addtolength{\textwidth}{100pt}
%\addtolength{\evensidemargin}{-45pt}
%\addtolength{\oddsidemargin}{-60pt}

\pagestyle{empty}
%\begin{document}\begin{itemize}

%\thispagestyle{empty}




\begin{document}
\showsol{0}
	
	\thispagestyle{empty}
	
	\section*{Classifying abelian groups, and others, up to isomorphism}
	
	
\begin{framed}
\textsc{Structure Theorem for Finite Generated Abelian Groups: Invariant Factors:} \\
Let $G$ be a finitely generated abelian group. There exist integers $r \geqslant 0$, and $n_i \geqslant 2$, satisfying $n_1 \mid n_2 \mid \cdots \mid n_t$ such that 
  $$
  G \cong \mathbb{Z}^r \times \mathbb{Z}/n_1 \times \dots \times \mathbb{Z}/n_t.
  $$
  Moreover, the list $r,n_1,\ldots , n_t$ is uniquely determined by $G$.


\

\textsc{Structure Theorem for Finite Generated Abelian Groups: Elementary Divisors:} 
Let $G$ be a finitely generated abelian group. Then there exist integers $r \geqslant 0$, not necessarily distinct positive prime integers $p_1, \cdots, p_s$, and integers $a_i \geqslant 1$ for $1 \leqslant i \leqslant s$ such that
$$G \cong \mathbb{Z}^r\times \mathbb{Z}/p_1^{a_1} \times \cdots \times \mathbb{Z}/p_s^{a_s}.$$
Moreover, $r$ and $s$ are uniquely determined by $G$, and the list of prime powers $p_1^{a_1}, \dots, p_s^{a_s}$ is unique up to the ordering. 

\end{framed}

	
	\begin{enumerate}
	\itemA Converting between forms:
\begin{enumerate}
\item[] To convert a cyclic group $\Z/a$ to elementary divisor form, write each $a=p_1^{e_1}\cdots p_s^{e_s}$ as a product of prime powers, and use CRT get
\[ \Z/a \cong \Z/{p_1^{e_1}} \times \cdots \times \Z/{p_s^{e_s}}.\]
\itema Convert $\Z^2 \times \Z/50 \times \Z/60$ to elementary divisor form.
\item[] To convert a group  from elementary divisor form to invariant factor form, 
\begin{itemize} 
\item For each distinct prime $p_j$ occurring, take the largest power $E_j$ it has in an elementary divisor, and combine and combine $\prod_j \Z/ p_j^{E_j} \cong \Z/(p_1^{E_1} \cdots p_\ell^{E_\ell})$ via CRT. If there's more than one copy of  $\Z/p_j^{E_j}$, just take one of the copies and leave the rest.
\item Repeat with the remaining factors.
\end{itemize}
\itema Convert $\Z^3 \times \Z/4 \times \Z/4 \times \Z/9 \times \Z/27 \times \Z/25$ to invariant factor form.
\end{enumerate}

\

	
	\itemA Which of the following groups of order $2160$ are isomorphic or not?
	\begin{itemize} 
	\item $\Z/5 \times \Z/12 \times \Z/36$
	\item $\Z/10 \times \Z/12 \times \Z/18$
	\item $\Z/30 \times \Z/54$
	\end{itemize}
	
	\
	
	
	
	\itemA Classify all \emph{abelian} groups of order $72$ up to isomorphism. For each isomorphism class,  give its expression in invariant factor form. 
	
	\
	
	\itemA Let $p<q$ be primes.
	\begin{enumerate}
	\itema Show that if $p$ does not divide $q-1$, then any group of order $pq$ is isomorphic to $C_{pq}$ by the following steps:
	\begin{itemize}
	\item Use Sylow's Theorem to count the number of Sylow subgroups.
	\item Apply the Recognition Theorem for direct products.
	\end{itemize} 
	\itema Show from that if $p$ does divide $q-1$, then there are exactly two groups of order $pq$ up to isomorphism by the following steps:
		\begin{itemize}
	\item Use Sylow's Theorem to count the number of Sylow subgroups.
	\item Apply the Recognition Theorem for semidirect products.
	\item Use an Exercise from class about when two semidirect products are isomorphic.
	\end{itemize} 
	\end{enumerate}
	
	\
	
	
	\itemB Let $p$ be a prime integer. Let $G$ be a group of order $p^2$.
	\begin{enumerate}
	\itemb Show\footnote{Hint: If not, what can you say about $Z(G)$ and $G/Z(G)$?} that $G$ is abelian.
	\itemb Classify all groups of order $p^2$ up to isomorphism.
	\end{enumerate}
	
	\
	
	\itemB Let $p,q$ be primes such that $q=p+2$ and $p\geq 5$. Show that any group of order $p^2q^2$ is either isomorphic to a cyclic group or a product of two cyclic groups.  
	


\end{enumerate}

\end{document}
