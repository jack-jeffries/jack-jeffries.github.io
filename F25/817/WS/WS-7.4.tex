\documentclass[12pt]{amsart}


\usepackage{times}
\usepackage[margin=1in]{geometry}
\usepackage{paralist,amsmath,amssymb,multicol,graphicx,framed,ifthen,color,xcolor,stmaryrd,colonequals}
\usepackage[shortlabels]{enumitem}
\usepackage[all]{xy}
\usepackage[outline]{contour}
\contourlength{.4pt}
\contournumber{10}
\newcommand{\Bold}[1]{\contour{black}{#1}}

\definecolor{chianti}{rgb}{0.6,0,0}
\definecolor{meretale}{rgb}{0,0,.6}
\definecolor{leaf}{rgb}{0,.35,0}
\newcommand{\Q}{\mathbb{Q}}
\newcommand{\N}{\mathbb{N}}
\newcommand{\Z}{\mathbb{Z}}
\newcommand{\R}{\mathbb{R}}
\newcommand{\C}{\mathbb{C}}
\newcommand{\e}{\varepsilon}
\newcommand{\inv}{^{-1}}
\newcommand{\dabs}[1]{\left| #1 \right|}
\newcommand{\ds}{\displaystyle}
\newcommand{\solution}[1]{\ifthenelse {\equal{\displaysol}{1}} {\begin{framed}{\color{meretale}\noindent #1}\end{framed}} { \ }}
\newcommand{\solutione}[1]{\ifthenelse {\equal{\displaysol}{1}} {\begin{framed}{\color{leaf}This solution is embargoed.}\end{framed}} { \ }}
\newcommand{\showsol}[1]{\def\displaysol{#1}}

\newcommand{\rsa}{\rightsquigarrow}


\newcommand\itemA{\stepcounter{enumi}\item[{\Bold{(\theenumi)}}]}
\newcommand\itemB{\stepcounter{enumi}\item[(\theenumi)]}
\newcommand\itemC{\stepcounter{enumi}\item[{\it{(\theenumi)}}]}
\newcommand\itema{\stepcounter{enumii}\item[{\Bold{(\theenumii)}}]}
\newcommand\itemb{\stepcounter{enumii}\item[(\theenumii)]}
\newcommand\itemc{\stepcounter{enumii}\item[{\it{(\theenumii)}}]}
\newcommand\itemai{\stepcounter{enumiii}\item[{\Bold{(\theenumiii)}}]}
\newcommand\itembi{\stepcounter{enumiii}\item[(\theenumiii)]}
\newcommand\itemci{\stepcounter{enumiii}\item[{\it{(\theenumiii)}}]}
\newcommand\ceq{\colonequals}


\DeclareMathOperator{\ord}{ord}

\DeclareMathOperator{\res}{res}
\setlength\parindent{0pt}
%\usepackage{times}

%\addtolength{\textwidth}{100pt}
%\addtolength{\evensidemargin}{-45pt}
%\addtolength{\oddsidemargin}{-60pt}

\pagestyle{empty}
%\begin{document}\begin{itemize}

%\thispagestyle{empty}




\begin{document}
\showsol{0}
	
	\thispagestyle{empty}
	
	\section*{Classifying groups up to isomorphism}
	
	
	\
	
	
	\begin{enumerate}
	\itemA Classify all \emph{abelian} groups of order $72$ up to isomorphism. For each isomorphism class,  give its expression in invariant factor form. 
	
	\
	
	\itemA Let $p<q$ be primes.
	\begin{enumerate}
	\itema Show that if $p$ does not divide $q-1$, then any group of order $pq$ is isomorphic to $C_{pq}$ by the following steps:
	\begin{itemize}
	\item Use Sylow's Theorem to count the number of Sylow subgroups.
	\item Apply the Recognition Theorem for direct products.
	\end{itemize} 
	\itema Show from that if $p$ does divide $q-1$, then there are exactly two groups of order $pq$ up to isomorphism by the following steps:
		\begin{itemize}
	\item Use Sylow's Theorem to count the number of Sylow subgroups.
	\item Apply the Recognition Theorem for semidirect products.
	\item Use an Exercise from class about when two semidirect products are isomorphic.
	\end{itemize} 
	\end{enumerate}
	
	\
	
	
	\itemB Let $p$ be a prime integer. Let $G$ be a group of order $p^2$.
	\begin{enumerate}
	\itemb Show\footnote{Hint: If not, what can you say about $Z(G)$ and $G/Z(G)$?} that $G$ is abelian.
	\itemb Classify all groups of order $p^2$ up to isomorphism.
	\end{enumerate}
	
	\
	
	\itemB Let $p,q$ be primes such that $q=p+2$ and $p\geq 5$. Show that any group of order $p^2q^2$ is either isomorphic to a cyclic group or a product of two cyclic groups.  
	


\end{enumerate}

\end{document}
