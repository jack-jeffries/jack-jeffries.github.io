\documentclass[12pt]{amsart}


\usepackage{times}
\usepackage[margin=1.1in]{geometry}
\usepackage{paralist,amsmath,amssymb,multicol,graphicx,framed,ifthen,color,xcolor,stmaryrd,colonequals}
\usepackage[shortlabels]{enumitem}
\usepackage[all]{xy}
\usepackage[outline]{contour}
\contourlength{.4pt}
\contournumber{10}
\newcommand{\Bold}[1]{\contour{black}{#1}}

\definecolor{chianti}{rgb}{0.6,0,0}
\definecolor{meretale}{rgb}{0,0,.6}
\definecolor{leaf}{rgb}{0,.35,0}
\newcommand{\Q}{\mathbb{Q}}
\newcommand{\N}{\mathbb{N}}
\newcommand{\Z}{\mathbb{Z}}
\newcommand{\R}{\mathbb{R}}
\newcommand{\C}{\mathbb{C}}
\newcommand{\e}{\varepsilon}
\newcommand{\inv}{^{-1}}
\newcommand{\dabs}[1]{\left| #1 \right|}
\newcommand{\ds}{\displaystyle}
\newcommand{\solution}[1]{\ifthenelse {\equal{\displaysol}{1}} {\begin{framed}{\color{meretale}\noindent #1}\end{framed}} { \ }}
\newcommand{\solutione}[1]{\ifthenelse {\equal{\displaysol}{1}} {\begin{framed}{\color{leaf}This solution is embargoed.}\end{framed}} { \ }}
\newcommand{\showsol}[1]{\def\displaysol{#1}}

\newcommand{\rsa}{\rightsquigarrow}


\newcommand\itemA{\stepcounter{enumi}\item[{\Bold{(\theenumi)}}]}
\newcommand\itemB{\stepcounter{enumi}\item[(\theenumi)]}
\newcommand\itemC{\stepcounter{enumi}\item[{\it{(\theenumi)}}]}
\newcommand\itema{\stepcounter{enumii}\item[{\Bold{(\theenumii)}}]}
\newcommand\itemb{\stepcounter{enumii}\item[(\theenumii)]}
\newcommand\itemc{\stepcounter{enumii}\item[{\it{(\theenumii)}}]}
\newcommand\itemai{\stepcounter{enumiii}\item[{\Bold{(\theenumiii)}}]}
\newcommand\itembi{\stepcounter{enumiii}\item[(\theenumiii)]}
\newcommand\itemci{\stepcounter{enumiii}\item[{\it{(\theenumiii)}}]}
\newcommand\ceq{\colonequals}


\DeclareMathOperator{\ord}{ord}

\DeclareMathOperator{\res}{res}
\setlength\parindent{0pt}
%\usepackage{times}

%\addtolength{\textwidth}{100pt}
%\addtolength{\evensidemargin}{-45pt}
%\addtolength{\oddsidemargin}{-60pt}

\pagestyle{empty}
%\begin{document}\begin{itemize}

%\thispagestyle{empty}




\begin{document}
\showsol{0}
	
	\thispagestyle{empty}
	
	\section*{Ideals}
	
	
\begin{framed}
\textsc{Definition:} Let $R$ be a ring. An \textbf{ideal} of $R$ (also called a \textbf{two-sided ideal}) is a nonempty subset of $R$ such that
\begin{enumerate}
\item $I$ is closed under addition: for all $a,b\in I$, we have $a+b\in I$.
\item $I$ absorbs multiplication: for all $r\in R$ and $a\in I$, we have $ra\in I$ and $ar\in I$.
\end{enumerate}
A \textbf{left ideal} of $R$ is a nonempty subset of $R$ such that
\begin{enumerate}
\item $I$ is closed under addition: for all $a,b\in I$, we have $a+b\in I$.
\item[(2$\ell$)] $I$ absorbs left multiplication: for all $r\in R$ and $a\in I$, we have $ra\in I$.
\end{enumerate}
The definition of \textbf{right ideal} is analogous

\

\textsc{Lemma 1 (General recipes for ideals):} Let $R$ be a ring.
\begin{enumerate}[(i)]
\item If $I,J$ are ideals, then $I+J \colonequals \{ a + b \ | \ a\in I, b\in J\}$ is an ideal.
\item If $I,J$ are ideals, then $IJ := \{ \sum_{i=1}^n a_i b_i \ | \ a_i \in I, b_i\in J\}$ is an ideal.
\item If $\{ I_{\alpha} \}_{\alpha\in A}$ is an arbitrary collection of ideals of $R$, then $\bigcap_{\alpha\in A} I_\alpha$ is an ideal.
\item If $\{ I_{\alpha} \}_{\alpha\in A}$ is a \emph{chain}\footnotemark\,of ideals, then $\bigcup_{\alpha\in A} I_\alpha$ is an ideal.
\end{enumerate}
\end{framed}
\footnotetext[1]{This means that for all $\alpha,\beta\in A$, either $I_\alpha\subseteq I_\beta$ or $I_\beta\subseteq I_\alpha$.}
	
	\begin{enumerate}
	\itemA Working with the definition: 
	\begin{enumerate}
	\itema If $I$ is an ideal (or left ideal) of $R$, explain why $0\in I$ and $(I,+)$ is a subgroup of $(R,+)$.
	\itema Very quickly explain why  $\{0\}$ and $R$ are ideals of $R$. We say that an ideal is \textbf{nontrivial} if $I\neq 0$ and proper if $I\neq R$.
	\itema Explain why an ideal $I\subseteq R$ is proper if and only if $1\notin I$.
	\itema Quickly explain why ``ideal,'' ``left ideal,'' and ``right ideal'' are identical notions in a commutative ring.
	\end{enumerate} 

	\
	
	
	\itemA Show that the subset 
	\[ \left\{ \begin{bmatrix} a & 0 \\ b & 0 \end{bmatrix} \ | \ a,b\in \mathbb{R} \right\} \subseteq \mathrm{Mat}_2(\mathbb{R})\] 
	is a left ideal, but is not a (two-sided) ideal.
	
	\
	
	\itemB Prove one or two parts of Lemma 1.
	
	\
	
	\itemB Show\footnote{Hint: Consider $2\mathbb{Z} , 3\mathbb{Z} \subseteq \mathbb{Z}$.} that the union of two ideals does not have to be an ideal in general. 


\end{enumerate}

\newpage

\begin{framed}

\textsc{Definition:} Let $R$ be a ring. and $S \subseteq R$ be a subset. The \textbf{ideal generated by $S$} is the ideal
\[ (S) = \bigcap\limits_{\substack{{I \ \mathrm{ideal}} \\ {I \supseteq S} }} I. \]
An ideal is \textbf{principal} if $I=(a)$ for a single element $a\in R$.

\

\textsc{Lemma 2 (Ideal generated a subset):} Let $R$ be a ring and $S\subseteq R$. 
\begin{enumerate}[(i)]
\item There is an equality $(S) = \{ \sum_{i=1}^n r_i a_i r'_i \ | \ r_i, r'_i \in R, a_i \in S\}$.
\item If $R$ is commutative, then $(S) = \{ \sum_{i=1}^n r_i a_i \ | \ r_i \in R, a_i \in S\}$.
\item If $R$ is commutative and $a\in R$, then $(a) = \{ ra \ | \ r\in R\}$.
\end{enumerate}
\end{framed}

\begin{enumerate}
\setcounter{enumi}{4}
\itemA Let $R=\mathbb{Z}[x]$. Use the Lemma to quickly explain the following:
\begin{enumerate}
\itema $(2)$ is the set of all integer polynomials with every coefficient even.
\itema $(x)$ is the set of all integer polynomials with zero constant term.
\itema $(2,x)$ is the set of all integer polynomials with even constant term.
\end{enumerate}

\

\itemB Show\footnote{Hint: If $(2,x) = (f)$, with $f=a_0 + a_1 x + \cdots +a_n x^n$, note that $2, x \in (f)$. What can you say about $a_0$?}  that the ideal $(2,x)$ in $\mathbb{Z}[x]$ is not principal.

\


\itemB Show that if $R$ is noncommutative then one can\footnote{Hint: You can use the fact that you will prove in HW\#11 that $\mathrm{Mat}_n(F)$ has no nontrivial proper ideals if $F$ is a field.} have $(a) \supsetneqq \{ r a r' \ | \ r, r' \in R\}$.
\end{enumerate}


\end{document}
