\documentclass[12pt]{amsart}


\usepackage{times}
\usepackage[margin=.65in]{geometry}
\usepackage{paralist,amsmath,amssymb,multicol,graphicx,framed,ifthen,color,xcolor,stmaryrd,enumitem,colonequals}
\usepackage[outline]{contour}
\contourlength{.4pt}
\contournumber{10}
\newcommand{\Bold}[1]{\contour{black}{#1}}

\definecolor{chianti}{rgb}{0.6,0,0}
\definecolor{meretale}{rgb}{0,0,.6}
\definecolor{leaf}{rgb}{0,.35,0}
\newcommand{\Q}{\mathbb{Q}}
\newcommand{\N}{\mathbb{N}}
\newcommand{\Z}{\mathbb{Z}}
\newcommand{\R}{\mathbb{R}}
\newcommand{\C}{\mathbb{C}}
\newcommand{\Zc}{Z}
\newcommand{\e}{\varepsilon}
\newcommand{\inv}{^{-1}}
\newcommand{\dabs}[1]{\left| #1 \right|}
\newcommand{\ds}{\displaystyle}
\newcommand{\solution}[1]{\ifthenelse {\equal{\displaysol}{1}} {\begin{framed}{\color{meretale}\noindent #1}\end{framed}} { \ }}
\newcommand{\solutione}[1]{\ifthenelse {\equal{\displaysol}{1}} {\begin{framed}{\color{leaf}This solution is embargoed.}\end{framed}} { \ }}
\newcommand{\showsol}[1]{\def\displaysol{#1}}

\newcommand{\rsa}{\rightsquigarrow}


\newcommand\itemA{\stepcounter{enumi}\item[{\Bold{(\theenumi)}}]}
\newcommand\itemB{\stepcounter{enumi}\item[(\theenumi)]}
\newcommand\itemC{\stepcounter{enumi}\item[{\it{(\theenumi)}}]}
\newcommand\itema{\stepcounter{enumii}\item[{\Bold{(\theenumii)}}]}
\newcommand\itemb{\stepcounter{enumii}\item[(\theenumii)]}
\newcommand\itemc{\stepcounter{enumii}\item[{\it{(\theenumii)}}]}
\newcommand\itemai{\stepcounter{enumiii}\item[{\Bold{(\theenumiii)}}]}
\newcommand\itembi{\stepcounter{enumiii}\item[(\theenumiii)]}
\newcommand\itemci{\stepcounter{enumiii}\item[{\it{(\theenumiii)}}]}
\newcommand\ceq{\colonequals}


\DeclareMathOperator{\ord}{ord}

\DeclareMathOperator{\res}{res}
\setlength\parindent{0pt}
%\usepackage{times}

%\addtolength{\textwidth}{100pt}
%\addtolength{\evensidemargin}{-45pt}
%\addtolength{\oddsidemargin}{-60pt}

\pagestyle{empty}
%\begin{document}\begin{itemize}

%\thispagestyle{empty}




\begin{document}
\showsol{0}
	
	\thispagestyle{empty}
	
	\section*{Subgroups}
	
	

\begin{framed}

\textsc{Definition:} Let $G$ be a group. A nonempty subset $H$ of $G$ is a \textbf{subgroup} of $G$ if $H$ is a group under the the same operation as $G$  (i.e., $h \cdot_H h' = h \cdot_G h'$ for $h,h'\in H$). We write $H\leq G$ to indicate that $H$ is a subgroup of $G$.

\

Any group $G$ has two \textbf{trivial subgroups} $\{e\}$ and $G$.

\

\textsc{Lemma 1:} Let $H$ be a subset of $G$.
\begin{itemize}
\item \textsc{Two step test:} If $H$ is nonempty, $H$ is closed under multiplication\footnotemark\,and $H$ is closed under inverses\footnotemark[1], then $H$ is a subgroup of $G$.
\item \textsc{One step test:} If $H$ is nonempty, and for all $x,y\in H$, $xy^{-1}\in H$, then $H$ is a subgroup of $G$.
\end{itemize}

\

\textsc{Lemma 2 (General recipes for subgroups):} Let $G$ be a group.
\begin{enumerate}
\item If $H\leq  G$ and $K\leq H$, then $K\leq G$.
\item If $\{ H_{\alpha} \}_{\alpha\in J}$ is a collection of subgroups of $G$, then $\bigcap_{\alpha\in J} H_\alpha \leq G$.
\item If $f:G\to H$ is a group homomorphism, then $\mathrm{im}(G) \leq H$.
\item If $f:G\to H$ is a group homomorphism, and $K\leq G$, then $f(K) = \{ f(k) \ | \ k\in K\} \leq H$.
\item If $f:G\to H$ is a group homomorphism, and $K\leq G$, then $\ker(f) \leq G$.
\item The center $Z(G)$ is a subgroup of $G$.
\end{enumerate}


\end{framed}
\footnotetext[1]{A subset $H\subseteq G$ is \textit{closed under multiplication} if $x,y\in H \Rightarrow xy\in H$ and \textit{closed under inverses} if  $x\in H \Rightarrow x^{-1}\in H$.}

\begin{enumerate}
\itemA Proving subsets are subgroups:
\begin{enumerate}
\itema Choose a couple of parts of Lemma 2 and prove them; you can use Lemma 1.
\solution{\begin{enumerate}
	\item By definition, $K$ is a group under the multiplication in $H$, and the multiplication in $H$ is the same as that in $G$, so $K$ is a subgroup of $G$.
	\item First, note that $H$ is nonempty since $e_G \in H_\alpha$ for all $\alpha\in J$. Moreover, given $x,y\in H$, for each $\alpha$ we have $x,y \in H_\alpha$ and hence $xy^{-1} \in H_\alpha$. It follows that $xy^{-1} \in H$. By the Two-Step test, $H$ is a subgroup of $G$. 
	\item Since $G$ is nonempty, then $\mathrm{image}(f)$ must also be nonemtpy; for example, it contains $f(e_G) = e_H$. If $x,y \in \mathrm{image}(f)$, then $x = f(a)$ and $y = f(b)$ for some $a,b \in G$, and hence 
	$$xy^{-1} =f(a)f(b)^{-1} = f(ab^{-1}) \in \mathrm{image}(f).$$
	By the Two-Step Test, $\mathrm{image}(f)$ is a subgroup of $H$. 
	
	\item The restriction $g\!: K \to H$ of $f$ to $K$ is still a group homomorphism, and thus $f(K) = \mathrm{image} g$ is a subgroup of $H$. 
	
	\item Using the One-step test, note that if $x, y \in \ker(f)$, meaning $f(x)=f(y)=e_G$, then 
	$$f(xy^{-1})=f(x)f(y)^{-1}=e_G.$$ 
	This shows that if $x,y\in \ker(f)$ then $xy^{-1}\in \ker(f)$, so $\ker(f)$ is closed for taking inverses. By the Two-Step test, $\ker(f)$ is a subgroup of $G$.
	\item The center $\Zc(G)$ is the kernel of the permutation representation $G\to \mathrm{Perm}(G)$ for the conjugation action, so $\Zc(G)$ is a subgroup of $G$ since the kernel of a homomorphism is a subgroup.
\end{enumerate} 
}
\itema Let $n\geq 3$ and consider the dihedral group $D_n$ of symmetries of the $n$-gon. 
\begin{enumerate}
\item Is the set of all reflections in $D_n$ a subgroup?
\solution{No; the composition of two reflections is not a reflection. Also, the identity is not a reflection.}
\item Is the set of all rotations in $D_n$ a subgroup?
\solution{Yes; the composition of two rotations is a rotation, as is the inverse of any rotation.}
\end{enumerate}
\itema Let $n\in \mathbb{Z}_{\geq 1}$, and define $\mathrm{SL}_n(\mathbb{R})$ to be the set of $n\times n$ real matrices with determinant $1$. Show\footnote{Hint: This becomes very quick with a proper use of Lemma 2.} that $\mathrm{SL}_n(\mathbb{R}) \leq \mathrm{GL}_n(\mathbb{R})$. ($\mathrm{SL}_n(\mathbb{R})$ is called the \textbf{special linear group}.)
\solution{Recall that $\det: \mathrm{GL}_n(\mathbb{R}) \to \mathbb{R}^\times$ is a homomorphism, and the identity of $\mathbb{R}^\times$ is $1$. Thus, this follows from part (5) of Lemma 2.}
\itema Let $n\in \mathbb{Z}_{\geq 1}$. Recall from linear algebra that an $n\times n$ matrix $Q$ is \textit{orthogonal} if $Q^T Q = I$, where~${}^T$ denotes transpose and $I$ denotes the identity matrix. Define $\mathrm{O}_n(\mathbb{R})$ to be the set of $n\times n$ real orthogonal matrices. Show that $\mathrm{O}_n(\mathbb{R}) \leq \mathrm{GL}_n(\mathbb{R})$. ($\mathrm{O}_n(\mathbb{R})$ is called the \textbf{orthogonal group}.)
\solution{We use the two-step test. Let $A,B\in O_n$, so $A^T A= I$ and $B^T B=I$. Since $A$ is square, note that $A A^T=I$ as well. Then $(AB)^T (AB) = B^T A^T AB = B^T B = I$, so $AB\in O_n$. Also, $(A^{-1})^T = (A^T)^{-1}$, so $(A^{-1})^T A^{-1} = (A^T)^{-1} A^{-1} = (A A^T)^{-1} = I^{-1} = I$, so $A^{-1}\in O_n$. Thus, $O_n$ is a group.}
\itema Define $\mathrm{SO}_n(\mathbb{R})$ to be the set of $n\times n$ real orthogonal matrices that have determinant $1$. Show that $\mathrm{SO}_n(\mathbb{R}) \leq \mathrm{GL}_n(\mathbb{R})$. ($\mathrm{SO}_n(\mathbb{R})$ is called the \textbf{special orthogonal group}.)
\solution{By definition, $SO_n = SL_n \cap O_n$, so by part 2 of Lemma 2, this is a subgroup.}
\end{enumerate}

\


\itemB Prove or disprove: The union of two subgroups of a group is a subgroup.

\

\itemB Prove Lemma 1.
\end{enumerate}

\newpage


\begin{framed}

\textsc{Definition:} Let $G$ be a group, and $S\subseteq G$ be a subset. The \textbf{subgroup of $G$ generated by $S$} is the intersection of all subgroups of $G$ that contain $S$:
\[ \langle S \rangle :=  \bigcap_{\substack{ H\leq G  \\ S \subseteq H}} H \]

\

\textsc{Proposition:}  Let $G$ be a group, and $S\subseteq G$ be a subset. Then
\[ \langle S \rangle = \{ x_1^{j_1} \cdots x_m^{j_m} \ | \ x_i \in S, \, j_i \in \mathbb{Z}\}.\]
\end{framed}

\

\begin{enumerate}
\setcounter{enumi}{3}
\itemA Explain why $\bigcap_{\substack{ H\leq G  \\ S \subseteq H}} H$ is a subgroup of $G$, and why it is the \emph{unique smallest} subgroup of $G$ that contains $S$.

\solution{It follows from the Lemma that this is a subgroup. Call this group $K$. If $H$ is a subgroup of $G$ containing $S$, then  $K$ is the intersection of $H$ with some other set, by definition of $K$, so $K \subseteq H$. This means that $K$ is the unique smallest subgroup containing $S$.}



\itemB \textsc{Proof of the Proposition:} Let $K=\{ x_1^{j_1} \cdots x_m^{j_m} \ | \ x_i \in S, \, j_i \in \mathbb{Z}\}$ as in the Proposition.
\begin{enumerate}
\itemb What concrete things do you need to show about $K$, $S$, and subgroups $H \leq G$ to prove this equality?
\itemb Complete the proof.
\end{enumerate}
\end{enumerate}

\

\begin{framed}

\textsc{Cayley's Theorem:} Let $G$ be a finite group of order $n$. Then $G$ is isomorphic to a subgroup of $S_n$.

\end{framed}

\

\begin{enumerate}
\setcounter{enumi}{5}

\itemA Prove\footnote{Hint: Let $G$ act on $G$ by left multiplication.} Cayley's Theorem.
\solution{Let $G$ act on $G$ by left multiplication. This action induces a permutation representation $\rho: G\to \mathrm{Perm}(G)$. We claim that $\rho$ is injective. Indeed, if $\rho(g)$ is the identity permutation, then $gh = g \cdot h = h$ for all $h\in H$, whence $g=e$. If $G$ has $n$ elements, we can label them $1$ through $n$, and identify $\mathrm{Perm}(G)$ with $S_n$; so we have an injective homomorphism $\rho$ from $G$ to $S_n$. Let $H$ be the image of $\rho$; we have an injective homomorphism$\rho'$  from $G$ to $H$, and by definition of image, this is also surjective. Thus $\rho'$ is an isomorphism so $G\cong H$. This is the isomorphism we seek.}

\end{enumerate}









\end{document}
