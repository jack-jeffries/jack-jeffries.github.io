\documentclass[12pt]{amsart}


\usepackage{times}
\usepackage[margin=.8in]{geometry}
\usepackage{paralist,amsmath,amssymb,multicol,graphicx,framed,ifthen,color,xcolor,stmaryrd,colonequals}
\usepackage[shortlabels]{enumitem}
\usepackage[all]{xy}
\usepackage[outline]{contour}
\contourlength{.4pt}
\contournumber{10}
\newcommand{\Bold}[1]{\contour{black}{#1}}

\definecolor{chianti}{rgb}{0.6,0,0}
\definecolor{meretale}{rgb}{0,0,.6}
\definecolor{leaf}{rgb}{0,.35,0}
\newcommand{\Q}{\mathbb{Q}}
\newcommand{\N}{\mathbb{N}}
\newcommand{\Z}{\mathbb{Z}}
\newcommand{\R}{\mathbb{R}}
\newcommand{\C}{\mathbb{C}}
\newcommand{\e}{\varepsilon}
\newcommand{\inv}{^{-1}}
\newcommand{\dabs}[1]{\left| #1 \right|}
\newcommand{\ds}{\displaystyle}
\newcommand{\solution}[1]{\ifthenelse {\equal{\displaysol}{1}} {\begin{framed}{\color{meretale}\noindent #1}\end{framed}} { \ }}
\newcommand{\solutione}[1]{\ifthenelse {\equal{\displaysol}{1}} {\begin{framed}{\color{leaf}This solution is embargoed.}\end{framed}} { \ }}
\newcommand{\showsol}[1]{\def\displaysol{#1}}

\newcommand{\rsa}{\rightsquigarrow}


\newcommand\itemA{\stepcounter{enumi}\item[{\Bold{(\theenumi)}}]}
\newcommand\itemB{\stepcounter{enumi}\item[(\theenumi)]}
\newcommand\itemC{\stepcounter{enumi}\item[{\it{(\theenumi)}}]}
\newcommand\itema{\stepcounter{enumii}\item[{\Bold{(\theenumii)}}]}
\newcommand\itemb{\stepcounter{enumii}\item[(\theenumii)]}
\newcommand\itemc{\stepcounter{enumii}\item[{\it{(\theenumii)}}]}
\newcommand\itemai{\stepcounter{enumiii}\item[{\Bold{(\theenumiii)}}]}
\newcommand\itembi{\stepcounter{enumiii}\item[(\theenumiii)]}
\newcommand\itemci{\stepcounter{enumiii}\item[{\it{(\theenumiii)}}]}
\newcommand\ceq{\colonequals}


\DeclareMathOperator{\ord}{ord}

\DeclareMathOperator{\res}{res}
\setlength\parindent{0pt}
%\usepackage{times}

%\addtolength{\textwidth}{100pt}
%\addtolength{\evensidemargin}{-45pt}
%\addtolength{\oddsidemargin}{-60pt}

\pagestyle{empty}
%\begin{document}\begin{itemize}

%\thispagestyle{empty}




\begin{document}
\showsol{0}
	
	\thispagestyle{empty}
	
	\section*{Maximal ideals and prime ideals}
	
	
\begin{framed}
\textsc{Definition:} Let $R$ be a ring.
\begin{enumerate}[(i)]
\item An ideal $I$ of $R$ is a \textbf{maximal ideal} if $I$ is proper and for any proper ideal $J$, $I\subseteq J$ implies $I=J$.
\item Let $R$ be commutative. An ideal $I$ of $R$ is a \textbf{prime ideal} if $I$ is proper and $ab\in I$ implies $a\in I$ or $b\in I$.
\end{enumerate}

\

\textsc{Theorem 1:} Let $R$ be a commutative ring and $I$ an ideal.
\begin{enumerate}[(i)]
\item The ideal $I$ is maximal if and only if $R/I$ is a field.
\item The ideal $I$ is prime if and only if $R/I$ is an integral domain.
\end{enumerate}


\end{framed}
	\begin{enumerate}
	\itemA Prime ideals vs maximal ideals:
	\begin{enumerate}
	\itema Use the theorem to quickly explain why every maximal ideal in a commutative ring is prime.
	\itema Show that the ideal $(2)$ in $\mathbb{Z}[x]$ is prime but not maximal.
	\end{enumerate}
	
	\
	
	\itemA Prove\footnote{Hint: For part (i), you might want use a HW problem characterizing fields in terms of ideals.} Theorem 1.
\end{enumerate}

\

\begin{framed}
\textsc{Theorem 2:} Let $R$ be a ring. Then $R$ has a maximal ideal.

\

\textsc{Definition:} Let $(P,\leq)$ be a partially ordered set.
\begin{enumerate}[(i)]
\item A \textbf{maximal element} of $P$ is an element $x\in P$ such that $x\geq y$ for all $y\in P$.
\item A subset $X$ of $P$ is a \textbf{chain} if for all $x,y\in X$ either $x\leq y$ or $y\leq x$.
\item A \textbf{upper bound} for a subset $X$ is an element $x$ such that $x\geq y$ for all $y\in X$.
\end{enumerate}

\

\textsc{Zorn's Lemma:} Let $(P,\leq)$ be a nonempty partially ordered set. If every chain $C\subseteq P$ has an upper bound $c\in C$, then $P$ has a maximal element.
\end{framed}

\begin{enumerate}
\setcounter{enumi}{2}
\itemA Understanding Zorn's Lemma:
\begin{enumerate}
\itema The most common use of Zorn's Lemma occurs in the following situation: $P$ is the collection of all subsets of some set $Y$ ordered by inclusion ($A\leq B$ if and only if $A\subseteq B$), and $S$ is some special family of subsets of $Y$. Rewrite\footnote{Meaning replace all $\leq$ with $\subseteq$ and unpackage the definitions of maximal element and upper bound.} the statement of Zorn's Lemma in this context.
\itema Let $Y=\mathbb{Z}$ and $P$ be the collection of all \emph{finite} subsets of $Y$. Explain why there is no maximal element of $P$.
\end{enumerate}

\

\itemA Prove the Theorem.

\

\itemB Prove that every proper ideal is contained in a maximal ideal.

\

\itemB Prove that every prime ideal contains a minimal prime ideal.

\end{enumerate}


\end{document}
