\documentclass[12pt]{amsart}


\usepackage{times}
\usepackage[margin=.5in]{geometry}
\usepackage{paralist,amsmath,amssymb,multicol,graphicx,framed,ifthen,color,xcolor,stmaryrd,colonequals}
\usepackage[shortlabels]{enumitem}
\usepackage[all]{xy}
\usepackage[outline]{contour}
\contourlength{.4pt}
\contournumber{10}
\newcommand{\Bold}[1]{\contour{black}{#1}}

\definecolor{chianti}{rgb}{0.6,0,0}
\definecolor{meretale}{rgb}{0,0,.6}
\definecolor{leaf}{rgb}{0,.35,0}
\newcommand{\Q}{\mathbb{Q}}
\newcommand{\N}{\mathbb{N}}
\newcommand{\Z}{\mathbb{Z}}
\newcommand{\R}{\mathbb{R}}
\newcommand{\C}{\mathbb{C}}
\newcommand{\e}{\varepsilon}
\newcommand{\inv}{^{-1}}
\newcommand{\dabs}[1]{\left| #1 \right|}
\newcommand{\ds}{\displaystyle}
\newcommand{\solution}[1]{\ifthenelse {\equal{\displaysol}{1}} {\begin{framed}{\color{meretale}\noindent #1}\end{framed}} { \ }}
\newcommand{\solutione}[1]{\ifthenelse {\equal{\displaysol}{1}} {\begin{framed}{\color{leaf}This solution is embargoed.}\end{framed}} { \ }}
\newcommand{\showsol}[1]{\def\displaysol{#1}}

\newcommand{\rsa}{\rightsquigarrow}


\newcommand\itemA{\stepcounter{enumi}\item[{\Bold{(\theenumi)}}]}
\newcommand\itemB{\stepcounter{enumi}\item[(\theenumi)]}
\newcommand\itemC{\stepcounter{enumi}\item[{\it{(\theenumi)}}]}
\newcommand\itema{\stepcounter{enumii}\item[{\Bold{(\theenumii)}}]}
\newcommand\itemb{\stepcounter{enumii}\item[(\theenumii)]}
\newcommand\itemc{\stepcounter{enumii}\item[{\it{(\theenumii)}}]}
\newcommand\itemai{\stepcounter{enumiii}\item[{\Bold{(\theenumiii)}}]}
\newcommand\itembi{\stepcounter{enumiii}\item[(\theenumiii)]}
\newcommand\itemci{\stepcounter{enumiii}\item[{\it{(\theenumiii)}}]}
\newcommand\ceq{\colonequals}


\DeclareMathOperator{\ord}{ord}

\DeclareMathOperator{\res}{res}
\setlength\parindent{0pt}
%\usepackage{times}

%\addtolength{\textwidth}{100pt}
%\addtolength{\evensidemargin}{-45pt}
%\addtolength{\oddsidemargin}{-60pt}

\pagestyle{empty}
%\begin{document}\begin{itemize}

%\thispagestyle{empty}




\begin{document}
\showsol{0}
	
	\thispagestyle{empty}
	
	\section*{Maximal ideals and prime ideals}
	
	
\begin{framed}
\textsc{Definition:} Let $R$ be a ring.
\begin{enumerate}[(i)]
\item An ideal $I$ of $R$ is a \textbf{maximal ideal} if $I$ is proper and for any proper ideal $J$, $I\subseteq J$ implies $I=J$. That is, $I$ is maximal under containment among all proper ideals of $R$.
\item Let $R$ be commutative. An ideal $I$ of $R$ is a \textbf{prime ideal} if $I$ is proper and $ab\in I$ implies $a\in I$ or $b\in I$.
\end{enumerate}

\

\textsc{Theorem 1:} Let $R$ be a commutative ring and $I$ an ideal.
\begin{enumerate}[(i)]
\item The ideal $I$ is maximal if and only if $R/I$ is a field.
\item The ideal $I$ is prime if and only if $R/I$ is an integral domain.
\end{enumerate}


\end{framed}
	\begin{enumerate}
	\itemA Prime ideals vs maximal ideals:
	\begin{enumerate}
	\itema Use Theorem 1 to quickly explain why every maximal ideal in a commutative ring is prime.
	\solution{We have $I$ is maximal implies $R/I$ is a field, which implies $R/I$ is a domain, which implies $I$ is prime.}
	\itema Show that the ideal $(2)$ in $\mathbb{Z}[x]$ is prime but not maximal.
		\solution{$\mathbb{Z}[x]/(2) \cong \mathbb{Z}/2[x]$, which is a domain but not a field.}
	\itema Identify a maximal ideal in $\mathbb{Z}[x]$.
			\solution{The ideal $(2,x)$ is maximal, since $\mathbb{Z}[x]/(2,x) \cong \mathbb{Z}/2$.}
	\end{enumerate}


\
	
	\itemA Prove\footnote{Hint: For part (i), you might want use a HW problem characterizing fields in terms of ideals.} Theorem 1.
\solution{ (i) By the Lattice Isomorphism Theorem, the ideals of $R/Q$ are of the form $I/Q$, where $I$ is an ideal in $R$ containing $Q$.

By an exercise, $R/Q$ is a field if and only if $R/Q$ has only two ideals, $\{ 0 \} = Q/Q$ and $R/Q$. Thus $R/Q$ is a field if and only if the only ideals that contain $Q$ are $Q$ and $R$.

(ii) Now suppose $Q$ is prime. If 
$$(r + I)(r' + I) = 0 + I,$$ 
then $rr' \in I$ and hence either $r \in I$ or $r' \in I$, so that either 
$$r + I = 0 \quad \text{or} \quad r'+ I  = 0.$$
Since $R$ is commutative, then $R/I$ is also commutative, and since $Q$ is a proper, then $R/I$ is not the zero ring. This proves that $R/Q$ is a domain.

Conversely, suppose that $R/Q$ is a domain. Since $R/Q$ is not the zero ring, $Q$ is proper. If $x,y \in R$ satisfy $xy \in I$, then 
$$(x + I)(y + I) = 0$$ 
in $R/Q$, and hence either $x+ Q = 0$ or $y + Q = 0$. It follows $x \in Q$ or $y \in Q$. This proves that $Q$ is prime.
}

\end{enumerate}

\

\begin{framed}
\textsc{Theorem 2:} Let $R$ be a ring. Then $R$ has a maximal ideal.

\

\textsc{Definition:} Let $(P,\leq)$ be a partially ordered set.
\begin{enumerate}[(i)]
\item A \textbf{maximal element} of $P$ is an element $x\in P$ such that for all $y\in P$,  one has $x\leq y$ implies $x=y$.
\item A \textbf{upper bound} for a subset $X$ is an element $x\in P$ such that for all $y\in X$, one has $y \leq x$.
\item A subset $X$ of $P$ is a \textbf{chain} if for all $x,y\in X$ either $x\leq y$ or $y\leq x$.
\end{enumerate}

\

\textsc{Zorn's Lemma:} Let $(P,\leq)$ be a nonempty partially ordered set. If every chain $C\subseteq P$ has an upper bound $c\in P$, then $P$ has a maximal element.
\end{framed}

\begin{enumerate}
\setcounter{enumi}{2}
\itemA Zorn's Lemma warmup.
\begin{enumerate}
\itema The most common use of Zorn's Lemma occurs in the following situation: $\mathcal{P}(Y)$ is the collection of all subsets of some set $Y$ ordered by inclusion ($A\leq B$ if and only if $A\subseteq B$), and $P$ is some special family of subsets of $\mathcal{P}(Y)$. Rewrite\footnote{Meaning replace all $\leq$ with $\subseteq$ and unpackage the definitions of maximal element and upper bound.} the statement of Zorn's Lemma in this context.
\solution{If, $P$ is nonempty and for any nested family of subsets $\{A_\alpha\}_{\alpha}$ with $A_\alpha\in P$ for all $\alpha$, there is some $B\in P$ such that $A_\alpha \subseteq B$ for all $\alpha$, then there is some element $X\in P$ that is not properly contained in any element of $P$.}
\itema In the context above, explain how to use Zorn's lemma to try to show the existence of a \emph{minimal element} of $P$. 
\solution{We can consider $P$ as a poset with the alternative partial order $A\leq B$ if and only if $A\supseteq B$. A maximal element of this poset corresponds to a minimal element of $P$ under containment.}
\end{enumerate}

\

\itemA Prove Theorem 2.

\solution{Fix a ring $R$. Let $S$ be the set of all proper ideals $J$ in $R$, which is partially ordered with the inclusion order $\subseteq$. We claim that {Zorn's Lemma} applies to $S$. First, $S$ is nonempty, since it contains $I$. Now consider a chain of proper ideals in $R$, say $\{ J_i \}_i$, all of which contain $I$. Notice that $J := \bigcup_i J_i$ is an ideal as well (exercise!), and moreover $J \neq R$ since $1 \notin J_i$ for all $i$.\footnote{Note that unions of ideals are not ideals in general, but a union of totally ordered ideals \emph{is} an ideal.} Since each $J_i \supseteq I$, we conclude that $J \supseteq I$. Thus we have checked that $J \in S$. Now this ideal $J \in S$ is an upper bound for our chain $\{ J_i \}_i$, and thus {Zorn's Lemma} applies to $S$. We conclude that $S$ has a maximal element. Such an element is, by definition, a maximal ideal of $R$.}

\itemB Prove or disprove: Any group $G$ has a maximal proper subgroup (meaning a proper subgroup that is maximal among all proper subgroups).

\

\itemB Prove that every prime ideal contains a minimal prime ideal.

\end{enumerate}


\end{document}
