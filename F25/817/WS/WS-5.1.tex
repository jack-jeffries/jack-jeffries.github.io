\documentclass[12pt]{amsart}


\usepackage{times}
\usepackage[margin=1in]{geometry}
\usepackage{paralist,amsmath,amssymb,multicol,graphicx,framed,ifthen,color,xcolor,stmaryrd,colonequals}
\usepackage[shortlabels]{enumitem}
\usepackage[all]{xy}
\usepackage[outline]{contour}
\contourlength{.4pt}
\contournumber{10}
\newcommand{\Bold}[1]{\contour{black}{#1}}

\definecolor{chianti}{rgb}{0.6,0,0}
\definecolor{meretale}{rgb}{0,0,.6}
\definecolor{leaf}{rgb}{0,.35,0}
\newcommand{\Q}{\mathbb{Q}}
\newcommand{\N}{\mathbb{N}}
\newcommand{\Z}{\mathbb{Z}}
\newcommand{\R}{\mathbb{R}}
\newcommand{\C}{\mathbb{C}}
\newcommand{\e}{\varepsilon}
\newcommand{\inv}{^{-1}}
\newcommand{\dabs}[1]{\left| #1 \right|}
\newcommand{\ds}{\displaystyle}
\newcommand{\solution}[1]{\ifthenelse {\equal{\displaysol}{1}} {\begin{framed}{\color{meretale}\noindent #1}\end{framed}} { \ }}
\newcommand{\solutione}[1]{\ifthenelse {\equal{\displaysol}{1}} {\begin{framed}{\color{leaf}This solution is embargoed.}\end{framed}} { \ }}
\newcommand{\showsol}[1]{\def\displaysol{#1}}

\newcommand{\rsa}{\rightsquigarrow}


\newcommand\itemA{\stepcounter{enumi}\item[{\Bold{(\theenumi)}}]}
\newcommand\itemB{\stepcounter{enumi}\item[(\theenumi)]}
\newcommand\itemC{\stepcounter{enumi}\item[{\it{(\theenumi)}}]}
\newcommand\itema{\stepcounter{enumii}\item[{\Bold{(\theenumii)}}]}
\newcommand\itemb{\stepcounter{enumii}\item[(\theenumii)]}
\newcommand\itemc{\stepcounter{enumii}\item[{\it{(\theenumii)}}]}
\newcommand\itemai{\stepcounter{enumiii}\item[{\Bold{(\theenumiii)}}]}
\newcommand\itembi{\stepcounter{enumiii}\item[(\theenumiii)]}
\newcommand\itemci{\stepcounter{enumiii}\item[{\it{(\theenumiii)}}]}
\newcommand\ceq{\colonequals}


\DeclareMathOperator{\ord}{ord}

\DeclareMathOperator{\res}{res}
\setlength\parindent{0pt}
%\usepackage{times}

%\addtolength{\textwidth}{100pt}
%\addtolength{\evensidemargin}{-45pt}
%\addtolength{\oddsidemargin}{-60pt}

\pagestyle{empty}
%\begin{document}\begin{itemize}

%\thispagestyle{empty}




\begin{document}
\showsol{0}
	
	\thispagestyle{empty}
	
	\section*{Orbit-Stabilizer Theorem}
	
	

\begin{framed}
\textsc{Definition:} Let $G$ be a group acting on a set $X$, and $x\in X$.
\begin{itemize}
\item The \textbf{orbit} of $x$ is $\mathrm{Orb}_G(x) = \{ g\cdot x  \ | \ g\in G\} \subseteq X$.
\item The \textbf{stabilizer} of $x$ is $\mathrm{Stab}_G(x) = \{ g\in G \ | \ g \cdot x = x\} \leq G$.
\end{itemize}

\

\textsc{Orbit-Stabilizer Theorem:} Let $G$ be a group acting on a set $X$, and $x\in X$. Then
\[ | \mathrm{Orb}_G(x) | = [ G : \mathrm{Stab}_G(x) ].\]

\

\textsc{Corollary of Orbit-Stabilizer Theorem:} Let $G$ be a finite group acting on a set $X$, and $x\in X$. Then
\[  | \mathrm{Orb}_G(x) | \cdot |  \mathrm{Stab}_G(x) | = | G |.\]
In particular, the size of any orbit divides the order of $G$.
\end{framed}

\

\begin{enumerate}
\itemA Use the Orbit-Stabilizer Theorem and/or its corollary above to quickly explain why the following are \textit{impossible}:
\begin{itemize}
\item $S_4 \curvearrowright X$ transitively for a set $X$ with $5$ elements.
\solution{This would imply that $X$ is a single orbit with 5 elements, but 5 does not divide the order of $S_4$.}
\item $G \curvearrowright X$ with $|G|=16$, $|X|$ odd, and the action has no fixed point\footnote{A \textbf{fixed point} of a group action is some $x\in X$ such that $g\cdot x= x$ for all $g\in G$.}.
\solution{Every orbit has order dividing 16, so is either equal to one (a fixed point) or has an even number of elements. If there are no fixed points, then $|X|$ must be even.}
\end{itemize}

\

\itemA Proof of Theorem/Corollary.
\begin{enumerate}
\itema Prove the Orbit-Stabilizer Theorem by showing that the map
\[ \begin{aligned} \{ \text{left cosets of $\mathrm{Stab}_G(x)$ in $G$}\} &\longrightarrow \mathrm{Orb}_G(x) \\ g \cdot \mathrm{Stab}_G(x) &\mapsto g \cdot x\end{aligned}\]
is a well-defined bijective function.
\solution{We have $g \mathrm{Stab}_G(x) = h \mathrm{Stab}_G(x) \Leftrightarrow h^{-1} g\in \mathrm{Stab}_G(x) \mathrm{Stab}_G(x) \Leftrightarrow h^{-1}g \cdot x = x \Leftrightarrow h^{-1} \cdot(g \cdot x) = x \Leftrightarrow g \cdot x = h \cdot x$, so this is well-defined and injective. It is surjective by construction and definition of orbit.}
\itema Deduce the Corollary from the Theorem.
\solution{Follows from Lagrange.}
\end{enumerate}

\

\itemA Let $G$ be the group of rotational symmetries of a cube.
\begin{enumerate}
\itema Explain very briefly why $G$ acts on the set $F$ of faces of the cube.
\solution{Any symmetry sends faces to other faces.}
\itema Explain why $G \curvearrowright F$ is transitive.
\solution{There is a rotation that takes any face to any other face.}
\itema Compute $\mathrm{Stab}_G(f)$ for $f\in F$.
\solution{If the top face stays on top, there are only four rotations.}
\itema Compute $|G|$.
\solution{By Orbit-Stabilizer, there are $6\cdot 4 = 24$ elements.}
\end{enumerate}

\

\itemB Let $G$ be the group of rotational symmetries of a cube.
\begin{enumerate}
\itemb Explain briefly why $G$ acts on the set of long diagonals $D$ (line segments between pairs of opposite vertices) of the cube.
\itemb Explain why, if we know that $G \curvearrowright D$ is faithful, then $G\cong S_4$.
\itemb Show that $G \curvearrowright D$ is faithful.
\end{enumerate}

\

\itemB For the other platonic solids, compute the order of the rotational symmetry group. Can you compute the rotational symmetry group up to isomorphism as a group we already know?


\end{enumerate}
\end{document}
