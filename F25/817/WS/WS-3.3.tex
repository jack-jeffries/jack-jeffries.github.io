\documentclass[12pt]{amsart}


\usepackage{times}
\usepackage[margin=.65in]{geometry}
\usepackage{paralist,amsmath,amssymb,multicol,graphicx,framed,ifthen,color,xcolor,stmaryrd,enumitem,colonequals}
\usepackage[outline]{contour}
\contourlength{.4pt}
\contournumber{10}
\newcommand{\Bold}[1]{\contour{black}{#1}}

\definecolor{chianti}{rgb}{0.6,0,0}
\definecolor{meretale}{rgb}{0,0,.6}
\definecolor{leaf}{rgb}{0,.35,0}
\newcommand{\Q}{\mathbb{Q}}
\newcommand{\N}{\mathbb{N}}
\newcommand{\Z}{\mathbb{Z}}
\newcommand{\R}{\mathbb{R}}
\newcommand{\C}{\mathbb{C}}
\newcommand{\e}{\varepsilon}
\newcommand{\inv}{^{-1}}
\newcommand{\dabs}[1]{\left| #1 \right|}
\newcommand{\ds}{\displaystyle}
\newcommand{\solution}[1]{\ifthenelse {\equal{\displaysol}{1}} {\begin{framed}{\color{meretale}\noindent #1}\end{framed}} { \ }}
\newcommand{\solutione}[1]{\ifthenelse {\equal{\displaysol}{1}} {\begin{framed}{\color{leaf}This solution is embargoed.}\end{framed}} { \ }}
\newcommand{\showsol}[1]{\def\displaysol{#1}}

\newcommand{\rsa}{\rightsquigarrow}


\newcommand\itemA{\stepcounter{enumi}\item[{\Bold{(\theenumi)}}]}
\newcommand\itemB{\stepcounter{enumi}\item[(\theenumi)]}
\newcommand\itemC{\stepcounter{enumi}\item[{\it{(\theenumi)}}]}
\newcommand\itema{\stepcounter{enumii}\item[{\Bold{(\theenumii)}}]}
\newcommand\itemb{\stepcounter{enumii}\item[(\theenumii)]}
\newcommand\itemc{\stepcounter{enumii}\item[{\it{(\theenumii)}}]}
\newcommand\itemai{\stepcounter{enumiii}\item[{\Bold{(\theenumiii)}}]}
\newcommand\itembi{\stepcounter{enumiii}\item[(\theenumiii)]}
\newcommand\itemci{\stepcounter{enumiii}\item[{\it{(\theenumiii)}}]}
\newcommand\ceq{\colonequals}


\DeclareMathOperator{\ord}{ord}

\DeclareMathOperator{\res}{res}
\setlength\parindent{0pt}
%\usepackage{times}

%\addtolength{\textwidth}{100pt}
%\addtolength{\evensidemargin}{-45pt}
%\addtolength{\oddsidemargin}{-60pt}

\pagestyle{empty}
%\begin{document}\begin{itemize}

%\thispagestyle{empty}




\begin{document}
\showsol{0}
	
	\thispagestyle{empty}
	
	\section*{Cyclic groups wrapup}
	
	

\begin{framed}

\textsc{Universal mapping theorem for cyclic groups:} Let $G = \langle x \rangle$ be a cyclic group and $H$ be an arbitrary group.
\begin{enumerate}
\item If $|x|=n<\infty$ and $y\in H$ is such that $y^n=e$, then there is a unique homomorphism $f:G\to H$ such that $f(x)=y$.
\item If $|x|=\infty$ and $y\in H$ is arbitrary, then there is a unique homomorphism $f:G\to H$ such that $f(x)=y$.
\end{enumerate}

\

\textsc{Definition:} \begin{itemize}
\item The \textbf{infinite cyclic group}  is the group $C_\infty = \{ a^j \ | \ j\in \mathbb{Z}\}$ with operation $a^j a^k = a^{j+k}$. Its presentation\footnotemark is $\langle a \ | \ \varnothing \rangle$. 
\item For any $n\in \mathbb{Z}_{\geq 1}$, the cyclic group of order $n$ is the group $C_n=\{ a^j \ | \ j\in \{0,1,\dots,n-1\} \, \}$ with operation $a^j a^k = a^{j+k \, (\mathrm{mod} \, n)}$. Its presentation is $\langle a \ | \ a^n = e\rangle$.
\end{itemize}

\

\textsc{Classification of cyclic groups:} Every infinite cyclic group is isomorphic to $C_\infty$. Every cyclic group of order $n$ is isomorphic to $C_n$.
\end{framed}
\footnotetext[1]{We write the empty set in the relations spot to indicate that there are no defining relations.}


\begin{enumerate}
\itemA Use the Universal Mapping Theorem for cyclic groups to prove the classification of cyclic groups.

\solution{Let $G=\langle x \rangle$ be an infinite cyclic group. By the UMP for cyclic groups, there is a homomorphism $f: G\to C_\infty$ mapping $x\mapsto a$. Conversely, by the UMP for cyclic groups, there is a homomorphism $g:C_\infty \to G$ mapping $a\mapsto x$. The composition $fg:C_\infty \to C_\infty$ maps $a\mapsto a$; the identity map is another such homomorphism, so by the uniqueness part of the UMP, $fg$ is the identity on $C_\infty$. For the same reason, $gf: G\to G$ is the identity. It follows that $f$ is an isomorphism.

Let $G=\langle x \rangle$ be a cyclic group of order $n$. Since $a\in C_n$ has order $n$, there is a homomorphism $f:G\to C_n$ mapping $x\mapsto a$. Likewise, there is a homomorphism $g:C_n \to G$ by the UMP. Following the same argument as above, we see that these are mutually inverse, so $f$ is an isomorphism.}

\itemA Prove the Universal mapping theorem for cyclic groups.

\solution{
We know that homomorphisms are uniquely determined by their images on a generating set, so in each case we just need to show existence.

In either case, define $f(x^i) = y^i$.  We must show this function is a well-defined group homomorphism.
To see that $f$ is well-defined, suppose $x^i=x^j$ for some $i,j\in \Z$. Then, since $x^{i-j}=e_G$, using earlier work, we have
$$\begin{cases}
n\mid i-j & \text{ if } |x|=n\\
i-j=0 & \text{ if } |x|=\infty\\
\end{cases}
\implies
\begin{cases}
y^ {i-j}=y^{nk} & \text{ if } |x|=n\\
y^{i-j}=y^0 & \text{ if } |x|=\infty\\
\end{cases}
\implies y^ {i-j}=e_H
\implies y^ i=y^j.
$$
Thus, if $x^i=x^j$ then $f(x^i)=y^i=y^j=f(x^j)$. In particular, if $x^k = e$, then $f(x^k) = e$, and $f$ is well-defined.

The fact that $f$ is a homomorphism is immediate: 
$$f(x^ix^j)=f(x^{i+j})=y^{i+j}=y^iy^j=f(x^i)f(x^j).$$


}

\itemB Classify all subgroups of $C_\infty$ and describe the subgroup lattice.
\end{enumerate}









\end{document}
