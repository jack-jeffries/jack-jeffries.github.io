\documentclass[12pt]{amsart}


\usepackage{times}
\usepackage[margin=0.7in]{geometry}
\usepackage{paralist,amsmath,amssymb,multicol,graphicx,framed,ifthen,color,xcolor,stmaryrd,enumitem,colonequals}
\usepackage{tikz}
\usepackage[outline]{contour}
\contourlength{.4pt}
\contournumber{10}
\newcommand{\Bold}[1]{\contour{black}{#1}}

\definecolor{chianti}{rgb}{0.6,0,0}
\definecolor{meretale}{rgb}{0,0,.6}
\definecolor{leaf}{rgb}{0,.35,0}
\newcommand{\Q}{\mathbb{Q}}
\newcommand{\N}{\mathbb{N}}
\newcommand{\Z}{\mathbb{Z}}
\newcommand{\R}{\mathbb{R}}
\newcommand{\C}{\mathbb{C}}
\newcommand{\e}{\varepsilon}
\newcommand{\inv}{^{-1}}
\newcommand{\dabs}[1]{\left| #1 \right|}
\newcommand{\ds}{\displaystyle}
\newcommand{\solution}[1]{\ifthenelse {\equal{\displaysol}{1}} {\begin{framed}{\color{meretale}\noindent #1}\end{framed}} { \ }}
\newcommand{\solutione}[1]{\ifthenelse {\equal{\displaysol}{1}} {\begin{framed}{\color{leaf}This solution is embargoed.}\end{framed}} { \ }}
\newcommand{\showsol}[1]{\def\displaysol{#1}}

\newcommand{\rsa}{\rightsquigarrow}


\newcommand\itemA{\stepcounter{enumi}\item[{\Bold{(\theenumi)}}]}
\newcommand\itemB{\stepcounter{enumi}\item[(\theenumi)]}
\newcommand\itemC{\stepcounter{enumi}\item[{\it{(\theenumi)}}]}
\newcommand\itema{\stepcounter{enumii}\item[{\Bold{(\theenumii)}}]}
\newcommand\itemb{\stepcounter{enumii}\item[(\theenumii)]}
\newcommand\itemc{\stepcounter{enumii}\item[{\it{(\theenumii)}}]}
\newcommand\itemai{\stepcounter{enumiii}\item[{\Bold{(\theenumiii)}}]}
\newcommand\itembi{\stepcounter{enumiii}\item[(\theenumiii)]}
\newcommand\itemci{\stepcounter{enumiii}\item[{\it{(\theenumiii)}}]}
\newcommand\ceq{\colonequals}


\DeclareMathOperator{\ord}{ord}

\DeclareMathOperator{\res}{res}
\setlength\parindent{0pt}
%\usepackage{times}

%\addtolength{\textwidth}{100pt}
%\addtolength{\evensidemargin}{-45pt}
%\addtolength{\oddsidemargin}{-60pt}

\pagestyle{empty}
%\begin{document}\begin{itemize}

%\thispagestyle{empty}




\begin{document}
\showsol{0}
	
	\thispagestyle{empty}
	
	\section*{Isomorphism wrapup}
	
	

\begin{framed}
\textsc{Proposition:} Let $f: G\to H$ be a group homomorphism. Then $f$ is an isomorphism if and only if $f$ is bijective\footnotemark.

\


\textsc{Lemma:} Let $f:G\to H$ be an isomorphism. Then for any $g\in G$, we have $|g| = |f(g)|$.

\


\textsc{Definition:} A property $\mathcal{P}$ of a group is an \textbf{isomorphism invariant} if whenever $G\cong H$ and $\mathcal{P}$ holds for $G$, then $\mathcal{P}$ also holds for $H$.

\


\textsc{Theorem:} The following are isomorphism invariants:
\begin{enumerate}
\item The order of the group.
\item The set of orders of elements of the group.
\item Being abelian.
\item The order of the center of the group.
\item Being finitely generated.
\end{enumerate}

\end{framed}
\footnotetext{Reminder: by definition a function is \textbf{bijective} if it is injective and surjective (i.e. a one-to-one correspondence). It is a theorem from set theory that a function $f:X\to Y$ is bijective if and only if there exists an inverse function $g:Y\to X$.}

\begin{enumerate}


\itemA Use the Theorem to show that none of the following groups are pairwise isomorphic:

\[  {S}_3 \qquad {S}_4\qquad \mathbb{Z}/6 \]


\itemA Prove the Proposition.

\

\itemB Prove the Lemma. 

\

\itemB Prove the Thoerem.

\end{enumerate}











\end{document}
