\documentclass[12pt]{amsart}


\usepackage{times}
\usepackage[margin=1in]{geometry}
\usepackage{paralist,amsmath,amssymb,multicol,graphicx,framed,ifthen,color,xcolor,stmaryrd,colonequals}
\usepackage[shortlabels]{enumitem}
\usepackage[all]{xy}
\usepackage[outline]{contour}
\contourlength{.4pt}
\contournumber{10}
\newcommand{\Bold}[1]{\contour{black}{#1}}

\definecolor{chianti}{rgb}{0.6,0,0}
\definecolor{meretale}{rgb}{0,0,.6}
\definecolor{leaf}{rgb}{0,.35,0}
\newcommand{\Q}{\mathbb{Q}}
\newcommand{\N}{\mathbb{N}}
\newcommand{\Z}{\mathbb{Z}}
\newcommand{\R}{\mathbb{R}}
\newcommand{\C}{\mathbb{C}}
\newcommand{\e}{\varepsilon}
\newcommand{\inv}{^{-1}}
\newcommand{\dabs}[1]{\left| #1 \right|}
\newcommand{\ds}{\displaystyle}
\newcommand{\solution}[1]{\ifthenelse {\equal{\displaysol}{1}} {\begin{framed}{\color{meretale}\noindent #1}\end{framed}} { \ }}
\newcommand{\solutione}[1]{\ifthenelse {\equal{\displaysol}{1}} {\begin{framed}{\color{leaf}This solution is embargoed.}\end{framed}} { \ }}
\newcommand{\showsol}[1]{\def\displaysol{#1}}

\newcommand{\rsa}{\rightsquigarrow}


\newcommand\itemA{\stepcounter{enumi}\item[{\Bold{(\theenumi)}}]}
\newcommand\itemB{\stepcounter{enumi}\item[(\theenumi)]}
\newcommand\itemC{\stepcounter{enumi}\item[{\it{(\theenumi)}}]}
\newcommand\itema{\stepcounter{enumii}\item[{\Bold{(\theenumii)}}]}
\newcommand\itemb{\stepcounter{enumii}\item[(\theenumii)]}
\newcommand\itemc{\stepcounter{enumii}\item[{\it{(\theenumii)}}]}
\newcommand\itemai{\stepcounter{enumiii}\item[{\Bold{(\theenumiii)}}]}
\newcommand\itembi{\stepcounter{enumiii}\item[(\theenumiii)]}
\newcommand\itemci{\stepcounter{enumiii}\item[{\it{(\theenumiii)}}]}
\newcommand\ceq{\colonequals}
\newcommand\blank[1]{\underline{#1}}
\newcommand\Ar{$\Longrightarrow$ }
\newcommand\Arr{$\Longleftrightarrow$ }

\DeclareMathOperator{\ord}{ord}

\DeclareMathOperator{\res}{res}
\setlength\parindent{0pt}
%\usepackage{times}

%\addtolength{\textwidth}{100pt}
%\addtolength{\evensidemargin}{-45pt}
%\addtolength{\oddsidemargin}{-60pt}

\pagestyle{empty}
%\begin{document}\begin{itemize}

%\thispagestyle{empty}




\begin{document}
\showsol{0}
	
	\thispagestyle{empty}
	
	\section*{Fill in the blank ring review}
	\begin{itemize}
	\item The kernel of a ring homomorphism is a(n) \blank{ideal}.
	\item The image of a ring homomorphism is a(n) \blank{subring}.
\item Use the candidates below to fill in the following:\\ \blank{field} \Ar \blank{Euclidean domain} \Ar \blank{PID} \Ar \blank{UFD} \Ar \blank{domain}.
\begin{itemize}
\item domain
\item Euclidean domain
\item field
\item PID
\item UFD
\end{itemize}
\item In a ring, unit \blank{\Ar not} zerodivisor.
\item A commutative ring has (exact) division by nonzero elements if it is a \blank{field}.
\item A\footnote{Be sure to give the most general correct answer.} commutative ring has cancellation by nonzero elements if it is a \blank{domain}.
\item A commutative ring has division with remainder by nonzero elements if it is a  \blank{Euclidean domain}.
\item In\footnote{Express in terms of divides.} a commutative ring, $(a) \subseteq (b)$ \Arr \blank{$b \, | \, a$}.
\item In\footnotemark[2] a commutative ring, $(a) = (b)$ \Arr \blank{$a \, | \, b$ and $b \, | \, a$}.
\item In\footnote{Express in terms of a word.} a \blank{domain}, $(a) = (b)$ \Arr \blank{associates}.
\item In\footnotemark[1] a \blank{UFD}, GCDs exist.
\item In\footnotemark[1] a \blank{PID}, the GCD of two elements is a linear combination of them.
\item In a \blank{domain}, GCDs are unique \blank{up to associates}.
\item In\footnotemark[1] a \blank{commutative ring}, maximal ideal \Ar prime ideal.
\item In a \blank{PID}, (nonzero) prime ideal \Ar maximal ideal.
\item In a commutative ring $R$, $I$ is a maximal ideal \Arr $R/I$ \blank{is a field}.
\item In a commutative ring $R$, $I$ is a prime ideal \Arr $R/I$ \blank{is a domain}.
\item In a \blank{domain}, prime element \Ar irreducible element.
\item In\footnotemark[1] a \blank{UFD}, irreducible element \Ar prime element.
\end{itemize}
	
	
\end{document}
