\documentclass[12pt]{amsart}


\usepackage{times}
\usepackage[margin=.75in]{geometry}
\usepackage{paralist,amsmath,amssymb,multicol,graphicx,framed,ifthen,color,xcolor,stmaryrd,enumitem,colonequals}
\usepackage[outline]{contour}
\contourlength{.4pt}
\contournumber{10}
\newcommand{\Bold}[1]{\contour{black}{#1}}

\definecolor{chianti}{rgb}{0.6,0,0}
\definecolor{meretale}{rgb}{0,0,.6}
\definecolor{leaf}{rgb}{0,.35,0}
\newcommand{\Q}{\mathbb{Q}}
\newcommand{\N}{\mathbb{N}}
\newcommand{\Z}{\mathbb{Z}}
\newcommand{\R}{\mathbb{R}}
\newcommand{\C}{\mathbb{C}}
\newcommand{\e}{\varepsilon}
\newcommand{\inv}{^{-1}}
\newcommand{\dabs}[1]{\left| #1 \right|}
\newcommand{\ds}{\displaystyle}
\newcommand{\solution}[1]{\ifthenelse {\equal{\displaysol}{1}} {\begin{framed}{\color{meretale}\noindent #1}\end{framed}} { \ }}
\newcommand{\solutione}[1]{\ifthenelse {\equal{\displaysol}{1}} {\begin{framed}{\color{leaf}This solution is embargoed.}\end{framed}} { \ }}
\newcommand{\showsol}[1]{\def\displaysol{#1}}

\newcommand{\rsa}{\rightsquigarrow}


\newcommand\itemA{\stepcounter{enumi}\item[{\Bold{(\theenumi)}}]}
\newcommand\itemB{\stepcounter{enumi}\item[(\theenumi)]}
\newcommand\itemC{\stepcounter{enumi}\item[{\it{(\theenumi)}}]}
\newcommand\itema{\stepcounter{enumii}\item[{\Bold{(\theenumii)}}]}
\newcommand\itemb{\stepcounter{enumii}\item[(\theenumii)]}
\newcommand\itemc{\stepcounter{enumii}\item[{\it{(\theenumii)}}]}
\newcommand\itemai{\stepcounter{enumiii}\item[{\Bold{(\theenumiii)}}]}
\newcommand\itembi{\stepcounter{enumiii}\item[(\theenumiii)]}
\newcommand\itemci{\stepcounter{enumiii}\item[{\it{(\theenumiii)}}]}
\newcommand\ceq{\colonequals}


\DeclareMathOperator{\ord}{ord}

\DeclareMathOperator{\res}{res}
\setlength\parindent{0pt}
%\usepackage{times}

%\addtolength{\textwidth}{100pt}
%\addtolength{\evensidemargin}{-45pt}
%\addtolength{\oddsidemargin}{-60pt}

\pagestyle{empty}
%\begin{document}\begin{itemize}

%\thispagestyle{empty}




\begin{document}
\showsol{0}
	
	\thispagestyle{empty}
	
	\section*{Group actions}
	
	

\begin{framed}

\textsc{Definition:} Let $G$ be a group and $X$ be a set. A \textbf{group action} of $G$ on $X$ is a function $G\times X \to X$ typically written as $(g,x)\mapsto g \cdot x$ such that
\begin{enumerate}
\item $g\cdot (h \cdot x) = (gh) \cdot x$ for all $g,h\in G$ and $x\in X$, and
\item $e_G \cdot x = x$ for all $x\in X$.
\end{enumerate}

\

Given a group action of $G$ on $X$ and $x\in X$, the \textbf{orbit} of $x$ is 
\[ \mathrm{Orb}_G(x) := \{ g\cdot x \ | \ g\in G\}.\]

\

\textsc{Lemma:} Given a group action of $G$ on $X$,
\begin{itemize}
\item for $x,y\in X$, either $\mathrm{Orb}_G(x) = \mathrm{Orb}_G(y)$ or  $\mathrm{Orb}_G(x) \cap \mathrm{Orb}_G(y)=\varnothing$.
\item $X= \bigcup_{x\in X} \mathrm{Orb}_G(x)$.
\end{itemize}

\

\textsc{Definition:} A group action of $G$ on $X$ is
\begin{itemize}
\item  \textbf{transitive} if $\mathrm{Orb}_G(x) = X$ for some $x\in X$.
\item \textbf{faithful} if $g \cdot x = x$ for all $x\in X$ implies that $g=e$.
\end{itemize}
\end{framed}

\begin{enumerate}
\itemA Let $G$ be a group acting on a set $X$. For $x,y\in X$, write $x\sim y$ if there exists $g\in G$ such that $g\cdot x = y$.
\begin{enumerate}
\itema Show that $\sim$ is an equivalence relation\footnote{Recall that a relation on a set is an \textbf{equivalence relation} if it is \emph{reflexive}, \emph{symmetric}, and \emph{transitive}.}.
\solution{Since $e\cdot x = x$, we have $x \sim x$, so $\sim$ is reflexive. If $x\sim y$, then $g\cdot x = y$ for some $g\in G$; then $g^{-1} \cdot y = g^{-1} \cdot (g \cdot x) = (g^{-1} g) \cdot x = e\cdot x = x$, so $y\sim x$; hence $\sim$ is symmetric. 

If $x\sim y$ and $y\sim z$, then we have $g\cdot x=y$ and $h \cdot y = z$ for some $g,h\in G$. Then $(hg) \cdot x = h \cdot (g \cdot x) =h \cdot y = z$, so $x\sim z$. This shows that $\sim$ is transitive.}
\itema Relate the previous part to the Lemma.
\solution{If $\sim$ is an equivalence relation on $X$, the equivalence classes form a partition of $X$. The conclusion of the Lemma is saying that the equivalnce classes (orbits) are a partition.}
\itema Suppose that $X$ is a finite set, and $X_1,\dots,X_\ell$ are the distinct orbits of $G$ acting on $X$. Explain:
\[ |X| = \sum_{i=1}^\ell |X_i|.\]
\solution{This follows immediately from the Lemma.}
\end{enumerate}

\

\itemA Dihedral group actions: Let $D_n$ be the group of symmetries of a regular $n$-gon $P_n$ in $\mathbb{R}^2$.
\begin{enumerate}
\itema Explain why/how $D_n$ acts naturally on $P_n$. Is this action transitive? Is it faithful?
\solution{By definition elements of $D_n$ are functions $f:\R^2 \to \R^2$ such that $f(P_n) \subseteq P_n$, so we may consider $f \cdot x = f(x)$ for $f\in D_n$ and $x\in P_n$. The identity $e$ of $D_n$ is the identity function on $P_n$, so $e\cdot x = x$ for all $x\in P_n$. The operation in $D_n$ is composition of functions, so for $g,h\in D_n$, $(gh) \cdot x = g(h(x)) = g\cdot (h\cdot x)$. This verifies that this is an action. It is not transitive, since the center of $P_n$ cannot be moved to a vertex of $P_n$, for example. It is faithful, since an isometry that fixes every point of $P_n$ must be the identity element of $P_n$.} 
\itema Explain why/how $D_n$ acts naturally on the set of vertices of $P_n$. Is this action transitive? Is it faithful?
\solution{The action of $D_n$ on $P_n$ restricts to an action on the set of vertices: this is because we proved that every isometry of $P_n$ sends vertices to vertices. This action is now transitive, as we can send any vertex to any other (e.g., by a rotation). It is still faithful.}
\end{enumerate}

\


\itemA Group actions on $X$ $\longleftrightarrow$ homomorphisms to $\mathrm{Perm}(X)$:
\begin{enumerate}
\itema Let $G$ be a group acting on a set $X$. For $g\in G$, let $\mu_g: X\to X$ be the function $\mu_g(x) = g\cdot x$, which we made out of the group action. Consider the function 
\[ \begin{aligned} \rho: G &\rightarrow \mathrm{Perm}(X)  \\ g &\mapsto \mu_g \end{aligned}\]
Show that $\rho$ is a group homomorphism\footnote{Warning: you should also show that $\mu_g$ is actually an element of $\mathrm{Perm}(X)$. One good way to do this is to show that $\mu_{g^{-1}}$ is the inverse function of $\mu_g$.}.
We call $\rho$ the \textbf{permutation representation} associated to the given group action.
\solution{We claim that $\mu_g \circ \mu_h = \mu_{gh}$. Indeed, for any $x\in X$, we have $\mu_g \mu_h (x) = \mu_g (h\cdot x) = g\cdot (h\cdot x) = (gh)\cdot x = \mu_{gh}(x)$. In particular, $\mu_g \circ \mu_{g^{-1}} = \mu_e = \mu_{g^{-1}} \circ \mu_g$, and $\mu_e$ is the identity function on $X$ (by the corresponding group action axiom). In particular, $\mu_g$ is invertible as a function, and hence is a permutation of $X$.

By the computation above, we have $\rho(g) \circ \rho(h) = \mu_g \circ \mu_h = \mu_{gh} = \rho(gh)$ for all $g,h\in G$, so $\rho$ is a group homomorphism.}
\itema Label the vertices of a square counterclockwise by $\{1,2,3,4\}$. Write out the induced homomorphism $D_4 \to S_4$ coming from the action of $D_4$ on the vertices as in (2.b) above.
\solution{It suffices to compute the images of our generators $r,s$, for a reflection $s$, e.g., the one over the line through $1$ and $3$. Since $r$ sends vertices $1,2,3,4$ to $2,3,4,1$ respectively, the corresponding permutation is $(1 \, 2\, 3\, 4)$. Since
$s$ sends vertices $1,2,3,4$ to $1,4,3,2$ respectively, the corresponding permutation is $(2\, 4)$.}
\itema Let $G$ be a group, $X$ a set, and $\rho:G\to \mathrm{Perm}(X)$ a group homomorphism. Give a natural recipe for a group action of $G$ on $X$, and verify that this is indeed a group action.
\solution{We can set $g\cot x = \rho(g)(x)$. Let us verify the axioms. We have $\rho(e)$ is the identity of $\mathrm{Perm}(X)$, so $\rho(e)(x)=x$ for all $x\in X$, and thus $e\cdot x= x$ for all $x\in X$. Given $g,h\in G$, $\rho(gh) = \rho(g)\rho(h)$, so $\rho(gh)(x) = \rho(g) \rho(h) (x) = \rho(g)(\rho(h)(x))$ for all $x\in X$. Thus, $(gh)\cdot x = g \cdot(h\cdot x)$ for all $x\in X$.}
\end{enumerate}

\

\itemB Let $G$ be a group acting on a set $X$. Complete the following sentence, and prove your answer: \\
The action of $G$ on $X$ is faithful if and only if the associated permutation representation ${\rho: G\to \mathrm{Perm}(X)}$ is $\underline{\phantom{INJECTIVE}}$.

\

\itemB Linear representations on $K^n$ $\longleftrightarrow$ homomorphisms to $\mathrm{GL}_n(K)$:
\begin{enumerate}
\item Let $G$ be a group and $K$ be a field (you can assume $K=\mathbb{R}$ if you want.) A \textbf{linear action} of $G$ on $K^n$ is a group action of $G$ on $K^n$ such that for each $g\in G$, the function $\mu_g: K^n \to K^n$ as in (3) is a linear transformation over $K$. Given a linear action of $G$ on $K^n$, show that there is natural group homomorphism $\rho: G\to \mathrm{GL}_n(K)$. 
\item Conversely, given a group homomorphism $\rho: G \to \mathrm{GL}_n(K)$, give a natural recipe for a linear action of $G$ on $K^n$.
\end{enumerate}
\end{enumerate}














\end{document}
