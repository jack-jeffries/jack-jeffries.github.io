\documentclass[12pt]{amsart}


\usepackage{times}
\usepackage[margin=.6in]{geometry}
\usepackage{paralist,amsmath,amssymb,multicol,graphicx,framed,ifthen,color,xcolor,stmaryrd,colonequals}
\usepackage[shortlabels]{enumitem}
\usepackage[all]{xy}
\usepackage[outline]{contour}
\contourlength{.4pt}
\contournumber{10}
\newcommand{\Bold}[1]{\contour{black}{#1}}

\definecolor{chianti}{rgb}{0.6,0,0}
\definecolor{meretale}{rgb}{0,0,.6}
\definecolor{leaf}{rgb}{0,.35,0}
\newcommand{\Q}{\mathbb{Q}}
\newcommand{\N}{\mathbb{N}}
\newcommand{\Z}{\mathbb{Z}}
\newcommand{\R}{\mathbb{R}}
\newcommand{\C}{\mathbb{C}}
\newcommand{\e}{\varepsilon}
\newcommand{\inv}{^{-1}}
\newcommand{\dabs}[1]{\left| #1 \right|}
\newcommand{\ds}{\displaystyle}
\newcommand{\solution}[1]{\ifthenelse {\equal{\displaysol}{1}} {\begin{framed}{\color{meretale}\noindent #1}\end{framed}} { \ }}
\newcommand{\solutione}[1]{\ifthenelse {\equal{\displaysol}{1}} {\begin{framed}{\color{leaf}This solution is embargoed.}\end{framed}} { \ }}
\newcommand{\showsol}[1]{\def\displaysol{#1}}

\newcommand{\rsa}{\rightsquigarrow}


\newcommand\itemA{\stepcounter{enumi}\item[{\Bold{(\theenumi)}}]}
\newcommand\itemB{\stepcounter{enumi}\item[(\theenumi)]}
\newcommand\itemC{\stepcounter{enumi}\item[{\it{(\theenumi)}}]}
\newcommand\itema{\stepcounter{enumii}\item[{\Bold{(\theenumii)}}]}
\newcommand\itemb{\stepcounter{enumii}\item[(\theenumii)]}
\newcommand\itemc{\stepcounter{enumii}\item[{\it{(\theenumii)}}]}
\newcommand\itemai{\stepcounter{enumiii}\item[{\Bold{(\theenumiii)}}]}
\newcommand\itembi{\stepcounter{enumiii}\item[(\theenumiii)]}
\newcommand\itemci{\stepcounter{enumiii}\item[{\it{(\theenumiii)}}]}
\newcommand\ceq{\colonequals}


\DeclareMathOperator{\ord}{ord}

\DeclareMathOperator{\res}{res}
\setlength\parindent{0pt}
%\usepackage{times}

%\addtolength{\textwidth}{100pt}
%\addtolength{\evensidemargin}{-45pt}
%\addtolength{\oddsidemargin}{-60pt}

\pagestyle{empty}
%\begin{document}\begin{itemize}

%\thispagestyle{empty}




\begin{document}
\showsol{0}
	
	\thispagestyle{empty}
	
	\section*{The Main Theorem of Sylow Theory}
	
	

\begin{framed}
\textsc{Recall:} Let $G$ be a finite group and $p$ be a prime number. Write $|G|=p^e m$ with $e\geq 0$ and $p\nmid m$.
\begin{itemize}
\item A $p$-subgroup of $G$ is a subgroup of order $p^k$ for some $k\geq 0$.
\item A Sylow $p$-subgroup of $G$ is a subgroup of order $p^e$.
\item We write $\mathrm{Syl}_p(G)$ for the set of Sylow $p$-subgroups of $G$. We often write $n_p$ for $\# \mathrm{Syl}_p(G)$.
\end{itemize}

\


\textsc{Main Theorem of Sylow Theory:} Let $G$ be a finite group and $p$ be a prime number. Write $|G|=p^e m$ with $e\geq 0$ and $p\nmid m$.
\begin{enumerate}
\item There exists a Sylow $p$-subgroup of $G$.
\item Every Sylow subgroup is conjugate. Moreover, for any $p$-subgroup $Q$ and any Sylow $p$-subgroup $P$, there is some $g\in G$ such that $Q\leq gPg^{-1}$.
\item The number of Sylow $p$-subgroups of $G$ is congruent to $1$ modulo $p$.
\item The number of Sylow $p$-subgroups of $G$ divides $m$.
\end{enumerate}

\

\textsc{Lemma:} Let $G$ be a finite group and $p$ be a prime number. Let $P$ be a Sylow $p$-subgroup of $G$ and $Q$ be any $p$-subgroup of $G$. Then $Q \cap N_G(P) = Q \cap P$.

\end{framed}

\begin{enumerate}

\itemA Let $p<q$ be distinct primes and $G$ be a group of order $pq$. Use the Sylow Theorem to show that $G$ is not simple.

\

\itemA Consider $G=S_4$.
\begin{enumerate}
\itema Show\footnote{Hint: $D_4$ acts on the vertices of a square.} that $G$ has a subgroup isomorphic to $D_4$, the symmetry group of the square.
\itema Show that $S_4$ has exactly three subgroups isomorphic to $D_4$, that these three are conjugate, and that any subgroup of $S_4$ of order $8$ is isomorphic to $D_4$.
\itema Describe the subgroups of order $3$ of $S_4$. 
\end{enumerate}

\

\itemA Proof of part (1) of Sylow's Theorem: Fix $p$. We will argue by induction on $n$ that every group of $n$ has a Sylow $p$-subgroup. 
\begin{enumerate}
\itema Write $n=p^e m$. Address the case $e=0$. Henceforth assume $e>0$, so $p\, |\, n$.
\itema Case 1: Assume that $p$ divides $|Z(G)|$. Explain why there is some $N\trianglelefteq G$ with $|N|=p$.
\itema Apply the induction hypothesis to $G/N$. How can you use this to find a Sylow $p$-subgroup in $G$?
\itema Case 2: Assume that $p$ does not divide $|Z(G)|$. Show that there is some $g\in G$ such that ${[G: C_G(g)]}$ is \emph{not} a multiple of $p$ and \emph{not} one. What does this say about $|C_G(g)|$? What do you get from the induction hypothesis?
\end{enumerate}

\

\itemB Proof of parts (2) and (3) of Sylow's Theorem: Fix a Sylow $p$-subgroup $P$. Let $\mathcal{S}_P$ be the set of conjugates of $P$, namely  $\{ g P g^{-1} \ | \ g\in G\}\subseteq \mathrm{Syl}_p(G)$. We need to show that (2) $\mathrm{Syl}_p(G) = \mathcal{S}_P$ and that (3) $\# \mathrm{Syl}_p(G) \equiv 1 \ \mathrm{mod} \ p$.
\begin{enumerate}
\itemb Let $Q$ be any $p$-subgroup of $G$, and let $Q$ act on $\mathcal{S}_P$ by conjugation. Use the Lemma to show that for any $P_i\in \mathcal{S}_P$, $\mathrm{Stab}_Q(P_i) = Q \cap P_i$.
\itemb Show that $|\mathcal{S}_P| = \sum_{i=1}^s [Q : Q \cap P_i]$ where $P_i$ ranges through a set of representatives of distinct orbits for the action of $Q$ on $\mathcal{S}_P$. 
\itemb Take $Q=P$ and WLOG $P_1=P$. Deduce that $| \mathcal{S}_P | \equiv 1 \ \mathrm{mod} \ p$.
\itemb To show (2) by contradiction, suppose that $Q$ is not contained in any conjugate of $P$. Observe that $Q\cap P_i \subsetneqq Q$ for all $i$. Revisit the equation in part (b) and the conclusion of part (c) to obtain a contradiction.
\itemb Deduce part (3) from part (c) and part (2).
\end{enumerate}

\end{enumerate}

\end{document}
