\documentclass[12pt]{amsart}


\usepackage{times}
\usepackage[margin=.8in]{geometry}
\usepackage{paralist,amsmath,amssymb,multicol,graphicx,framed,ifthen,color,xcolor,stmaryrd,colonequals}
\usepackage[shortlabels]{enumitem}
\usepackage[outline]{contour}
\contourlength{.4pt}
\contournumber{10}
\newcommand{\Bold}[1]{\contour{black}{#1}}

\definecolor{chianti}{rgb}{0.6,0,0}
\definecolor{meretale}{rgb}{0,0,.6}
\definecolor{leaf}{rgb}{0,.35,0}
\newcommand{\Q}{\mathbb{Q}}
\newcommand{\N}{\mathbb{N}}
\newcommand{\Z}{\mathbb{Z}}
\newcommand{\R}{\mathbb{R}}
\newcommand{\C}{\mathbb{C}}
\newcommand{\e}{\varepsilon}
\newcommand{\inv}{^{-1}}
\newcommand{\dabs}[1]{\left| #1 \right|}
\newcommand{\ds}{\displaystyle}
\newcommand{\solution}[1]{\ifthenelse {\equal{\displaysol}{1}} {\begin{framed}{\color{meretale}\noindent #1}\end{framed}} { \ }}
\newcommand{\solutione}[1]{\ifthenelse {\equal{\displaysol}{1}} {\begin{framed}{\color{leaf}This solution is embargoed.}\end{framed}} { \ }}
\newcommand{\showsol}[1]{\def\displaysol{#1}}

\newcommand{\rsa}{\rightsquigarrow}


\newcommand\itemA{\stepcounter{enumi}\item[{\Bold{(\theenumi)}}]}
\newcommand\itemB{\stepcounter{enumi}\item[(\theenumi)]}
\newcommand\itemC{\stepcounter{enumi}\item[{\it{(\theenumi)}}]}
\newcommand\itema{\stepcounter{enumii}\item[{\Bold{(\theenumii)}}]}
\newcommand\itemb{\stepcounter{enumii}\item[(\theenumii)]}
\newcommand\itemc{\stepcounter{enumii}\item[{\it{(\theenumii)}}]}
\newcommand\itemai{\stepcounter{enumiii}\item[{\Bold{(\theenumiii)}}]}
\newcommand\itembi{\stepcounter{enumiii}\item[(\theenumiii)]}
\newcommand\itemci{\stepcounter{enumiii}\item[{\it{(\theenumiii)}}]}
\newcommand\ceq{\colonequals}


\DeclareMathOperator{\ord}{ord}

\DeclareMathOperator{\res}{res}
\setlength\parindent{0pt}
%\usepackage{times}

%\addtolength{\textwidth}{100pt}
%\addtolength{\evensidemargin}{-45pt}
%\addtolength{\oddsidemargin}{-60pt}

\pagestyle{empty}
%\begin{document}\begin{itemize}

%\thispagestyle{empty}




\begin{document}
\showsol{0}
	
	\thispagestyle{empty}
	
	\section*{Normal subgroups}
	
	

\begin{framed}
\textsc{Definition:} A subgroup $N$ of a group $G$ is \textbf{normal} if $gNg^{-1} = N$ for all $g\in G$, where \newline $gNg^{-1} = \{ gng^{-1} \ | \ n\in N\}$. We write $N\trianglelefteq G$ to indicate that $N$ is a normal subgroup of $G$.

\


\textsc{Lemma:} Let $N$ be a subgroup of a group $G$. The following are equivalent:
\begin{enumerate}[(1)]
\item $N$ is a normal subgroup of $G$.
\item For all $g\in G$, $gNg^{-1} \subseteq N$.
\item For all $g\in G$, the \emph{left coset} $gN$ is equal to the \emph{right coset} $Ng$.
\item For all $g\in G$, $gN \subseteq Ng$.
\item For all $g\in G$, $Ng \subseteq gN$.
\end{enumerate}
\end{framed}

\smallskip

\begin{enumerate}
\itemA Examples of normal subgroups: Use the definition and/or the Lemma to show the following:
\begin{enumerate}
\itema If $G$ is an abelian group and $H\leq G$, then $H\trianglelefteq G$.
\itema The center $Z(G)$ of a group $G$ is a normal subgroup\footnote{Recall that we have already shown that $Z(G)\leq G$.} of $G$.
\itema The\footnote{Hint: Recall from HW 1 that $\tau (i \ j) \tau^{-1}= (\tau(i) \ \tau(j))$.} group $K = \{ e, (12)(34), (13)(24), (14)(23) \} \leq S_4$ is normal.
\itema Let $H= \{ e, (12)(34) \} \leq K$, with $K$ as above. Check that $H \trianglelefteq K$ and $K \trianglelefteq S_4$, but $H \not\trianglelefteq S_4$. Draw a moral from this example.
\itemb Is the subgroup of all rotations a normal subgroup of $D_n$?
\itemb Is the subgroup generated by one reflection a normal subgroup of $D_n$?
\end{enumerate}

\


\itemB Prove the Lemma.

\

\itemB Let $G$ be a group and $H\leq G$ a subgroup of index $2$. Show that $H$ must be normal.
\end{enumerate}

\smallskip


\begin{framed}
\textsc{Recall:}
\begin{itemize}
\item An equivalence relation $\sim$ on a group is \textbf{compatible with multiplication} if $x \sim y$ implies ${xz \sim yz}$ and ${zx \sim zy}$ for all $x,y,z\in G$. If $\sim$ is compatible with multiplication, then the equivalence classes of $\sim$ obtain a well-defined group structure via the rule $[x][y] = [xy]$.
\item For a subgroup $H$, we define an equivalence relation on $G$ by $x\sim_H y$ if and only $hx = y$ for some $h\in H$. The equivalence classes are the right cosets $Hx$.
\end{itemize}

\


\textsc{Theorem:} Let $G$ be a group. An equivalence relation $\sim$ is compatible with multiplication if and only if ${\sim \ =\ \sim_N}$ for some $N\trianglelefteq G$.

\

\textsc{Corollary:} If $G$ is a group and $N$ is a normal subgroup, the collection of left cosets $\{gN \ | \ g\in G\}$ of $N$ forms a group by the rule $gN \cdot hN = ghN$.
\end{framed}

\smallskip

\begin{enumerate}\setcounter{enumi}{3}
\itemA Explain why the Corollary follows from the Theorem.

\

\itemA Prove the $(\Leftarrow)$ direction of the Theorem.

\

\itemB Prove\footnote{Hint: The main issue here is to find a candidate $N$. Think first about how you would reconstruct $N$ from $\sim_N$.} the $(\Rightarrow)$ direction of the Theorem.
\end{enumerate}





\end{document}
