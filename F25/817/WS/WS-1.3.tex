\documentclass[12pt]{amsart}


\usepackage{times}
\usepackage[margin=0.7in]{geometry}
\usepackage{paralist,amsmath,amssymb,multicol,graphicx,framed,ifthen,color,xcolor,stmaryrd,enumitem,colonequals}
\usepackage[outline]{contour}
\contourlength{.4pt}
\contournumber{10}
\newcommand{\Bold}[1]{\contour{black}{#1}}

\definecolor{chianti}{rgb}{0.6,0,0}
\definecolor{meretale}{rgb}{0,0,.6}
\definecolor{leaf}{rgb}{0,.35,0}
\newcommand{\Q}{\mathbb{Q}}
\newcommand{\N}{\mathbb{N}}
\newcommand{\Z}{\mathbb{Z}}
\newcommand{\R}{\mathbb{R}}
\newcommand{\C}{\mathbb{C}}
\newcommand{\e}{\varepsilon}
\newcommand{\inv}{^{-1}}
\newcommand{\dabs}[1]{\left| #1 \right|}
\newcommand{\ds}{\displaystyle}
\newcommand{\solution}[1]{\ifthenelse {\equal{\displaysol}{1}} {\begin{framed}{\color{meretale}\noindent #1}\end{framed}} { \ }}
\newcommand{\solutione}[1]{\ifthenelse {\equal{\displaysol}{1}} {\begin{framed}{\color{leaf}This solution is embargoed.}\end{framed}} { \ }}
\newcommand{\showsol}[1]{\def\displaysol{#1}}

\newcommand{\rsa}{\rightsquigarrow}


\newcommand\itemA{\stepcounter{enumi}\item[{\Bold{(\theenumi)}}]}
\newcommand\itemB{\stepcounter{enumi}\item[(\theenumi)]}
\newcommand\itemC{\stepcounter{enumi}\item[{\it{(\theenumi)}}]}
\newcommand\itema{\stepcounter{enumii}\item[{\Bold{(\theenumii)}}]}
\newcommand\itemb{\stepcounter{enumii}\item[(\theenumii)]}
\newcommand\itemc{\stepcounter{enumii}\item[{\it{(\theenumii)}}]}
\newcommand\itemai{\stepcounter{enumiii}\item[{\Bold{(\theenumiii)}}]}
\newcommand\itembi{\stepcounter{enumiii}\item[(\theenumiii)]}
\newcommand\itemci{\stepcounter{enumiii}\item[{\it{(\theenumiii)}}]}
\newcommand\ceq{\colonequals}


\DeclareMathOperator{\ord}{ord}

\DeclareMathOperator{\res}{res}
\setlength\parindent{0pt}
%\usepackage{times}

%\addtolength{\textwidth}{100pt}
%\addtolength{\evensidemargin}{-45pt}
%\addtolength{\oddsidemargin}{-60pt}

\pagestyle{empty}
%\begin{document}\begin{itemize}

%\thispagestyle{empty}




\begin{document}
\showsol{0}
	
	\thispagestyle{empty}
	
	\section*{Dihedral Groups}
	
	

\begin{framed}
\begin{itemize}
\item A \textbf{isometry} of $\mathbb{R}^2$ is a bijective function $f:\mathbb{R}^2 \to \mathbb{R}^2$ that preserves distances between pairs of points; these include rotations around a point, translations, and reflections over a line.

\item Let $X\subseteq \mathbb{R}^2$. A \textbf{symmetry} of $X$ is an isometry of $\mathbb{R}^2$ such that $f(X)=X$ as a set.

\item The \textbf{dihedral group} $D_n$ is the group of symmetries of a regular $n$-gon $P_n$ in the plane, with composition of functions as the group operation.
\end{itemize}


\


\textsc{Theorem:} The dihedral group $D_n$ is indeed a group. It has exactly $2n$ elements consisting of:
\begin{itemize} 
\item The identity map $e$,
\item $n-1$ rotations $r, r^2, \dots, r^{n-1}$, where $r$ is counterclockwise rotation by $2\pi/n$ (so $r^i$ is counterclockwise rotation by $2\pi i/n$),
\item $n$ reflections. More precisely, 
\begin{itemize} 
\item when $n$ is odd, there are $n$ distinct reflections over a line between a vertex and an opposite edge;
\item when $n$ is even, there are $n/2$ distinct reflections between opposite pairs of vertices, and another $n/2$ distinct reflections between opposite pairs of edges.
\end{itemize}
\end{itemize}
\end{framed}

\begin{enumerate}
\itemA Orders of these elements?
\itemA Proof
\end{enumerate}



\begin{framed}

\textsc{Lemma:} Let $v\in P_n$ be a vertex, and $s\in D_n$ the reflection through the axis containing $s$. Let $r\in D_n$ be counterclockwise rotation by $2\pi/n$. Then $s r s^{-1} = r^{-1}$.


\

\textsc{Theorem:} Let $v\in P_n$ be a vertex, and $s\in D_n$ the reflection through the axis containing $s$. Let $r\in D_n$ be counterclockwise rotation by $2\pi/n$.
\begin{enumerate}
\item Every element of $D_n$ can be written uniquely in the form 
\[ r^j \quad \text{for} \ j=0,\dots,n-1, \qquad \text{or} \qquad r^j s \quad \text{for} \ j=0,\dots,n-1.\]
\item $D_n$ is generated by $r,s$.
\item $D_n$ has the group presentation $\langle r,s \ | \ r^n=e, s^2=e, srs^{-1} = r^{-1}\rangle$.
\end{enumerate}



\end{framed}



Symmetries of the circle.

Symmetries of a line.












\end{document}
