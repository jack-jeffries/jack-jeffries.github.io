\documentclass[11pt]{article}
\usepackage[margin=1in]{geometry}
\usepackage{amsmath,amsfonts,amssymb,amsthm,enumerate}
\usepackage[]{graphicx}
\usepackage{color,subfigure}
\definecolor{scarlet}{rgb}{0.81,0,0}
\usepackage{multicol}
\usepackage{float}
\usepackage[all]{xypic}
\usepackage[colorlinks=true,citecolor=scarlet,linkcolor=scarlet]{hyperref}
\usepackage{colonequals}

\usepackage{fancyhdr, lastpage}
\pagestyle{fancy}
\fancyfoot[C]{{\thepage} of \pageref{LastPage}}



\DeclareMathOperator{\mSpec}{mSpec}
\DeclareMathOperator{\Spec}{Spec}
\DeclareMathOperator{\Ass}{Ass}
\DeclareMathOperator{\Supp}{Supp}
\DeclareMathOperator{\height}{height}
\DeclareMathOperator{\Hom}{Hom}
\DeclareMathOperator{\ann}{ann}
\DeclareMathOperator{\End}{End}
\DeclareMathOperator{\coker}{coker}
%\DeclareMathOperator{\ker}{ker}
\DeclareMathOperator{\rank}{rank}
\DeclareMathOperator{\im}{im}
\DeclareMathOperator{\M}{M}
\DeclareMathOperator{\Tor}{Tor}
\DeclareMathOperator{\id}{id}
\DeclareMathOperator{\ch}{char}
\DeclareMathOperator{\Aut}{Aut}

%\DeclareMathOperator{\dim}{dim}

\DeclareMathOperator{\lcm}{lcm}

\def\ra{\rightarrow}
\newcommand{\m}{\mathfrak{m}}
\newcommand{\C}{\mathbb{C}}
\newcommand{\Q}{\mathbb{Q}}
\newcommand{\Z}{\mathbb{Z}}
\newcommand{\ZZ}{\mathbb{Z}}
\newcommand{\R}{\mathbb{R}}
\newcommand{\N}{\mathbb{N}}
\newcommand{\ov}[1]{\overline{#1}}
\newcommand{\norm}{\trianglelefteq}

\def\ov#1{\overline{#1}}


\title{}
\date{\vspace{-0.5in}}

\makeatletter
\g@addto@macro\@floatboxreset\centering
\makeatother

\theoremstyle{definition}
\newtheorem{problem}{Problem}


\begin{document}

\thispagestyle{fancy}
\pagestyle{fancy}
\rhead{UNL $\mid$ Fall 2025}
\lhead{Introduction to Modern Algebra I}

\vspace{3em}

\begin{center}
	{\LARGE Problem Set 4 \\}
	Due Thursday, September 25
\end{center}

\

\noindent
{\bf Instructions:}
You are encouraged to work together on these problems, but each student should hand in their own final draft, written in a way that indicates their individual understanding of the solutions. Never submit something for grading that you do not completely understand. You cannot use any resources besides me, your classmates, and our course notes.


I will post the .tex code for these problems for you to use if you wish to type your homework. If you prefer not to type, please  {\em write neatly}. As a matter of good proof writing style, please use complete sentences and correct grammar. You may use any result stated or proven in class or in a homework problem, provided you reference it appropriately by either stating the result or stating its name (e.g. the definition of ring or Lagrange's Theorem). Please do not refer to theorems by their number in the course notes, as that can change.


\smallskip



\begin{problem} 
Let $G$ be a group, and let $H$ and $K$ be finite subgroups of $G$ of relatively prime order; i.e., $\mathrm{gcd}(|H|, |K|) = 1$. Show that $H \cap K = \{e\}$.
\end{problem} 

\smallskip


\begin{problem} 
For $k\in \mathbb{Z}_{\geq 2}$, let $C_k$ denote the cyclic group of order $k$. Show that for any relatively prime $m,n\geq 2$, there is an isomorphism $C_m \times C_n \cong C_{mn}$.
\end{problem} 

\smallskip


\begin{problem} 
Let $S_n$ denote the symmetric group on $n$ symbols.
\begin{enumerate}[(3.1)]
\item Show that\footnote{Your proof should be no more than a few lines.} the sign map $S_n \to \{\pm 1\}$ is a group homomorphism, where $\{\pm 1\}$ is considered as a subgroup of $\mathbb{R}^\times$. The kernel of this map is called the \textbf{alternating group} on $n$ symbols and denoted $A_n$.
\item Let $n\geq 3$. Show that $A_n$ is generated by the set of $3$-cycles $(i \ j\ k)$ and disjoint pairs\footnote{For $n=3$, there are no disjoint pairs of transpositions.} of transpositions $(i \ j) (k \ \ell)$ in $S_n$. 
\end{enumerate}
\end{problem} 

\smallskip


\noindent \textsc{Defintion:} Let $G$ be a group and $N$ be a subgroup. We say that $N$ is a \textbf{normal} subgroup of $G$ if for all $g$ in $G$, we have $gNg^{-1} \subseteq N$; that is, for any $g\in G$ and any $n\in N$, we have that $gng^{-1}\in N$. We write $N \trianglelefteq G$ to say $N$ is a normal subgroup of $G$.

\begin{problem} Let $f: G\to H$ be a group homomorphism.
\begin{enumerate}[(4.1)]
\item Show that $\ker(f) \trianglelefteq G$.
\item Show that if $K \trianglelefteq H$, then $f^{-1}(K) \trianglelefteq G$.
\end{enumerate}
\end{problem}

\smallskip

\begin{problem}
Let $G$ be a group, $S$ a subset of $G$, and $H=\langle S \rangle$.  

\begin{enumerate}[(5.1)]
\item Prove that $H \norm G$ if and only if $gsg^{-1}\in H$ for every $s\in S$ and $g\in G$.

\item Consider the commutator subgroup of $G$
$$[G,G] \colonequals \langle aba^{-1}b^{-1}\mid a,b\in G \rangle$$ 
generated by all the commutators of elements in $G$.
Prove that $[G,G] \norm G$.


\end{enumerate}
\end{problem}





\end{document}