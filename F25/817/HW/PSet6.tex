\documentclass[11pt]{article}
\usepackage[margin=1in]{geometry}
\usepackage{amsmath,amsfonts,amssymb,amsthm,enumerate}
\usepackage[]{graphicx}
\usepackage{color,subfigure}
\definecolor{scarlet}{rgb}{0.81,0,0}
\usepackage{multicol}
\usepackage{float}
\usepackage[all]{xypic}
\usepackage[colorlinks=true,citecolor=scarlet,linkcolor=scarlet]{hyperref}
\usepackage{colonequals}

\usepackage{fancyhdr, lastpage}
\pagestyle{fancy}
\fancyfoot[C]{{\thepage} of \pageref{LastPage}}



\DeclareMathOperator{\mSpec}{mSpec}
\DeclareMathOperator{\Spec}{Spec}
\DeclareMathOperator{\Ass}{Ass}
\DeclareMathOperator{\Supp}{Supp}
\DeclareMathOperator{\height}{height}
\DeclareMathOperator{\Hom}{Hom}
\DeclareMathOperator{\ann}{ann}
\DeclareMathOperator{\End}{End}
\DeclareMathOperator{\coker}{coker}
%\DeclareMathOperator{\ker}{ker}
\DeclareMathOperator{\rank}{rank}
\DeclareMathOperator{\im}{im}
\DeclareMathOperator{\M}{M}
\DeclareMathOperator{\Tor}{Tor}
\DeclareMathOperator{\id}{id}
\DeclareMathOperator{\ch}{char}
\DeclareMathOperator{\Aut}{Aut}

%\DeclareMathOperator{\dim}{dim}

\DeclareMathOperator{\lcm}{lcm}

\def\ra{\rightarrow}
\newcommand{\m}{\mathfrak{m}}
\newcommand{\C}{\mathbb{C}}
\newcommand{\Q}{\mathbb{Q}}
\newcommand{\Z}{\mathbb{Z}}
\newcommand{\ZZ}{\mathbb{Z}}
\newcommand{\R}{\mathbb{R}}
\newcommand{\N}{\mathbb{N}}
\newcommand{\ov}[1]{\overline{#1}}
\newcommand{\norm}{\trianglelefteq}

\def\ov#1{\overline{#1}}


\title{}
\date{\vspace{-0.5in}}

\makeatletter
\g@addto@macro\@floatboxreset\centering
\makeatother

\theoremstyle{definition}
\newtheorem{problem}{Problem}


\begin{document}

\thispagestyle{fancy}
\pagestyle{fancy}
\rhead{UNL $\mid$ Fall 2025}
\lhead{Introduction to Modern Algebra I}

\vspace{3em}

\begin{center}
	{\LARGE Problem Set 6 \\}
	Due Thursday, October 9
\end{center}

\

\noindent
{\bf Instructions:}
You are encouraged to work together on these problems, but each student should hand in their own final draft, written in a way that indicates their individual understanding of the solutions. Never submit something for grading that you do not completely understand. You cannot use any resources besides me, your classmates, and our course notes.


I will post the .tex code for these problems for you to use if you wish to type your homework. If you prefer not to type, please  {\em write neatly}. As a matter of good proof writing style, please use complete sentences and correct grammar. You may use any result stated or proven in class or in a homework problem, provided you reference it appropriately by either stating the result or stating its name (e.g. the definition of ring or Lagrange's Theorem). Please do not refer to theorems by their number in the course notes, as that can change.


\smallskip

\begin{problem}
Let $G$ be a group. Show that $G/Z(G) \cong \mathrm{Inn}(G)$.
\end{problem}

\

\begin{problem} \phantom{ }
\begin{enumerate}[(2.1)]
\item Let $f:A \to B$ be a homomorphism of groups. Show that if $B$ is finite, then $|\mathrm{im}(f)|$ divides~$|B|$.
\item  Let $G$ be a finite group, $H$ and $N$ subgroups of $G$ such that $|H|$ and $[G : N]$ are relatively prime. Prove that if $N \trianglelefteq G$ then $H \subseteq N$.
\end{enumerate}
\end{problem}


\



\begin{problem} A finite group $G$ is called \textbf{solvable} if there exists a finite chain of subgroups
  $$
  \{e\} = H_m \leq H_{m-1} \leq \cdots \leq H_1 \leq H_0 = G
  $$
  such that $H_{j+1} \trianglelefteq H_j$ and $H_j/H_{j+1}$ is cyclic of prime order, for all $0 \leq j \leq m-1$. 

  Prove\footnote{Hint: Consider the chain  of subgroups of $M$ obtained from the given chain for $G$ by
    intersecting with $M$.} that if $G$ is solvable and $M \leq G$, then $M$ is also
  solvable.
    
    \end{problem}
    
    \
    
    
    \begin{problem} Let $n\geq 2$ and $S_n$ be the permutation group on $n$ elements. Show that the abelianization of $S_n$ is isomorphic to $\mathbb{Z}/2$.
    \end{problem} 
    
    
    
    
    
    \
    
    
    \begin{problem}
    Find, with proof, a presentation of the quaternion group $Q_8$ using two generators.
    \end{problem}





\end{document}