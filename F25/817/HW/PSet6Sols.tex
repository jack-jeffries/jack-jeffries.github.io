\documentclass[11pt]{article}
\usepackage[margin=1in]{geometry}
\usepackage{amsmath,amsfonts,amssymb,amsthm,enumerate}
\usepackage[]{graphicx}
\usepackage{color,subfigure}
\definecolor{scarlet}{rgb}{0.81,0,0}
\usepackage{multicol}
\usepackage{float}
\usepackage[all]{xypic}
\usepackage[colorlinks=true,citecolor=scarlet,linkcolor=scarlet]{hyperref}
\usepackage{colonequals}

\usepackage{fancyhdr, lastpage}
\pagestyle{fancy}
\fancyfoot[C]{{\thepage} of \pageref{LastPage}}



\DeclareMathOperator{\mSpec}{mSpec}
\DeclareMathOperator{\Spec}{Spec}
\DeclareMathOperator{\Ass}{Ass}
\DeclareMathOperator{\Supp}{Supp}
\DeclareMathOperator{\height}{height}
\DeclareMathOperator{\Hom}{Hom}
\DeclareMathOperator{\ann}{ann}
\DeclareMathOperator{\End}{End}
\DeclareMathOperator{\coker}{coker}
%\DeclareMathOperator{\ker}{ker}
\DeclareMathOperator{\rank}{rank}
\DeclareMathOperator{\im}{im}
\DeclareMathOperator{\M}{M}
\DeclareMathOperator{\Tor}{Tor}
\DeclareMathOperator{\id}{id}
\DeclareMathOperator{\ch}{char}
\DeclareMathOperator{\Aut}{Aut}

%\DeclareMathOperator{\dim}{dim}

\DeclareMathOperator{\lcm}{lcm}

\def\ra{\rightarrow}
\newcommand{\m}{\mathfrak{m}}
\newcommand{\C}{\mathbb{C}}
\newcommand{\Q}{\mathbb{Q}}
\newcommand{\Z}{\mathbb{Z}}
\renewcommand{\a}{\alpha}
\newcommand{\sign}{\mathrm{sign}}
\newcommand{\ZZ}{\mathbb{Z}}
\newcommand{\R}{\mathbb{R}}
\newcommand{\N}{\mathbb{N}}
\newcommand{\ov}[1]{\overline{#1}}
\newcommand{\norm}{\trianglelefteq}
\newcommand{\nsg}{\trianglelefteq}

\def\ov#1{\overline{#1}}


\title{}
\date{\vspace{-0.5in}}

\makeatletter
\g@addto@macro\@floatboxreset\centering
\makeatother

\theoremstyle{definition}
\newtheorem{problem}{Problem}


\begin{document}

\thispagestyle{fancy}
\pagestyle{fancy}
\rhead{UNL $\mid$ Fall 2025}
\lhead{Introduction to Modern Algebra I}

\vspace{3em}

\begin{center}
	{\LARGE Problem Set 6 \\}
	Due Thursday, October 9
\end{center}

\

\noindent
{\bf Instructions:}
You are encouraged to work together on these problems, but each student should hand in their own final draft, written in a way that indicates their individual understanding of the solutions. Never submit something for grading that you do not completely understand. You cannot use any resources besides me, your classmates, and our course notes.


I will post the .tex code for these problems for you to use if you wish to type your homework. If you prefer not to type, please  {\em write neatly}. As a matter of good proof writing style, please use complete sentences and correct grammar. You may use any result stated or proven in class or in a homework problem, provided you reference it appropriately by either stating the result or stating its name (e.g. the definition of ring or Lagrange's Theorem). Please do not refer to theorems by their number in the course notes, as that can change.


\smallskip

\begin{problem}
Let $G$ be a group. Show that $G/Z(G) \cong \mathrm{Inn}(G)$.
\end{problem}

\begin{proof}
Define $f:G\to \operatorname{Inn}(G)$ by $f(g)=\psi_g$.  In the process of proving that $\operatorname{Inn}(G)$ is a subgroup in an earlier problem set, we proved that $\psi_g\psi_h=\psi_{gh}$ for all $g,h\in G$.  Hence, $f(gh)=\psi_{gh}=\psi_g\psi_h=f(g)f(h)$, so $f$ is a homomorphism.  Clearly, $f$ is surjective.  Note that $g\in \ker f$ if and only if $\psi_g=\operatorname{id}_G$, which is if and only if $gxg^{-1}=x$ for all $x\in G$.  This last condition is equivalent to saying $g\in \operatorname{Z}(G)$.  Thus, $\ker f=\operatorname{Z}(G)$.  The result now follows from the first isomorphism theorem.
\end{proof}

\begin{problem} \phantom{ }
\begin{enumerate}[(2.1)]
\item Let $f:A \to B$ be a homomorphism of groups. Show that if $B$ is finite, then $|\mathrm{im}(f)|$ divides~$|B|$.

\begin{proof}
By the First Isomorphism Theorem, $\im(f) \cong A/\ker(f)$, and hence 
$$|\im(f)| = |A/\ker(f)| = \frac{|A|}{|\ker(f)|}$$ 
where the last equality follows from Lagrange's Theorem.
Thus $|\im(f)|$ divides $|A|$.
\end{proof}


\item  Let $G$ be a finite group, $H$ and $N$ subgroups of $G$ such that $|H|$ and $[G : N]$ are relatively prime. Prove that if $N \trianglelefteq G$ then $H \subseteq N$.

\begin{proof}
Let $i\!: H \to G$ be the inclusion homomorphism and $\pi\!: G \to G/N$ be the canonical projection. Then
$$f = \pi \circ i \!: H \to G/N$$ 
is also a homomorphism, as the composition of homomorphisms is a homomorphism. Note that $f(h) = hN$ for any $h \in H$. By part (a), $|\im(f)|$ divides $|H|$. Moreover, $\im(f)$ is a subgroup of $G/N$, so by Lagrange's Theorem, $|\im(f)|$ also divides $|G/N| = [G:N]$. Thus $|\im(f)|$ divides both $|H|$ and $[G:N]$. Since $|H|$ and $[G:N]$ are relatively prime, we conclude that $|\im(f)| = 1$ and hence $f$ is the trivial map. Therefore, for all $h \in H$ we have $hN = f(h) = N$, which implies that $h \in N$. We conclude that $H \subseteq N$.
\end{proof}

\begin{proof}[Alternative proof]
We can instead apply the Second
      Isomorphism Theorem to get that 
      $$H/(H \cap N) \cong HN/N$$ 
      and hence $|H/(H \cap N)| = |HN/N|$. Since $HN/N$ is a subgroup of $G/N$, its order divides $|G/N| = [G: N]$.
      On the other hand,
      $$|H/(H \cap N)| = [H: H \cap N],$$ 
      which divides $|H|$. Since $[G:N]$ and $|H|$ are relatively prime, we must have $[H: H \cap N] = 1$ and hence $H \cap N = H$. This implies $H \subseteq N$.   
    \end{proof}
    
\end{enumerate}
\end{problem}


\



\begin{problem} A finite group $G$ is called \textbf{solvable} if there exists a finite chain of subgroups
  $$
  \{e\} = H_m \leq H_{m-1} \leq \cdots \leq H_1 \leq H_0 = G
  $$
  such that $H_{j+1} \trianglelefteq H_j$ and $H_j/H_{j+1}$ is cyclic of prime order, for all $0 \leq j \leq m-1$. 

  Prove\footnote{Hint: Consider the chain  of subgroups of $M$ obtained from the given chain for $G$ by
    intersecting with $M$.} that if $G$ is solvable and $M \leq G$, then $M$ is also
  solvable.
    
    \end{problem}
    
    \begin{proof} Set $M_j = H_j \cap M$. Then $M_m = \{e\}$, $M_0 = M$, and $M_{j+1} \nsg M_j$ for all $j$.
      By the Second Isomorphism Theorem, we have
      $$
      M_j/M_{j+1} \cong (H_j \cap M)H_j/H_{j+1}
      $$
      and we note that $(H_j \cap M)H_j/H_{j+1} \leq H_j/H_{j+1}$. Since $H_j/H_{j+1}$ has prime order, this proves that $M_j/M_{j+1}$ is either trivial or cyclic of prime order. If it is
      trivial, then $M_{j+1} = M_j$. So, upon discarding all $M_{j+1}$'s such that $M_{j+1} = M_j$, we obtain a chain
  $$
  \{e\} = M_s \leq M_{m-1} \leq \cdots \leq M_1 \leq M_0 = G
  $$
  such that $M_j/M_{j+1}$ has prime order for all $j$. That is, $M$ is solvable.
\end{proof}

    
    
    \begin{problem} Let $n\geq 2$ and $S_n$ be the permutation group on $n$ elements. Show that the abelianization of $S_n$ is isomorphic to $\mathbb{Z}/2$.
    \end{problem} 
    
     \begin{proof} 
       Assume $n \geq 2$. Let $1 \leq i < j \leq n$ and $1 \leq a < b \leq n$. There is a permutation $\a$ such that $\a(i) = a$ and $\a(j) = b$. Thus
       $\a(i \, j) \a^{-1} = (\a(i) \, \a(j)) = (a \, b)$ and hence
       $$
       [\a, (i \, j)] =  (a \, b) (i \, j).
       $$
       This proves that every product of two transpositions is in $S_n'$ and, since $S'_n$ is subgorup, it follows that every product of an even number of transpositions is in
       $S'_n$. Thus $A_n \subseteq S_n'$.

       For the opposite containment, merely note $\sign([\a,\b]) = [\sign(\a), \sign(\b)] = e$ since $\{\pm 1\}$ is abelian, and thus every commutator is even. 

So,        the abelianization of $S_n$ is $S_n/A_n \cong \{\pm 1 \}$ (by the First Isomorphism Theorem), which has order two.
     \end{proof}

    
    
    
    
    
    \newpage 
    
    
    \begin{problem}
    Find, with proof, a presentation of the quaternion group $Q_8$ using two generators.
    \end{problem}
    
    
    \begin{proof}
    We claim that $\langle x, y \mid x^4=1, x^2y^2=1, yxy^3x=1\rangle$ is a presentation for $Q_8$.  Let $F$ be the free group on $\{x,y\}$ and $N$ the normal subgroup of $F$ generated by $\{x^4, x^2y^2, yxy^3x\}$.  Then, by the definition of free group, there exists a (unique) group homomorphism $f:F\to Q_8$ by $f(x)=i$ and $f(y)=j$.  As $Q_8=\langle i,j\rangle$, we see that $f$ must be surjective.  It is easily seen that $f(x^4)=i^4=1$, $f(x^2y^2)=i^2j^2=1$ and $f(yxy^3x)=jij^3i=1$.
Hence, $N\subseteq \ker f$.  Thus, there exists an induced homomorphism $\overline{f}:F/N\to Q_8$.  Note that in $F/N$, $\overline{x}^4=1$, $\overline{x}^2=\overline{y}^{-2}$ (and so $\overline{y}^4=1$ and $\overline{y}^2=\overline{x}^2$) and $\overline{y}\overline{x}=\overline{x^3}\overline{y}$.  Thus, every element in $F/N$ can be expressed in the form $\overline{x}^i\overline{y}^j$ for $0\le i\le 3$ and $0\le j\le 1$.  Thus, $|F/N|\le 8$.  Since $\overline{f}$ is surjective and $|Q_8|=8$, we must have $f$ is bijective and thus an isomorphism.
\end{proof}




\end{document}