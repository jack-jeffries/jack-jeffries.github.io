\documentclass[11pt]{article}
\usepackage[margin=1in]{geometry}
\usepackage{amsmath,amsfonts,amssymb,amsthm,enumerate}
\usepackage[]{graphicx}
\usepackage{color,subfigure}
\definecolor{scarlet}{rgb}{0.81,0,0}
\usepackage{multicol}
\usepackage{float}
\usepackage[all]{xypic}
\usepackage[colorlinks=true,citecolor=scarlet,linkcolor=scarlet]{hyperref}
\usepackage{colonequals}

\usepackage{fancyhdr, lastpage}
\pagestyle{fancy}
\fancyfoot[C]{{\thepage} of \pageref{LastPage}}



\DeclareMathOperator{\mSpec}{mSpec}
\DeclareMathOperator{\Spec}{Spec}
\DeclareMathOperator{\Ass}{Ass}
\DeclareMathOperator{\Supp}{Supp}
\DeclareMathOperator{\height}{height}
\DeclareMathOperator{\Hom}{Hom}
\DeclareMathOperator{\ann}{ann}
\DeclareMathOperator{\End}{End}
\DeclareMathOperator{\coker}{coker}
%\DeclareMathOperator{\ker}{ker}
\DeclareMathOperator{\rank}{rank}
\DeclareMathOperator{\im}{im}
\DeclareMathOperator{\M}{M}
\DeclareMathOperator{\Tor}{Tor}
\DeclareMathOperator{\id}{id}
\DeclareMathOperator{\ch}{char}
\DeclareMathOperator{\Aut}{Aut}
%\DeclareMathOperator{\dim}{dim}

\DeclareMathOperator{\lcm}{lcm}

\def\ra{\rightarrow}
\newcommand{\m}{\mathfrak{m}}
\newcommand{\C}{\mathbb{C}}
\newcommand{\Q}{\mathbb{Q}}
\newcommand{\Z}{\mathbb{Z}}
\newcommand{\R}{\mathbb{R}}
\newcommand{\N}{\mathbb{N}}
\newcommand{\ov}[1]{\overline{#1}}

\def\ov#1{\overline{#1}}


\title{}
\date{\vspace{-0.5in}}

\makeatletter
\g@addto@macro\@floatboxreset\centering
\makeatother

\theoremstyle{definition}
\newtheorem{problem}{Problem}


\begin{document}

\thispagestyle{fancy}
\pagestyle{fancy}
\rhead{UNL $\mid$ Fall 2025}
\lhead{Introduction to Modern Algebra I}

\vspace{3em}

\begin{center}
	{\LARGE Problem Set 2 \\}
	Due Wednesday, September 10
\end{center}

\

\noindent
{\bf Instructions:}
You are encouraged to work together on these problems, but each student should hand in their own final draft, written in a way that indicates their individual understanding of the solutions. Never submit something for grading that you do not completely understand. You cannot use any resources besides me, your classmates, and our course notes.


I will post the .tex code for these problems for you to use if you wish to type your homework. If you prefer not to type, please  {\em write neatly}. As a matter of good proof writing style, please use complete sentences and correct grammar. You may use any result stated or proven in class or in a homework problem, provided you reference it appropriately by either stating the result or stating its name (e.g. the definition of ring or Lagrange's Theorem). Please do not refer to theorems by their number in the course notes, as that can change.




\begin{problem}
Find $\mathrm{Z}(D_{n})$ for $n \geqslant 3$.  

\noindent Hint: your answer will depend on whether $n$ is even or odd.
\end{problem}


\begin{problem}
List\footnote{Hint: Reuse your work from Problem Set \#1.} all of the orders of elements of $S_5$ and how many elements have each such order and justify your answer.
\end{problem}



\begin{problem} Let $G$ be a group.
 \begin{enumerate}[(3.1)]
 \item Show that if $g^n=e$ for some $n \geq 1$, then $|g|$ divides $n$.
\item Let $g \in G$ be an element of finite order. Show that $g^m$ has finite order for any integer $m \geq 0$, and in fact
\[
|g^m| = \frac{\lcm(m,|g|)}{m} = \frac{|g|}{\gcd(m, |g|))}.
\]
\item Prove that for all $g, h$ in  $G$, $|gh| = |hg|$ holds.
\end{enumerate}
\end{problem}


\begin{problem}
	Let $C$ denote a circle in the plane $\R^2$. One can show, along similar lines to our analysis of $D_n$, that the group $G$ of symmetries of $C$ consists exactly of the following elements:
	\[\begin{aligned}  r_\alpha &= \ \text{rotation counterclockwise by } \ 2 \pi \alpha \quad &\text{for} \ 0\leq \alpha <1 \\
s_\alpha &= \ \text{reflection over the line through the center with angle} \ 2 \pi \alpha \qquad &\text{for} \ 0\leq \alpha <1\end{aligned}\]
You do not have to prove this. 
\begin{enumerate}[(4.1)]
\item Compute the order of each element of $G$.
\item Find two elements $g,h\in G$ of finite order with product $gh$ of infinite order.
\end{enumerate}
\end{problem}

\begin{problem}
Show that for every integer $n \geqslant 2$, there is no nontrivial group homomorphism $\Z/n \longrightarrow \Z$.
\end{problem}







\end{document}