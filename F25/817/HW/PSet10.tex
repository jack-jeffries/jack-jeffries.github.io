\documentclass[11pt]{article}
\usepackage[margin=1in]{geometry}
\usepackage{amsmath,amsfonts,amssymb,amsthm,enumerate}
\usepackage[]{graphicx}
\usepackage{color,subfigure}
\definecolor{scarlet}{rgb}{0.81,0,0}
\usepackage{multicol}
\usepackage{float}
\usepackage[all]{xypic}
\usepackage[colorlinks=true,citecolor=scarlet,linkcolor=scarlet]{hyperref}
\usepackage{colonequals}

\usepackage{fancyhdr, lastpage}
\pagestyle{fancy}
\fancyfoot[C]{{\thepage} of \pageref{LastPage}}



\DeclareMathOperator{\mSpec}{mSpec}
\DeclareMathOperator{\Spec}{Spec}
\DeclareMathOperator{\Ass}{Ass}
\DeclareMathOperator{\Supp}{Supp}
\DeclareMathOperator{\height}{height}
\DeclareMathOperator{\Hom}{Hom}
\DeclareMathOperator{\ann}{ann}
\DeclareMathOperator{\End}{End}
\DeclareMathOperator{\coker}{coker}
%\DeclareMathOperator{\ker}{ker}
\DeclareMathOperator{\rank}{rank}
\DeclareMathOperator{\im}{im}
\DeclareMathOperator{\M}{M}
\DeclareMathOperator{\Tor}{Tor}
\DeclareMathOperator{\id}{id}
\DeclareMathOperator{\ch}{char}
\DeclareMathOperator{\Aut}{Aut}

%\DeclareMathOperator{\dim}{dim}

\DeclareMathOperator{\lcm}{lcm}

\def\ra{\rightarrow}
\newcommand{\m}{\mathfrak{m}}
\newcommand{\C}{\mathbb{C}}
\newcommand{\Q}{\mathbb{Q}}
\newcommand{\Z}{\mathbb{Z}}
\newcommand{\ZZ}{\mathbb{Z}}
\newcommand{\R}{\mathbb{R}}
\newcommand{\N}{\mathbb{N}}
\newcommand{\ov}[1]{\overline{#1}}
\newcommand{\norm}{\trianglelefteq}

\def\ov#1{\overline{#1}}


\title{}
\date{\vspace{-0.5in}}

\makeatletter
\g@addto@macro\@floatboxreset\centering
\makeatother

\theoremstyle{definition}
\newtheorem{problem}{Problem}


\begin{document}

\thispagestyle{fancy}
\pagestyle{fancy}
\rhead{UNL $\mid$ Fall 2025}
\lhead{Introduction to Modern Algebra I}

\vspace{3em}

\begin{center}
	{\LARGE Problem Set 10 \\}
	Due Thursday, November 13
\end{center}

\

\noindent
{\bf Instructions:}
You are encouraged to work together on these problems, but each student should hand in their own final draft, written in a way that indicates their individual understanding of the solutions. Never submit something for grading that you do not completely understand. You cannot use any resources besides me, your classmates, and our course notes.


I will post the .tex code for these problems for you to use if you wish to type your homework. If you prefer not to type, please  {\em write neatly}. As a matter of good proof writing style, please use complete sentences and correct grammar. You may use any result stated or proven in class or in a homework problem, provided you reference it appropriately by either stating the result or stating its name (e.g. the definition of ring or Lagrange's Theorem). Please do not refer to theorems by their number in the course notes, as that can change.

\


\begin{problem}
Let $p$ be a prime. Classify\footnote{Hint: Consider the center.} all groups of order $p^2$ up to isomorphism.
\end{problem}

\

\begin{problem}
	Consider a group $G$ of order $75 = 5^2 \cdot 3$.
	\begin{enumerate}
	\item[(a)] Show that if $G$ contains an element of order $25$ then $G$ is cyclic.
	\item[(b)] Show that there exists some group $G_1$ of order $75$ that is abelian but not cyclic, and there exists some group $G_2$ of order $75$ that is not abelian.
	\end{enumerate} 
\end{problem}

\

\begin{problem}
Let $G$ be a group of order $231 = 3\cdot 7\cdot 11$.  Prove that there is a unique Sylow $11$-subgroup of $G$, and that it is contained in $\operatorname{Z}(G)$.
\end{problem}


\

\begin{problem}
Prove\footnote{Hint: You may want to show that (1) there is either a unique Sylow $5$-subgroup or a unique Sylow $7$-subgroup of $G$, and (2) $G$ has a cyclic subgroup of order $35$. \\You can also without proof the Exercise from class giving a sufficient condition for two semidirect products to be isomorphic.}
that there are precisely two groups of order $105 = 3 \cdot 5 \cdot 7$ up to isomorphism. 



\end{problem}






\end{document}