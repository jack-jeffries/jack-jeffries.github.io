\documentclass[11pt]{article}
\usepackage[margin=1in]{geometry}
\usepackage{amsmath,amsfonts,amssymb,amsthm,enumerate}
\usepackage[]{graphicx}
\usepackage{color,subfigure}
\definecolor{scarlet}{rgb}{0.81,0,0}
\usepackage{multicol}
\usepackage{float}
\usepackage[all]{xypic}
\usepackage[colorlinks=true,citecolor=scarlet,linkcolor=scarlet]{hyperref}
\usepackage{colonequals}

\usepackage{fancyhdr, lastpage}
\pagestyle{fancy}
\fancyfoot[C]{{\thepage} of \pageref{LastPage}}



\DeclareMathOperator{\mSpec}{mSpec}
\DeclareMathOperator{\Spec}{Spec}
\DeclareMathOperator{\Ass}{Ass}
\DeclareMathOperator{\Supp}{Supp}
\DeclareMathOperator{\height}{height}
\DeclareMathOperator{\Hom}{Hom}
\DeclareMathOperator{\ann}{ann}
\DeclareMathOperator{\End}{End}
\DeclareMathOperator{\coker}{coker}
%\DeclareMathOperator{\ker}{ker}
\DeclareMathOperator{\rank}{rank}
\DeclareMathOperator{\im}{im}
\DeclareMathOperator{\M}{M}
\DeclareMathOperator{\Tor}{Tor}
\DeclareMathOperator{\id}{id}
\DeclareMathOperator{\ch}{char}
\DeclareMathOperator{\Aut}{Aut}

%\DeclareMathOperator{\dim}{dim}

\DeclareMathOperator{\lcm}{lcm}

\def\ra{\rightarrow}
\newcommand{\m}{\mathfrak{m}}
\newcommand{\C}{\mathbb{C}}
\newcommand{\Q}{\mathbb{Q}}
\newcommand{\Z}{\mathbb{Z}}
\newcommand{\ZZ}{\mathbb{Z}}
\newcommand{\R}{\mathbb{R}}
\newcommand{\N}{\mathbb{N}}
\newcommand{\ov}[1]{\overline{#1}}
\newcommand{\norm}{\trianglelefteq}

\def\ov#1{\overline{#1}}


\title{}
\date{\vspace{-0.5in}}

\makeatletter
\g@addto@macro\@floatboxreset\centering
\makeatother

\theoremstyle{definition}
\newtheorem{problem}{Problem}


\begin{document}

\thispagestyle{fancy}
\pagestyle{fancy}
\rhead{UNL $\mid$ Fall 2025}
\lhead{Introduction to Modern Algebra I}

\vspace{3em}

\begin{center}
	{\LARGE Problem Set 4 \\}
	Due Thursday, September 25
\end{center}

\

\noindent
{\bf Instructions:}
You are encouraged to work together on these problems, but each student should hand in their own final draft, written in a way that indicates their individual understanding of the solutions. Never submit something for grading that you do not completely understand. You cannot use any resources besides me, your classmates, and our course notes.


I will post the .tex code for these problems for you to use if you wish to type your homework. If you prefer not to type, please  {\em write neatly}. As a matter of good proof writing style, please use complete sentences and correct grammar. You may use any result stated or proven in class or in a homework problem, provided you reference it appropriately by either stating the result or stating its name (e.g. the definition of ring or Lagrange's Theorem). Please do not refer to theorems by their number in the course notes, as that can change.


\smallskip



\begin{problem} 
Let $G$ be a group, and let $H$ and $K$ be finite subgroups of $G$ of relatively prime order; i.e., $\mathrm{gcd}(|H|, |K|) = 1$. Show that $H \cap K = \{e\}$.
\end{problem} 

	\begin{proof}
	Let $g\in H\cap K$. By Lagrange's Theorem, $|g|$ both divides $|H|$ and $|K|$, and hence their GCD, which is one. Thus $g=e$, so $H\cap K=\{e\}$.
	\end{proof}


\begin{problem} 
For $k\in \mathbb{Z}_{\geq 2}$, let $C_k$ denote the cyclic group of order $k$. Show that for any relatively prime $m,n\geq 2$, there is an isomorphism $C_m \times C_n \cong C_{mn}$.
\end{problem} 

\begin{proof}
Fix generators $x$ for $C_m$ and $y$ for $C_n$ so that we have $|x|=m$ and $|y|=n$. We claim that $|(x,y)|=mn$. Indeed,
\[
(x,y)^d=(x^d,y^d)=(e_{C_m},e_{C_n})\footnote{The identity element of $G_1\times G_2$ is $(e_{G_1},e_{G_2})$} \iff \begin{cases} x^d=e_{C_m}\\ y^d=e_{C_n} \end{cases} \iff 
\begin{cases} m\mid d\\ n\mid d
\end{cases}
\iff \lcm(m,n)\mid d.
\] 
Since $\gcd(m,n)=1$, we have $\lcm(m,n)=mn$ hence the smallest $d\geq 1$ so that $(x,y)^d=(e_{C_m},e_{C_n})$ is $mn$. This shows $|(x,y)|=mn$ in $C_m\times C_n$. 

BY the UMP of the cyclic group there is a group homomorphism $f:C_{mn}\to C_m\times C_n$ so that $f(a^i)=(x,y)^i$, where $a$ is a generator of $C_{mn}$ and $i\in\Z$ is arbitrary. We see that $f(a^i)=(e_{C_m}, e_{C_n})$ if and only if $(x,y)^i=(e_{C_m}, e_{C_n})$ if and only if $mn\mid i$ (by the argument above) if an only  if $a^i=e_{C_{mn}}$.
Thus $\ker(f)=\{e_{C_{mn}}\}$ and thus $f$ is injective. Since $f$ maps between sets of the same cardinality $mn$ and is injective it must be bijective, hence an isomorphism.
\end{proof}


\begin{problem} 
Let $S_n$ denote the symmetric group on $n$ symbols.
\begin{enumerate}[(3.1)]
\item Show that\footnote{Your proof should be no more than a few lines.} the sign map $S_n \to \{\pm 1\}$ is a group homomorphism, where $\{\pm 1\}$ is considered as a subgroup of $\mathbb{R}^\times$. The kernel of this map is called the \textbf{alternating group} on $n$ symbols and denoted $A_n$.

\begin{proof} To verify that this is a homomorphism, let $\sigma, \tau\in S_n$; we proceed by cases. If $\mathrm{sign}(\sigma) = \mathrm{sign}(\tau)= 1$, then we can write $\sigma$ as a product of $2m$ transpositions for some $m$ and $\tau$ as a product of $2n$ transpositions; then $\sigma \tau$ can be written as a product of $2(m+n)$ transpositions, so $\mathrm{sign}(\sigma\tau) = 1 = \mathrm{sign}(\sigma) \cdot \mathrm{sign}(\tau)$. If $\mathrm{sign}(\sigma) = 1$ and $\mathrm{sign}(\tau)=-1$, then  we can write $\sigma$ as a product of $2m$ transpositions for some $m$ and $\tau$ as a product of $2n+1$ transpositions; then $\sigma \tau$ can be written as a product of $2(m+n)+1$ transpositions, so $\mathrm{sign}(\sigma\tau) = -1 = \mathrm{sign}(\sigma) \cdot \mathrm{sign}(\tau)$. The other cases are similar.
\end{proof}




\item Let $n\geq 3$. Show that $A_n$ is generated by the set of $3$-cycles $(i \ j\ k)$ and disjoint pairs\footnote{For $n=3$, there are no disjoint pairs of transpositions.} of transpositions $(i \ j) (k \ \ell)$ in $S_n$. 

\begin{proof} 


\end{proof}

\end{enumerate}
\end{problem} 

\smallskip


\noindent \textsc{Defintion:} Let $G$ be a group and $N$ be a subgroup. We say that $N$ is a \textbf{normal} subgroup of $G$ if for all $g$ in $G$, we have $gNg^{-1} \subseteq N$; that is, for any $g\in G$ and any $n\in N$, we have that $gng^{-1}\in N$. We write $N \trianglelefteq G$ to say $N$ is a normal subgroup of $G$.

\begin{problem} Let $f: G\to H$ be a group homomorphism.
\begin{enumerate}[(4.1)]
\item Show that $\ker(f) \trianglelefteq G$.


\begin{proof}
	We already know that $\ker f$ is a subgroup of $G$, so we only need to prove normality. Consider $g \in \ker f$, and any $h \in G$. Then
	$$\begin{aligned}
	f(hgh^{-1}) & = f(h)f(g)f(h)^{-1} & \textrm{since } f \textrm{ is a homomorphism} \\
	& = f(h)f(h)^{-1} & \textrm{since } f(g) = e_H \\
	& = e_H,
	\end{aligned}$$
	so $hgh^{-1} \in H$. We conclude that $H$ is normal.
\end{proof}

\item Show that if $K \trianglelefteq H$, then $f^{-1}(K) \trianglelefteq G$.


\begin{proof}
We have already shown that the preimage of a subgroup is a subgroup. We justify the normality of the preimage.
Let $g\in G$ and $\ell\in f^{-1}(K)$. Then $f(\ell)\in K$ and $f(g\ell g^{-1})=f(g)f(\ell)f(g)^{-1}\in K$ by the normality of $K$. Therefore $g\ell g^{-1}\in  f^{-1}(K)$ for all $g\in G$ and so  $gf^{-1}(K)g^{-1}\subseteq  f^{-1}(K)$ for all $g\in G$. This suffices to prove that  $f^{-1}(K)\norm G$.
\end{proof}

\end{enumerate}
\end{problem}

\smallskip

\begin{problem}
Let $G$ be a group, $S$ a subset of $G$, and $H=\langle S \rangle$.  

\begin{enumerate}[(5.1)]
\item Prove that $H \norm G$ if and only if $gsg^{-1}\in H$ for every $s\in S$ and $g\in G$.

\begin{proof}	
Suppose $gsg^{-1}\in H$ for all $s\in S$.  Let $h\in H$ and $g\in G$. Then $h=s_1^{e_1}s_2^{e_2}\cdots s_n^{e_n}$ for some $s_1,\dots,s_n\in S$ and $e_1,\dots,e_n\in \{\pm 1\}$.  Let $g\in G$.  Then
$$ghg^{-1}= g (s_1^{e_1}s_2^{e_2}\cdots s_n^{e_n}) g^{-1} = (gs_1^{e_1}g^{-1})(gs_2^{e_2}g^{-1})\cdots (gs_n^{e_n}g^{-1}).$$
Note that if $e_i=-1$ then $gs_i^{-1}g^{-1}=(gs_ig^{-1})^{-1}\in H$.   Thus, $gs_i^{e_i}g^{-1}\in H$ for all $i$, and hence $ghg^{-1}\in H$.  Therefore, $H\triangleleft G$.  The reverse implication is true by definition of normal subgroup.
\end{proof}


\item Consider the commutator subgroup of $G$
$$[G,G] \colonequals \langle aba^{-1}b^{-1}\mid a,b\in G \rangle$$ 
generated by all the commutators of elements in $G$.
Prove that $[G,G] \norm G$.

\begin{proof}
Let $g\in G$ and $s=aba^{-1}b^{-1}$. Set $x=gag^{-1}$ and $y=gbg^{-1}$, and note that 
$$gsg^{-1}=xyx^{-1}y^{-1}\in S\subseteq H.$$
Hence, $H\norm G$ by (5.1).
\end{proof}

\end{enumerate}
\end{problem}





\end{document}