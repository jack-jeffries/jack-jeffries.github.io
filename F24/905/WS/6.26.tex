\documentclass[12pt]{amsart}


\usepackage{times}
\usepackage[margin=1in]{geometry}
\usepackage{amsmath,amssymb,multicol,graphicx,framed,ifthen,color,xcolor,stmaryrd,enumitem,colonequals,bbm}
\usepackage[all]{xy}

\usepackage[outline]{contour}
\contourlength{.4pt}
\contournumber{10}
\newcommand{\Bold}[1]{\contour{black}{#1}}


\definecolor{chianti}{rgb}{0.6,0,0}
\definecolor{meretale}{rgb}{0,0,.6}
\definecolor{leaf}{rgb}{0,.35,0}
\newcommand{\Q}{\mathbb{Q}}
\newcommand{\N}{\mathbb{N}}
\newcommand{\Z}{\mathbb{Z}}
\newcommand{\R}{\mathbb{R}}
\newcommand{\C}{\mathbb{C}}
\newcommand{\e}{\varepsilon}
\newcommand{\m}{\mathfrak{m}}
\newcommand{\p}{\mathfrak{p}}
\newcommand{\q}{\mathfrak{q}}
\newcommand{\ord}{\mathrm{ord}}
\newcommand{\ann}{\mathrm{ann}}
\newcommand{\Min}{\mathrm{Min}}
\newcommand{\Max}{\mathrm{Max}}
\newcommand{\Spec}{\mathrm{Spec}}
\newcommand{\Ass}{\mathrm{Ass}}
\renewcommand{\1}{\mathbbm{1}}
\newcommand{\cZ}{\mathcal{Z}}

\newcommand{\inv}{^{-1}}
\newcommand{\dabs}[1]{\left| #1 \right|}
\newcommand{\ds}{\displaystyle}
\newcommand{\solution}[1]{\ifthenelse {\equal{\displaysol}{1}} {\begin{framed}{\color{meretale}\noindent #1}\end{framed}} { \ }}
\newcommand{\showsol}[1]{\def\displaysol{#1}}
\newcommand{\rsa}{\rightsquigarrow}

\newcommand\itemA{\stepcounter{enumi}\item[{\Bold{(\theenumi)}}]}
\newcommand\itemB{\stepcounter{enumi}\item[(\theenumi)]}
\newcommand\itemC{\stepcounter{enumi}\item[{\it{(\theenumi)}}]}
\newcommand\itema{\stepcounter{enumii}\item[{\Bold{(\theenumii)}}]}
\newcommand\itemb{\stepcounter{enumii}\item[(\theenumii)]}
\newcommand\itemc{\stepcounter{enumii}\item[{\it{(\theenumii)}}]}
\newcommand\itemai{\stepcounter{enumiii}\item[{\Bold{(\theenumiii)}}]}
\newcommand\itembi{\stepcounter{enumiii}\item[(\theenumiii)]}
\newcommand\itemci{\stepcounter{enumiii}\item[{\it{(\theenumiii)}}]}
\newcommand\ceq{\colonequals}

\DeclareMathOperator{\res}{res}
\setlength\parindent{0pt}
%\usepackage{times}

%\addtolength{\textwidth}{100pt}
%\addtolength{\evensidemargin}{-45pt}
%\addtolength{\oddsidemargin}{-60pt}

\pagestyle{empty}
%\begin{document}\begin{itemize}

%\thispagestyle{empty}

\usepackage[hang,flushmargin]{footmisc}


\begin{document}
\showsol{0}
	
	\thispagestyle{empty}
	
	\section*{\S6.26: More associated primes}
	
	\begin{framed}

\noindent \textsc{Lemma:} Let $R$ be a ring, and $N\subseteq M$ be modules. Then
\[ \Ass_R(N) \subseteq \Ass_R(M) \subseteq \Ass_R(N) \cup \Ass_R(M/N).\]

\

\noindent \textsc{Existence of Prime filtrations:} Let $R$ be a Noetherian ring and $M$ be a finitely generated module. Then there exists a finite chain of submodules
\[ M= M_t \supsetneqq M_{t-1} \supsetneqq \cdots \supsetneqq M_{1} \supsetneqq M_0 = 0 \]
such that for each $i=1,\dots,t$, there is some $\p_i\in \Spec(R)$ such that $M_i / M_{i-1} \cong R/\p_i$.
Such a chain of submodules is called a \textbf{prime filtration} of $M$.

\

\noindent \textsc{Corollary 1:} Let $R$ be a Noetherian ring and $M$ be a finitely generated module. Then for any prime filtration of $M$, $\Ass_R(M)$ is a subset of the prime factors that occur in the filtration. In particular, $\Ass_R(M)$ is finite.

\

\noindent \textsc{Prime avoidance:} Let $R$ be a ring, $J$ an ideal, and $I_1,I_2,I_3,\dots,I_t$ a finite collection of ideals with $I_i$ prime for $i>2$ (that is, \emph{at most} two $I_i$ are not prime). If $J\not\subseteq I_i$ for all $i$, then $J \not\subseteq \bigcup_i I_i$.

\

\noindent \textsc{Corollary 2:} Let $R$ be a Noetherian ring, $M$ a finitely generated module, and $I$ an ideal. If every element of $I$ is a zerodivisor on $M$, then there is some nonzero $m\in M$ such that $Im=0$.

\end{framed}


	

\begin{enumerate}

\itemA Let $R=K[X,Y]$ and $M=R/(X^2Y,XY^2)$. 
\begin{enumerate}
\itema Verify that $0 \subseteq R xy \subseteq R x \subseteq M$ is a prime filtration of $M$.
\itema In an earlier problem, we more or less showed that $\{(x),(y), (x,y) \}\subseteq \Ass_R(M)$. Use  Corollary~1 to deduce that this is an equality.
\end{enumerate} 


\solution{
\begin{enumerate}
\itema We have $Rxy\cong (XY)/(X^2Y,XY^2)$. The elements that multiply $XY$ into $(X^2Y,XY^2)$ are the elements in $(X,Y)$, so this is isomorphic to $R/(X,Y)$ and $(X,Y)$ is prime. Then $Rx/Rxy\cong (X)/(XY,X^2Y,XY^2) = (X)/(XY)\cong R/(Y)$ and $Y$ is prime. Finally, the last quotient is isomorphic to $R/(X)$, and $(X)$ is prime.
\itema Yes, it gives the other containment!
\end{enumerate} }

\itemA Proving some Corollaries:
\begin{enumerate}
\itema Show that Corollary 1 follows from the Lemma (and Existence of Prime Filtrations).
\itema Write the contrapositive of the conclusion of Prime Avoidance.
\itema Show that Corollary 2 follows from Prime Avoidance and Corollary 1.
\end{enumerate}

\solution{
\begin{enumerate}
\itema We just need to show the first statement. By the Lemma, we have $\Ass_R(M) = \Ass_R(M_t) \subseteq \Ass_R(M_{t-1}) \cup \Ass_R(M_t/M_{t-1}) = \Ass_R(M_{t-1}) \cup \{\p_t\}$. Then $\Ass_R(M_{t-1}) \subseteq \Ass_R(M_{t-2}) \cup \{p_{t-1}\}$. Continuing like so we obtain the conclusion.
\itema If $J \subseteq \bigcup_i I_i$ then $J\subseteq I_i$ for some $i$.
\itema From last time, we know that the set of zerodivisors is $\bigcup_{\p\in \Ass_R(M)} \p$. If $I$ contained in this \emph{finite} union of primes, it is contained in one of them by Prime avoidance. But if $I$ is contained in an associated prime, take a witness $m$, and $Im=0$.
\end{enumerate}}


\itemA Proof of Existence of Prime Filtrations: Let $R$ be a Noetherian ring and $M$ a finitely generated $R$-module.
\begin{enumerate}
\itema If $M\neq 0$, explain why you can always choose $M \supseteq M_1$ with $M_1\cong R/\p$ for some prime $\p$.
\itema If $M\neq M_1$, explain why\footnote{Hint: Consider $M/M_1$ and go back to the previous step.} you can always choose $M \supseteq M_2 \supseteq M_1$ with ${M_2/M_1 \cong R/\p}$ for some prime $\p$.
\itema If $M\neq M_{i-1}$ and you already have $M_1,\dots,M_{i-1}$, explain why you can always choose\\ ${M \supseteq M_i \supsetneqq M_{i-1}}$ with $M_i/M_{i-1} \cong R/\p$ for some prime $\p$.
\itema Explain why this process has to stop, and if it stops at $i=t$, we must have $M_t=M$.
\end{enumerate}

\solution{
\begin{enumerate}
\itema $M$ has an associated prime, since $R$ is Noetherian and $M\neq 0$. An associated prime is a recipe for exactly such a submodule.
\itema Apply the previous to $M/M_1$. By the lattice theorem, we can write this as $M_2/M_1$ for some $M_2$ containing $M_1$.
\itema Same thing.
\itema $M$ is a Noetherian module, so an ascending chain of submodules terminates. It must terminate with $M_t=M$ by what we just said in the previous step.
\end{enumerate}}

\begin{samepage}
\itemA Lemma 1:
\begin{enumerate}
\itema Let $K$ be a field and $R=K[X]$. Explain why
\begin{itemize}
\item $\Ass_R(R) = \{(0)\}$
\item $(X) \cong R$, so $\Ass_R((X)) = \{(0)\}$, 
\item $\Ass_R(R/(X)) = \{(X)\}.$
\end{itemize}
Does this contradict the Lemma?
\itema Show that $\Ass_R(N) \subseteq \Ass_R(M)$.
\itema Suppose that $\p\in \Ass_R(M) \smallsetminus \Ass_R(N)$ with witness $m$. Show\footnote{Note that $Rm\cong R/\p$ so every nonzero element has annihilator $\p$.} that $Rm\cap N=0$, so the map $Rm \to M/N$ is injective. Deduce that $\p\in \Ass_R(M/N)$ and complete the proof.
\end{enumerate}
\end{samepage}

\solution{
\begin{enumerate}
\itema 
\begin{itemize}
\item This is an example of $\Ass_R(R/\p)=\{\p\}$.
\item The map $R\to (X)$ given by $r\mapsto rX$ is $R$-linear and bijective, so an isomorphism of $R$-modules.
\item This is an example of $\Ass_R(R/\p)=\{\p\}$.
\end{itemize}
This does not contradict the lemma.
\itema A witness of $\p$ in $N$ is a witness for $\p$ in $M$.
\itema Note that $Rm\cong R/\p$ so every nonzero element has annihilator $\p$. Since $\p\notin \Ass_R(N)$, no element of $N$ has annihilator $\p$, so nonzero element of $Rm$ is also an element of $N$. Thus the induced map $R/\p \cong Rm \hookrightarrow M/N$, so $\p\in \Ass_R(M/N)$.
\end{enumerate}}


\itemB Prove\footnote{By induction, you can find elements $a_i \in J \smallsetminus \bigcup_{j\neq i} I_j$. Now consider $x=a_n + a_1\cdots a_{n-1}$.} the prime avoidance lemma.

\

\itemB Let $K$ be a field and $R=K[X^2,XY,Y^2]\subseteq K[X,Y]$.
\begin{enumerate}
\item Mark all\footnote{Well, enough to get the pattern at least\dots} of the points in the plane corresponding to exponent vectors of elements of~$R$.
\item Is $I=(X^2)$ a prime ideal? Is $J=(X^2,XY)$?
\item Mark all of the points in the plane corresponding to exponent vectors of elements of $(X^2)\subseteq R$.
\item Find and illustrate a prime filtration of $R/I$. Compute $\Ass_R(R/I)$.
\item Find and illustrate a prime filtration of $R/J^2$. Compute $\Ass_R(R/J^2)$.
\end{enumerate}

\


\itemB More facts about associated primes: Let $R$ be a Noetherian ring.
\begin{enumerate}
\itemb Let $I\subseteq J$ be ideals. Show that $I=J$ if and only if $IR_\p = JR_\p$ for all $\p\in \Ass_R(R/I)$.
\itemb Let $I,J$ be ideals. Show that $I \subseteq J$ if and only if $IR_\p \subseteq JR_\p$ for all $\p\in \Ass_R(R/J)$.
\itemb Let $r$ be a nonzerodivisor. Show that $\Ass_R(R/r^n) = \Ass_R(R/r)$ for all $n \geq 1$.
\end{enumerate}


\end{enumerate}

\end{document}
