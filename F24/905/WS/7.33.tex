\documentclass[12pt]{amsart}


\usepackage{times}
\usepackage[margin=1.3in]{geometry}
\usepackage{amsmath,amssymb,multicol,graphicx,framed,ifthen,color,xcolor,stmaryrd,enumitem,colonequals,bbm}
\usepackage[all]{xy}

\usepackage[outline]{contour}
\contourlength{.4pt}
\contournumber{10}
\newcommand{\Bold}[1]{\contour{black}{#1}}


\definecolor{chianti}{rgb}{0.6,0,0}
\definecolor{meretale}{rgb}{0,0,.6}
\definecolor{leaf}{rgb}{0,.35,0}
\newcommand{\Q}{\mathbb{Q}}
\newcommand{\N}{\mathbb{N}}
\newcommand{\Z}{\mathbb{Z}}
\newcommand{\R}{\mathbb{R}}
\newcommand{\C}{\mathbb{C}}
\newcommand{\e}{\varepsilon}
\newcommand{\m}{\mathfrak{m}}
\newcommand{\p}{\mathfrak{p}}
\newcommand{\q}{\mathfrak{q}}
\newcommand{\ord}{\mathrm{ord}}
\newcommand{\ann}{\mathrm{ann}}
\newcommand{\hgt}{\mathrm{height}}
\newcommand{\Min}{\mathrm{Min}}
\newcommand{\Max}{\mathrm{Max}}
\newcommand{\Spec}{\mathrm{Spec}}
\newcommand{\Ass}{\mathrm{Ass}}
\renewcommand{\1}{\mathbbm{1}}
\newcommand{\cZ}{\mathcal{Z}}

\newcommand{\inv}{^{-1}}
\newcommand{\dabs}[1]{\left| #1 \right|}
\newcommand{\ds}{\displaystyle}
\newcommand{\solution}[1]{\ifthenelse {\equal{\displaysol}{1}} {\begin{framed}{\color{meretale}\noindent #1}\end{framed}} { \ }}
\newcommand{\showsol}[1]{\def\displaysol{#1}}
\newcommand{\rsa}{\rightsquigarrow}

\newcommand\itemA{\stepcounter{enumi}\item[{\Bold{(\theenumi)}}]}
\newcommand\itemB{\stepcounter{enumi}\item[(\theenumi)]}
\newcommand\itemC{\stepcounter{enumi}\item[{\it{(\theenumi)}}]}
\newcommand\itema{\stepcounter{enumii}\item[{\Bold{(\theenumii)}}]}
\newcommand\itemb{\stepcounter{enumii}\item[(\theenumii)]}
\newcommand\itemc{\stepcounter{enumii}\item[{\it{(\theenumii)}}]}
\newcommand\itemai{\stepcounter{enumiii}\item[{\Bold{(\theenumiii)}}]}
\newcommand\itembi{\stepcounter{enumiii}\item[(\theenumiii)]}
\newcommand\itemci{\stepcounter{enumiii}\item[{\it{(\theenumiii)}}]}
\newcommand\ceq{\colonequals}

\DeclareMathOperator{\res}{res}
\setlength\parindent{0pt}
%\usepackage{times}

%\addtolength{\textwidth}{100pt}
%\addtolength{\evensidemargin}{-45pt}
%\addtolength{\oddsidemargin}{-60pt}

\pagestyle{empty}
%\begin{document}\begin{itemize}

%\thispagestyle{empty}

\usepackage[hang,flushmargin]{footmisc}


\begin{document}
\showsol{0}
	
	\thispagestyle{empty}
	
	\section*{\S7.33: Transcendence degree and dimension}
	
	\begin{framed}
	
	\noindent	\textsc{Definition:} Let $K\subseteq L$ be an extension of fields and let $S$ be a subset of $L$.
	\begin{enumerate}
	\item The \textbf{subfield of $L$ generated by $K$ and $S$}, denoted $K(S)$, is the smallest subfield of $L$ containing $K$ and $S$. Equivalently, $K(S)$ is the set of elements in $L$ that can be written as rational function expressions in $S$ with coefficients in $K$.
%	\item We say that $S$ generates $L$ over $K$ if $L=K(S)$.
	\item We say that $S$ is \textbf{algebraically independent} over $K$ if there are nonzero polynomial relations on any finite subset of $S$.
	Equivalently, $S$ is algebraically independent over $K$ if, for a set of indeterminates $X=\{X_s \ | \ s\in S\}$, there is an isomorphism of field extensions of $K$ between the field of rational functions $K(S)$ and $K(X)$ via $s\mapsto X_s$.
	\item We say that $S$ is a \textbf{transcendence basis} for $L$ over $K$ if $S$ is algebraically independent over $K$ and the field extension $K(S) \subseteq L$ is algebraic.
	\end{enumerate}
	
	\
	
	\noindent \textsc{Lemma:} Let $K\subseteq L$ be an extension of fields.
\begin{enumerate}
\item Every $K$-algebraically independent subset of $L$ is contained in a transcendence basis. In particular, there exists a transcendence basis for $L$ over $K$.
\item Every transcendence basis for $L$ over $K$ has the same cardinality.
\end{enumerate}
	
	\
	
\noindent	\textsc{Definition:} Let $K\subseteq L$ be an extension of fields. The \textbf{transcendence degree} of $L$ over $K$ is the cardinality of a transcendence basis for $L$ over $K$.
	

	\
	
\noindent	\textsc{Theorem:} Let $K$ be a field, and $R$ be a domain that is algebra-finite over $K$. Then, the dimension of $R$ is equal to the transcendence degree of $\mathrm{Frac}(R)$ over $K$.
\end{framed}




\begin{enumerate}
\itemA Let $K$ be a field, and $R$ be a domain that is algebra-finite over $K$.
\begin{enumerate}
\itema Explain why, if $R=K[f_1,\dots,f_m]$, then $\mathrm{Frac}(R) = K ( f_1,\dots, f_m)$.
\itema Show\footnote{Hint: Recall that every nonzero $r\in R$ has a nonzero multiple in $A$.} that if $A=K[z_1,\dots,z_t]$ is a Noether normalization for $R$, then $\{z_1,\dots,z_t\}$ forms a transcendence basis for $\mathrm{Frac}(R)$.
\itema Deduce the Theorem.
\end{enumerate}
\solution{
\begin{enumerate}
\itema Since $f_1,\dots,f_m \in \mathrm{Frac}(R)$, the containment $\mathrm{Frac}(R) \supseteq K ( f_1,\dots, f_m)$ holds. Conversely, every element of $\mathrm{Frac}(R)$ can be written as a fraction of elements of $R$, and an element of $R$ can be written as a polynomial expression in $f_i$, so each element of $\mathrm{Frac}(R)$ is a rational expression in the $f_i$'s.
\itema By definition, the $z_i$ are algebraically independent. Write $R=\sum A r_i$. We claim that $\mathrm{Frac}(R) = \sum \mathrm{Frac}(A) r_i$. Indeed, given $r/s$ for $r,s\in R$, we can write $st=a$ for some $a\in A$ nonzero and $t\in R$. Then for some $s_i\in R$, we have $r/s=rt/a = (\sum r_i s_i)/a = \sum (s_i/a) r_i$, so $r/s\in \sum \mathrm{Frac}(A) r_i$.
\itema Follows from the Theorem that in this setting the dimension equals the cardinality of the variables in a Noether normalization, and that the transcendence degree of the fraction field of a NN is the number of elements in the NN.
\end{enumerate}
}



\itemA Let $K$ be a field. Use the Theorem to compute the dimension of 
\[ R=K[UX,UY,UZ,VX,VY,VZ]\subseteq K[U,V,X,Y,Z].\]

\solution{
We have $\mathrm{Frac}(R)= K(UX,UY,UZ,VX,VY,VZ) = K(UX,Y/X,Z/X,V/U)$, which has transcendence degree four.
}

\itemA Let $R\subseteq S$ be domains.
 \begin{enumerate}
\itema Use the Theorem to prove that if $R\subseteq S$ are finitely generated algebras over some field $K$, then ${\dim(R) \leq \dim(S)}$.
\itema Give an example where $\dim(R)>\dim(S)$.
\end{enumerate}

\solution{
 \begin{enumerate}
\itema This follows from the transcendence degree characterization, since a maximal algebraically independent subset of $\mathrm{Frac}(R)$ is contained in a maximal algebraically independent subset of $\mathrm{Frac}(S)$.
\itema $\Z\subseteq \Q$.
\end{enumerate}
}
\newpage 
\itemB Proof of Lemma: Let $K\subseteq L$ be fields, and $S$ a subset of $L$.
\begin{enumerate}
\itemb Show that $S$ is a transcendence basis for $L$ over $K$ if and only if it is a maximal $K$-algebraically independent subset of $L$.
\itemb Deduce part (1) of the Lemma.
\itemb Show that, to prove part (2) (in the case of two finite transcendence bases), it suffices to show the following \\
\textsc{Exchange Lemma:} If $\{x_1,\dots, x_m\}$ and $\{y_1,\dots,y_n\}$ are two transcendence bases, then there is some $j$ such that \\$\{x_j, y_2,\dots,y_n\}$ is a transcendence basis.
\itemb In the setting of the Exchange Lemma, explain why for each $j$, there is some nonzero ${p_j(t)\in K[y_1,\dots,y_n][t]}$ such that $p_j(x_j)=0$.
\itemb In the setting of the previous part, explain why there is some $j$ such that \\${p_j(t)\notin K[y_2,\dots,y_n][t]}$.
\itemb Show that the conclusion of the Exchange Lemma holds for $j$ as in the previous part.
\end{enumerate}

\solution{
\begin{enumerate}
\itemb If  $\{l_\lambda\}$ and $l\in L$, then $l$ is algebraic over $K(\{l_\lambda\})$, so there is a nonzero polynomial relation $l^n + {r_1} l^{n-1} + \cdots + r_n = 0$ with $r_i\in K(\{l_\lambda\})$. Writing $r_i=\frac{p_i}{q_i}$ and multiplying by the product of the $q_i$'s gives a nonzero polynomial relation on the $l_\lambda$'s and $l$. Thus, $\{l_\lambda\}$ is a maximal algebraic subset. The converse is similar.
\itemb Given a nested union of algebraically independent subsets, the union is as well, since a relation on one of these sets involves finitely many elements, all of which must occur in one of the sets in the chain. The claim then follows from Zorn's Lemma.
\itemb  If $\{x_1,\dots,x_m\}$ and $\{y_1,\dots,y_n\}$ are two transcendence bases,  say that $m\leq n$.  If the intersection has $s<m$ elements, then without loss of generality $y_1\notin \{x_1,\dots,x_m\}$. Then, for some $i$,  $\{x_i, y_2\dots,y_n\}$ is a transcendence basis, and $\{x_1,\dots,x_m\} \cap \{x_i, y_2\dots,y_n\}$ has $s+1$ elements. Replacing $\{y_1,\dots,y_n\}$ with  $\{x_i, y_2\dots,y_n\}$ and repeating this process, we obtain a transcendence basis with $n$ elements such that $\{x_1,\dots,x_m\} \subseteq \{y_1,\dots,y_n\}$. But we must then have that these two transcendence bases are equal, so $m=n$.

\itemb  Since $L$ is algebraic over $K(y_1,\dots,y_n)$, for each $i$ there is some $p_i(t)\in K(y_1,\dots,y_n)[t]$ such that $p_i(x_i)=0$. We can clear denominators to assume without loss of generality that $p_i(x_i)\in K[y_1,\dots,y_n][t]$. 
\itemb If not, so $p_i(t) \in K[y_2,\dots,y_n][t]$ for all $i$, note that each $x_i$ is algebraic over $K(y_2,\dots,y_n)$.  Thus, $K(x_1,\dots,x_m)$ is algebraic over $K(y_2,\dots,y_n)$, and since $L$ is algebraic over $K(x_1,\dots,x_m)$, $y$ is algebraic over $K(y_2,\dots,y_n)$, which contradicts that $\{y_1,\dots,y_n\}$ is a transcendence basis. This shows the claim.
\itemb Thinking of the equation $p_i(x_i)=0$ as a polynomial expression in $K[x_i,y_2,\dots,y_n][y_1]$, $y_1$ is algebraic over $K(x_i,y_2,\dots,y_n)$, hence $K(y_1,\dots,y_n)$ is algebraic over $K(x_i,y_2,\dots,y_n)$, and $L$ as well.

If $\{x_i,y_2,\dots,y_n\}$ were algebraically dependent, take a polynomial equation $p(x_i,y_2,\dots,y_n)=0$. Note that this equation must involve $x_i$, since $y_2,\dots,y_n$ are algebraically independent. We would then have $K(x_i,y_2,\dots,y_n)$ is algebraic over $K(y_2,\dots,y_n)$. But since  $y_1$ is algebraic over $K(x_i,y_2,\dots,y_n)$, we would have that $K(y_1,\dots,y_n)$ is algebraic over $K(y_2,\dots,y_n)$, which would contradict that $y_1,\dots,y_n$ is a transcendence basis.
\end{enumerate}

}



\end{enumerate}
\end{document}
