\documentclass[12pt]{amsart}


\usepackage{times}
\usepackage[margin=0.8in]{geometry}
\usepackage{amsmath,amssymb,multicol,graphicx,framed,ifthen,color,xcolor,stmaryrd,enumitem,colonequals}
\usepackage[outline]{contour}
\contourlength{.4pt}
\contournumber{10}
\newcommand{\Bold}[1]{\contour{black}{#1}}

\definecolor{chianti}{rgb}{0.6,0,0}
\definecolor{meretale}{rgb}{0,0,.6}
\definecolor{leaf}{rgb}{0,.35,0}
\newcommand{\Q}{\mathbb{Q}}
\newcommand{\N}{\mathbb{N}}
\newcommand{\Z}{\mathbb{Z}}
\newcommand{\R}{\mathbb{R}}
\newcommand{\C}{\mathbb{C}}
\newcommand{\e}{\varepsilon}
\newcommand{\m}{\mathfrak{m}}
\newcommand{\p}{\mathfrak{p}}
\newcommand{\ord}{\mathrm{ord}}
\newcommand{\im}{\mathrm{im}}

\newcommand{\inv}{^{-1}}
\newcommand{\dabs}[1]{\left| #1 \right|}
\newcommand{\ds}{\displaystyle}
\newcommand{\solution}[1]{\ifthenelse {\equal{\displaysol}{1}} {\begin{framed}{\color{meretale}\noindent #1}\end{framed}} { \ }}
\newcommand{\showsol}[1]{\def\displaysol{#1}}
\newcommand{\rsa}{\rightsquigarrow}

\newcommand\itemA{\stepcounter{enumi}\item[{\Bold{(\theenumi)}}]}
\newcommand\itemB{\stepcounter{enumi}\item[(\theenumi)]}
\newcommand\itemC{\stepcounter{enumi}\item[{\it{(\theenumi)}}]}
\newcommand\itema{\stepcounter{enumii}\item[{\Bold{(\theenumii)}}]}
\newcommand\itemb{\stepcounter{enumii}\item[(\theenumii)]}
\newcommand\itemc{\stepcounter{enumii}\item[{\it{(\theenumii)}}]}
\newcommand\itemai{\stepcounter{enumiii}\item[{\Bold{(\theenumiii)}}]}
\newcommand\itembi{\stepcounter{enumiii}\item[(\theenumiii)]}
\newcommand\itemci{\stepcounter{enumiii}\item[{\it{(\theenumiii)}}]}
\newcommand\ceq{\colonequals}

\DeclareMathOperator{\res}{res}
\setlength\parindent{0pt}
%\usepackage{times}

%\addtolength{\textwidth}{100pt}
%\addtolength{\evensidemargin}{-45pt}
%\addtolength{\oddsidemargin}{-60pt}

\pagestyle{empty}
%\begin{document}\begin{itemize}

%\thispagestyle{empty}

\usepackage[hang,flushmargin]{footmisc}


\begin{document}
\showsol{1}
	
	\thispagestyle{empty}
	
	\section*{\S1.3: Algebras}	

\begin{framed}

\noindent \textsc{Definition:} Let $A$ be a ring. An \textbf{$A$-algebra} is a ring $R$ equipped with a ring homomorphism ${\phi:A\to R}$; we call $\phi$ the \textbf{structure morphism} of the algebra\footnotemark. A \textbf{homomorphism} of $A$-algebras is a ring homomorphism that is compatible with the structure morphisms; i.e., if $\phi:A\to R$ and $\psi:A\to S$ are $A$-algebras, then $\alpha:R\to S$ is an $A$-algebra homomorphism if $\alpha\circ \phi = \psi$.


\

\noindent \textsc{Universal property of polynomial rings:} Let\footnotemark \ $A$ be a ring, and $T=A[X_1,\dots,X_n]$ be a polynomial ring. For any $A$-algebra $R$, and any collection of elements $r_1,\dots,r_n\in R$, there is a unique $A$-algebra homomorphism $\alpha: T\to R$ such that $\alpha(X_i) = r_i$.


\


\noindent \textsc{Definition:} Let $A$ be a ring, and $R$ be an $A$-algebra. Let $S$ be a subset of $R$. The \textbf{subalgebra generated by $S$}, denoted $A[S]$, is the smallest $A$-subalgebra of $R$ containing $S$. Equivalently\footnotemark,

\[ A[r_1,\dots,r_n] = \left\{ \sum_{\mathrm{finite}} a r_1^{d_1} \cdots r_n^{d_n} \ | \ a\in \phi(A) \right\}.\]

\

%\noindent \textsc{Proposition:} $A[S]$ is the image of the $A$-algebra homomorphism $A[X] \to R$\dots

\noindent \textsc{Definition:} Let $R$ be an $A$-algebra. Let $r_1,\dots,r_n\in R$. The ideal of \textbf{$A$-algebraic relations} on $r_1,\dots,r_n$ is the set of polynomials $f(X_1,\dots,X_n)\in A[X_1,\dots,X_n]$ such that ${f(r_1,\dots,r_n)=0}$ in $R$. Equivalently, the ideal of $A$-algebraic relations on $r_1,\dots,r_n$ is the kernel of the homomorphism ${\alpha: A[X_1,\dots,X_n]\to R}$ given by $\alpha(X_i)=r_i$. We say that a set of elements in an $A$-algebra is \textbf{algebraically independent over $A$} if it has  no nonzero $A$-algebraic relations.

\

\noindent \textsc{Definition:} A \textbf{presentation} of an $A$-algebra $R$ consists of a set of generators $r_1,\dots,r_n$ of $R$ as an $A$-algebra and a set of generators $f_1,\dots,f_m\in A[X_1,\dots,X_n]$ for the ideal of $A$-algebraic relations on $r_1,\dots,r_n$. We call $f_1,\dots,f_m$ a set of \textbf{defining relations} for $R$ as an $A$-algebra.

\

\noindent \textsc{Proposition:} If $R$ is an {$A$-algebra}, and $f_1,\dots,f_m$ is a set of defining relations for $R$ as an \mbox{$A$-algebra}, then $R\cong A[X_1,\dots,X_n]/(f_1,\dots,f_m)$.

 \end{framed}
 
 \footnotetext[1]{Note: the same $R$ with different $\phi$'s yield different $A$-algebras. Despite this we often say ``Let $R$ be an $A$-algebra'' without naming the structure morphism.}
 \footnotetext[2]{This is equally valid for polynomial rings in infinitely many variables $T=A[X_{\lambda} \ | \ \lambda\in \Lambda]$ with a tuple of elements of  $\{r_\lambda\}_{\lambda\in \Lambda}$ in $R$ in bijection with the variable set. I just wrote this with finitely many variables to keep the notation for getting too overwhelming.}
 \footnotetext[3]{Again written with a finite set just for convenience.}
 
\begin{enumerate}
\itemA Let $R$ be an $A$-algebra and $r_1,\dots,r_n\in R$.
\begin{enumerate}
\itema Discuss why the equivalent characterizations in the definition of $A[r_1,\dots,r_n]$ are equivalent.
\itema Explain why $A[r_1,\dots,r_n]$ is the image of the $A$-algebra homomorphism ${\alpha: A[X_1,\dots,X_n]\to R}$ such that $\alpha(X_i) = r_i$.

\itema Suppose that $R=A[r_1,\dots,r_n]$ and let $f_1,\dots,f_m$ be a set of generators for the kernel of the map $\alpha$. Explain why $R\cong A[X_1,\dots,X_n]/(f_1,\dots,f_m)$, i.e., why the Proposition above is true.
\itema Suppose that $R$ is generated as an $A$-algebra by a set $S$. Let $I$ be an ideal of $R$. Explain why $R/I$ is generated as an $A$-algebra by the image of $S$ in $R/I$.
\itema Let $R=A[X_1,\dots,X_n]/(f_1,\dots,f_m)$, where $A[X_1,\dots,X_n]$ is a polynomial ring over $A$. Find a presentation for $R$.
%\itema Use (b) to briefly explain why $r_1,\dots,r_n$ generates $R$ as an $A$-algebra if and only if $r_1,\dots,r_n$ generates $R$ as a $\phi(A)$-algebra.
\end{enumerate}

\solution{
\begin{enumerate}
\itema Clearly $\im(\alpha) \subseteq R$ is an $A$-subalgebra that contains $r_1,\dots,r_n$, so $A[r_1,\dots,r_n]\subseteq \im(\alpha)$. On the other hand, since $r_1,\dots,r_n\in A[r_1,\dots,r_n]$, we have $\alpha(X_i)\in A[r_1,\dots,r_n]$, so we can consider $\alpha$ as an $A$-algebra homomorphism from $A[X_1,\dots,X_n] \to A[r_1,\dots,r_n]$, and hence $\im(\alpha)\subseteq  A[r_1,\dots,r_n]$.
\itema This is just another way of thinking about $\im(\alpha)$: $\alpha( \sum a_i X_1^{i_1} \cdots X_n^{i_n}) = \sum \phi(a_i) r_1^{i_1} \cdots r_n^{i_n}$.
\itema This is just the First Isomorphism Theorem applied along with (a).
\itema If $K[\{X_\lambda\}] \to R$ where the variables map to the elements of $S$ is surjective, then composing with the quotient map gives a surjection $K[\{X_\lambda\}] \to R \to R/I$ where the variables map to the images of elements of $S$.
\itema $R$ is generated by $[X_1],\dots,[X_n]$, with defining relations $f_1,\dots,f_m$.
\end{enumerate}
}

\begin{samepage}
\itemA Presentations of some subrings:
\begin{enumerate}
\itema Consider the $\Z$-subalgebra of $\C$ generated by $\sqrt{2}$. Write the notation for this ring. Is there a more compact description of the set of elements in this ring? Find a presentation.
\itema Same as (a) with $\sqrt[3]{2}$ instead of $\sqrt{2}$.
\itema Let $K$ be a field, and $T=K[X,Y]$. Come up with a concrete description of the ring\sloppy \ ${R=K[X^2,XY,Y^2]\subseteq T}$,  (i.e., describe in simple terms which polynomials are elements of~$R$), and give a presentation as a $K$-algebra.
%Let $R\subseteq T$ be the ring of polynomials that only have even degree terms (e.g., $X^2+\pi X^3Y+5$ is in while $X^2 + \pi XY^2+5$ is out). Show that $R$ is a $K$-subalgebra of $T$, and find a presentation for $R$.
\end{enumerate}
\end{samepage}

\solution{
\begin{enumerate}
\itema $\Z[\sqrt{2}] = \{ a + b \sqrt{2} \ | \ a,b\in \Z\} \cong \Z[X]/(X^2-2)$
\itema $\Z[\sqrt[3]{2}] = \{ a + b \sqrt[3]{2} + c \sqrt[3]{4} \ | \ a,b,c\in \Z\} \cong \Z[X]/(X^3-2)$.
\itema $K[X^2,XY,Y^2]$ is the collection of polynomials that only have even degree terms. We computed the kernel of the presenting map last time, in slightly different words and letters, and saw that the kernel is generated by $X_2^2-X_1X_3$.
\end{enumerate}
}






\itemA Infinitely generated algebras:
\begin{enumerate}
\itema Show that $\Q = \Z[ 1/p \ | \ p \ \text{is a prime number}]$.
\itema True or false: It is a direct consequence of the conclusion of (a) and the fact that there are infinitely many primes that $\Q$ is not a finitely generated $\Z$-algebra.
\itema Given $p_1,\dots,p_m$ prime numbers, describe the elements of $\Z[1/p_1,\dots,1/p_m]$ in terms of their prime factorizations. Can you ever have $\Z[1/p_1,\dots,1/p_m] = \Q$ for a finite set of primes?
\itema Show that $\Q$ is not a finitely generated $\Z$-algebra.
\itemb Show that, for a field $K$, the algebra $K[X,XY, XY^2, XY^3,\dots] \subseteq K[X,Y]$ is not a finitely generated $K$-algebra.
\itemb Show that, for a field $K$, the algebra $K[X,Y/X, Y/X^2, Y/X^3,\dots] \subseteq K(X,Y)$ is not a finitely generated $K$-algebra.
\end{enumerate}


\solution{
\begin{enumerate}
\itema The $\supseteq$ containment is clear. For the other, take $a/b\in \Q$, and write $b=p_1^{e_1}\cdots p_n^{e_n}$. Then $a/b=a(1/p_1)^{e_1}\cdots (1/p^n)^{e_n}$ exhibits $a/b$ in the right hand side.
\itema False! There could be a different finite generating set.
\itema An element of $\Z[1/p_1,\dots,1/p_m]$ can be written as $\sum_{\alpha} a_\alpha (1/p_1)^{\alpha_1} \cdots (1/p_m)^{\alpha_m}$ so has a denominator that is a product of powers of $p_i$'s. This can never equal $\Q$, since $1/(p_1 \cdots p_m +1)$ can't be written in this form: if so, and in lowest terms with numerator $a$, after clearing denominators we would have $p_1^{\alpha_1} \cdots p_n^{\alpha_n} = (p_1 \cdots p_m +1)a$, which contradicts the expression in lowest terms.
\itema If $\Q=\Z[a_1/b_1,\dots,a_n/b_n]$ (in lowest terms) let $p_1,\dots,p_m$ be the prime factors of $b_1,\dots,b_n$. Then $\Z[a_1/b_1,\dots,a_n/b_n] \subseteq \Z[1/p_1,\dots,1/p_m]$, so $\Z[1/p_1,\dots,1/p_m] = \Q$ contradicting what we just showed.
\itemb Suppose otherwise that $K[X,XY, XY^2, XY^3,\dots]=K[f_1,\dots,f_n]$. Since each $f_i$ is a polynomial expression of $X,XY, XY^2, XY^3,\dots$, and there are finitely many $XY^j$ that appear in (fixed expressions for) each of the finitely many $f_i$, we have $K[X,XY, XY^2, XY^3,\dots] \subseteq K[f_1,\dots,f_n]\subseteq K[X,XY,\dots,XY^m]$ for some $m$,  and equality holds for this same $m$. We claim that $XY^{m+1}\notin K[X,XY,\dots,XY^m]$, which will yield the desired contradiction. Indeed, one can see that every monomial in $K[X,XY,\dots,XY^m]$ has its $y$-exponent is less than or equal to $m$ times its $x$-exponent, which is not true of $XY^{m+1}$. This is the desired contradiction.
\itemb Similar to the previous.
\end{enumerate}
}

\itemB More algebras:
\begin{enumerate}
\itemb Give two different nonisomorphic $\C[X]$-algebra structures on $\C$.
\itemb Find a $\C$-algebra generating set for the ring of polynomials in $\C[X,Y]$ that only have terms whose total degree ($X$-exponent plus $Y$-exponent) is a multiple of three (e.g., $X^3+\pi X^5Y+5$ is in while $X^3 + \pi X^4Y+5$ is out).
\itemb Find a $\C$-algebra presentation for $\C \times \C$.
\end{enumerate}

\solution{
\begin{enumerate}
\itemb We can write $\C\cong \C[X]/(X)$ or $\C \cong \C[X]/(X-1)$, for example. These are not isomorphic as $\C[X]$-algebras, since such a morphism would send $[0]$ to $[0]$ and $[X]$ to $[X]$, but $[X]=[0]$ in $\C[X]/(X)$ while $[X]=[1]$ in $\C[X]/(X-1)$.
\itemb The set $X^3,X^2Y,XY^2,Y^3$ works. We can write any polynomial in this ring as a sum of monomials of total degree three. From such a monomial, we can factor out powers of $X^3$ and $Y^3$ until we get either a constant or $X^2Y$, or $XY^2$. Then putting everything back together, we get that any polynomial in our ring is a polynomial expression in the four things we named.
\itemb We need a generator for $(1,0)$; then $(0,1)$ comes for free as $1-(1,0)$, and we're set on generators. Let's map $X$ to $(1,0)$ for our presentation. Then $X(1-X)$ maps to $(1,0)(0,1)=0$ so this is in the kernel; one can show with a division argument along the lines of many we've discussed that this generates the kernel.
\end{enumerate}
}


\itemB Let $K$ be a field. Describe which elements are in the $K$-algebra $K[X,X^{-1}]\subseteq K(X)$, and find an element of $K(X)$ not in $K[X,X^{-1}]$. Then compute\footnote{Hint: Note that Division does not apply. Say $X_1 \mapsto X$ and $X_2 \mapsto Y$. Show that the top $X_2$-degree coefficient of an algebraic relation is a multiple of $X_1$, and use this to set an induction on the top $X_2$-degree.} a presentation for $K[X,X^{-1}]$ as a $K$-algebra.

\solution{The elements of $K[X,X^{-1}]$ are rational functions that can be written with a power of $X$ as a denominator. The rational function $1/(X-1)$ is not in this algebra.

We claim that $K[X,X^{-1}]\cong K[X_1,X_2]/(X_1 X_2-1)$. Clearly $X_1 X_2-1$ is a relation on $X$ and $X^{-1}$. If it does not generate, take a relation not in the ideal among which has lowest $X_2$-degree. Let $f(X_1,X_2)= f_n(X_1) X_2^n + f_{n-1}(X_1) X_2^{n-1} + \cdots + f_0(X_1)$ be an algebraic relation, and consider the top $X_2$-degree coefficient $f_n(X_1)$ of $f$. Note that $f_n$ is a multiple of $X_1$ since, mapping $X_1\mapsto X$ and $X_2\mapsto X^{-1}$, we get $f_n(X) X^{-n} + f_{n-1}(X) X^{-n+1} + \cdots +f_0(X)=0$, so $f_n(X) = X(-f_{n-1}(X) - X f_{n-2}(X) - \cdots - X^n f_0(X))$. Write $f_n = X_1 f'_n$. Then 
\[ \begin{aligned}
f(X_1,X_2)&= f_n(X_1) X_2^n + f_{n-1}(X_1) X_2^{n-1} + \cdots + f_0(X_1) \\
&= X_1 f'_n(X_1) X_2^n + f_{n-1}(X_1) X_2^{n-1} + \cdots + f_0(X_1) \\ 
&= (X_1 X_2 - 1) f'_n(X_1) X_2^{n-1} + (f'_n(X_1) + f_{n-1}) X_2^{n-1} + \cdots + f_0(X_1).
\end{aligned}\]
Subtracting off a multiple of $X_1 X_2 - 1$, we obtain a relation of lower $X_2$-degree, contradicting the choice of our relation, and hence the existence of a relation that is not a multiple of $X_1 X_2 - 1$.
}



\itemB Can you guess defining relations for the ring in (4b)? Can you prove your guess?

\solution{ Since $X^3,X^2Y,XY^2,Y^3\in R$, we have $K[X^3,X^2Y,XY^2,Y^3] \subseteq R$. To show equality, note that we can write $f\in R$ as a sum of monomials of degree a multiple of three, so it suffices to show that any such monomial is in the algebra generated by $X^3,X^2Y,XY^2,Y^3$. Given $X^i Y^j$, if $i\geq 3$ or $j\geq 3$, we can write $X^i Y^j= X^3 \mu$ or $Y^3 \mu$ with $\mu$ a smaller monomial of degree a multiple of three. Continuing like so, we can assume $i,j<3$, in which case we must have $X^2 Y$ or $XY^2$. Thus, $K[X^3,X^2Y,XY^2,Y^3] =R$.

Now we compute the ideal of relations. We can check directly that each relation is in the defining ideal. To see that they generate, we show that any polynomial in the kernel of the presenting map is equivalent to zero modulo the ideal generated by the given three. Write $T=X_1, U=X_2, V=X_3, W=Y^3$. Given a relation $F$, we think of it as a polynomial in $V$. We can use division via $V^2-UW$ to get rid of the $V^{\geq 2}$ terms, and the other relations to rewrite the coefficient of the $V^1$ term as a polynomial in $W$ alone, so $F\equiv f_1(W) V + f_0(T,U,W)$. Then we have $f_1(Y^3) XY^2 + f_0(X^3, X^2Y, Y^3)=0$. The first term only produces $Y^1$-terms, while the second produces only other powers of $Y$, so the two parts must be zero. This implies that $f_1$ is the zero polynomial, and that $f_0$ is a relation on $X^3, X^2Y, Y^3$. A similar division argument shows that any polynomial in $T,U,W$ that vanishes upon mapping $T\mapsto X^3$, $U\mapsto X^2 Y$, $W\mapsto Y^3$ is a multiple of $U^3-T^2W$, but $U^3-T^2W=U(U^2-TV)-T(TW-UV)$. This completes the proof.
}



%\itemB Let $R\subseteq \Q[X]$ be the set of polynomials $f(X)$ such that $f(n)\in \Z$ for all $n\in \Z$. Find a 


%\itemB Jacobian criterion for algebraic independence: Let $K$ be a field of characteristic zero, $R=K[X_1,\dots,X_n]$ be a polynomial ring, and ${f_1,\dots,f_n\in R}$ be $n$ polynomials. Show that $f_1,\dots,f_n$ are algebraically independent over $K$ if and only if 
%\[ \det \begin{bmatrix} \frac{\partial f_1}{\partial X_1} & \cdots & \frac{\partial f_n}{\partial X_1} \\
%\vdots & \ddots & \vdots \\
% \frac{\partial f_1}{\partial X_n} & \cdots & \frac{\partial f_n}{\partial X_n} \end{bmatrix}  \neq 0.\]
% Use this to show that the $2\times 2$ minors of a $2\times 3$ matrix of indeterminates are algebraically independent.
 







\end{enumerate}


\vfill





\end{document}
