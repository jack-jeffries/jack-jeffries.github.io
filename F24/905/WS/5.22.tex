\documentclass[12pt]{amsart}


\usepackage{times}
\usepackage[margin=0.8in]{geometry}
\usepackage{amsmath,amssymb,multicol,graphicx,framed,ifthen,color,xcolor,stmaryrd,enumitem,colonequals,bbm}
\usepackage[all]{xy}

\usepackage[outline]{contour}
\contourlength{.4pt}
\contournumber{10}
\newcommand{\Bold}[1]{\contour{black}{#1}}


\definecolor{chianti}{rgb}{0.6,0,0}
\definecolor{meretale}{rgb}{0,0,.6}
\definecolor{leaf}{rgb}{0,.35,0}
\newcommand{\Q}{\mathbb{Q}}
\newcommand{\N}{\mathbb{N}}
\newcommand{\Z}{\mathbb{Z}}
\newcommand{\R}{\mathbb{R}}
\newcommand{\C}{\mathbb{C}}
\newcommand{\e}{\varepsilon}
\newcommand{\m}{\mathfrak{m}}
\newcommand{\p}{\mathfrak{p}}
\newcommand{\q}{\mathfrak{q}}
\newcommand{\ord}{\mathrm{ord}}
\newcommand{\ann}{\mathrm{ann}}
\newcommand{\Min}{\mathrm{Min}}
\newcommand{\Spec}{\mathrm{Spec}}
\renewcommand{\1}{\mathbbm{1}}
\newcommand{\cZ}{\mathcal{Z}}

\newcommand{\inv}{^{-1}}
\newcommand{\dabs}[1]{\left| #1 \right|}
\newcommand{\ds}{\displaystyle}
\newcommand{\solution}[1]{\ifthenelse {\equal{\displaysol}{1}} {\begin{framed}{\color{meretale}\noindent #1}\end{framed}} { \ }}
\newcommand{\showsol}[1]{\def\displaysol{#1}}
\newcommand{\rsa}{\rightsquigarrow}

\newcommand\itemA{\stepcounter{enumi}\item[{\Bold{(\theenumi)}}]}
\newcommand\itemB{\stepcounter{enumi}\item[(\theenumi)]}
\newcommand\itemC{\stepcounter{enumi}\item[{\it{(\theenumi)}}]}
\newcommand\itema{\stepcounter{enumii}\item[{\Bold{(\theenumii)}}]}
\newcommand\itemb{\stepcounter{enumii}\item[(\theenumii)]}
\newcommand\itemc{\stepcounter{enumii}\item[{\it{(\theenumii)}}]}
\newcommand\itemai{\stepcounter{enumiii}\item[{\Bold{(\theenumiii)}}]}
\newcommand\itembi{\stepcounter{enumiii}\item[(\theenumiii)]}
\newcommand\itemci{\stepcounter{enumiii}\item[{\it{(\theenumiii)}}]}
\newcommand\ceq{\colonequals}

\DeclareMathOperator{\res}{res}
\setlength\parindent{0pt}
%\usepackage{times}

%\addtolength{\textwidth}{100pt}
%\addtolength{\evensidemargin}{-45pt}
%\addtolength{\oddsidemargin}{-60pt}

\pagestyle{empty}
%\begin{document}\begin{itemize}

%\thispagestyle{empty}

\usepackage[hang,flushmargin]{footmisc}


\begin{document}
\showsol{0}
	
	\thispagestyle{empty}
	
	\section*{\S5.22: Localization of modules}	

\begin{framed}
\noindent \textsc{Definition:} Let $R$ be a ring, $M$ an $R$-module, and $W$ a multiplicatively closed subset. 
The \mbox{\textbf{localization}} $W^{-1}M$ is the $W^{-1}R$-module\footnotemark \ with
\begin{itemize}
\item  elements  equivalence classes of $(m,w)\in M\times W$, with the class of $(m,w)$ denoted as $\ds \frac{m}{w}$.
\smallskip
\item with equivalence relation $\ds \frac{m}{u} = \frac{n}{v}$ if there is some $w\in W$ such that $w(vm-un)=0$,
\smallskip
\item addition given by $\ds\frac{m}{u} + \frac{n}{v} = \frac{vm+un}{uv}$, and
\smallskip
\item action given by $\ds\frac{r}{u}  \frac{m}{v} = \frac{rm}{uv}$.
\end{itemize}
If $\alpha:M\to N$ is a homomorphism of $R$-modules, then the $W^{-1}R$-module homomorphism ${W^{-1}\alpha: W^{-1}M\to W^{-1}N}$ is defined by $W^{-1}\alpha (\frac{m}{w}) = \frac{\alpha(m)}{w}$.


\

\noindent \textsc{Definition:} Let $R$ be a ring and $M$ a module.
\begin{itemize}
\item If $f\in R$, then $M_f \ceq \{1,f,f^2,\dots\}^{-1} M$.
\item If $\p \subseteq R$ is a prime ideal then $M_{\p} \ceq (R\smallsetminus \p)^{-1} M$.
\end{itemize}

\

\noindent \textsc{Proposition:} Let $R$ be a ring, $W$ a multiplicatively closed set, and $N\subseteq M$ be modules. Then 
\begin{itemize}
\item $W^{-1}N$ is a submodule of $W^{-1}M$, and 
\item $\ds W^{-1}(M/N) \cong \frac{W^{-1}M}{W^{-1}N}$.
\end{itemize}

\

\noindent \textsc{Corollary:} Let $R$ be a ring, $I$ an ideal, and $W$ a multiplicatively closed subset. Then the map $R\to W^{-1}(R/I)$ induces an order preserving bijection
\[ \Spec(W^{-1}(R/I)) \stackrel{\sim}{\longrightarrow} \{ \p\in \Spec(R) \ | \ \p \supseteq I \ \text{and} \ \p \cap W=\varnothing\}.\]
\end{framed}
\footnotetext[1]{If $0\in W$, then $W^{-1}R=0$ is not a ring.}

 
\begin{enumerate}
\itemA Let $M$ be an $R$-module and $W$ be a multiplicatively closed set.
\begin{enumerate}
\itema What is the natural map from $M\to W^{-1}M$?
\itema If $S$ is a generating set for $M$, explain why $\frac{S}{1}=\{ \frac{s}{1} \ | \ s\in S\}$ is a generating set for $W^{-1}M$.
\itema Let $m\in M$. Show that $\frac{m}{u}$ is zero in $W^{-1}M$ if and only if there is some $w\in W$ such that $wm=0$ in $M$.
\itema Let $m_1,\dots,m_t \in M$ be a finite set of elements. Show that $\frac{m_1}{u_1},\dots,\frac{m_t}{u_t}\in W^{-1}M$ are all zero if and only if there is some $w\in W$ that such that $wm_i=0$ in $M$ for all $i$.
\itema Let $M$ be a finitely generated module. Show that $W^{-1}M=0$ if and only if $M_w=0$ for some $w\in W$.
\itema Let $m\in M$ and $\p$ be a prime ideal. Show that $\frac{m}{1}\neq 0$ in $M_{\p}$ if and only if $\p \supseteq \ann_R(m)$.
\end{enumerate}

\solution{
\begin{enumerate}
\item $m\mapsto \frac{m}{1}$
\item We can write $\frac{m}{w}=\frac{\sum_i r_i m_i}{w} = \sum_i \frac{r_i}{w} \frac{m_i}{1}$.
\item $\frac{m}{u} = \frac{0}{1}$ iff $\exists w$ such that $0= w(1m-0u)= wm$.
\item The ``if'' is clear; for the only if, we have $w_1 m_1 = \cdots w_t m_t=0$ so we can take $w=w_1\cdots w_t$.
\item Take a finite generating set for $M$. Then $W^{-1}M=0$ iff each generator maps to $0$ iff there is a $w$ that kills each $m_i$ iff the corresponding $M_w=0$.
\item $\frac{m}{1} = 0$ if and only if there is some $w\notin \p$ with $wm=0$, which happens if and only if $\p \not\supseteq \ann_R(m)$.
\end{enumerate}}



\itemA Prove the Proposition.

\solution{For the first part, we need to show that a nonzero element in $W^{-1}N$ is nonzero in $W^{-1}M$. If $\frac{n}{u}\neq 0$, in $W^{-1}M$ then there is some $w\in W$ such that $wn=0$, which is the same as the condition to be zero in $W^{-1}N$.

For the second part, consider the map from $W^{-1}M$ to $W^{-1}(M/N)$ given by $\frac{m}{u} \mapsto \overline{m}{u}$. Clearly, $W^{-1}N$ is contained in the kernel. An element is in the kernel if and only if there is some $w\in W$ such that $w \overline{m}=0$ in $M/N$, which means $wm \in N$. Then $\frac{m}{u}=\frac{wm}{wu}\in W^{-1}N$.
}

\itemA Corollary.
\begin{enumerate}
\itema Rewrite the Corollary in the special case $W=R\smallsetminus \p$ for some prime $\p$.
\itema Use the Proposition\footnote{Hint: You may want to show that, for $W\cap \p = \varnothing$, $I\subseteq \p$ if and only if $W^{-1}I \subseteq W^{-1}\p$. For this, it may help to observe that $W^{-1}\p \cap R = \p$. You can also use that the isomorphism from the Proposition is a ring isomorphism when $R$ is a ring and $I$ is an ideal.} to justify the Corollary.
%\itema Use the Proposition to show that $\left(\frac{K[X,Y]}{(XY)}\right)_{\!\!(x)} \cong \left(\frac{K[X,Y]}{(X)}\right)_{\!\!(x)} \cong K(Y)$.
\end{enumerate}

\solution{\begin{enumerate}
\itema There is a bijection between $\Spec((R/I)_{\p})$ and primes of $R$ containing $I$ but also contained in $\p$.
\itema We have $W^{-1}(R/I) \cong W^{-1}R/W^{-1}I$. Fromt he Proposition, this is an isomorphism of $R$-modules, but it is easy to see that the map is in fact a ring isomorphism. The primes in $W^{-1}R$ are of the form $W^{-1}\p$ for $\p\in \Spec(R)$ such that $\p\cap W=\varnothing$. By the lattice isomorphism theorem, the primes in $W^{-1}R/W^{-1}I$ correspond to primes $W^{-1}\p$ that contain $W^{-1}I$. But if $\p\supseteq I$ then $W^{-1}\p \supseteq W^{-1}I$, and if $W^{-1}\p \supseteq W^{-1}I$, then since $W^{-1}\p \cap R = \p$ (from definition of prime) $I \subseteq W^{-1}I \cap R \subseteq W^{-1}\p \cap R = \p$. Thus, there is a bijection between primes containing $I$ and not intersecting $W$ with primes of $W^{-1}(R/I)$.
\end{enumerate}}

\itemB Invariance of base: Let $\phi: R\to S$ be a ring homomorphism, and $V\subseteq R$ and $W\subseteq S$ be multiplicatively closed sets such that $\phi(V)=W$.  Show that for any $S$-module $M$, ${V^{-1}M \cong W^{-1}M}$.

\

\begin{samepage}
\itemB I'm already local!
\begin{enumerate}
\item Suppose that the action of each $w\in W$ on $M$ is invertible: for every $w\in W$ the map $m\mapsto mw$ is bijective. Show that $M\cong W^{-1}M$ via the natural map.
\item Let $R$ be a ring, $\m$ a maximal ideal (so $R/\m$ is a field), and $M$ a module such that $\m M=0$. Show that $M\cong M_\m$ by the natural map.
\item More generally, show that\footnote{Hint: Note that $R/\m^n$ is local with maximal ideal (the image of) $\m$.}  if for every $m\in M$ there is some $n$ such that $\m^n m=0$, then $M\cong M_\m$.
\end{enumerate}
\solution{
\begin{enumerate}
\item The map is injective, since $wm=0$ implies $m=0$, and surjective since $\frac{m}{w} = \frac{m' w}{w} = \frac{m'}{1}$ for some $m'$.
\item Let $u\in R\smallsetminus \m$. Then since $R/\m$ is a field, there is some $v\in R$ such that $uv \equiv 1 \mod \m$. Then for any $m\in M$, we have $uvm = (1+ a) m = m$ for some $a\in \m$. In particular the action of $v$ is the inverse of $u$.
\item  Because $R/\m^n$ is local with maximal ideal $\m$, every element not in $\m$ in this ring is a unit. Thus, given $u\in R\smallsetminus \m$, there is some $v\in R$ such that $uv \equiv 1 \mod \m^n$. This shows that the action of $u$ on $M$ is bijective and the first part applies.
\end{enumerate}}
\end{samepage}



\itemB Prove the following: \\
\textsc{Lemma:}  Let $R$ be a ring, $W$ a multiplicatively closed set. Let $M$ be a finitely presented\footnote{This means that $M$ admits a finite generating set for which the module of relations is also finitely generated.}
 \ $R$-module, and $N$ an arbitrary $R$-module. Then for any homomorphism of $W^{-1}R$-modules  ${\beta: W^{-1}M \to W^{-1}N}$, there is some $w\in W$ and some $R$-module homomorphism $\alpha:M\to N$ such that ${\beta = \frac{1}{w}  W^{-1} \alpha}$.
\begin{enumerate}
\item Given $\beta$, show that there exists some $u\in W$ such that for every $m\in M$, $\frac{u}{1}\beta(\frac{M}{1}) \subseteq \frac{N}{1}$.
\item Let $m_1,\dots,m_a$ be a (finite) set of generators for $M$, and $A=[r_{ij}]$ be a corresponding (finite) matrix of relations. Let $n_1,\dots,n_a$ be an $a$-tuple of elements of $N$. Justify: There exists an $R$-module homomorphism $\alpha:M\to N$ such that $\alpha(m_i)=n_i$ if and only if $[n_1, \cdots , n_a] A = 0$.
\item Complete the proof.
\end{enumerate}

\solution{
\begin{enumerate}
\item Let $m_1,\dots,m_a$ be a (finite) set of generators for $M$. We have $\beta(\frac{m_i}{1}) = \frac{t_i}{w_i}$ for some $t_i\in N$ and $w_i\in W$. Take $u=w_1\cdots w_a$.
\item For $\alpha$ to be well-defined means that relations map to zero; it suffices to show that any defining relation maps to zero, and the condition above just says this.
\item In the notation of the above, let $\frac{n'_i}{u} = \beta(m_i)$. Note that 
\[ [\frac{n'_1}{u}, \cdots , \frac{n'_a}{u}] A = [\beta{m_1}, \cdots ,\beta{m_a}] A = \beta( [m_1,\dots,m_a] A) = 0 \qquad \text{in} \ W^{-1}N. \] But this just means that there is some $v\in W$ such that $v$ kills each entry of $ [\frac{n'_1}{u}, \cdots , \frac{n'_a}{u}] A$. But then
\[ [vn'_1, \cdots , vn'_a] A = (uv)   [\frac{n'_1}{u}, \cdots , \frac{n'_a}{u}] A = 0.\]
This means that the map $\alpha$ given by $\alpha(m_i)=vn'_i$ is well defined, and $\beta = \frac{1}{uv}W^{-1}\alpha$ since it is true for each generator $m_i$.
\end{enumerate}}



\end{enumerate}
\vfill





\end{document}
