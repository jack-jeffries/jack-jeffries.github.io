\documentclass[12pt]{amsart}


\usepackage{times}
\usepackage[margin=.86in]{geometry}
\usepackage{amsmath,amssymb,multicol,graphicx,framed,ifthen,color,xcolor,stmaryrd,enumitem,colonequals,bbm}
\usepackage[all]{xy}

\usepackage[outline]{contour}
\contourlength{.4pt}
\contournumber{10}
\newcommand{\Bold}[1]{\contour{black}{#1}}


\definecolor{chianti}{rgb}{0.6,0,0}
\definecolor{meretale}{rgb}{0,0,.6}
\definecolor{leaf}{rgb}{0,.35,0}
\newcommand{\Q}{\mathbb{Q}}
\newcommand{\N}{\mathbb{N}}
\newcommand{\Z}{\mathbb{Z}}
\newcommand{\R}{\mathbb{R}}
\newcommand{\C}{\mathbb{C}}
\newcommand{\e}{\varepsilon}
\newcommand{\m}{\mathfrak{m}}
\newcommand{\p}{\mathfrak{p}}
\newcommand{\q}{\mathfrak{q}}
\newcommand{\ord}{\mathrm{ord}}
\newcommand{\Min}{\mathrm{Min}}
\newcommand{\Spec}{\mathrm{Spec}}
\renewcommand{\1}{\mathbbm{1}}
\newcommand{\cZ}{\mathcal{Z}}

\newcommand{\inv}{^{-1}}
\newcommand{\dabs}[1]{\left| #1 \right|}
\newcommand{\ds}{\displaystyle}
\newcommand{\solution}[1]{\ifthenelse {\equal{\displaysol}{1}} {\begin{framed}{\color{meretale}\noindent #1}\end{framed}} { \ }}
\newcommand{\showsol}[1]{\def\displaysol{#1}}
\newcommand{\rsa}{\rightsquigarrow}

\newcommand\itemA{\stepcounter{enumi}\item[{\Bold{(\theenumi)}}]}
\newcommand\itemB{\stepcounter{enumi}\item[(\theenumi)]}
\newcommand\itemC{\stepcounter{enumi}\item[{\it{(\theenumi)}}]}
\newcommand\itema{\stepcounter{enumii}\item[{\Bold{(\theenumii)}}]}
\newcommand\itemb{\stepcounter{enumii}\item[(\theenumii)]}
\newcommand\itemc{\stepcounter{enumii}\item[{\it{(\theenumii)}}]}
\newcommand\itemai{\stepcounter{enumiii}\item[{\Bold{(\theenumiii)}}]}
\newcommand\itembi{\stepcounter{enumiii}\item[(\theenumiii)]}
\newcommand\itemci{\stepcounter{enumiii}\item[{\it{(\theenumiii)}}]}
\newcommand\ceq{\colonequals}

\DeclareMathOperator{\res}{res}
\setlength\parindent{0pt}
%\usepackage{times}

%\addtolength{\textwidth}{100pt}
%\addtolength{\evensidemargin}{-45pt}
%\addtolength{\oddsidemargin}{-60pt}

\pagestyle{empty}
%\begin{document}\begin{itemize}

%\thispagestyle{empty}

\usepackage[hang,flushmargin]{footmisc}


\begin{document}
\showsol{0}
	
	\thispagestyle{empty}
	
	\section*{\S5.21: Localization of rings}	

\begin{framed}

%\noindent \textsc{Recall:} A subset $W$ of a ring $R$ is \textbf{multiplicatively closed} if $1\in W$ and $u,v\in W$ implies $uv\in W$.

%\

\noindent \textsc{Definition:} Let $R$ be a ring and $W$ a multiplicatively closed subset with $0\notin W$. The \textbf{localization} $W^{-1}R$ is the ring with
\begin{itemize}
\item  elements equivalence classes of $(r,w)\in R\times W$, with the class of $(r,w)$ denoted as $\ds \frac{r}{w}$.
\smallskip
\item with equivalence relation $\ds \frac{s}{u} = \frac{t}{v}$ if there is some $w\in W$ such that $w(sv-tu)=0$,
\smallskip
\item addition given by $\ds\frac{s}{u} + \frac{t}{v} = \frac{sv+tu}{uv}$, and
\smallskip
\item multiplication given by $\ds\frac{s}{u}  \frac{t}{v} = \frac{st}{uv}$.
\end{itemize}
(If $0\in W$, then $W^{-1}R\ceq 0$, which by our convention is not a ring.)

\

\noindent \textsc{Definition:} Let $R$ be a ring.
\begin{itemize}
\item If $f\in R$ is nonnilpotent\footnotemark, then $R_f \ceq \{1,f,f^2,\dots\}^{-1} R$.
\item If $\p \subseteq R$ is a prime ideal then $R_{\p} \ceq (R\smallsetminus \p)^{-1} R$.
\item The \textbf{total quotient ring} of $R$ is $\mathrm{Frac}(R)\ceq \{ w\in R \ | \ w \ \text{is a nonzerodivisor}\}^{-1}R$.
\end{itemize}

\

%\noindent \textsc{Universal property of localization\footnotemark:} The localization $W^{-1}R$ is an $R$-algebra (with structure map $l$) satisfying the following property: for any ring homomorphism $\phi:R \to S$ such that $\phi(W)$ consists of units in $S$, there is a unique ring homomorphism $\widetilde{\phi}: W^{-1}R \to S$ such that $\phi = \widetilde{\phi} \circ l$.

%This property characterizes the localization up to isomorphism.

\noindent For a ring $R$, multiplicative set $W\not\ni 0$, and an ideal $I$, we define 
\[W^{-1}I \ceq \left\{\ds \frac{a}{w} \in W^{-1}R \ | \ a\in I\right\}.\]

\

\noindent \textsc{Theorem:} Let $R$ be a ring and $W$ be a multiplicatively closed subset. Then the map induced on $\Spec$ corresponding to the natural map $R\to W^{-1}R$ yields a homeomorphism into its image:
\[ \Spec(W^{-1} R) \cong \{ \p\in \Spec(R) \ | \ \p \cap W = \varnothing\}.\]

\

\noindent \textsc{Lemma:} Let $R$ be a ring and $W$ be a multiplicatively closed subset.
\begin{enumerate}
\item For any ideal $I\subseteq R$, \ $W^{-1}I= I(W^{-1}R)$.
\item For any ideal $I\subseteq R$, \ $W^{-1}I \cap R = \{ r\in R \ | \ \exists w\in W : wr\in I\}$.
\item For any ideal $J\subseteq W^{-1}R$, \ $W^{-1}(J\cap R) = J$.
\item For any prime ideal $\p \subseteq R$ with\footnotemark \ $\p\cap W=\varnothing$, \ $W^{-1}\p$ is prime.
\end{enumerate}
\end{framed}
\footnotetext[1]{If $f$ is nilpotent, $0\in \{1,f,f^2,\dots\}$ so $R_f=0$.}
\footnotetext[2]{If $W\cap \p\ni a$, then $W^{-1}\p \ni \frac{a}{a}=\frac{1}{1}$, so $W^{-1}\p=W^{-1}R$ is the improper ideal!}

 
\begin{enumerate}
\itemA Computing localizations
\begin{enumerate}
\itema What is the natural ring homomorphism $R \to W^{-1}R$?
\itema Show that the kernel of $R \to W^{-1}R$ is ${}^W\! 0 \ceq\{r\in R \ | \ \exists w\in W : wr=0\}$.
\itema If every element of $W$ is a nonzerodivisor, explain why the equivalence relation on $W^{-1}R$ simplifies to $\frac{s}{u} = \frac{t}{v}$ if and only if $sv=tu$.
\itema If $R$ is a domain, explain why $\mathrm{Frac}(R)$ is the usual fraction field of $R$.
\itema If $R$ is a domain, explain why $W^{-1}R$ is a subring of the fraction field of $R$. Which subring?
\itema Let $\overline{R}= R/ {}^W \!0$ and $\overline{W}$ be the image of $W$ in $\overline{R}$. Show that $W^{-1}R \cong \overline{W}^{-1} \overline{R}$.
\end{enumerate}

\solution{\begin{enumerate}
\itema $r\mapsto \frac{r}{1}$.
\itema $\frac{r}{1} = \frac{0}{1}$ if and only if $\exists w\in W : rw = w(1r-0)=0$.
\itema $w(sv-tu)=0$ and $w$ a nonzerdivisor implies $sv-tu=0$; i.e., $sv=tu$.
\itema In light of the above, it's just the definition.
\itema The equivalence relation on the fractions is the same as that in the fraction field, so the map is injective; the operations are definitely the same. It is the subring consisting of fractions that can be written with denominator in $W$.
\itema We define a map from $W^{-1}R \to \overline{W}^{-1} \overline{R}$ by $\frac{r}{w} \mapsto \frac{\overline{r}}{\overline{w}}$. It is clear from the construction that this is a surjective homomorphism. Suppose that $\frac{r}{w}$ is in the kernel, so $ \frac{\overline{r}}{\overline{w}}= \frac{\overline{0}}{\overline{1}}$. This means that there is some $\overline{v}\in \overline{W}$ such that $\overline{v}\overline{r}=\overline{0}$; i.e.,  $vr\in {}^W \!0$ for some $v\in W$. Then there is some $u\in W$ such that $uvr=0$, but $uv\in W$, so $\frac{r}{w}=\frac{0}{1}$ in $W^{-1}R$. 
\end{enumerate}}

\begin{samepage}
\itemA Ideals in localizations: Let $R$ be a ring and $W$ a multiplicatively closed set.
\begin{enumerate}
\itema Use the Theorem to show that, if $f\in R$ is nonnilpotent, then \[{\Spec(R_f) \cong D(f) \subseteq \Spec(R)}.\]
\itema Use the Theorem to show that, if $\p \subseteq R$ is prime, then \[{\Spec(R_{\p})\cong \{ \q\in \Spec(R) \ | \ \q \subseteq \p\}} \equalscolon \Lambda(\p).\] Deduce that $R_{\p}$ is always a \emph{local} ring.
\itema Draw\footnote{Recall that $\Spec(\frac{\C[X,Y]}{(XY)})$ consists of $\{(x),(y), (x,y-\alpha), (x-\beta,y) \ | \ \alpha,\beta\in \C\}$. } a picture of $\Spec(\frac{\C[X,Y]}{(XY)}_{\! \!(x,y)})$.
\itema Use Part (3) of the Lemma to show that every ideal of $W^{-1}R$ is of the form $W^{-1} I$ for some ideal $I\subseteq R$.
\itema Use Part (3) of the Lemma to show that any localization of a Noetherian ring is Noetherian.
\end{enumerate}
\end{samepage}


\solution{
\begin{enumerate}
\itema The condition $\p\cap \{1,f,f^2,\dots\}=\varnothing$ is equivalent to $f\notin \p$; i.e., $f\in D(\p)$.
\itema The condition $\q\cap (R\smallsetminus \p)=\varnothing$ is equivalent to $\q \subseteq \p$; i.e., $\q\in \Lambda(\p)$. There is a unique maximal element in this set, namely $\p$, so $R_{\p}$ is local.
\itema \[ \xymatrix@C-1pc{ & (x,y) & \\ 
 (x)\ar@{-}[ur] & &  (y)\ar@{-}[ul]}\]
\itema Clear.
\itema Given an ideal of $W^{-1}R$, write it as $I(W^{-1}R)$ for some ideal $I$ of $R$. Then $I=(f_1,\dots,f_t)$ by Noetherianity, whence $I(W^{-1}R)$ is generated by the images $\frac{f_1}{1},\dots,\frac{f_t}{1}$.
\end{enumerate}}

\itemA Examples of localizations
\begin{enumerate}
\itema Describe as concretely as possible the rings $\Z_{2}$ and $\Z_{(2)}$ as defined above. 
\itema Describe as concretely as possible the rings $K[X]_X$ and $K[X]_{(X)}$.
\itema Describe as concretely as possible the rings $K[X,Y]_X$ and $K[X,Y]_{(X)}$.
\itema Describe as concretely as possible the rings $\left(\frac{K[X,Y]}{(XY)}\right)_{\! x}$ and $\left(\frac{K[X,Y]}{(XY)}\right)_{\! (x)}$.
\itema Describe as concretely as possible  $\left(\frac{K[X,Y]}{(X^2)}\right)_{\! x}$ and $\left(\frac{K[X,Y]}{(X^2)}\right)_{\! (x)}$.
\end{enumerate}

\solution{\begin{enumerate}
\itema $\Z_{2}=\{ a/b\in \Q \ | \ b=2^n\}$ and $\Z_{(2)}= \{ a/b \in \Q \ | \ 2\nmid b\}$. 
\itema $K[X]_X=\{ f/g\in K(X) \ | \ g=X^n\}$ and $K[X]_{(X)}=\{ f/g\in K(X) \ | \ X\nmid g\}$.
\itema $K[X,Y]_X=\{ f/g\in K(X,Y) \ | \ g=X^n\}$ \\ and ${K[X,Y]_{(X)}=\{ f/g\in K(X,Y) \ | \ X\nmid g\}}$.
\itema $\left(\frac{K[X,Y]}{(XY)}\right)_{\! x}\cong K[X,X^{-1}]$ and $\left(\frac{K[X,Y]}{(XY)}\right)_{\! (x)}\cong K(Y)$.
\itema $\left(\frac{K[X,Y]}{(X^2)}\right)_{\! x}\cong K[Y]$ and $\left(\frac{K[X,Y]}{(X^2)}\right)_{\! (x)}\cong K(Y)[X]/(X^2)$.
\end{enumerate}}



\itemB Prove the Lemma and the Theorem.

\solution{Lemma:
\begin{enumerate}
\item For the containment $\subseteq$, we have $\frac aw = \frac a1 \frac 1w$. For the other, given $\sum_i \frac{a_i}{1} \frac{r_i}{w_i}$, take $w = w_1\cdots w_t$ and $w'_i$ to be the product of all $w$'s except $w_i$; then
\[ \sum_i \frac{a_i}{1} \frac{r_i}{w_i} = \sum_i \frac{a_i}{1} \frac{w'_i r_i}{w} = \sum_i \frac{a_i w'_i r_i}{w} \in W^{-1}I.\]
\item We have $r \in W^{-1}I \cap R$ if and only if $\frac{r}{1} \in W^{-1}I$, so $\frac{r}{1} = \frac{a}{w}$ some $a\in I, w\in W$. Then there is some $u\in W$ such that $u(wr-a)=0$, so $(uw)r\in I$, as claimed.
\item Let $j=\frac{r}{w}\in J$. Then $\frac{r}{1} = w j\in J \cap R$, $\frac{r}{w} = \frac{1}{w} \frac{r}{1} \in W^{-1}(J\cap R)$. Conversely, if $\frac{a}{w} \in W^{-1}(J\cap R)$ so $a\in J\cap R$, then $\frac{a}{1}\in J$, and $\frac{a}{w}=\frac{1}{w}\frac{a}{1}\in J$.
\item Let $\frac{a}{u}, \frac{b}{v}\in W^{-1}R$, and $\frac{ab}{uv}\in W^{-1}\p$. Then there are some $w\in W$ and $p\in \p$ such that $\frac{ab}{uv} = \frac{p}{w}$, so there is $t\in W$ with $t(wab-uvp)=0$, so $(tw)ab\in \p$. Since $W\cap \p=\varnothing$, $tw\notin \p$ so $a\in \p$ or $b\in \p$, and hence $\frac{a}{u}\in W^{-1}\p$ or $\frac{b}{v}\in W^{-1}\p$.
\end{enumerate}

Theorem:
 Suppose that $\q$ is a prime ideal in $W^{-1}R$ and $\q \cap R=\p$. Then $W^{-1}\p = W^{-1}(\q \cap R) = \q$. This shows that the only ideal (in particular, the only prime ideal) that contracts to $\p$ is $W^{-1}\p$, so this map is injective. Since $W^{-1}\p$ is prime for any $\p \cap W=\varnothing$, and is the bogus ideal otherwise, the image is exactly the primes with $\p \cap W=\varnothing$. To see that it induces a homeomorphism onto its image, it suffices to show that the image of a closed set is closed. One checks from the definition that the image of $V(W^{-1} I)$ is $V(I) \cap \{\p\in \Spec(R) \ | \ \p\cap W=\varnothing\}$. 
}

%\itemB Use the Lemma to give a second proof of the \\
%\textsc{Proposition:} If $I$ is an ideal and $W$ is a multiplicatively closed subset such that $W\cap I = \varnothing$, then there is a prime ideal $\p \supseteq I$ such that $W\cap \p = \varnothing$.


%\solution{}

\itemB Prove the following \textsc{Lemma:} If $V,W$ are multiplicatively closed sets, then $(VW)^{-1} R\cong (\frac{V}{1})^{-1}(W^{-1} R)$, where $(\frac{V}{1})^{-1}$ is the image of $V$ in $W^{-1}R$.

\solution{Check that the map $(r/w)/(v/1) \mapsto r/(wv)$ is an isomorphism: it is clearly a ring homomorphism, and clearly surjective. If $r/(wv)$ is zero, then there is some $u\in VW$ with $ur=0$. We can write $u=st$ with $s\in V$ and $t\in W$, so $str=0$. But this implies that $s(r/w)=0$ in $W^{-1}R$ (because there is some $t\in W$ such that $str=0$), and this means that $(r/w)/(v/1)=0$.}

%\itemB 
%\begin{enumerate}
%\itemb Show that $R_f \cong R[X]/(fX-1)$.
%\itemb Show that $W^{-1}R$ is generated as an $R$-algebra by the elements $\{1/w \ | \ w\in W\}$ with defining relations $\{w X_w - 1 \ | \ w\in W\}$.
%\end{enumerate}
%
%\

\begin{samepage}
\itemB Minimal primes.
\begin{enumerate}
\item Let $\p$ be a minimal prime of $R$. Show that for any $a\in \p$, there is some $u\notin \p$ and $n\geq 1$ such that $ua^n=0$.
\item Show that the set of minimal\footnote{$\Min(R)$ denotes the set of primes of $R$ that are minimal. This is the same as $\Min(0)$ in our notation of minimal primes of an ideal; this conflict of notation is standard.} primes $\Min(R)$ with the induced topology from $\Spec(R)$ is Hausdorff.
\item Let $R= K[X_1,X_2,X_3,\dots]/(\{X_i X_j \ | \ i \neq j\})$. Describe $\Min(R)$ as a topological space.
\end{enumerate}
\end{samepage}




\




\end{enumerate}
\vfill





\end{document}
