\documentclass[12pt]{amsart}


\usepackage{times}
\usepackage[margin=0.95in]{geometry}
\usepackage{amsmath,amssymb,multicol,graphicx,framed,ifthen,color,xcolor,stmaryrd,enumitem,colonequals,bbm}
\usepackage[all]{xy}

\usepackage[outline]{contour}
\contourlength{.4pt}
\contournumber{10}
\newcommand{\Bold}[1]{\contour{black}{#1}}


\definecolor{chianti}{rgb}{0.6,0,0}
\definecolor{meretale}{rgb}{0,0,.6}
\definecolor{leaf}{rgb}{0,.35,0}
\newcommand{\Q}{\mathbb{Q}}
\newcommand{\N}{\mathbb{N}}
\newcommand{\Z}{\mathbb{Z}}
\newcommand{\R}{\mathbb{R}}
\newcommand{\C}{\mathbb{C}}
\newcommand{\e}{\varepsilon}
\newcommand{\m}{\mathfrak{m}}
\newcommand{\p}{\mathfrak{p}}
\newcommand{\q}{\mathfrak{q}}
\newcommand{\ord}{\mathrm{ord}}
\newcommand{\ann}{\mathrm{ann}}
\newcommand{\Min}{\mathrm{Min}}
\newcommand{\Max}{\mathrm{Max}}
\newcommand{\Spec}{\mathrm{Spec}}
\newcommand{\Ass}{\mathrm{Ass}}
\renewcommand{\1}{\mathbbm{1}}
\newcommand{\cZ}{\mathcal{Z}}

\newcommand{\inv}{^{-1}}
\newcommand{\dabs}[1]{\left| #1 \right|}
\newcommand{\ds}{\displaystyle}
\newcommand{\solution}[1]{\ifthenelse {\equal{\displaysol}{1}} {\begin{framed}{\color{meretale}\noindent #1}\end{framed}} { \ }}
\newcommand{\showsol}[1]{\def\displaysol{#1}}
\newcommand{\rsa}{\rightsquigarrow}

\newcommand\itemA{\stepcounter{enumi}\item[{\Bold{(\theenumi)}}]}
\newcommand\itemB{\stepcounter{enumi}\item[(\theenumi)]}
\newcommand\itemC{\stepcounter{enumi}\item[{\it{(\theenumi)}}]}
\newcommand\itema{\stepcounter{enumii}\item[{\Bold{(\theenumii)}}]}
\newcommand\itemb{\stepcounter{enumii}\item[(\theenumii)]}
\newcommand\itemc{\stepcounter{enumii}\item[{\it{(\theenumii)}}]}
\newcommand\itemai{\stepcounter{enumiii}\item[{\Bold{(\theenumiii)}}]}
\newcommand\itembi{\stepcounter{enumiii}\item[(\theenumiii)]}
\newcommand\itemci{\stepcounter{enumiii}\item[{\it{(\theenumiii)}}]}
\newcommand\ceq{\colonequals}

\DeclareMathOperator{\res}{res}
\setlength\parindent{0pt}
%\usepackage{times}

%\addtolength{\textwidth}{100pt}
%\addtolength{\evensidemargin}{-45pt}
%\addtolength{\oddsidemargin}{-60pt}

\pagestyle{empty}
%\begin{document}\begin{itemize}

%\thispagestyle{empty}

\usepackage[hang,flushmargin]{footmisc}


\begin{document}
\showsol{0}
	
	\thispagestyle{empty}
	
	\section*{\S6.25: Associated primes}
	
	\begin{framed}

\noindent \textsc{Definition:} Let $R$ be a ring and $M$ be a module. A prime ideal $\p$ of $R$ is an \textbf{associated prime} of $M$ if $\p = \ann_R(m)$ for some $m\in M$. The element $m$ is called a \textbf{witness} for the associated prime $\p$. We write $\Ass_R(M)$ for the set of associated primes of a module.

\

\noindent \textsc{Lemma:} Let $R$ be a Noetherian ring and $M$ be a module. 
For any nonzero element $m\in M$, the ideal $\ann_R(m)$ is contained in an associated prime of $M$. In particular, if $M\neq 0$, then $M$ has an associated prime.

\

\noindent \textsc{Definition:} Let $R$ be a ring and $M$ be an $R$-module. We say that an element $r\in R$ is a \textbf{zerodivisor} on $M$ if there is some $m \in M\smallsetminus 0$ such that $rm = 0$.

\

\noindent \textsc{Proposition:} Let $R$ be a Noetherian ring and $M$ an $R$-module. The set of zerodivisors on $M$ is the union of the associated primes of $M$.

\

\noindent \textsc{Theorem:} Let $R$ be a Noetherian ring, $W$ be a multiplicatively closed set, and $M$ be a module. Then
\[ \Ass_{W^{-1}R}(W^{-1}M) = \{ W^{-1}\p \ | \ \p\in \Ass_R(M), \p\cap W=\varnothing\}.\]

\

\noindent \textsc{Corollary:} Let $R$ be a Noetherian ring and  $I$ be an ideal. Then $\Min(I) \subseteq \Ass_R(R/I)$.
\end{framed}


	

\begin{enumerate}

\itemA Proof of Lemma and Proposition: Let $R$ be a Noetherian ring and $M$ be a nonzero module.
\begin{enumerate}
\itema Let $\mathcal{S}=\{ \ann_R(m) \ | \ m\in M \smallsetminus 0\}$. Explain why $\mathcal{S}$ has a maximal element $J$.
\itema Let $J=\ann_R(m)$ and suppose that $rs\in J$ but $s\notin J$. Explain why ${J=\ann_R(sm)}$.
\itema Conclude the proof of the Lemma.
\itema Deduce the Proposition from the Lemma.
\itema What does the Proposition say in the special case when $M=R$?
\end{enumerate} 

\solution{\begin{enumerate}
\itema Because this is a nonempty collection of ideals in a Noetherian ring.
\itema First, $\ann_R(sm) \supseteq \ann_R(m)$ since $rm=0$ implies $rsm=0$. Since $s\notin J$, $\ann_R(sm)\neq R$, so by maximality we have equality.
\itema Suppose $s\notin J$ and $rs\in J$. Then $rsm=0$ implies that $r\in \ann_R(sm) = J$. Thus $J$ is prime. Since any element of $\mathcal{S}$ is contained in a maximal element, the claim follows.
\itema If $r$ is a zerodivisor on $M$, then $r$ is contained in some ideal of $\mathcal{S}$, and then it is contained in an associated prime. Conversely, any element in an associated prime is a zerodivisor on $M$ by definition.
\itema The zerodivisors in $R$ are the elements in some associated prime.
\end{enumerate} }

\itemA Working with associated primes.
\begin{enumerate}
\itema Let $R$ be a domain and $M$ be a torsionfree module. Show that $\Ass_R(M)=\{(0)\}$.
\itema Let $R$ be a ring and $\p$ be a prime ideal. Show that for any nonzero element $\overline{r} \in R/\p$ that $\ann_R(\overline{r}) = \p$ and use the definition to deduce that $\Ass_R(R/\p) = \{\p\}$.
\itema Let $K$ be a field and $R=K[X,Y]/(X^2Y,XY^2)$. Use\footnote{Hint: Consider $xy$ and $y^2$.} the definition to show that $(x,y)$, $(x)$, and $(y)$ are associated primes of $R$. 
\itema Let $M$ be a module. Explain why $\p\in \Ass_R(M)$ if and only if there is an injective $R$-module homomorphism $R/\p \hookrightarrow M$.
\end{enumerate}

\solution{
\begin{enumerate}
\itema By definition, any nonzero element has annihilator zero.
\itema Clearly $\p\subseteq \ann_R(\overline{r})$. Let $r$ be a representative of $\overline{r}$; we have $r\notin \p$. The annihilator of $\overline{r}\in R/\p$ is the set of $s\in R$ such that $sr\in \p$. By definition of prime, $s\in \p$, so $\ann_R(\overline{r}) \subseteq \p$ and equality holds.
\itema Since $x \cdot xy = x^2y = 0$ and $y \cdot xy = xy^2$, the annihilator of $xy$ contains $(x,y)$; any element not in $(x,y)$ has some/every representative with a nonzero constant term, and hence does not kill $xy$. Thus $\ann_R(xy)=(x,y)$.

We claim that $\ann_R(y^2) = (x)$. Indeed, $x \cdot y^2 = 0$, and if $f\notin (x)$, then some/every representative $f$ has a nonzero term that only involves $Y$, and $f \cdot Y^2$ has a noznero term only involving $Y$, and hence nonzero modulo $(X^2Y,XY^2)$. The claim follows. Along similar lines, $\ann_R(x^2)= (y)$.
\itema If $\ann_R(m)=\p$, then the map $R\to M$ sending $1\mapsto m$ has kernel $\p$, so one has an injection $R/\p \to M$. Conversely, if $R/\p \hookrightarrow M$, then the image of $1$ has annihilator $\p$.
\end{enumerate}
}


\itemA Using the Theorem. Let $R$ be a Noetherian ring.
\begin{enumerate}
\itema Restate the Theorem in the special case $W = R\smallsetminus \p$ with our standard notation for this setting.
\itema Show (either using the Theorem or 2(d) above) that $\Ass_R(M)\subseteq \mathrm{Supp}_R(M)$.
\itema Use the Theorem (and the previous part or otherwise) to prove the Corollary.
\itema Show the more general statement: if $M$ is a nonzero module, then the primes that are minimal within the support of $I$ are associated to $M$.
\end{enumerate}


\solution{\begin{enumerate}
\itema $\Ass_{R_\p}(M_\p) = \{ \q R_\p \ | \ \q \in \Ass_R(M) \ \text{and} \ \q \subseteq \p\}$.
\itema Suppose that $\p\in \Ass_R(M)$. Then $\p R_\p \in \Ass_{R_\p}(M_\p)$ so $M_\p\neq 0$.
\itema Let $M=R/I$ and $\p\in \Min(I)$. Then $M_\p\neq 0$ (for various reasons as previously discussed in localizations), so $M_\p\neq 0$. But the support of $M_\p$ is $V(I R_\p) = \{\p R_\p\}$, so $\p R_\p\in \Ass_{R_\p}(M_\p)$ and hence $\p\in \Ass_R(M)$.
\itema The previous argument shows this.
\end{enumerate}}

\itemB The ring of Puiseux series is $R=\bigcup_{n\geq 1} \C\llbracket X^{1/n} \rrbracket$: elements consist of power series with fractional exponents that have a common denominator (though different elements can have different common denominators).
\begin{enumerate}
\item Show that every nonzero element of $R$ can be written in the form $X^{m/n} \cdot u$ for some unit~$u$.
\item Show that the $R$-module $R/(X)$ is nonzero but has no associated primes.
\end{enumerate}

\



\itemB Proof of Theorem: Let $R$ be a Noetherian ring, $W$ be a multiplicatively closed set, and $M$ be a module.
\begin{enumerate}
\itemb Suppose that $\p$ is an associated prime of $M$ with $W\cap \p = \varnothing$, and let $m$ be a witness for $\p$ as an associated prime of $M$. Show that $W^{-1}\p$ is an associated prime of $W^{-1}M$ with witness~$\frac{m}{1}$.
\itemb Suppose that $W^{-1}\p \in \Spec(W^{-1}R)$ is an associated prime of $W^{-1}M$. Explain why there is a witness of the form $\frac{m}{1}$.
\itemb Let $\p=(f_1,\dots,f_t)$. Explain why there exist $w_1,\dots,w_t\in W$ such that $w_i f_i m=0$ in $M$ for all $i$.
\itemb Show that $w_1 \cdots w_t m$ is a witness for $\p$ as an associated prime of $M$.
\end{enumerate}

\

\itemB Let $R$ be a Noetherian ring and $M$ be a module. Show that $\p\in \Ass_R(M)$ if and only if for every $r\in \p$ and every nonzero $m\in M$, there exists some $u\notin \p$ such that $urm=0$.

\

\itemB Let $R$ be a Noetherian ring. Is every minimal prime of a zerodivisor a minimal prime of $R$?

\end{enumerate}

\end{document}
