\documentclass[12pt]{amsart}


\usepackage{times}
\usepackage[margin=.65in]{geometry}
\usepackage{amsmath,amssymb,multicol,graphicx,framed,ifthen,color,xcolor,stmaryrd,enumitem,colonequals,bbm}
\usepackage[all]{xy}

\usepackage[outline]{contour}
\contourlength{.4pt}
\contournumber{10}
\newcommand{\Bold}[1]{\contour{black}{#1}}


\definecolor{chianti}{rgb}{0.8,0,0}
\definecolor{meretale}{rgb}{0,0,.6}
\definecolor{leaf}{rgb}{0,.35,0}
\newcommand{\Q}{\mathbb{Q}}
\newcommand{\N}{\mathbb{N}}
\newcommand{\Z}{\mathbb{Z}}
\newcommand{\R}{\mathbb{R}}
\newcommand{\C}{\mathbb{C}}
\newcommand{\e}{\varepsilon}
\newcommand{\m}{\mathfrak{m}}
\newcommand{\p}{\mathfrak{p}}
\newcommand{\q}{\mathfrak{q}}
\newcommand{\ord}{\mathrm{ord}}
\newcommand{\hgt}{\mathrm{height}}
\newcommand{\ann}{\mathrm{ann}}
\newcommand{\Min}{\mathrm{Min}}
\newcommand{\Max}{\mathrm{Max}}
\newcommand{\Spec}{\mathrm{Spec}}
\newcommand{\Ass}{\mathrm{Ass}}
\renewcommand{\1}{\mathbbm{1}}
\newcommand{\cZ}{\mathcal{Z}}

\newcommand{\inv}{^{-1}}
\newcommand{\dabs}[1]{\left| #1 \right|}
\newcommand{\ds}{\displaystyle}
\newcommand{\solution}[1]{\ifthenelse {\equal{\displaysol}{1}} {\begin{framed}{\color{meretale}\noindent #1}\end{framed}} { \ }}
\newcommand{\solutione}[1]{\ifthenelse {\equal{\displaysol}{1}} {\begin{framed}{\color{leaf}This solution is embargoed.}\end{framed}} { \ }}
\newcommand{\showsol}[1]{\def\displaysol{#1}}
\newcommand{\rsa}{\rightsquigarrow}

\newcommand\itemA{\stepcounter{enumi}\item[{\Bold{(\theenumi)}}]}
\newcommand\itemB{\stepcounter{enumi}\item[(\theenumi)]}
\newcommand\itemC{\stepcounter{enumi}\item[{\it{(\theenumi)}}]}
\newcommand\itema{\stepcounter{enumii}\item[{\Bold{(\theenumii)}}]}
\newcommand\itemb{\stepcounter{enumii}\item[(\theenumii)]}
\newcommand\itemc{\stepcounter{enumii}\item[{\it{(\theenumii)}}]}
\newcommand\itemai{\stepcounter{enumiii}\item[{\Bold{(\theenumiii)}}]}
\newcommand\itembi{\stepcounter{enumiii}\item[(\theenumiii)]}
\newcommand\itemci{\stepcounter{enumiii}\item[{\it{(\theenumiii)}}]}
\newcommand\ceq{\colonequals}

\DeclareMathOperator{\res}{res}
\setlength\parindent{0pt}
%\usepackage{times}

%\addtolength{\textwidth}{100pt}
%\addtolength{\evensidemargin}{-45pt}
%\addtolength{\oddsidemargin}{-60pt}

\pagestyle{empty}
%\begin{document}\begin{itemize}

%\thispagestyle{empty}

\usepackage[hang,flushmargin]{footmisc}


\begin{document}
\showsol{0}
	
	\thispagestyle{empty}
	
	\section*{\S7.29: Dimension and height}
	
	\begin{framed}

\noindent \textsc{Definition:} Let $R$ be a ring. 
\begin{itemize}
\item A \textbf{chain of primes of length $n$} is
\[ \p_0 \subsetneqq \p_1 \subsetneqq \cdots \subsetneqq \p_n \qquad \text{with} \ \p_i\in \Spec(R).\]
We may say this chain is \textbf{from $\p_0$} and/or \textbf{to $\p_n$} to indicate the minimal and/or maximal elements.
\item A chain of primes as above is \textbf{saturated} if for each $i$, there is no prime $\q$ such that ${\p_i \subsetneqq \q \subsetneqq \p_{i+1}}$.
\item The \textbf{dimension} of $R$ is 
\[ \dim(R) \ceq \sup\{ n\geq 0 \ | \ \text{there is a chain of primes of length $n$ in $\Spec(R)$}\}.\]
\item The \textbf{height} of a prime ideal $\p\in \Spec(R)$ is
\[ \hgt(\p)\ceq  \sup\{ n\geq 0 \ | \ \text{there is a chain of primes to $\p$ of length $n$ in $\Spec(R)$}\}.\]
\item The \textbf{height} of an arbitrary proper ideal $I\subseteq R$ is
\[ \hgt(I)\ceq  \inf\{ \hgt(\p) \ | \ \p\in \Min(I) \}.\]
\end{itemize}
\end{framed}



\begin{enumerate}
\itemA Let $K$ be field. Use the definition of dimension to prove the following:
\begin{enumerate}
\itema $\dim(K)=0$.
\itema If $R$ is a PID, but not a field, then $\dim(R)=1$.
\itema $\dim(K[X_1,\dots,X_n])\geq n$.
\itema $\dim(K\llbracket X_1,\dots,X_n \rrbracket)\geq n$.
\itema $\dim(K[X_1,X_2,X_3,\dots])=\infty$.
\end{enumerate}

\solution{
\begin{enumerate}
\itema The only prime is $(0)$ so every chain has length zero.
\itema Every nonzero prime is maximal, so the longest chains have length one.
\itema There is a chain $(0) \subsetneqq (X_1)  \subsetneqq (X_1,X_2)  \subsetneqq \cdots  \subsetneqq (X_1,\dots,X_n)$. 
\itema Same as above.
\itema Same as above by keep going.
\end{enumerate}
}


\itemA Let $R$ be a ring, $I$ an ideal, and $\p$ a prime ideal. Use the definitions to prove the following:
\begin{enumerate}
\itema $\hgt(\p)=0$ if and only if $\p\in \Min(R)$.
\itema $\hgt(I)=0$ if and only if $I\subseteq \p$ for some $\p\in \Min(R)$.
\itema If $R$ is a domain and $I\neq 0$, then $\hgt(I)>0$.
\itema $\dim(R/\p) = \sup\{ n\geq 0 \ | \ \text{there is a chain of primes of length $n$ in $V(\p)$}\}.$
\itema $\dim(R/I) = \sup\{ n\geq 0 \ | \ \text{there is a chain of primes of length $n$ in $V(I)$}\}.$
\itema If $R$ is a domain and $I\neq 0$, and $\dim(R)<\infty$, then $\dim(R/I)<\dim(R)$.
\itema $\dim(R) = \sup \{ \dim(R/\p) \ | \ \p\in \Min(R)\}$.
\itema $\dim(R_\p) = \hgt(\p)$.
\itema $\dim(R) = \sup\{\dim(R_\m) \ | \ \m\in \Max(R)\}$.
\itema $\hgt(\p) + \dim(R/\p) = \sup\left\{ n\geq 0 \ \Big| \ \  \begin{aligned} &\text{there is a chain of primes of length $n$} \\  &\text{in $\Spec(R)$ such that  $\p_i=\p$ for some $i$} \end{aligned}\right\}$
\itema $\hgt(\p) + \dim(R/\p) \leq \dim(R)$.
\itema $\hgt(I) + \dim(R/I)\, {\color{chianti}\leq} \,\sup\left\{ n\geq 0 \ \Big| \ \  \begin{aligned} &\text{there is a chain of primes of length $n$} \\  &\text{in $\Spec(R)$ such that  $\p_i\in \Min(I)$ for some $i$} \end{aligned}\right\}$.
\itema $\hgt(I) + \dim(R/I) \leq \dim(R)$.
\end{enumerate}

\solution{
\begin{enumerate}
\itema Height zero means it can't contain any other primes, because that would be a recipe for a chain of positive length.
\itema Height zero means some minimal prime of it is a minimal prime of $R$. That is the same as being contained in a minimal prime of $R$.
\itema The only minimal prime of a domain is zero; see above.
\itema Primes in $R/\p$ correspond to primes of $R$ containing $\p$.
\itema Primes of $R/I$ correspond to primes of $R$ containing $I$.
\itema If $R$ is a domain and $I\neq 0$, then any prime in $V(I)$ properly contains zero, so a chain in $V(I)$ can be made one longer by throwing in $(0)$ at the bottom.
\itema $(\geq)$ is clear since $V(\p) \subseteq \Spec(R)$. $(\leq)$ follows since any chain of primes in $R$ can be extended to a chain from a minimal prime.
\itema Primes in $R_\p$ correspond to primes of $R$ that are contained in $\p$; thus any chain of primes to a prime contained in $\p$ corresponds to a chain of primes in $R_\p$ and conversely.
\itema $(\geq)$ is clear since $\Lambda(\m) \subseteq \Spec(R)$. $(\leq)$ follows since any chain of primes in $R$ can be extended to a chain to a maximal ideal.
\itema As above, we identify chains of primes in $R/\p$ with chains in $V(\p)$.
For $(\geq)$, given such a chain, break it at $\p$ to get a chain to $\p$ and a chain from $\p$; the first has length at most $\hgt(\p)$ and the second has length at most $\dim(R/\p)$. For $(\leq)$, given a chain of primes to $\p$ and a chain in $V(\p)$, we obtain by concatenation a chain in $R$ whose length is at least the sum of the lengths. 
\itema Clear from the previous.
\itema For $(\leq)$, if $\hgt(I)\geq a$ and $\dim(R/I)\geq b$ , then for every $\p\in \Min(I)$, there is a chain of primes of $\p$ of length at least $a$, and there exists $\p_0 \in \Min(I)$ and a chain of primes from $\p_0$ of length $b$. Concatenating, we get a chain of primes through $\p_0$ of length at least $a+b$. This shows the inequality. 
\itema Clear from the previous.
\end{enumerate}
}


\itemB Dimension vs height
\begin{enumerate}
\itemb Let $K$ be a field and $R=K[X,Y,Z]/(XY,XZ)$. Let $\p=(y,z)$. Compute $\dim(R/\p)$ and $\hgt(\p)$, and show that $\dim(R)\geq 2$. 
\itemb Let $R=\Z_{(2)}[X]$. Let $\p= (2X-1)$. Compute $\dim(R/\p)$ and\footnote{You can use the next problem if you like.} $\hgt(\p)$, and show that $\dim(R)\geq 2$. 
\end{enumerate}

\solution{ 
\begin{enumerate}
\itemb $R/\p\cong K[X]$ so its dimension is $1$. $\p$ is minimal so its height is $0$. But $(x) \subseteq (x,y) \subseteq (x,y,z)$ shows that $\dim(R)\geq 2$.
\itemb $R/\p \cong \Z_{(2)}[1/2] \cong \Q$ so $\dim(R/\p)=0$. $\p$ has height $1$ since $R$ is a UFD; see below. But $R$ has dimension at least $2$ since one has $(0) \subseteq (2) \subseteq (2,X)$. \end{enumerate}
}


\itemB Let $R$ be a domain. Show that $R$ is a UFD if and only if every prime ideal of height one is principal.

\solutione{}

\itemB Does it follow from the definition that in a Noetherian ring, every prime has finite height?

\solution{No, there could be distinct chains that get longer and longer.}

\itemB In this problem we will construct a Noetherian ring of infinite dimension. Let $K$ be a field, \\${S=K[X_{1,1},X_{2,1},X_{2,2},X_{3,1},X_{3,2},X_{3,3},\dots]}$, and $W= S \smallsetminus \bigcup_{t} (X_{t,1},\dots,X_{t,t})$.
\begin{enumerate}
\item Let $A$ be a ring. Suppose that $\Max(A)$ is finite, $A_\m$ is Noetherian for every $\m\in \Max(A)$, and every nonzero element is contained in finitely many maximal ideals. Show that $A$ is Noetherian.
\item Let $\p_t = (X_{t,1},\dots,X_{t,t,})$ for $t\geq 1$. Let $I$ be an ideal. Show that if $I\subseteq \bigcup_{t\geq 1} \p_t$, then there is\footnote{Note that this looks similar to prime avoidance, but with an infinite set of primes. For $f\in S$, let $v(f)\ceq\{ t \ | \ f\in \p_t\}$. Show that for any $f,g\in I$, there is some $h\in I$ with $v(h)\subseteq v(f)\cup v(g)$. Then apply prime avoidance.} some $t\geq 1$ such that $I \subseteq \p_t$.
\item Show that $R\ceq W^{-1}S$ is Noetherian and infinite dimensional.
 \end{enumerate}
  \solution{
  
  \begin{enumerate}
\item Let $I_1\subseteq I_2 \subseteq I_3 \subseteq \cdots$ be an ascending chain of ideals; without loss of generality, $I_1$ is nonzero. By hypothesis, $V_{\max}(I_1)$ is finite, and $V_{\max}(I_i) \supseteq V_{\max}(I_{i+1})$ for every $i$ by definition. A descending chain of finite sets stabilizes, so $X=V_{\max}(I_i)$ stabilizes.  Then for each $\m\in X$, the chain 
\[(I_1)_{\m} \subseteq (I_2)_{\m} \subseteq (I_3)_{\m} \subseteq \cdots\]
stabilizes. In particular, there is some $t$ such that $(I_i)_{\m} = (I_{i+1})_{\m}$ for all $i\geq t$ and all maximal ideals containing $I_{i+1}$. Thus, $\mathrm{Supp}(I_{i+1}/I_i)$ contains no maximal ideals, hence is empty, so $I_i=I_{i+1}$ for all $i\geq t$; i.e., the chain stabilizes.
\item  If $I=0$ this is clear, so suppose $I\neq 0$, that $I\subseteq \bigcup_{i\in \N} \p_i$.
For $s\in S$, set 
\[ v(s)\ceq \{i \ | \ f\in \p_i\}.\]
Since $s$ involves finitely many variables, $v(s)$ is finite for each nonzero $s\in S$. Our hypothesis translates to saying $v(f)$ is nonempty for each $f\in I$.

We claim that for any $f,g\in I$, there is some $h\in I$ with $v(h) \subseteq v(f) \cap v(g)$. Namely, let $k$ be larger than the first index of any variable in $f$ or $g$, and $t$ be an integer greater than the degree of $f$ and set $h=f+x_k^t g$. Then $f$ and $x_k^t g$ have no monomials in common (since the degrees of all the monomials in $x_k^t g$ are at least $t$ and the degree of the monomials in $f$ are all less than $t$) so none can cancel from each other. In particular, if $x_\ell$ divides $h$ in $T$, then $x_\ell$  divides both $f$ and $x_k^t g$ in $T$; i.e., $v(h) \subseteq v(f) \cap v(g)$ as claimed.

Thus, fixing some nonzero $f\in I$, for every $g\in I$, $v(f) \cap v(g)$ is nonempty. That means that every $g\in I$ is in some $\p_i$ for $i  \in v(f)$, so $I \subseteq \bigcup_{i\in v(f)} \p_i$, which is a finite union of primes. By the usual version of prime avoidance, $I \subseteq \p_i$ for some~$i$.
\item Clearly $R$ is infinite dimensional, since for any $n$, there is a chain of primes contained in $\p_n$ of length $n$, which yields a chain of primes of length $n$ in $R$. To see that $R$ is Noetherian, note first that by the previous part, any ideal of $S$ that does not intersect $W$ is contained in some $\p_t$, so every ideal $W^{-1} R$ is contained in some $W^{-1}\p_t$, so these are the maximal ideals of $R$. Now note that any element considered as a fraction has a numerator in at most finitely many $\p_n$. Moreover, localizing at $\p_t$ yields ring isomorphic to a localization of polynomial ring in $t$ variables over a field, which is Noetherian. Thus, by the Lemma, $R$ is Noetherian.
 \end{enumerate}
 }
  
  



\end{enumerate}
\end{document}
