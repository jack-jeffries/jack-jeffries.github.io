\documentclass[12pt]{amsart}


\usepackage{times}
\usepackage[margin=.8in]{geometry}
\usepackage{amsmath,amssymb,multicol,graphicx,framed,ifthen,color,xcolor,stmaryrd,enumitem,colonequals,bbm}
\usepackage[all]{xy}

\usepackage[outline]{contour}
\contourlength{.4pt}
\contournumber{10}
\newcommand{\Bold}[1]{\contour{black}{#1}}


\definecolor{chianti}{rgb}{0.6,0,0}
\definecolor{meretale}{rgb}{0,0,.6}
\definecolor{leaf}{rgb}{0,.35,0}
\newcommand{\Q}{\mathbb{Q}}
\newcommand{\N}{\mathbb{N}}
\newcommand{\Z}{\mathbb{Z}}
\newcommand{\R}{\mathbb{R}}
\newcommand{\C}{\mathbb{C}}
\newcommand{\e}{\varepsilon}
\newcommand{\m}{\mathfrak{m}}
\newcommand{\p}{\mathfrak{p}}
\newcommand{\q}{\mathfrak{q}}
\newcommand{\ord}{\mathrm{ord}}
\newcommand{\ann}{\mathrm{ann}}
\newcommand{\hgt}{\mathrm{height}}
\newcommand{\Min}{\mathrm{Min}}
\newcommand{\Max}{\mathrm{Max}}
\newcommand{\Spec}{\mathrm{Spec}}
\newcommand{\Ass}{\mathrm{Ass}}
\renewcommand{\1}{\mathbbm{1}}
\newcommand{\cZ}{\mathcal{Z}}

\newcommand{\inv}{^{-1}}
\newcommand{\dabs}[1]{\left| #1 \right|}
\newcommand{\ds}{\displaystyle}
\newcommand{\solution}[1]{\ifthenelse {\equal{\displaysol}{1}} {\begin{framed}{\color{meretale}\noindent #1}\end{framed}} { \ }}
\newcommand{\showsol}[1]{\def\displaysol{#1}}
\newcommand{\rsa}{\rightsquigarrow}

\newcommand\itemA{\stepcounter{enumi}\item[{\Bold{(\theenumi)}}]}
\newcommand\itemB{\stepcounter{enumi}\item[(\theenumi)]}
\newcommand\itemC{\stepcounter{enumi}\item[{\it{(\theenumi)}}]}
\newcommand\itema{\stepcounter{enumii}\item[{\Bold{(\theenumii)}}]}
\newcommand\itemb{\stepcounter{enumii}\item[(\theenumii)]}
\newcommand\itemc{\stepcounter{enumii}\item[{\it{(\theenumii)}}]}
\newcommand\itemai{\stepcounter{enumiii}\item[{\Bold{(\theenumiii)}}]}
\newcommand\itembi{\stepcounter{enumiii}\item[(\theenumiii)]}
\newcommand\itemci{\stepcounter{enumiii}\item[{\it{(\theenumiii)}}]}
\newcommand\ceq{\colonequals}

\DeclareMathOperator{\res}{res}
\setlength\parindent{0pt}
%\usepackage{times}

%\addtolength{\textwidth}{100pt}
%\addtolength{\evensidemargin}{-45pt}
%\addtolength{\oddsidemargin}{-60pt}

\pagestyle{empty}
%\begin{document}\begin{itemize}

%\thispagestyle{empty}

\usepackage[hang,flushmargin]{footmisc}


\begin{document}
\showsol{0}
	
	\thispagestyle{empty}
	
	\section*{\S7.32: Noether normalization and dimension}
	
	\begin{framed}
\noindent	\textsc{Theorem:} Let $K$ be a field, and $R$ be a domain that is algebra-finite over $K$. Let $K[f_1,\dots,f_n]$ be a Noether normalization of $R$. Any saturated chain of primes from $0$ to a maximal ideal $\m$ of $R$ has length $n$.

\

\noindent	\textsc{Corollary:} Let $K$ be a field, and $R$ be a finitely generated $K$-algebra. Then
\begin{enumerate}
\item For any primes $\p \subseteq \q$ of $R$, every saturated chain of primes from $\p$ to $\q$ has the same length. (That is, $R$ is \textbf{catenary}).
\item If $R$ is a domain, and $I$ is an arbitrary ideal, then $\dim(R) = \dim(R/I) + \mathrm{height}(I)$.
\end{enumerate}
\end{framed}




\begin{enumerate}
\itemA Consequences of the Theorem: Let $K$ be a field.
\begin{enumerate}
\itema Use the Theorem to deduce that $\dim(K[X_1,\dots,X_n])=n$.
\itema Use the Theorem to deduce that every Noether normalization has the same number of elements.
\itema Use part (a) above to show that the dimension of a $K$-algebra is at most the number of generators in an $K$-algebra generating set.
\itema Use the Theorem to prove part (1) of the Corollary.
\end{enumerate}

\solution{
\begin{enumerate}
\itema Note that $K[X_1,\dots,X_n]$ is a Noether normalization of itself. So, any saturated chain of primes from $0$ to a maximal ideal has length $n$.
\itema Follows because every Noether normalization has cardinality equal to the length of a saturated chain from $0$ to any maximal ideal.
\itema If $R$ is a $K$ algebra with $n$ generators, it is a quotient of $K[X_1,\dots,X_n]$, which has dimension $n$, so $R$ has dimension at most $n$.
\itema Take two saturated chains of primes from $\p$ to $\q$. Let $S=R/\p$, which is a domain. We get  two saturated chains of primes from $0$ to $\q/\p$ in $S$. Fix a maximal ideal of $S$ containing $\q/\p$. By concatenation, we get two saturated chains from $0$ to fixed maximal ideals in $S$, which must have the same length. So the chains from $0$ to $\q/\p$ have the same length, and hence the chains from $\p$ to $\q$ have the same length.
\end{enumerate}
}


\itemA Let $K$ be a field. Use the Theorem and previous computations to compute the dimension of each of the following rings:
\begin{enumerate}
\itema $\ds \frac{K[X,Y,Z]}{(X^3+Y^3+Z^3)}$.
\itema $\ds \frac{K[X,Y]}{(XY)}$.
\itema $\ds K[X^4,X^3 Y, XY^3, Y^4]$.
\end{enumerate}

\solution{
\begin{enumerate}
\itema A Noether normalization is $K[x,y]$, so the dimension is $2$.
\itema A Noether normalization is $K[x+y]$, so the dimension is $1$.
\itema A Noether normalization is $K[X^4,Y^4]$, so the dimension is $2$.
\end{enumerate}
}

\begin{samepage}
\itemA Proof of Theorem: Induce on the number of elements $n$ in a Noether normalization.
\begin{enumerate}
\itema Explain the case $n=0$.
\itema For the general case, let $A=K[z_1,\dots,z_n] \subseteq R$ be a Noether normalization, and take a saturated chain of primes of $R$:
\[ (0) = \p_0 \subsetneqq \p_1 \subsetneqq \cdots  \subsetneqq \p_s = \m.\]
Explain why $\p_1$ has height $1$.
\itema Explain why $\p_1 \cap A$ has height $1$.
\itema Explain why $\p_1 \cap A$ is principal.
\itema Explain why, after a change of coordinates, we can assume that $K[z_1,\dots,z_{n-1}]$ is a Noether normalization of $R/\p_1$.
\itema Finish the proof.
\end{enumerate}
\end{samepage}

\solution{
\begin{enumerate}
\itema If $n=0$, then $R$ is a domain module-finite over a field, so a field. Then there are no chains of primes.
\itema This follows from the definition of saturated.
\itema This is a Corollary of Going Down.
\itema From the homework, every height one prime in a UFD is principal.
\itema Let $\p_1 \cap A = (f)$. After a change of coordinates in the $z_i$'s, we can assume that $f$ is monic in $z_n$. We have $K[z_1,\dots,z_{n}]/(f) \hookrightarrow R/\p_1$, and this is integral since $A \to R$ is. Then $K[z_1,\dots,z_{n-1}] \hookrightarrow K[z_1,\dots,z_{n}]/(f)$ is module-finite, and then the composition is a Noether normalization.
\itema By the induction hypothesis, any saturated chain from $\p_1/\p_1$ to a maximal ideal of $R/\p_1$ has length $n-1$. So $s=n$.
\end{enumerate}


}


\itemB Use the Theorem to prove part (2) of the Corollary.

\

\itemB Let $R=K[X_1,\dots,X_n]$ and $f_{m+1},\dots,f_n$ be polynomials such that $f_{m+1}\in K[X_1,\dots,X_{m+1}]$ is monic in $X_{m+1}$, \dots, $f_n\in K[X_1,\dots,X_n]$ is monic in $X_n$.
 Show that $K[x_1,\dots,x_m]$ is a Noether normalization for $S=R/(f_{m+1},\dots,f_n)$, and deduce that $\dim(S)=m$, and that ${\mathrm{height}(f_{m+1},\dots,f_n)=n-m}$.
 
 \
 
 \itemB Let $K$ be a field, and let $R\subseteq S$ be an inclusion of finitely generated $K$-algebras that are both domains. Show that for any $\q\in \Spec(S)$, $\hgt(\q) = \hgt (\q \cap R)$.
 
 \
 
 \itemB Let $K$ be a field. Show that $K\llbracket X_1,\dots,X_n\rrbracket$ is a domain of dimension $n$.

\end{enumerate}
\end{document}
