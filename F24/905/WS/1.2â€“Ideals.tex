\documentclass[12pt]{amsart}


\usepackage{times}
\usepackage[margin=0.65in]{geometry}
\usepackage{amsmath,amssymb,multicol,graphicx,framed,ifthen,color,xcolor,stmaryrd,enumitem,colonequals}
\definecolor{chianti}{rgb}{0.6,0,0}
\definecolor{meretale}{rgb}{0,0,.6}
\definecolor{leaf}{rgb}{0,.35,0}
\newcommand{\Q}{\mathbb{Q}}
\newcommand{\N}{\mathbb{N}}
\newcommand{\Z}{\mathbb{Z}}
\newcommand{\R}{\mathbb{R}}
\newcommand{\C}{\mathbb{C}}
\newcommand{\e}{\varepsilon}
\newcommand{\m}{\mathfrak{m}}
\newcommand{\p}{\mathfrak{p}}
\newcommand{\ord}{\mathrm{ord}}

\newcommand{\inv}{^{-1}}
\newcommand{\dabs}[1]{\left| #1 \right|}
\newcommand{\ds}{\displaystyle}
\newcommand{\solution}[1]{\ifthenelse {\equal{\displaysol}{1}} {\begin{framed}{\color{meretale}\noindent #1}\end{framed}} { \ }}
\newcommand{\showsol}[1]{\def\displaysol{#1}}
\newcommand{\rsa}{\rightsquigarrow}

\newcommand\itemA{\stepcounter{enumi}\item[{\bf{(\theenumi)}}]}
\newcommand\itemB{\stepcounter{enumi}\item[(\theenumi)]}
\newcommand\itemC{\stepcounter{enumi}\item[{\it{(\theenumi)}}]}
\newcommand\itema{\stepcounter{enumii}\item[{\bf{(\theenumii)}}]}
\newcommand\itemb{\stepcounter{enumii}\item[(\theenumii)]}
\newcommand\itemc{\stepcounter{enumii}\item[{\it{(\theenumii)}}]}
\newcommand\itemai{\stepcounter{enumiii}\item[{\bf{(\theenumiii)}}]}
\newcommand\itembi{\stepcounter{enumiii}\item[(\theenumiii)]}
\newcommand\itemci{\stepcounter{enumiii}\item[{\it{(\theenumiii)}}]}
\newcommand\ceq{\colonequals}

\DeclareMathOperator{\res}{res}
\setlength\parindent{0pt}
%\usepackage{times}

%\addtolength{\textwidth}{100pt}
%\addtolength{\evensidemargin}{-45pt}
%\addtolength{\oddsidemargin}{-60pt}

\pagestyle{empty}
%\begin{document}\begin{itemize}

%\thispagestyle{empty}

\usepackage[hang,flushmargin]{footmisc}


\begin{document}
\showsol{0}
	
	\thispagestyle{empty}
	
	\section*{\S1.2: Ideals}	

\begin{framed}
\textsc{Definition}: Let $S$ be a subset of a ring $R$. The \textbf{ideal generated by $S$}, denoted $(S)$, is the smallest ideal containing $S$. Equivalently,
\[ ( S )  = \left\{ \sum r_i s_i \ | \ r_i\in R, s_i\in S\right\} \quad \text{is the set of $R$-linear combinations\footnotemark\,of elements of $S$}.\]
We say that $S$ \textbf{generates} an ideal $I$ if $(S)=I$.

\


\textsc{Definition}: Let $I, J$ be ideals of a ring $R$. The following are ideals:
\begin{itemize}
\item $IJ\colonequals ( ab \ | \ a\in I, b\in J)$.
\item $I^n \colonequals \underbrace{I \cdot I\cdots I}_{\text{$n$ times}}= ( a_1 \cdots a_n \ | \ a_i\in I )$ for $n\geq 1$.
\item $I+J \colonequals  \{ a + b \ | \ a\in I, b\in J \} = ( I \cup J)$.
\item $rI \colonequals (r)I = \{ ra \ | \ a\in I\}$ for $r\in R$.
\item $I : J \ceq \{ r\in R \ | \ rJ \subseteq I\}$.
\end{itemize}

\

\textsc{Definition}: Let $I$ be an ideal in a ring $R$. The \textbf{radical} of $I$ is $\sqrt{I} \ceq \{ f\in R \ | \ f^n\in I \ \text{for some} \ n\geq1\}$.
An ideal $I$ is \textbf{radical} if $I=\sqrt{I}$.

\

\textsc{Division Algorithm}: Let $A$ be a ring, and $R=A[X]$ be a polynomial ring. Let $g\in R$ be a \textbf{monic} polynomial; i.e., the leading coefficient of $f$ is a unit. Then for any $f\in R$, there exist unique polynomials $q,r\in R$ such that $f=gq+r$ and the top degree of $r$ is less than the top degree of $g$.
 
 \end{framed}
\footnotetext{Linear combinations always means \emph{finite} linear combinations: the axioms of a ring can only make sense of finite sums.}
 
\begin{enumerate}
\itemA Briefly discuss why the two characterizations of $(S)$ in Definition~2.1 are equal.

\solution{The set of linear combinations of elements of $S$ is an ideal: 
\begin{itemize}
\item $0 = 0 s_1$ (we also consider $0$ to be the empty combination);
\item given two linear combinations, by including zero coefficients, we can assume our combinations involve the same elements of $S$, and then $\sum_i a_i s_i + \sum_i b_i s_i = \sum_i(a_i+b_i) s_i$;
\item $r ( \sum_i a_i s_i) = \sum_i r a_i s_i$.
\end{itemize}
Any ideal that contains $S$ must contain all of the linear combinations of $S$, using the definition of ideal. These two facts mean that the set of linear combinations is the smallest ideal containing~$S$.}

\itemA Finding generating sets for ideals: Let $S$ be a subset of a ring $R$, and $I$ an ideal.
\begin{enumerate}
\itema To show that $(S) = I$, which containment do you think is easier to verify? How would you~check?
\itema To show that $(S) = I$ given $(S)\subseteq I$, explain why it suffices to show that $I/(S) = 0$ in $R/(S)$; i.e., that every element of $I$ is equivalent to $0$ modulo $S$.
\itema Let $K$ be a field, $R=K[U,V,W]$ and $S=K[X,Y]$ be polynomial rings. Let $\phi:R\to S$ be the ring homomorphism that is constant on $K$, and maps $U\mapsto X^2, V\mapsto XY, W\mapsto Y^2$. Show that the kernel $\phi$ is generated by $V^2-UW$ as follows:
\begin{itemize}
\item Show that $(V^2-UW)\subseteq \ker(\phi)$.
\item Think of $R$ as $K[U,W][V]$. Given $F\in \ker(\phi)$, use the Division Algorithm to show that ${F\equiv F_1 V + F_0}$ modulo $(V^2-UW)$ for some $F_1, F_0\in K[U,W]$ with $F_1 V + F_0 \in \ker(\phi)$.
\item Use $\phi(F_1 V + F_0)=0$ to show that $F_1 = F_0 = 0$, and conclude that $F \in \ker(\phi)$.
\end{itemize}
\end{enumerate}

\solution{
\begin{enumerate}
\itema Showing $(S)\subseteq I$ is the easier containment: it suffices to show that $S\subseteq I$.
\itema This follows from the Second Isomorphism Theorem.
\itema \begin{itemize}
\item We check $\phi(V^2-UW) = (XY)^2 - X^2 Y^2 = 0$, so $V^2-UW\in \ker(\phi)$. This implies $(V^2-UW) \subseteq \ker(\phi)$.
\item By Division, we have $F = (V^2-UW)Q + R$, with the top degree (in $V$) of $R$ at most $1$. Then $F\equiv R = F_1 V + F_0 $ modulo $(V^2-UW)$. Since $F, V^2-UW \in \ker(\phi)$, we must have $F_1 V + F_0\in \ker(\phi)$.
\item We have $0 = \phi(F_1 V + F_0) = F_1(X^2,Y^2) XY + F_0(X^2,Y^2)$. The $F_1(X^2,Y^2) XY$ terms only have monomials whose $X$-degree is odd, and the $F_0(X^2,Y^2)$ terms only have monomials whose $X$-degree is even, so none can cancel with each other. This means that $F_1(X^2,Y^2)=0$ and $F_0(X^2,Y^2)=0$, so $F_1(U,W)=F_0(U,W)=0$. Thus, $F\equiv 0$ modulo $(V^2-UW)$, and as above, we conclude $\ker(\phi)=(V^2-UW)$.
\end{itemize}
\end{enumerate}}


\itemA Radical ideals:
\begin{enumerate}
\itema Fill in the blanks and convince yourself:
\[ \begin{array}{llll}
\bullet & R/I \ \text{is a field} & \Longleftrightarrow & I \ \text{is \underline{\phantom{RADICAL!!!!!!!!!!!!}}}\\
\bullet & R/I \ \text{is a domain} & \Longleftrightarrow & I \ \text{is \underline{\phantom{RADICAL!!!!!!!!!!!!}}}\\
\bullet & R/I \ \text{is reduced} & \Longleftrightarrow & I \ \text{is \underline{\phantom{RADICAL!!!!!!!!!!!!}}}
\end{array}\]
\itema Show that the radical of an ideal is an ideal.
\itema Show that a prime ideal is radical.
\itema Let $K$ be a field and $R=K[X,Y,Z]$. Find a generating set\footnote{Hint: To show your set generates, you might consider the bottom degree of $F$ considered as a polynomial in $X$ and $Y$.} for $\sqrt{(X^2, XYZ, Y^2)}$.
\end{enumerate}

\solution{
\begin{enumerate}
\itema \vspace{-3mm} \[ \begin{array}{llll}
\bullet & R/I \ \text{is a field} & \Longleftrightarrow & I \ \text{is \underline{maximal}}\\
\bullet & R/I \ \text{is a domain} & \Longleftrightarrow & I \ \text{is \underline{prime}}\\
\bullet & R/I \ \text{is reduced} & \Longleftrightarrow & I \ \text{is \underline{radical}}\\
\end{array}\]
\itema Let $f,g\in \sqrt{I}$. Then there are $m,n\geq 1$ such that $f^m,g^n\in I$. Then 
\[ (f+g)^{m+n-1} = \sum_{i+j=m+n-1} \binom{m+n-1}{i,j} f^{i} g^j,\]
and for each term in the sum either $i\geq m$ or $j\geq n$, so each term is in $I$, hence the whole sum is in $I$.
Now let $r\in R$. Then $(rf)^m = r^m f^m\in I$.
\itema Suppose $I$ is prime. If $x\in \sqrt{I}$, then $x^n\in I$ for some $n$. Then, by the definition of prime, $x\in I$. Thus, $\sqrt{I}=I$.
\itema Since $X^2$ and $Y^2$ are in $(X^2, XYZ, Y^2)$, we have $X,Y\in \sqrt{(X^2, XYZ, Y^2)}$ by definition, so $(X,Y)\subseteq \sqrt{(X^2, XYZ, Y^2)}$. For the other containment, if $F(X,Y,Z)\notin (X,Y)$, consider $F$ as a polynomial in $X,Y$ with coefficients in $K[Z]$; the condition means that the top degree of $F$ is zero, and hence the top degree of $F^n$ is zero for all $n$, so $F\notin \sqrt{(X^2, XYZ, Y^2)}$.
\end{enumerate}}



\itemA Evaluation ideals in polynomial rings: Let $K$ be a field and $R=K[X_1,\dots,X_n]$ be a polynomial ring. Let $\alpha=(\alpha_1,\dots,\alpha_n)\in K^n$.
\begin{enumerate}
\itema Let $\mathrm{ev}_{\alpha}: R \to K$ be the map of evaluation at $\alpha$: $\mathrm{ev}_{\alpha}(f)=f(\alpha_1,\dots,\alpha_n)$, or $f(\alpha)$ for short. Show that $\m_\alpha \colonequals \ker \mathrm{ev}_{\alpha}$ is a maximal ideal and $R/\m_\alpha\cong K$.
\itema Apply division repeatedly to show that $\m_\alpha= (X_1-\alpha_1,\dots,X_n-\alpha_n)$.
\itema For $K=\R$ and $n=1$, find a maximal ideal that is not of this form. Same question with $n=2$.
\itema With $K$ arbitrary again, show that every maximal ideal $\m$ of $R$ for which $R/\m\cong K$ is of the form $\m_\alpha$ for some $\alpha \in K^n$. 
Note: this is \emph{not} a theorem with a fancy German name.
\end{enumerate}

\solution{
\begin{enumerate}
\itema The evaluation map is surjective, since for any $k\in K$, the constant function $k$ maps to $k$. By the First Isomorphism Theorem, $R/\m_\alpha \cong K$, so $\m_\alpha$ is maximal.
\itema We have $\mathrm{ev}_\alpha(X_i - \alpha_i) = \alpha_i - \alpha_i=0$, so $(X_1-\alpha_1,\dots,X_n-\alpha_n) \subseteq \m_\alpha$. Given some $F\in \m_\alpha$, consider $F$ as a polynomial in $X_1$ and apply division by $X_1-\alpha_1$, to get $F\equiv F_1$ modulo $(X_1-\alpha_1,\dots,X_n-\alpha_n)$, for some $F_1$ not involving $X_1$. Continue with $X_2-\alpha_2$, \dots to get the $F$ is equivalent to a constant, which must be zero. This shows that $F \in (X_1-\alpha_1,\dots,X_n-\alpha_n)$, so $\m_\alpha=(X_1-\alpha_1,\dots,X_n-\alpha_n)$.
\itema $(X^2+1)$; $(X^2+1,Y)$.
\itema Let $\phi: R \to R/\m \cong K$ be quotient map followed by the given isomorphism. Set $\alpha_i\ceq \phi(X_i)$. Then $X_i-\alpha_i\in \ker(\phi)$, so $\m_\alpha = (X_1-\alpha_1,\dots,X_n-\alpha_n) \subseteq \ker(\phi)$. Since $\m_\alpha$ is maximal, we must have equality.
\end{enumerate}

}









\itemB Lots of generators:
\begin{enumerate}
\itemb Let $K$ be a field and $R=K[X_1,X_2,\dots]$ be a polynomial ring in countably many variables. Explain\footnote{Hint: You might find it convenient to show that $(f_1,\dots,f_m) \subseteq (X_1,\dots,X_n)$ for some $n$, and then show that~${(X_1,\dots,X_n)\subsetneqq \m}$} why the ideal $\m=(X_1,X_2,\dots)$ cannot be generated by a finite set.
\itemb Show that the ideal $(X^n,X^{n-1} Y, \dots, X Y^{n-1}, Y^n) \subseteq K[X,Y]$ cannot be generated by fewer than $n+1$ generators.

\itemb Let $R=\mathcal{C}([0,1], \R)$ and $\alpha\in(0,1)$. Show that for any element $g\in (f_1,\dots,f_n)\subseteq \m_\alpha$, there is some $\varepsilon>0$ and some $C>0$ such that $|g| < C \max_i\{ |f_i|\}$ on $(\alpha-\varepsilon,\alpha+\varepsilon)$. Use this to show that $\m_\alpha$ cannot be generated by a finite set.
%\itemb Let $K$ be a field and $\displaystyle R \ceq \left\{ \frac{f(X,Y)}{X^n}  \ \Big| \ f(X,Y)\in K[X,Y], \ \text{$Y$ divides $f$ or $n=0$}\right\}$.~Show~that: 
%\begin{enumerate}
%\itembi $R$ is a subring of the field of rational functions $K(X,Y)$.
%\itembi The ideal $\displaystyle \p \ceq \left\{ \frac{f(X,Y)}{X^n}  \in R \ \Big| \ Y \ \text{divides $f$}\right\}$ is not finitely generated.
%\itembi The ideal $\m\ceq (X)$ is a maximal ideal that contains $\p$.
%\end{enumerate}
\end{enumerate}

\solution{\begin{enumerate}

\itemb Suppose $\m=(f_1,\dots,f_m)$. Since each polynomial involves only finitely many variables, only finitely many variables occur in $\{f_1,\dots,f_m\}$, and since each $f_i$ has no constant term, these polynomials are linear combinations of those variables $X_1,\dots,X_n$; i.e., $(f_1,\dots,f_m) \subseteq (X_1,\dots,X_n)$. It suffices to show that $\m \neq (X_1,\dots,X_n)$. To see it, take $X_{n+1}$ and note that $X_{ n+1}= \sum_{i=1}^n g_i X_i$ is impossible, since the monomial $X_{n+1}$ can't occur in any summand of the right hand side.

\item Note that this ideal is the set of all polynomial whose bottom degree is at least $n$. Given a generating set $f_1,\dots,f_m$ for $I$, consider the degree $n$ terms of the polynomials $f_i$. We claim that the degree $n$ terms of $f_1,\dots,f_m$ must span the space of degree $n$ polynomials as a vector space. Indeed, given $h$ of degree $n$, we have $h\in I$, so $h=\sum_i g_i f_i$. But every term of $f_i$ has degree at least $n$, so the only things of degree $n$ on the right hand side come from the degree $n$ piece of $f_i$ and the degree zero piece of $g_i$. This shows the claim. Then the statement is clear, since the degree $n$ terms form an $n+1$ dimensional vector space.

\itemb Let $g=\sum g_i f_i \in (f_1,\dots,f_n)$. By continuity, there is some $\varepsilon>0$ and some $C>0$ such that $|g_i| < C/n$ on $(\alpha-\varepsilon,\alpha+\varepsilon)$, so $|g| < |\sum_i g_i f_i | \leq \sum_i |g_i| |f_i| \leq \sum_i C/n \max_i\{|f_i|\} \leq C \max_i\{|f_i|\}$ on $(\alpha-\varepsilon,\alpha+\varepsilon)$. 

Now, given $f_1,\dots,f_n\in \m_\alpha$, let $g=\sqrt{ \max_i\{ |f_i|\} }$. Then $g$ is continuous and $g(\alpha)=0$, so $g\in \m_\alpha$, but $g/  \max_i\{ |f_i|\} = 1 / g \to \infty$ as $x\to\alpha$, so there is no constant $C>0$ and no interval $(\alpha-\varepsilon,\alpha+\varepsilon)$ on which $|g| < C \max_i\{ |f_i|\}$. Thus, $\m_\alpha$ is not finitely generated.
%\item \begin{enumerate}
%\item 
%\itembi We have $1\in R$. By some combination of cases, we check that $R$ is closed under addition and multiplication.
%\itembi  Suppose that $\p=(g_1,\dots,g_n)$. We can write $g_i = \frac{Y f_i}{X^n}$ for some $n$; we can take the same $n$ by taking the maximum. %We claim that $\frac{Y}{X^{n+1}}\notin (g_1,\dots,g_n)$. Indeed, if so, we would have
%$\frac{Y}{X^{n+1}}= \sum_i (r_i + \frac{Ys_i}{X^m}) \frac{Y f_i}{X^n}  = \sum_i  \frac{Y f_i r_i}{X^n} + \sum_i \frac{Y^2 s_i f_i}{X^{m+n}}) =  \sum_i  \frac{XY f_i r_i}{X^{n+1}} + \sum_i \frac{Y^2 s_i f_i}{X^{m+n}})$.
%But this is impossible, since if the right-hand side can be written with denominator $X^{n+1}$, the numerator must be in the ideal $(XY,Y^2)$.
%\itembi Note that any element of $\p$ is a multiple of $X$, since $Y f/X^n = X Yf/X^{n+1}$. Thus, any element of $R$ can be written as a constant plus a multiple of $X$ in $R$, so $R/(X)\cong K$, and $(X)$ is a maximal ideal containing $\p$.
%\end{enumerate}
\end{enumerate}}

\itemB Evaluation ideals in function rings: Let $R =\mathcal{C}([0,1],\R)$. Let $\alpha\in [0,1]$.
\begin{enumerate}
\itemb Let $\mathrm{ev}_{\alpha}: \mathcal{C}([0,1]) \to \R$ be the map of evaluation at $\alpha$: $\mathrm{ev}_{\alpha}(f)=f(\alpha)$. Show that $\m_\alpha \colonequals \mathrm{ev}_{\alpha}$ is a maximal ideal and $R/\m_\alpha\cong \R$.
\itemb Show that $(x-\alpha) \subseteq \m_\alpha$.
\itemc Show that every maximal ideal $R$ is of the form $\m_\alpha$ for some $\alpha \in [0,1]$. You may want to argue by contradiction: if not, there is an ideal $I$ such that the sets $U_f  \ceq \{ x\in [0,1] \ | \ f(x)\neq 0\}$ for $f\in I$ form an open cover of $[0,1]$. Take a finite subcover $U_{f_1},\dots,U_{f_t}$ and consider $f_1^2 + \cdots + f_t^2$.
\end{enumerate}
\solution{
\begin{enumerate}
\itemb $\mathrm{ev}_{\alpha}: \mathcal{C}([0,1]) \to \R$ is a surjective ring homomorphism, since $\mathrm{ev}_{\alpha}(r)=r$ for any $r\in\R$. Thus, by the First Isomorphism Theorem, $R/\m_\alpha\cong \R$, and hence $\m_\alpha$ is a maximal ideal.
\itemb It suffices to note that $\mathrm{ev}_{\alpha}(x-\alpha) = 0$.
\itemc Argue by contradiction: if not, there is a proper ideal $I$ that is not contained in some $\m_\alpha$; this means that for every $\alpha$, some element of $I$ does not vanish at $\alpha$. Since for any continuous $f$, the set $U_f \colonequals \{ x\in [0,1] \ | \ f(x)\neq 0\}$ is open, the collection $\{ U_f \ | \ f\in I \}$ is an open cover of $[0,1]$. Since $[0,1]$ is compact, there is a finite subcover $U_{f_1},\dots,U_{f_t}$. For these $f_i$'s consider $h=f_1^2 + \cdots + f_t^2$. Each $f_i^2$ is nonnegative, and for any $\alpha$, one of these is strictly positive at $\alpha$. This means that $h(x)\neq 0$ for all $x\in [0,1]$, so $h$ is a unit, and hence $I = R$, a contradiction.
\end{enumerate}}




\itemB Division Algorithm.
\begin{enumerate}
\itemb What fails in the Division Algorithm when $g$ is not monic? Uniqueness? Existence? Both?
\itemb Review the proof of the Division Algorithm.
\end{enumerate}

\solution{}





%\itemB Let $K$ be a field, $R=K[X,Y]$ be a polynomial ring, and $I=(X,Y)$. 
%\begin{enumerate}
%\itemb Show that $I^n = (X^n, X^{n-1} Y, \ldots, Y^n)$.
%\itemc Show that\footnote{Hint: Let $V\subseteq R$ be the vector space of homogeneous polynomials of degree $2$. Show that for $f_1,\dots,f_t\subseteq I$, ${\dim_K( V\cap (f_1,\dots,f_t)) \leq t}$.} any generating set for $I^n$ has at least $n+1$ elements.
%\end{enumerate}
%
%\solution{}

%\itemB Let $K$ be a field, and $S=K[X,Y]$ be a polynomial ring. 
%\begin{enumerate}
%\item  Let $\phi:K[U,V,W]\to S$ be the ring homomorphism that is constant on $K$, and maps ${U\mapsto X^3},$ ${V\mapsto X^2Y}, {W\mapsto Y^3}$.
%Show that $\ker(\phi)=(V^3-U^2W)$.
%\item Let $\phi:K[T,U,V,W]\to S$ be the ring homomorphism that is constant on $K$, and maps ${T\mapsto X^3}$, $U\mapsto X^2Y, V\mapsto XY^2, W\mapsto Y^3$. Show that\footnote{Hint: Use the proposed generators to take a given relation and rewrite as $f(T) U + g(T,V,W)$ and plug back in.} $\ker(\phi)=(U^2-TV, V^2-UW, TW-UV)$.
%\itemc Let $\phi:K[T,U,V,W]\to S$ be the ring homomorphism that is constant on $K$, and maps ${T\mapsto X^4}$, $U\mapsto X^3Y, V\mapsto XY^3, W\mapsto Y^4$. Find a generating set for $\ker(\phi)$.
%\end{enumerate}
%\solution{
%\begin{enumerate}
%\item Let $I$ be this ideal. To show $(V^3-U^2W)\subseteq I$ we check that $V^3-U^2W\in I$ by verifying it is a relation: $(X^2Y)^3 - (X^3)^2(Y^3)=0$, so this holds. For the other containment, we show that $I/(V^3-U^2W)=0$ by showing that any relation reduces to $0$ modulo $(V^3-U^2W)$. Suppose $F\in I$. Thinking of $F$ as a polynomial in $V$ and applying division, we have that $F\equiv f_2(U,W)V^2 + f_1(U,W)V + f_0(U,W)$. Now, plugging in $U=X^3, V=X^2Y, W=Y^3$, we get $f_2(X^3,Y^3) X^4 Y^2 + f_1(X^3,Y^3) X^2 Y + f_0(X^3,Y^3)=0$. Each of the three terms produce distinct monomials: look at the exponents modulo $3$. Thus, each $f_i(X^3,Y^3)$ has to be zero. But this implies that the polynomial $f_i=0$, since $X^3,Y^3$ are algebraically independent. This shows that $I/(V^3-U^2W)=0$, so $I=(V^3-U^2W)$.
%\item We proceed along similar lines. It is easy to check that these are relations. For the other containment, we again show that the quotient is zero. Given a relation $F$, we think of it as a polynomial in $V$. We can use division via $V^2-UW$ to get rid of the $V^{\geq 2}$ terms, and the other relations to rewrite the coefficient of the $V^1$ term as a polynomial in $W$ alone, so $F\equiv f_1(W) V + f_0(T,U,W)$. Then we have $f_1(Y^3) XY^2 + f_0(X^3, X^2Y, Y^3)=0$. The first term only produces $Y^1$-terms, while the second produces only other powers of $Y$, so the two parts must be zero. This implies that $f_1$ is the zero polynomial, and that $f_0$ is a relation on $X^3, X^2Y, Y^3$. But we computed those relations, and (changing the letters to match) $U^3-T^2W=U(U^2-TV)-T(TW-UV)$. It follows that the relation is in the proposed ideal.
%\item $I=(UV-TW,V^3-U^2W,TV^2-U^2W,U^3-T^2V)$, verified by similar games.
%\end{enumerate}}


\itemC Let $K$ be a field and $R=K\llbracket X_1,\dots,X_n\rrbracket$ be a power series ring in $n$ indeterminates. Let \\${R'= K\llbracket X_1,\dots,X_{n-1}\rrbracket}$, so we can also think of $R=R'\llbracket X_{n}\rrbracket$.  In this problem we will prove the useful analogue of division in power series rings:

\vspace{3mm}

\noindent \textsc{Weierstrass Division Theorem:} Let $r\in R$, and write $g= \sum_{i\geq 0} a_i X_n^i$ with $a_i\in R'$. For some $d\geq 0$, suppose that $a_d\in R'$ is a unit, and that $a_i \in R'$ is \emph{not} a unit for all $i<d$. Then, for any $f\in R$, there exist unique $q\in R$ and $r\in R'[X_n]$ such that $f=gq+r$ and the top degree of $r$ as a polynomial in $X_n$ is less than $d$. 

\vspace{3mm}


\begin{enumerate}
\itemc Show the theorem in the very special case $g= X_n^d$.
\itemc Show the theorem in the special case $a_i=0$ for all $i<d$. 
\itemc Show the uniqueness part of the theorem.\footnote{Hint: For an element of $R'$ or of $R$, write $\ord'$ for the order in the $X_1,\dots,X_{n-1}$ variables; that is, the lowest total $X_1,\dots,X_{n-1}$-degree of a nonzero term (not counting $X_n$ in the degree). If $qg+r=0$, write $q=\sum_i b_i X_n^i$. You might find it convenient to pick $i$ such that $\ord'(b_i)$ is minimal, and in case of a tie, choose the smallest such $i$ among these.}
\itemc Show the existence part of the theorem.\footnote{Hint: Write $g_- = \sum_{i=0}^{t-1} a_i X_n^i$ and $g_+ = \sum_{i=t}^\infty a_i X_n^i$. Apply (b) with $g_+$ instead of $g$, to get some $q_0,r_0$; write $f_1=f-(q_0 g + r_0)$, and keep repeating to get a sequence of $q_i$'s and $r_i$'s. Show that $\ord'(q_i), \ord'(r_i) \geq i$, and use this to make sense of $q=\sum_i q_i$ and $r=\sum_i r_i$.}
\end{enumerate}

\solution{
\begin{enumerate}
\item Given $f$, write $f=\sum_{i\geq 0} b_i X_n^i$ with $b_i\in R'$. For existence, just take $r=\sum_{i=0}^{d-1} b_i X_n^i$ and $q=\sum_{i=d}^\infty b_i X_n^{i-d}$. For uniqueness, note that if $f=gq+r=gq'+r'$ with the top degree of $r$ and $r'$ as polynomials in $X_n$ are less than $d$. Then $0=g(q-q') + (r-r')$, so the uniqueness claim reduces to the case $f=0$; we will use this in the other parts without comment. Every term of $r$ has $X_n$-degree less than $d$, whereas every term of $qg$ has $X_n$-degree at least $d$, so no terms can cancel. Thus $qg+r=0$ implies $q=r=0$ (here and henceforth, we assume $r$ is as in the statement when we write $qg+r$).

\item If $a_i=0$ for $i<d$, then $g=X_n^d u$ where $u=\sum_{i\geq 0} a_{i-d} X_n^i$. Since the constant coefficient of $u$ is $a_d$, which is a unit in $R'$, $u$ is a unit in $R$. Thus, we can apply (a) to $f$ and $X_n^d$ to get $f=q_0 X_n^d + r_0 =  (q_0 u^{-1}) g +r_0$; thus, $q=q_0 u^{-1}$ and $r=r_0$ satisfy the existence clause of the theorem. For uniqueness, if $f= q'g+ r'$, then $f = q' u X_n^d + r'$, so by the uniqueness part of (a), we must have $q' u = q_0$ and $r'=r_0$, and thus $q' = q$ and $r'=r$.

\item For an element of $R'$ or of $R$, write $\ord'$ for the order in the $X_1,\dots,X_{n-1}$ variables; that is, the lowest total $X_1,\dots,X_{n-1}$-degree of a nonzero term (not counting $X_n$ in the degree). Suppose that $qg+r=0$, and write $q=\sum_i b_i X_n^i$. Suppose that $q$ is nonzero, so $b_i\neq 0$ for some $i$. Pick $i$ such that $\ord'(b_i) \leq \ord'(b_j)$ for all $j$ with $b_j\neq 0$, and $\ord'(b_i)=\ord'(b_j)$ implies $i<j$; we can do this by well ordering of $\N$. Say $\ord'(b_i)=t$. Consider the coefficient of $X_n^{d+i}$ in $0=qg+r$. Byt he degree constraint on $r$, this is the same as the coefficient of $X_n^{d+i}$ in $qg$. Multiplying out, this is $\sum_{j=0}^{d+i} a_{d+i-j} b_{j}$. For $j=i$, the order of $a_{d} b_{i}$ is $t$. For $j<i$, we have $\ord'(a_{d+i-j} b_{j}) \geq \ord'(b_j) > t$ by choice of $i$. For $j>i$, since $\ord'(a_{d+i-j})>0$ and \ord'(b_j)\geq t$, we have $\ord'(a_{d+i-j} b_{j})>t$. Thus, the no term can cancel the $a_d b_i$ term, so $qg+r\neq 0$. On the other hand, if $q=0$ and $r\neq 0$, clearly $qg+r\neq 0$. It follows there there are unique $q,r$ such that $qg+r=0$.

\item First, we observe that in the context of (b), if $\ord'(f)=t$, then $\ord'(q), \ord'(r) \geq t$. This is clear in the setting of (a), and following the proof of (b), we just need to observe that if $u$ is a unit in $R$, then $\ord'(q_0 u^{-1}) \geq \ord'(q_0)$, which is clear since any coefficient of the product $q_0 u^{-1}$ is a sum of multiples of the coefficients of $q_0$.

Now we begin the main proof. Write $g_- = \sum_{i=0}^{t-1} a_i X_n^i$ and $g_+ = \sum_{i=t}^\infty a_i X_n^i$. Apply (b) with $g_+$ to write $f=q_0 g_+ + r_0$, and set $f_1= f- (q_0 g + r_0) = -q_0 g_-$. Repeat with $f_1$ to write $f_1= q_1 g_+ + r_1$, and $f_2= f_1 - (q_1 g + r_1) = -q_1 g_-$. Continue like so to obtain a sequence of series $q_0,q_1,\dots$ and $r_0,r_1,\dots$. From the observation above, we have that $\ord'(q_i), \ord'(r_i)\geq \ord'(f_i) \geq \ord'(q_{i_1}) +1$, since the constant term of each coefficient of $g_-$ vanishes. It follows that $\ord'(q_i), \ord'(r_i) \geq i$ for each $i$.

For a series $h$, write $[h]_i$ for the degree $i$ part of $h$, and $[h]_{{\leq i}}$ for the sum of all parts of degree $\leq i$. Define $q$ to be the series such that $[q]_i = \sum_{j=0}^i [q_j]_i$, and likewise with $r$. Note that $r$ is a still a polynomial  in $X_n$ of top degree less than $d$. We claim that $f=qg+r$. To show this, it suffices to show that $[f]_i=[qg+r]_i$. Note that to compute $[qg+r]_i$, we can replace $q,g,r$ by $[q]_{\leq i}$, and similarly for the others. But $[q]_{\leq i}= [ \sum_{j=0}^i q_j ]_{\leq i}$ (and likewise with $r$), so $[qg+r]_i = [ (\sum_{j=0}^i q_j) g + (\sum_{j=0}^i r_j)]_i$. Then, by construction of the sequences $\{q_i\},\{r_i\}, \{f_i\}$, we have $[ f- (qg+r) ]_i = [f_{i+1}]_i$ and since $\ord'(f_{i+1}) \geq i+1$, we have $[f_{i+1}]_i=0$. It follows that $f- (qg+r)=0$; i.e., $f=qg+r$.
\end{enumerate}
}







\end{enumerate}


\vfill





\end{document}
