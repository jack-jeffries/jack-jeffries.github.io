\documentclass[12pt]{amsart}


\usepackage{times}
\usepackage[margin=0.7in]{geometry}
\usepackage{amsmath,amssymb,multicol,graphicx,framed,ifthen,color,xcolor,stmaryrd,enumitem,colonequals,bbm}
\usepackage[outline]{contour}
\contourlength{.5pt}
\contournumber{10}
\newcommand{\Bold}[1]{\contour{black}{#1}}

\definecolor{chianti}{rgb}{0.6,0,0}
\definecolor{meretale}{rgb}{0,0,.6}
\definecolor{leaf}{rgb}{0,.35,0}
\newcommand{\Q}{\mathbb{Q}}
\newcommand{\N}{\mathbb{N}}
\newcommand{\Z}{\mathbb{Z}}
\newcommand{\R}{\mathbb{R}}
\newcommand{\C}{\mathbb{C}}
\newcommand{\e}{\varepsilon}
\newcommand{\m}{\mathfrak{m}}
\newcommand{\p}{\mathfrak{p}}
\newcommand{\ord}{\mathrm{ord}}
\newcommand{\1}{\mathbbm{1}}

\newcommand{\inv}{^{-1}}
\newcommand{\dabs}[1]{\left| #1 \right|}
\newcommand{\ds}{\displaystyle}
\newcommand{\solution}[1]{\ifthenelse {\equal{\displaysol}{1}} {\begin{framed}{\color{meretale}\noindent #1}\end{framed}} { \ }}
\newcommand{\showsol}[1]{\def\displaysol{#1}}
\newcommand{\rsa}{\rightsquigarrow}

\newcommand\itemA{\stepcounter{enumi}\item[{\Bold{(\theenumi)}}]}
\newcommand\itemB{\stepcounter{enumi}\item[(\theenumi)]}
\newcommand\itemC{\stepcounter{enumi}\item[{\it{(\theenumi)}}]}
\newcommand\itema{\stepcounter{enumii}\item[{\Bold{(\theenumii)}}]}
\newcommand\itemb{\stepcounter{enumii}\item[(\theenumii)]}
\newcommand\itemc{\stepcounter{enumii}\item[{\it{(\theenumii)}}]}
\newcommand\itemai{\stepcounter{enumiii}\item[{\Bold{(\theenumiii)}}]}
\newcommand\itembi{\stepcounter{enumiii}\item[(\theenumiii)]}
\newcommand\itemci{\stepcounter{enumiii}\item[{\it{(\theenumiii)}}]}
\newcommand\ceq{\colonequals}

\DeclareMathOperator{\res}{res}
\setlength\parindent{0pt}
%\usepackage{times}

%\addtolength{\textwidth}{100pt}
%\addtolength{\evensidemargin}{-45pt}
%\addtolength{\oddsidemargin}{-60pt}

\pagestyle{empty}
%\begin{document}\begin{itemize}

%\thispagestyle{empty}

\usepackage[hang,flushmargin]{footmisc}


\begin{document}
\showsol{0}
	
	\thispagestyle{empty}
	
	\section*{\S3.12: Graded Modules}	

\begin{framed}

\noindent \textsc{Definition:} Let $R$ be an $\N$-graded ring with graded pieces $R_i$. A \textbf{$\Z$-grading} on an $R$-module $M$ is
\begin{itemize}
\item a decomposition of $M$ as additive groups $M= \bigoplus_{e\in \Z} M_e$
\item such that $r\in R_d$ and $m\in M_e$ implies $rm\in M_{d+e}$. 
\end{itemize}
An \textbf{$\Z$-graded module} is a module with a $\Z$-grading. As with rings, we have the notions of \textbf{homogeneous} elements of $M$, the \textbf{degree} of a homogeneous element, \textbf{homogeneous decomposition} of an arbitrary element of $M$. A homomorphism $\phi:M\to N$ between graded modules is \textbf{degree-preserving} if $\phi(M_e) \subseteq N_e$.

\


\noindent \textsc{Graded NAK 1:} Let $R$ be an $\N$-graded ring, and $R_+$ be the ideal generated by the homogeneous elements of positive degree. Let $M$ be a $\Z$-graded module. Suppose that $M_{\ll 0}=0$; that is, there is some $n\in \Z$ such that $M_t=0$ for $t\leq n$. Then $M= R_+ M$ implies $M=0$.

\

\noindent \textsc{Graded NAK 2:} Let $R$ be an $\N$-graded ring and $M$ be a $\Z$-graded module with  $M_{\ll 0}=0$. Let $N$ be a graded submodule of $M$. Then $M=N+R_+ M$ if and only if $M=N$.


\

\noindent \textsc{Graded NAK 3:} Let $R$ be an $\N$-graded ring and $M$ be a $\Z$-graded module with $M_{\ll 0}=0$. Then a set of homogeneous elements $S\subseteq M$ generates $M$ if and only if the image of $S$ in $M/R_+ M$ generates  $M/R_+M$ as a module over $R_0 \cong R/R_+$.

\

\noindent \textsc{Definition:} Let $R$ be an $\N$-graded ring with $R_0=K$ a field. Let $M$ be a a $\Z$-graded module with $M_{\ll 0}=0$. A set $S$ of homogeneous elements of $M$ is a \textbf{minimal generating set} for $M$ if the image of $S$ in $M/R_+ M$ is an $K$-vector space basis.

 \end{framed}
 


\begin{enumerate}

\itemA Warmup with minimal generating sets.
\begin{enumerate}
\itema Note that the definition of ``minimal generating set'' does not say that it is a generating set. Use Graded NAK 3 to explain why it is!
\itema Let $K$ be a field and $S=K[X,Y]$. Verify that $\{X^2,XY,Y^2\}$ is a minimal generating set of the ideal $I$ it generates in $S$.
\itema Let $K$ be a field. Find a minimal generating set of $S=K[X,Y]$ as a module over the \mbox{$K$-subalgebra} $R=K[X+Y,XY]$.
\end{enumerate}
 
 \solution{
 \begin{enumerate}
\itema A basis is a generating set; it is then the $(\Leftarrow)$ of Graded NAK 3.
\itema We need to show that the images of $X^2,XY,Y^2$ form a basis for $I/R_+ I$; write lowercase for images in this quotient. To see that they span, take $F\in I$, so $F=A X^2 + B XY + C Y^2$ for $A,B,C\in R$; then going modulo $R_+$ we have $f = a x^2 + b xy + c y^2$, so $x^2,xy,y^2$ span the quotient. For linear independence, $a x^2 + b xy + c y^2 = 0$ implies $A X^2 + B XY + C Y^2 \in R_+ I$, and by comparing degrees, $A,B,C$ have bottom degree one, hence are in $R_+$, so $a,b,c=0$.

Alternatively, note that $I$ consists of all polynomials of bottom degree at least two, and $R_+ I$ consists of all polynomials of bottom degree at least three. Then the quotient is isomorphic as a vector space to the collection of polynomials of degree two, and $X^2,XY,Y^2$ is indeed a basis.
\itema We compute $S/R_+ S = K[X,Y]/(X+Y,XY) \cong K[Y]/(-Y^2) \cong K[Y]/(Y^2)$, so the classes of $1,Y$ generate. Thus $\{1,Y\}$ forms a minimal generating set.
\end{enumerate}}

\itemA Proofs of graded NAKs:
\begin{enumerate}
\itema Prove Graded NAK 1.
\itema Use Graded NAK 1 to prove Graded NAK 2. 
\itema Use Graded NAK 2 to prove Graded NAK 3.
\end{enumerate}

\solution{
\begin{enumerate}
\itema Suppose that $M\neq 0$. Take a nonzero homogeneous element $m$ of minimal degree $d$ in $M$, which exists by the hypothesis. Then since $m\in R_+ M$, we can write $r=\sum_i r_i m_i$ with $r_i\in R_+$, so the bottom degree of $r_i$ is at least one. Thus, we can take the top degree of $m_i$ to be $<d$. But then each $m_i=0$, so $m=0$, a contradiction.
\itema The $(\Leftarrow)$ direction is clear. For the other, we can apply Graded NAK 1 to $M/N$ since it is graded and its degrees are bounded below. We have $\frac{M}{N}= \frac{N+R_+ M}{N} = R_+ \frac{M}{N}$ so $M/N=0$; i.e., $M=N$.
\itema Apply Graded NAK 2 to the submodule $N=\sum_{s\in S} R s$: to do so, we need to note that a submodule generated by homogeneous elements is a graded submodule, which follows along similar lines to the corresponding statement we showed for ideals. 
\end{enumerate}


}

\itemA The hypotheses:
\begin{enumerate}
\itema Examine your proofs from the previous problem and verify that one direction (each) of Graded NAK 2 and Graded NAK 3 hold without assuming that $R$ or $M$ is graded.
\itema Let $K$ be a field and $R=K[X]$ with the standard grading. Let $M=K[X]/(X-1)$. Analyze the hypotheses and conclusion of Graded NAK 1 for this example.
\itema Let $K$ be a field and $R=K[X]$ with the standard grading. Let $M=K[X,X^{-1}]$. Analyze the hypotheses and conclusion of Graded NAK 1 for this example. 
\itema Find counterexamples to Graded NAK 3 with $M$ is not graded or not bounded below in degree.
\end{enumerate}

\solution{\begin{enumerate}
\itema The $(\Leftarrow)$ direction of Graded NAK 2 and the $(\Rightarrow)$ direction of Graded NAK 3 hold without assuming that $R$ or $M$ is graded.
\itema $M$ is not a graded module; any element is of the form $\overline{\lambda}$ for $\lambda\in K$; if such an element was homogeneous, then \[ \deg(\overline{\lambda}) =   \deg(\overline{X\lambda})= \deg(X) + \deg(\overline{\lambda}) = 1 + \deg(\overline{\lambda}),\]
a contradiction. We also have $M= (X) M = R_+ M$.
\itema $M$ is graded, but not bounded below.  We also have $M= (X) M = R_+ M$.
\itema For a cheap example, take either of the previous with $S=\varnothing$.
\end{enumerate}}

\vfill

\begin{samepage}
\itemB Minimal generating sets: Let $R$ be an $\N$-graded ring with $R_0=K$ a field. Let $M$ be a a $\Z$-graded module with $M_{\ll 0}=0$.
\begin{enumerate}
\itemb Explain why every minimal generating set for $M$ has the same cardinality.
\itemb Explain why every homogeneous generating set for $M$ contains a minimal generating set for $M$. Moreover, explain why any generating set (homogeneous or not) has cardinality at least that of a minimal generating set.
\itemb Explain why ``minimal generating set'' is equivalent to ``homogeneous generating set such that no proper subset generates''.
\itemb Give an example of a finitely generated module $N$ over $K[X,Y]$ and two generating set $S_1,S_2$ for $N$ such that no proper subset of $S_i$ generates $N$, but $|S_1| \neq |S_2|$. Compare to the statements above.
\end{enumerate}
\end{samepage}

\solution{
\begin{enumerate}
\itemb Because all bases of a vector space do.
\itemb If $S$ is a homogeneous generating set for $M$, then the images span $M/R_+ M$, so the images must contain a basis; the elements of $S$ that map to a basis form a minimal generating set. For a general generating set, its images still contain a basis of $M/R_+ M$.
\itemb This just follows from the fact that a basis of a vector space is the same as a minimal spanning set.
\itemb One could take the two generating sets of the ideal $I=((X-1)Y,XY) = (Y)$.
\end{enumerate}}


\itemB  Let $R$ be an $\N$-graded ring with $R_0=K$ a field. Suppose that $R_{\mathrm{red}} = R/ \sqrt{0}$ is a domain, and that $f\in R$ is a homogeneous nonnilpotent element of positive degree. Show that $R/(f)$ is reduced implies that $R$ is a reduced, and hence a domain.


\solution{
\item Let $r\in \sqrt{0}$ be a homogeneous nilpotent element. Then for some $e\in \N$ we have $r^e=0 \in (f)$, and since $R/(f)$ is reduced, $r\in (f)$. Thus, we can write $r=fs$ for some homogeneous $s$. But $r\in \sqrt{0}$, $f\notin \sqrt{0}$, and $\sqrt{0}$ prime implies that $s\in \sqrt{0}$. This implies that $\sqrt{0} = f \sqrt{0} \subseteq R_+ \sqrt{0}$, so $\sqrt{0}=0$; i.e., $R$ is reduced.
}


\end{enumerate}


\vfill





\end{document}
