\documentclass[12pt]{amsart}


\usepackage{times}
\usepackage[margin=0.8in]{geometry}
\usepackage{amsmath,amssymb,multicol,graphicx,framed,ifthen,color,xcolor,stmaryrd,enumitem,colonequals}
\usepackage[outline]{contour}
\contourlength{.4pt}
\contournumber{10}
\newcommand{\Bold}[1]{\contour{black}{#1}}


\definecolor{chianti}{rgb}{0.6,0,0}
\definecolor{meretale}{rgb}{0,0,.6}
\definecolor{leaf}{rgb}{0,.35,0}
\newcommand{\Q}{\mathbb{Q}}
\newcommand{\N}{\mathbb{N}}
\newcommand{\Z}{\mathbb{Z}}
\newcommand{\R}{\mathbb{R}}
\newcommand{\C}{\mathbb{C}}
\newcommand{\e}{\varepsilon}
\newcommand{\m}{\mathfrak{m}}
\newcommand{\p}{\mathfrak{p}}
\newcommand{\ord}{\mathrm{ord}}
\newcommand{\tr}{\mathrm{tr}}
\newcommand{\im}{\mathrm{im}}

\newcommand{\inv}{^{-1}}
\newcommand{\dabs}[1]{\left| #1 \right|}
\newcommand{\ds}{\displaystyle}
\newcommand{\solution}[1]{\ifthenelse {\equal{\displaysol}{1}} {\begin{framed}{\color{meretale}\noindent #1}\end{framed}} { \ }}
\newcommand{\showsol}[1]{\def\displaysol{#1}}
\newcommand{\rsa}{\rightsquigarrow}

\newcommand\itemA{\stepcounter{enumi}\item[{\Bold{(\theenumi)}}]}
\newcommand\itemB{\stepcounter{enumi}\item[(\theenumi)]}
\newcommand\itemC{\stepcounter{enumi}\item[{\it{(\theenumi)}}]}
\newcommand\itema{\stepcounter{enumii}\item[{\Bold{(\theenumii)}}]}
\newcommand\itemb{\stepcounter{enumii}\item[(\theenumii)]}
\newcommand\itemc{\stepcounter{enumii}\item[{\it{(\theenumii)}}]}
\newcommand\itemai{\stepcounter{enumiii}\item[{\Bold{(\theenumiii)}}]}
\newcommand\itembi{\stepcounter{enumiii}\item[(\theenumiii)]}
\newcommand\itemci{\stepcounter{enumiii}\item[{\it{(\theenumiii)}}]}
\newcommand\ceq{\colonequals}

\DeclareMathOperator{\res}{res}
\setlength\parindent{0pt}
%\usepackage{times}

%\addtolength{\textwidth}{100pt}
%\addtolength{\evensidemargin}{-45pt}
%\addtolength{\oddsidemargin}{-60pt}

\pagestyle{empty}
%\begin{document}\begin{itemize}

%\thispagestyle{empty}

\usepackage[hang,flushmargin]{footmisc}


\begin{document}
\showsol{0}
	
	\thispagestyle{empty}
	
	\section*{\S1.4: Modules}	

\begin{framed}
\noindent \textsc{Example:} For a ring $R$, the following are sources of modules:
\begin{enumerate}
\item The free module of $n$-tuples $R^n$, or more generally, for a set $\Lambda$, the free module
\[ R^{\oplus \Lambda} = \{ (r_\lambda)_{\lambda\in\Lambda} \ | \ r_\lambda\neq 0 \ \text{for at most finitely many} \ \lambda\in\Lambda\}.\]
\item Every ideal $I\subseteq R$ is a submodule of $R$.
\item Every quotient ring $R/I$ is a quotient module of $R$.
\item If $S$ is an $R$-algebra, (i.e., there is a ring homomorphism $\alpha: R\to S$), then $S$ is an $R$-module by \textbf{restriction of scalars}:
$r \cdot s \ceq \alpha(r) s$.
\item More generally, if $S$ is an $R$-algebra and $M$ is an $S$-module, then $M$ is also an $R$-module by \textbf{restriction of scalars}: $r \cdot m \ceq \alpha(r) \cdot m$.
\item Given an $R$-module $M$ and $m_1,\dots,m_n\in M$, the \textbf{module of $R$-linear relations} on $m_1,\dots,m_n$ is the set of $n$-tuples $[r_1,\dots,r_n]^\tr \in R^n$ such that $\sum_i r_i m_i=0$ in $R$.
\end{enumerate}
 
 \
 
 \noindent \textsc{Definition:} Let $M$ be an $R$-module. Let $S$ be a subset of $M$. The \textbf{submodule generated by $S$}, denoted\footnotemark \, $\sum_{m\in S} Rm$, is the smallest $R$-submodule of $M$ containing $S$. Equivalently, 
\[ \sum_{m\in S} Rm = \big\{ \sum r_i m_i \ | \ r_i \in R, m_i \in S\big\} \quad \text{is the set of $R$-linear combinations of elements of $S$.}\]
We say that $S$ \textbf{generates} $M$ if $M=\sum_{m\in S} Rm$.

\

\noindent \textsc{Definition:} A\footnotemark\ \textbf{presentation} of an $R$-algebra $M$ consists of a set of generators $m_1,\dots,m_n$ of $M$ as an $R$-module and a set of generators $v_1,\dots, v_m\in R^n$ for the submodule of $R$-linear relations on $m_1,\dots,m_n$. We call the $n \times m$ matrix with columns $v_1,\dots, v_m$ a \textbf{presentation matrix} for $M$. 

\

\noindent \textsc{Lemma:}\mbox{ If $M$ is an $R$-module, and $A$ an $n\times m$ presentation matrix\footnotemark\, for $M$,~then~${M\cong R^n / \im(A)}$.} We call the module $R^n / \im(A)$ the \textbf{cokernel} of the matrix $A$.


 \end{framed}

 %\footnotetext{One immediate advantage of thinking about modules is that we have a common framework to compare $I$ and $R/I$.}
 \footnotetext[1]{{If $S=\{m\}$ is a singleton, we just write $Rm$, and if $S=\{m_1,\dots,m_n\}$, we may write $\sum_i R m_i$.}}
 \footnotetext[2]{As written, there is a finite set of generators, and a finite set of generators for their relations. This is called a \textbf{finite presentation}.  One could do the same thing with an infinite generating set and/or infinite generating set for the relations.}
  \footnotetext{$\im(A)$ denotes the \textbf{image} or column space of $A$ in $R^n$. This is equal to the module generated by the columns of $A$.}

 
\begin{enumerate}
\itemA Let $M$ be an $R$-module and $m_1,\dots,m_n\in M$.
\begin{enumerate}
\itema Briefly explain why the characterizations of the submodule generated by $S$ are equivalent.
\itema Briefly explain why $\sum_i Rm_i$ is the image of the $R$-module homomorphism ${\beta: R^n \to M}$ such\footnote{where $e_i$ is the vector with $i$th entry one and all other entries zero.}
 that $\beta(e_i) = m_i$.
% \itema Discuss the following: $\sum_i Rm_i$ is the set of elements of $M$ that can be written as $R$-linear combinations of the elements $m_i$.
 \itema Let $I$ be an ideal of $R$. How does a generating set of $I$ as an ideal compare to a generating set of $I$ as an $R$-module?
\itema Explain why the Lemma above is true.
%\itema Briefly explain why if $S$ generates $M$ and $M/N$ is a quotient module of $M$, then the image of $S$ in $M/N$ generates $M/N$.
\itema If $M$ has an $a\times b$ presentation matrix $A$, how many generators and how many (generating) relations are in the presentation corresponding to $A$?
\itema What is a presentation matrix for a free module?
\end{enumerate}

\solution{
\begin{enumerate}
\itema $(\subseteq):$ The elements of the form $\sum r_i m_i$ form a submodule of $M$ that contains $S$.
$(\supseteq):$ A submodule that contains $S$ must also contain the elements of the form $\sum r_i m_i$.
\itema This is just unpackaging $\im(\beta)$: $\beta( (r_1,\dots,r_n) ) = \beta( \sum_i r_i e_i) = \sum_i r_i m_i$.
\itema They are the same.
\itema Follows from (b) and First Isomorphism Theorem.
%\itema Given an element $[m]$ of $M/N$, take a representative $m$ and write it $m=\sum r_i s_i$; then $[m]=\sum r_i [s_i]$. 
\itema There are $a$ generators and $b$ relations.
\itema A matrix is free if and only if it has zero presentation matrix.
\end{enumerate}
}

\

\itemA Describe $\Z[\sqrt{2}]$ as a $\Z$-module.

\solution{$Z[\sqrt{2}]$ is a free $\Z$-module with basis $1,\sqrt{2}$.}

\begin{samepage}
\itemA Module structure for polynomial rings and quotients:
\begin{enumerate}
\itema Let $R=A[X]$ be a polynomial ring. Give a generating set for $R$ as an $A$-module. Is $R$ a free $A$-module?
\itema Let $R=A[X,Y]$ be a polynomial ring. Give a generating set for $R$ as an $A$-module. Is $R$ a free $A$-module?
\itema Let $R=A[X] / (f)$, where $f$ is a monic polynomial of top degree $d$. Apply the Division Algorithm to show that $R$ is a free $A$-module with basis $[1],[X],\dots,[X^{d-1}]$.
\itema Let $R=\C[X,Y]/(Y^3- i X Y + 7 X^4)$. Describe $R$ as a $\C[X]$-module, and then give a $\C$-vector space basis.
\end{enumerate}
\end{samepage}
\solution{
\begin{enumerate}
\itema $R$ is free on basis $1, X, X^2, \dots$.
\itema $R$ is free on basis $1, X, X^2, \dots, Y, XY, XY^2, \dots, Y^2, XY^2, X^2Y^2, \dots \dots$.
\itema We need to show that any $[g] \in R$ has a unique expression as an $A$-linear combination of $[1],\dots,[X^{d-1}]$. Given $[g]$, take a represenatative $g$; use the division algorithm to write $g=qf+r$ with top deg $r$ < $d$. Thus $[g] = [r]$, and since $r\in A 1 + A X + \cdots + A X^{d-1}$, $[g]=[r] \in A[1] + \cdots+ A[X^{d-1}]$. For uniqueness, it suffices to show linear independence of $[1],\dots,[X^{d-1}]$; a nontrivial relation would yield a multiple of $f$ in $A[X]$ of degree less than $d$, which cannot happen.
\itema $R$ is free over $\C[X]$ on $[1], [Y], [Y^2]$. It has as a vector space basis ${\{ [X^i Y^j] \ | \ i\geq 0, j\in\{0,1,2\}.\}}$.
\end{enumerate}}

\itemA Let $R=\C[X]$ and $S=\C[X,X^{-1}] \subseteq \C(X)$. Find a generating set for $S$ as an $R$-module. Does there exist a finite generating set for $S$ as an $R$-module? Is $S$ a free $R$-module?

\solution{$S$ is generated by $\{ 1/X^n \ | \ n\geq 0\}$. $S$ cannot be generated by a finite set: if ${S=R f_1 + \cdots + Rf_n}$, among $f_1,\dots,f_n$ there is a largest power of $X$ in the denominator, say $m$. Then $S\subseteq R \frac{1}{X^m}$, but $\frac{1}{X^{m+1}}\in S \smallsetminus R \frac{1}{X^m}$. $S$ is not free: if it were, there would be a basis element $s$, and $s\notin xS$, as this would lead to a nontrivial relation with other basis elements, but $S=xS$, so this is impossible.}


\itemB Presentations of modules: Let $K$ be a field, and $R=K[X,Y]$ be a polynomial ring.
\begin{enumerate}
\itemb Consider the quotient ring $K\cong R/(X,Y)$ as an $R$-module. Find a presentation for $K$ as an $R$-module.
\itemb Consider the ideal $I=(X,Y)$ as an $R$-module. Find a presentation for $I$ as an $R$-module.
\itemb Consider the ideal $J=(X^2,XY,Y^2)$ as an $R$-module. Find a presentation for $J$ as an $R$-module.
\end{enumerate}

\solution{
\begin{enumerate}
\itemb $[1]$ generates $K$, and $X,Y$ are the defining relations. So, a presentation matrix is $[X, Y]$.
\itemb A generating set is $\{X,Y\}$. To find the relations, suppose that $fX+gY=0$. Then $fX=-gY$. Writing out $f,-g$ in terms of monomials, one sees that $-g$ must be a multiple of $X$ and $f$ must be a multiple of $Y$ so $f=hY$, $-g=jX$. Then $hXY=jXY$, so $j=h$. Thus, the relation $\begin{bmatrix} f \\ g\end{bmatrix}$ can be written as $h \begin{bmatrix} Y \\ -X \end{bmatrix}$. A defining relation (and hence the presentation matrix) is $ \begin{bmatrix} Y \\ -X \end{bmatrix}$.
\itemb A generating set is $\{X^2,XY,Y^2\}$. We have relations $\begin{bmatrix} Y \\ -X \\ 0\end{bmatrix}$ and $\begin{bmatrix} 0 \\ Y \\ -X \end{bmatrix}$ corresponding to $Y (X^2) - X (XY) = 0$ and $Y (XY) - X (Y^2) = 0$. We claim that these generate. Suppose that $a X^2 + b XY + C Y^2=0$; we want to show that  $\begin{bmatrix} a \\ b \\ c\end{bmatrix} \in \mathrm{im}  \begin{bmatrix} Y & 0 \\ -X & Y\\ 0& -X\end{bmatrix}$. We can write $a= a' Y + a''$ with $a''\in K[X]$ and subtracting $a' \begin{bmatrix} Y \\ -X \\ 0\end{bmatrix}$, we obtain a relation with $a\in K[X]$; similarly, we can assume $c\in K[Y]$. Then plugging in $a(X) X^2 + b(X,Y) XY + c(Y) Y^2$, since each sum has no possible monomials in common, we must have $a=b=c=0$. This shows the claim.
\end{enumerate}}




\itemB Let $M$ be an $R$-module, $S\subseteq M$ a generating set, and $r\in R$. Show that $rM=0$ if and only if $rm=0$ for all $m\in S$.

\solution{The forward direction is clear. For the other, writing $m=\sum_i r_i m_i$ with $m_i\in S$, if $rm_i=0$, then $rm=0$.}

\itemB Let $K$ be a field, $S=K[X,Y]$ be a polynomial ring, and $R=K[X^2,XY,Y^2] \subseteq S$. Find an \mbox{$R$-module} $M$ such that $S=R \oplus M$ as $R$-modules. Given a presentations for $S$ and $M$ as \mbox{$R$-modules}.

\solution{We can take $M$ to be the collection of polynomials all of whose terms have odd degree. Note that $M$ is indeed closed under multiplication by $R$. A presentation matrix for $M$ is $\begin{bmatrix}  XY & Y^2 \\ -X^2 & -XY \end{bmatrix}$ and for $S$ is $\begin{bmatrix}  0 & 0 \\ XY & Y^2 \\ -X^2 & -XY \end{bmatrix}$.}

\itemB Messing with presentation matrices: Let $M$ be a module with an $n\times m$ presentation matrix $A$.
\begin{enumerate}
\itemb If you add a column of zeroes to $A$, how does $M$ change?
\itemb If you add a row of zeroes to $A$, how does $M$ change?
\itemb If you add a row and column to $A$, with a $1$ in the corner and zeroes elsewhere in the new row and column, how does $M$ change?
\itemb If $A$ is a block matrix $\begin{bmatrix} B & 0 \\ 0 & C\end{bmatrix}$, what does this say about $M$?
\end{enumerate}

\solution{
\begin{enumerate}
\itemb It doesn't.
\itemb Corresponds to adding a free copy of $R$ as a direct sum.
\itemb  It doesn't.
\itemb $M\cong \mathrm{coker}(B) \oplus \mathrm{coker}(C)$\end{enumerate}}

\end{enumerate}


\vfill





\end{document}
