\documentclass[12pt]{amsart}


\usepackage{times}
\usepackage[margin=0.7in]{geometry}
\usepackage{amsmath,amssymb,multicol,graphicx,framed,ifthen,color,xcolor,stmaryrd,enumitem,colonequals,bbm}
\usepackage[outline]{contour}
\contourlength{.4pt}
\contournumber{10}
\newcommand{\Bold}[1]{\contour{black}{#1}}

\definecolor{chianti}{rgb}{0.6,0,0}
\definecolor{meretale}{rgb}{0,0,.6}
\definecolor{leaf}{rgb}{0,.35,0}
\newcommand{\Q}{\mathbb{Q}}
\newcommand{\N}{\mathbb{N}}
\newcommand{\Z}{\mathbb{Z}}
\newcommand{\R}{\mathbb{R}}
\newcommand{\C}{\mathbb{C}}
\newcommand{\e}{\varepsilon}
\newcommand{\m}{\mathfrak{m}}
\newcommand{\p}{\mathfrak{p}}
\newcommand{\ord}{\mathrm{ord}}
\newcommand{\1}{\mathbbm{1}}
\newcommand{\tr}{\mathrm{tr}}

\newcommand{\inv}{^{-1}}
\newcommand{\dabs}[1]{\left| #1 \right|}
\newcommand{\ds}{\displaystyle}
\newcommand{\solution}[1]{\ifthenelse {\equal{\displaysol}{1}} {\begin{framed}{\color{meretale}\noindent #1}\end{framed}} { \ }}
\newcommand{\showsol}[1]{\def\displaysol{#1}}
\newcommand{\rsa}{\rightsquigarrow}

\newcommand\itemA{\stepcounter{enumi}\item[{\Bold{(\theenumi)}}]}
\newcommand\itemB{\stepcounter{enumi}\item[(\theenumi)]}
\newcommand\itemC{\stepcounter{enumi}\item[{\it{(\theenumi)}}]}
\newcommand\itema{\stepcounter{enumii}\item[{\Bold{(\theenumii)}}]}
\newcommand\itemb{\stepcounter{enumii}\item[(\theenumii)]}
\newcommand\itemc{\stepcounter{enumii}\item[{\it{(\theenumii)}}]}
\newcommand\itemai{\stepcounter{enumiii}\item[{\Bold{(\theenumiii)}}]}
\newcommand\itembi{\stepcounter{enumiii}\item[(\theenumiii)]}
\newcommand\itemci{\stepcounter{enumiii}\item[{\it{(\theenumiii)}}]}
\newcommand\ceq{\colonequals}

\DeclareMathOperator{\res}{res}
\setlength\parindent{0pt}
%\usepackage{times}

%\addtolength{\textwidth}{100pt}
%\addtolength{\evensidemargin}{-45pt}
%\addtolength{\oddsidemargin}{-60pt}

\pagestyle{empty}
%\begin{document}\begin{itemize}

%\thispagestyle{empty}

\usepackage[hang,flushmargin]{footmisc}


\begin{document}
\showsol{0}
	
	\thispagestyle{empty}
	
	\section*{\S1.5: Determinants}	

\begin{framed}
Recall that given matrices $A$ and $B$, the matrix product $AB$ consists of linear combinations, namely:
Each column of $AB$ is a linear combinations of the columns of $A$, with coefficients/weights coming from the corresponding columns of~$B$. That is,
 \[ \big(\mathrm{col} \ j \ \text{of} \ AB\big) = \sum_{i=1}^t b_{ij} \cdot  \big(\mathrm{col} \ i \ \text{of} \ A);\]
 note that $b_{1j},\dots,b_{tj}$ is the $j$-th column of $B$.

\

\noindent \textsc{Properties of $\det$:} For a ring $R$, the determinant is a function $\det: \mathrm{Mat}_{n\times n}(R) \to R$ such that:
  \begin{enumerate}
  \item $\det$ is a polynomial expression of the entries of $A$ of degree $n$.
  \item $\det$ is a linear function of each column.
  \item $\det(A)=0$ if the columns are linearly dependent.
  \item $\det(AB)=\det(A)\det(B)$.
  \item $\det$ can be computed by Laplace expansion along a row/column.
  \item $\det(A) = \det(A^\tr)$.
  \item If $\phi:R\to S$ is a ring homomorphism, and $\phi(A)$ is the matrix obtained from $A$ by applying $\phi$ to each entry, then $\det(\phi(A)) = \phi(\det(A))$.
  \end{enumerate} 


\

\textsc{Adjoint Trick:} For an $n\times n$ matrix $A$ over $R$, 
\[\det(A) \mathbbm{1}_n = A^\mathrm{adj} A = A \, A^\mathrm{adj},\] where 
  $(A^\mathrm{adj})_{ij}=(-1)^{i+j} \det( \text{matrix obtained from $A$ by removing row $j$ and column $i$}).$

\


 \noindent \textsc{Eigenvector Trick:} Let $A$ be an $n\times n$ matrix, $v\in R^n$, and $r\in R$. If $Av=rv$, then ${\det(r \1_n - A) v = 0}$. Likewise, if instead $v$ is a row vector and $v A = rv$, then $\det(r \1_n - A) v = 0$.

\

\noindent \textsc{Definition:} Given an $n\times m$ matrix $A$ and $1\leq t \leq \min\{m,n\}$ the \textbf{ideal of $t\times t$ minors of $A$}, denoted $I_t(A)$, is the ideal generated by the determinants of all $t\times t$ submatrices of $A$ given by choosing $t$ rows and $t$ columns. For $t=0$, we set $I_0(A)=R$ and for $t>\min\{m,n\}$ we set $I_t(A)=0$.


\

\noindent \textsc{Lemma:} If $A$ is an $n\times m$ matrix, $B$ is an $m \times \ell$ matrix, and $t\leq 1$, then
\begin{itemize}
\item $I_{t+1}(A) \subseteq I_t(A)$
\item ${I_t(AB) \subseteq I_t(A) \cap I_t(B)}$.
\end{itemize}



\

\noindent \textsc{Proposition:} Let $M$ be a finitely presented module. Suppose that $A$ is an $n\times m$ presentation matrix for $M$. Then $I_n(A) M = 0$. Conversely, if $f M=0$, then $f^n\in I_n(A)$.

 
 \end{framed}
 

 
\begin{enumerate}
\itemA Let $M$ be a module. Suppose that $m_1,\dots,m_n$ is a generating set with corresponding presentation matrix $A$. Which of the following is true:
\[  A \begin{bmatrix} m_1 \\ \vdots \\ m_n \end{bmatrix} \stackrel{?}{=} 0 \qquad \qquad \begin{bmatrix} m_1 & \cdots & m_n \end{bmatrix} A  \stackrel{?}{=} 0. \]
Explain your answer in terms of the recollection on matrix multiplication above.

\solution{The second one!}

\begin{samepage}
\itemA Eigenvector Trick:
\begin{enumerate}
\itema What familiar fact/facts from linear algebra (over fields) is/are related to the Eigenvector Trick?
\itema Use the Adjoint Trick to prove the Eigenvector Trick.
\end{enumerate}
\end{samepage}

\solution{
\begin{enumerate}
\itema Over a field,  an eigenvalue of a matrix is a root of the characteristic polynomial. 
\itema If $Av=rv$, then $(A-r\1_n) v=0$, so multiply by $(A-r\1_n)^\mathrm{adj}$ to get $\det(A-r\1_n) v =(A-r\1_n)^\mathrm{adj} (A-r\1_n) v = 0$. Likewise on the other side.
\end{enumerate}}


\itemA Show that a square matrix over a ring $R$ is invertible if and only if its determinant is a unit.

\solution{If $AB=\1_n$, then $\det(A) \det(B) = \det(\1_n) =1$, so $\det(A)$ is a unit. On the other hand, if $\det(A)$ is a unit, then $B=\det(A)^{-1} A^\mathrm{adj}$ is an inverse of $A$ by the adjoint trick.}

\itemA Proof of Proposition:
\begin{enumerate}
\itema First consider the case $m=n$. Show that $\det(A)$ kills each generator $m_i$, and conclude that $I_n(A) M =0$.
\itema Now consider the case $n\leq m$. Show that for any $n\times n$ submatrix $A'$ of $A$ that $\det(A') M = 0$, and conclude that $I_n(A) M =0$. What's the deal when $m<n$?
\itema For the ``conversely'' statement, show that if $fM=0$ then there is some matrix $B$ such that $AB=f \1_n$, and deduce that $f\in I_n(A)^n$.
\end{enumerate}

\solution{
\begin{enumerate}
\itema Since $A$ is a presentation matrix for $M$, with the corresponding generating set $m_1,\dots,m_n$, we have $\begin{bmatrix} m_1 & \dots & m_n \end{bmatrix} A=0$. By the adjoint trick, $ \det(A) \begin{bmatrix} m_1 & \dots & m_n \end{bmatrix} = 0$, so $\det(A)$ kills each generator of $M$. Thus, $\det(A)$ kills $M$. By definition $I_n(A) = (\det(A))$, so we are done.

\itema Suppose $n\leq m$ and fix $m$ columns of $A$ to form an $n\times n$ submatrix $A'$. The columns of $A'$ are still relations on $m_1,\dots,m_n$, so the same argument shows that $\det(A')$ kills $M$. Now, by definition, $I_n(A)$ is generated by the determinants of the submatrices $A'$, so $I_n(A) M=0$.

When $m<n$, $I_n(A)=0$, which very much kills $M$.


\itema If $fM=0$, then the vector with $f$ in the $i$th entry and zeroes elsewhere is a relation on the generators, so by definition of presentation matrix, this vector is a linear combination of the columns of $A$. Thus each column $f\1_n$ is a linear combination of the columns of $A$, which means that we can write $f\1_n= AB$ for some matrix $B$ following the discussion above. By the Lemma, we have $f^n = \det(f \1_n) \in I_n(AB) \subseteq I_n(A)$. This completes the proof.
\end{enumerate}
}




\itemB Prove the Lemma above.

\solution{The first statement follows from Laplace expansion. For the second, it suffices to show that the determinant of any $t\times t$ submatrix of $AB$ is a linear combination of determinants of $t\times t$ submatrices of $A$; the claim for $B$ follows by applying transposes. We can restrict to the relevant rows of $A$ and columns of $B$, so we can assume that $A$ is $t\times n$ and $B$ is $n \times t$ for some $n\geq t$. Then $AB$ is a matrix whose columns are linear combinations of the columns of $A$. Then using linearity of $\det$ in each column, we can write $\det(AB)$ as a linear combination of the determinants of matrices with columns from $A$, which shown the claim.}


\itemC Prove\footnote{Hint: First consider the case when the two presentations have the same generating sets, but different generating sets for the relations. Reduce to the case where $B = [ A | v]$ for a single column $v$.} \textsc{Fitting's Lemma:}  If $A$ and $B$ are presentation matrices for the same $R$-module $M$ of size $n\times m$ and $n' \times m'$ (respectively), and $t\geq 0$, then $I_{n-t}(A) = I_{n'-t}(B)$.

\solution{}



\end{enumerate}


\vfill





\end{document}
