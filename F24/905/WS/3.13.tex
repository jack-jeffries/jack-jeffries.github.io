\documentclass[12pt]{amsart}


\usepackage{times}
\usepackage[margin=0.7in]{geometry}
\usepackage{amsmath,amssymb,multicol,graphicx,framed,ifthen,color,xcolor,stmaryrd,enumitem,colonequals,bbm}
\usepackage[outline]{contour}
\contourlength{.5pt}
\contournumber{10}
\newcommand{\Bold}[1]{\contour{black}{#1}}

\definecolor{chianti}{rgb}{0.6,0,0}
\definecolor{meretale}{rgb}{0,0,.6}
\definecolor{leaf}{rgb}{0,.35,0}
\newcommand{\Q}{\mathbb{Q}}
\newcommand{\N}{\mathbb{N}}
\newcommand{\Z}{\mathbb{Z}}
\newcommand{\R}{\mathbb{R}}
\newcommand{\C}{\mathbb{C}}
\newcommand{\e}{\varepsilon}
\newcommand{\m}{\mathfrak{m}}
\newcommand{\p}{\mathfrak{p}}
\newcommand{\ord}{\mathrm{ord}}
\newcommand{\1}{\mathbbm{1}}

\newcommand{\inv}{^{-1}}
\newcommand{\dabs}[1]{\left| #1 \right|}
\newcommand{\ds}{\displaystyle}
\newcommand{\solution}[1]{\ifthenelse {\equal{\displaysol}{1}} {\begin{framed}{\color{meretale}\noindent #1}\end{framed}} { \ }}
\newcommand{\showsol}[1]{\def\displaysol{#1}}
\newcommand{\rsa}{\rightsquigarrow}

\newcommand\itemA{\stepcounter{enumi}\item[{\Bold{(\theenumi)}}]}
\newcommand\itemB{\stepcounter{enumi}\item[(\theenumi)]}
\newcommand\itemC{\stepcounter{enumi}\item[{\it{(\theenumi)}}]}
\newcommand\itema{\stepcounter{enumii}\item[{\Bold{(\theenumii)}}]}
\newcommand\itemb{\stepcounter{enumii}\item[(\theenumii)]}
\newcommand\itemc{\stepcounter{enumii}\item[{\it{(\theenumii)}}]}
\newcommand\itemai{\stepcounter{enumiii}\item[{\Bold{(\theenumiii)}}]}
\newcommand\itembi{\stepcounter{enumiii}\item[(\theenumiii)]}
\newcommand\itemci{\stepcounter{enumiii}\item[{\it{(\theenumiii)}}]}
\newcommand\ceq{\colonequals}

\DeclareMathOperator{\res}{res}
\setlength\parindent{0pt}
%\usepackage{times}

%\addtolength{\textwidth}{100pt}
%\addtolength{\evensidemargin}{-45pt}
%\addtolength{\oddsidemargin}{-60pt}

\pagestyle{empty}
%\begin{document}\begin{itemize}

%\thispagestyle{empty}

\usepackage[hang,flushmargin]{footmisc}


\begin{document}
\showsol{0}
	
	\thispagestyle{empty}
	
	\section*{\S3.13: Finiteness Theorem for Invariant rings}	

\begin{framed}

\textsc{Hilbert's finiteness Theorem:} Let $K$ be a field of characteristic zero, and $R=K[X_1,\dots,X_n]$ be a polynomial ring. Let $G$ be a finite group acting on $R$ by degree-preserving $K$-algebra automorphisms. Then the invariant ring $R^G$ is algebra-finite over $K$.

\



\noindent \textsc{Theorem:} Let $R$ be an $\N$-graded ring. Then $R$ is Noetherian if and only if $R_0$ is Noetherian and $R$ is algebra-finite over $R_0$.

\


\noindent \textsc{Definition:} Let $R\subseteq S$ be an inclusion of rings. We say that $R$ is a \textbf{direct summand} of $S$ if there is an $R$-module homomorphism $\pi:S\to R$ such that $\pi|_R=\1_R$.

\

\noindent \textsc{Proposition:} A direct summand of a Noetherian ring is Noetherian. 

\

\noindent \textsc{Lemma:} Let $R$ be a polynomial ring over a field $K$. If $G$ is a group acting on $R$ by degree-preserving $K$-algebra automorphisms, then
\begin{enumerate}
\item $R^G$ is an $\N$-graded $K$-subalgebra of $R$ with $(R^G)_0=K$.
\item If in addition, $G$ is finite, and $|G|$ is invertible in $K$, then $R^G$ is a direct summand of $R$.
\end{enumerate}
 \end{framed}


 
\begin{enumerate}
\itemA Use the Lemma, Proposition, and Theorem to deduce Hilbert's finiteness Theorem.

\solution{By the Lemma, $R^G$ is a direct summand of $R$. Since $R$ is Noetherian, so is $R^G$. By the Lemma, $R^G$ is graded with $(R^G)_0=K$. Then, by the Theorem, since $R^G$ is Noetherian, and $R^G$ is algebra-finite over $(R^G)_0$, and it remains to note that $(R^G)_0=K$.}

\itemA Proof of Theorem:
\begin{enumerate}
\itema Explain the direction $(\Leftarrow)$.
\itema Show that $R$ Noetherian implies $R_0$ is Noetherian.
\itema Let $f_1,\dots,f_t$ be a homogeneous generating set for $R_+$, the ideal generated by positive degree elements of $R$. Show\footnote{Hint: Start by writing $h\in R_d$ as $h=\sum_i r_i f_i$ with $d = \deg(r_i) +\deg(f_i)$ for all $i$.} by (strong) induction on $d$ that every element of $R_d$ is contained in $R_0[f_1,\dots,f_t]$.
\itema Conclude the proof of the Theorem.
\end{enumerate}


\solution{\begin{enumerate}
\item This follows from the Hilbert Basis Theorem.
\item $R_0\cong R/R_+$.
\item For $d=0$ there is nothing to show. For $d>0$, take $h\in R_d$. Since $R_d\subseteq R_+$, write $h=\sum_i r_i f_i$ for some $r_i\in R$. If we replace $r_i$ by $r'_i$ its homogeneous component of degree $d-\deg(f_i)$, we claim that $h=\sum_i r'_i f_i$. Indeed, writing each $r_i$ as a sum of homogeneous components and multiplying out, all of the other terms are homogeneous of some other degree, so the claim follows by uniqueness of homogeneous decomposition. So suppose $r_i$ is homogeneous of degree $d-\deg(f_i)$. By induction, we have $r_i\in R_0[f_1,\dots,f_t]$. But then this plus $h=\sum_i r_i f_i$ show $h\in R_0[f_1,\dots,f_t]$.
\item If $R$ is Noetherian then $R_+$ is finitely generated as an ideal; since $R_+$ is homogeneous, it is generated by the (fintely many) components of these generators so has a finite homogeneous generating set, and a such generating set of $R_+$ generates $R$ as an algebra over $R_0$ by the previous part.
\end{enumerate}}

\itemA Proof of Proposition:
\begin{enumerate}
\itema Show that if $R$ is a direct summand of $S$, and $I$ is an ideal of $R$, then $IS \cap R = I$.
\itema Complete the proof of the proposition.
\end{enumerate}

\solution{
\begin{enumerate}
\item We always have $I \subseteq IS \cap R$. Let $f\in IS \cap R$, so $f=\sum_i a_i s_i$ with $a_i\in I$, $s_i\in S$. Apply the map $\pi$. Since $f\in R$, we have $\pi(f)=f$. Since $\pi$ is $R$-linear, we also have $\pi\left( \sum_i a_i s_i\right) = \sum_i a_i \pi(s_i)$, with $\pi(s_i)\in R$. But this is an element of $I$, so $f\in I$.
\item Let $I_1 \subseteq I_2 \subseteq I_3 \subseteq \cdots$ be a chain of ideals in $R$. Then $I_1 S \subseteq I_2 S \subseteq I_3 S \subseteq \cdots$ is a chain of ideals in $S$, which necessarily stabilizes. But the chain $(I_1 S \cap R) \subseteq (I_2 S \cap R) \subseteq (I_3 S \cap R) \subseteq \cdots$ stabilizes, but this is our original chain!
\end{enumerate}
}

\itemB Proof of Lemma part (2): Consider $r \mapsto \frac{1}{|G|} \sum_{g\in G} g\cdot r$.

\solution{One checks directly that this map is $R^G$-linear and restricts to the identity on $R^G$.}

\itemB Let $\mathcal{S}_3$ denote the symmetric group on $3$ letters, and let $\mathcal{S}_3$ act on $R=\C[X_1,X_2,X_3]$ by permuting variables; i.e., $\sigma$ is the $\C$-algebra homomorphism given by $\sigma \cdot X_i= X_{\sigma(i)}$. Show\footnote{Hint: Order the monomials of $R$ by lexicographic (dictionary) order. Given a homogeneous invariant, can you find an element of $\C[X_1+X_2+X_3,X_1X_2+X_1X_3+X_2X_3,X_1X_2X_3]$ with the same ``first'' monomial in that order?} that
\[R^{\mathcal{S}_3} = \C[X_1+X_2+X_3,X_1X_2+X_1X_3+X_2X_3,X_1X_2X_3]\]
and that $X_1+X_2+X_3,X_1X_2+X_1X_3+X_2X_3,X_1X_2X_3$ are algebraically independent over $\C$. What about replacing $3$ with $n$?

\solution{}

\itemB Show that a direct summand of a normal ring is normal.

\solution{}


\end{enumerate}



\vfill





\end{document}
