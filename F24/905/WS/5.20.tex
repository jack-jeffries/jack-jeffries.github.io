\documentclass[12pt]{amsart}


\usepackage{times}
\usepackage[margin=0.8in]{geometry}
\usepackage{amsmath,amssymb,multicol,graphicx,framed,ifthen,color,xcolor,stmaryrd,enumitem,colonequals,bbm}


\usepackage[outline]{contour}
\contourlength{.4pt}
\contournumber{10}
\newcommand{\Bold}[1]{\contour{black}{#1}}


\definecolor{chianti}{rgb}{0.6,0,0}
\definecolor{meretale}{rgb}{0,0,.6}
\definecolor{leaf}{rgb}{0,.35,0}
\newcommand{\Q}{\mathbb{Q}}
\newcommand{\N}{\mathbb{N}}
\newcommand{\Z}{\mathbb{Z}}
\newcommand{\R}{\mathbb{R}}
\newcommand{\C}{\mathbb{C}}
\newcommand{\e}{\varepsilon}
\newcommand{\m}{\mathfrak{m}}
\newcommand{\p}{\mathfrak{p}}
\newcommand{\q}{\mathfrak{q}}
\newcommand{\ord}{\mathrm{ord}}
\newcommand{\Spec}{\mathrm{Spec}}
\newcommand{\1}{\mathbbm{1}}
\newcommand{\cZ}{\mathcal{Z}}

\newcommand{\inv}{^{-1}}
\newcommand{\dabs}[1]{\left| #1 \right|}
\newcommand{\ds}{\displaystyle}
\newcommand{\solution}[1]{\ifthenelse {\equal{\displaysol}{1}} {\begin{framed}{\color{meretale}\noindent #1}\end{framed}} { \ }}
\newcommand{\showsol}[1]{\def\displaysol{#1}}
\newcommand{\rsa}{\rightsquigarrow}

\newcommand\itemA{\stepcounter{enumi}\item[{\Bold{(\theenumi)}}]}
\newcommand\itemB{\stepcounter{enumi}\item[(\theenumi)]}
\newcommand\itemC{\stepcounter{enumi}\item[{\it{(\theenumi)}}]}
\newcommand\itema{\stepcounter{enumii}\item[{\Bold{(\theenumii)}}]}
\newcommand\itemb{\stepcounter{enumii}\item[(\theenumii)]}
\newcommand\itemc{\stepcounter{enumii}\item[{\it{(\theenumii)}}]}
\newcommand\itemai{\stepcounter{enumiii}\item[{\Bold{(\theenumiii)}}]}
\newcommand\itembi{\stepcounter{enumiii}\item[(\theenumiii)]}
\newcommand\itemci{\stepcounter{enumiii}\item[{\it{(\theenumiii)}}]}
\newcommand\ceq{\colonequals}

\DeclareMathOperator{\res}{res}
\setlength\parindent{0pt}
%\usepackage{times}

%\addtolength{\textwidth}{100pt}
%\addtolength{\evensidemargin}{-45pt}
%\addtolength{\oddsidemargin}{-60pt}

\pagestyle{empty}
%\begin{document}\begin{itemize}

%\thispagestyle{empty}

\usepackage[hang,flushmargin]{footmisc}


\begin{document}
\showsol{0}
	
	\thispagestyle{empty}
	
	\section*{\S5.20: Local rings and NAK}	

\begin{framed}




\noindent \textsc{Definition:} A ring is \textbf{local} if it has a unique maximal ideal. We write $(R,\m)$ for a local ring to denote the ring $R$ and the maximal ideal $\m$; we many also write $(R,\m,k)$ to indicate the residue field $k\ceq R/\m$.


\

\noindent \textsc{General NAK:} Let $R$ be a ring, $I$ an ideal, and $M$ be a finitely generated module. If $IM=M$, then there is some $a\in R$ such that $a\equiv 1 \ \mathrm{mod} \ I$ and  $aM=0$.

\

\noindent \textsc{Local NAK 1:} Let $(R,\m)$ be a local ring and $M$ be a finitely generated module. If $M=\m M$, then $M=0$.

\

\noindent \textsc{Local NAK 2:} Let $(R,\m)$ be a local ring and $M$ be a finitely generated module. Let $N$ be a submodule of $M$. Then $M = N + \m M$ if and only if $M=N$.

\

\noindent \textsc{Local NAK 3:} Let $(R,\m,k)$ be a local ring and $M$ be a finitely generated module. Then a set of elements $S\subseteq M$ generates $M$ if and only if the image of $S$ in $M/\m M$ generates $M/\m M$ as a $k$-vector space.


\

\noindent \textsc{Definition:} Let $(R,\m,k)$ be a local ring and $M$ be a finitely generated module. A set of elements $S$ of $M$ is a \textbf{minimal generating set} for $M$ if the image of $S$ in $M/\m M$ is a basis for $M/\m M$ as a $k$-vector space.
\end{framed}

 
\begin{enumerate}
\itemA Local rings.
\begin{enumerate}
\itema Show that for a ring $R$ the following are equivalent:
\begin{itemize}
\item $R$ is a local ring.
\item The set of all nonunits forms an ideal.
\item The set of all nonunits is closed under addition.
\end{itemize}
\itema Show that if $A$ is a domain then $A[X]$ is \emph{not} a local ring.
\itema Show that if $K$ is a field, the power series ring $R=K\llbracket X_1,\dots,X_n\rrbracket$ is a local ring.
\itema Let $p\in \Z$ be a prime number, and $\Z_{(p)} \subseteq \Q$ be the set of rational numbers that can be written with denominator \emph{not} a multiple of $p$. Show that $(\Z_{(p)}, p\Z_{(p)})$ is a local ring.
\itema Show that any quotient of a local ring is also a local ring.
\end{enumerate}
\solution{
\begin{enumerate}
\itema Since any element times a nonunit is a nonunit, the last two are equivalent. Recall that an element is a unit if and only if it is not in any maximal ideal. So, if $(R,\m)$ is local, the nonunits are the elements of $\m$, which is an ideal; conversely, if the nonunits form an ideal, then this ideal must be the unique maximal ideal.

\itema $X$ and $X+1$ are nonunits, but $1= (X+1)-X$ is a unit.
\itema The set of nonunits is the elements with zero constant term, which is the ideal $(X_1,\dots,X_n)$.
\itema First, check that this is a ring. Then note that the units in this ring are the fractions $a/b$ with $p\nmid a,b$, which is the same as the ideal $p\Z_{(p)}$.
\itema This follows from the Lattice Isomorphism Theorem.
\end{enumerate}
}


\itemA General NAK implies Local NAKs
\begin{enumerate}
\itema Show that General NAK implies Local NAK 1.
\itema Briefly\footnote{Reuse an old argument in a similar setting.} explain why Local NAK 1 implies Local NAK 2.
\itema Briefly\footnote{It's d\'{e}j\`{a} vu all over again.} explain why Local NAK 2 implies Local NAK 3.
\itema Use Local NAK 3 to briefly explain why a minimal generating set is a generating set, and that, in this setting, any generating set contains a minimal generating set.
\end{enumerate}

\solution{
\begin{enumerate}
\itema If $\m M = M$, then by General NAK, there is some $a\in \m$ such that $a\equiv 1 \, \mathrm{mod}\, \m$ and $aM=0$. But $a$ must be a unit, so $M=0$!
\itema Same as the graded case: apply NAK 1 to $M/N$.
\itema Same as the graded case: apply NAK 2 to $N=\sum_{s\in S} Rs$.
\itema Same as the graded case: a $k$-basis for $M/\m M$ is a $k$-spanning set for $M/\m M$, and any $k$-spanning set for $M/\m M$ contains a $k$-basis.
\end{enumerate}

}

\itemA Proof of General NAK: Let $M=\sum_{i=1}^n  R m_i$. Set $v$ to be the row vector $[m_1, \dots,m_n]$. 
\begin{enumerate}
\itema Suppose that $IM=M$. Explain why there is an $n\times n$ matrix $A$ with entries in $I$ such that $vA=v$.
\itema Apply a \textsc{Trick} and complete the proof.
\end{enumerate}
\solution{
\begin{enumerate}
\itema Each $m_i$ is an element of $IM$, so we can write $m_i = \sum_j b_j n_j$ with $n_j\in M$ and $b_j\in I$. We can then write $n_j$ as a linear combination of the $m_i$'s. Combining all together, we can write $m_i = \sum_j a_j m_j$ with $a_j\in I$. These linear combinations are the columns of a matrix $A$ as desired.
\itema By the Eigenvector trick, ${\det(A - \1)}$ kills $v$, so kills $M$. Going mod $I$ we have ${\det(A-\1) \equiv \det(-\1) \equiv \pm 1}$; up to sign, $a= \det(A - \1)$ is the element we seek.
\end{enumerate}
}

\itemB Let $(R,\m)$ be a local ring, $f\in R$ not a unit, and $M$ be a nonzero finitely generated module. Show that there is some element of $M$ that is \emph{not} a multiple of $f$.

\solution{Suppose otherwise. Then $M=fM$. We have $f\in \m$, so $M=fM \subseteq \m M \subseteq M$, so $M=\m M$. But by NAK, we then have $M=0$, a contradiction.
}

\itemB Applications of NAK.
\begin{enumerate}
\itemb Let $R$ be a ring and $I$ be a finitely generated ideal. Show that if $I^2=I$ then there is some idempotent $e$ such that $I=(e)$.
\itemb Find a counterexample to (a) if $I$ is \emph{not} assumed to be finitely generated.
\itemb Let $(R,\m)$ be a Noetherian local ring and $M$ be a finitely generated module. Show that $\bigcap_{n\geq 1} \m^n M = 0$.
\itemb Find a counterexample to (c) if $(R,\m)$ is still Noetherian local but $M$ is not finitely generated.
\itemb Find a counterexample to (c) if $(R,\m)$ with $M=R$, $\m$ is a maximal ideal, but $R$ is not necessarily Noetherian and local.
\itemb Let $R$ be a Noetherian ring, and $M$ a finitely generated module. Let $\phi:M\to M$ be a surjective $R$-module homomorphism. Show\footnote{Hint: Take a page from the 818 playbook and give $M$ an $R[X]$-module structure.} that $\phi$ must also be injective.
\itemb Let $(R,\m)$ be a local ring. Suppose that $R_{\mathrm{red}}\ceq R/\sqrt{0}$ is a domain, and that there is some $f\in R$ such that $R/fR$ is reduced (and nonzero). Show that $R$ is reduced (and hence a domain).
\end{enumerate}




\end{enumerate}
\vfill





\end{document}
