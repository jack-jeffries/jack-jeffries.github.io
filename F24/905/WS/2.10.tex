\documentclass[12pt]{amsart}


\usepackage{times}
\usepackage[margin=0.575in]{geometry}
\usepackage{amsmath,amssymb,multicol,graphicx,framed,ifthen,color,xcolor,stmaryrd,enumitem,colonequals,bbm}
\usepackage[outline]{contour}
\contourlength{.4pt}
\contournumber{10}
\newcommand{\Bold}[1]{\contour{black}{#1}}

\definecolor{chianti}{rgb}{0.6,0,0}
\definecolor{meretale}{rgb}{0,0,.6}
\definecolor{leaf}{rgb}{0,.35,0}
\newcommand{\Q}{\mathbb{Q}}
\newcommand{\N}{\mathbb{N}}
\newcommand{\Z}{\mathbb{Z}}
\newcommand{\R}{\mathbb{R}}
\newcommand{\C}{\mathbb{C}}
\newcommand{\e}{\varepsilon}
\newcommand{\m}{\mathfrak{m}}
\newcommand{\p}{\mathfrak{p}}
\newcommand{\ord}{\mathrm{ord}}
\newcommand{\1}{\mathbbm{1}}

\newcommand{\inv}{^{-1}}
\newcommand{\dabs}[1]{\left| #1 \right|}
\newcommand{\ds}{\displaystyle}
\newcommand{\solution}[1]{\ifthenelse {\equal{\displaysol}{1}} {\begin{framed}{\color{meretale}\noindent #1}\end{framed}} { \ }}
\newcommand{\showsol}[1]{\def\displaysol{#1}}
\newcommand{\rsa}{\rightsquigarrow}

\newcommand\itemA{\stepcounter{enumi}\item[{\Bold{(\theenumi)}}]}
\newcommand\itemB{\stepcounter{enumi}\item[(\theenumi)]}
\newcommand\itemC{\stepcounter{enumi}\item[{\it{(\theenumi)}}]}
\newcommand\itema{\stepcounter{enumii}\item[{\Bold{(\theenumii)}}]}
\newcommand\itemb{\stepcounter{enumii}\item[(\theenumii)]}
\newcommand\itemc{\stepcounter{enumii}\item[{\it{(\theenumii)}}]}
\newcommand\itemai{\stepcounter{enumiii}\item[{\Bold{(\theenumiii)}}]}
\newcommand\itembi{\stepcounter{enumiii}\item[(\theenumiii)]}
\newcommand\itemci{\stepcounter{enumiii}\item[{\it{(\theenumiii)}}]}
\newcommand\ceq{\colonequals}

\DeclareMathOperator{\res}{res}
\setlength\parindent{0pt}
%\usepackage{times}

%\addtolength{\textwidth}{100pt}
%\addtolength{\evensidemargin}{-45pt}
%\addtolength{\oddsidemargin}{-60pt}

\pagestyle{empty}
%\begin{document}\begin{itemize}

%\thispagestyle{empty}

\usepackage[hang,flushmargin]{footmisc}


\begin{document}
\showsol{0}
	
	\thispagestyle{empty}
	
	\section*{\S2.10: Noetherian Modules}	

\begin{framed}

\noindent \textsc{Definition:} A module is \textbf{Noetherian} if every ascending chain of submodules $M_1 \subseteq M_2 \subseteq M_3 \subseteq \cdots$ eventually stabilizes: i.e., there is some $N$ such that $M_n=M_N$ for all $n\geq N$.

\

\noindent \textsc{Theorem:} If $R$ is a Noetherian ring, then an $R$-module $M$ is Noetherian if and only $M$ is finitely generated.

\

\noindent \textsc{Corollary:} If $R$ is a Noetherian ring, then a submodule of a finitely generated $R$-module is finitely generated.


\

\noindent \textsc{Lemma:} Let $M$ be an $R$-module and $N\subseteq M$ a submodule. Let $L,L'$ be two more submodules of $M$. Then $L=L'$ if and only if $L\cap N= L'\cap N$ and $\frac{L+N}{N}=\frac{L'+N}{N}$.

 \end{framed}
 

 
\begin{enumerate}
\itemA Equivalences for Noetherianity.
\begin{enumerate}
\itema Explain why $M$ is Noetherian if and only if every submodule of $M$ is finitely generated.

\itema Explain why $M$ is Noetherian if and only if every nonempty collection of submodules has a maximal element.
\end{enumerate}


\solution{
\begin{enumerate}
\itema Analogous to what we did with ideals.
\itema Analogous to what we did with ideals.
\end{enumerate}
}

\itemA Submodules and quotient modules: Let $N \subseteq M$.
\begin{enumerate}
\itema Show that if $M$ is a Noetherian $R$-module, then $N$ is a Noetherian $R$-module.
\itema Show that if $M$ is a Noetherian $R$-module, then $M/N$ is a Noetherian $R$-module.
\itema Use the Lemma above to show that if $N$ and $M/N$ are Noetherian $R$-modules, then $M$ is a Noetherian $R$-module.
\end{enumerate}

\solution{
\begin{enumerate}
\itema A chain of submodules of $N$ is a chain of submodules of $M$, so by hypothesis must stabilize.
\itema The submodules of $M/N$ are in containment-preserving bijection with the submodules of $M$ that contain $N$, so a chain of submodules of $M/N$ must stabilize.
\itema Suppose we have a chain of submodules $M_i$ of $M$. By intersecting with $N$, we get a chain of submodules of $M_i \cap N$ of $N$, which by hypothesis, must stabilize at some $n=a$. By taking images in $M/N$, we get a chain of submodules  $\frac{M_i+N}{N}$ of $M/N$ that must stabilize at some $n=b$. Then for $n\geq \max\{a,b\}$ by the Lemma, we must have that the chain $M_i$ stabilizes.
\end{enumerate}
}


\itemA Proof of Theorem: Let $R$ be a Noetherian ring.
\begin{enumerate}
\itema Explain why $R$ is a Noetherian $R$-module.
\itema Show that $R^n$ is a Noetherian $R$-module for every $n$.

\itema Deduce the Theorem above.
\itema Deduce the Corollary above.

\end{enumerate}


\solution{
\begin{enumerate}
\itema The submodules of $R$ are just the ideals of $R$.
\itema There is a copy of $R^{n-1}$ in $R^n$ (where the last coordinate is zero) with quotient $R^1$, so it follows by induction on $n$.
\itema If $M$ is Noetherian, then every submodule of $M$ including $M$ itself is finitely generated. Conversely, if $M$ is finitely generated, then $M$ is a quotient of $R^n$ for some $n$, so it follows from (3b) and (2b). 
\itema Follows from (3c) and (2a).
\end{enumerate}

}

\itemA Proof of Hilbert Basis Theorem for $R[X]$: Let $R$ be a Noetherian ring.
\begin{enumerate}
\itema Let $I$ be an ideal of $R[X]$. Given a nonzero element $f\in R[X]$, set $\mathrm{LT}(f)$ to be the leading coefficient\footnote{That is, if $f=\sum_i a_i X^i$ and $k=\max\{ i \ | \ a_i\neq 0\}$, then $\mathrm{LT}(f)=a_k$.} of $f$ and $\mathrm{LT}(0)=0$, and let $\mathrm{LT}(I)= \{ \mathrm{LT}(f) \ | \ f\in I \}$. Is $\mathrm{LT}(I)$ an ideal of $R$?
\itema Let $f_1,\dots,f_n\in R[X]$ be such that $\mathrm{LT}(f_1), \dots, \mathrm{LT}(f_n)$ generate $\mathrm{LT}(I)$. Let $N$ be the maximum of the top degrees of $f_i$. Show that every element of $I$ can be written as $\sum_i r_i f_i + g$ with $r_i, g\in R[X]$ and the top degree of $g\in I$ is less than $N$.
\itema Write $R[X]_{<N}$ for the $R$-submodule of $R[X]$ consisting of polynomials with top degree $<N$. Show that $I\cap R[X]_{<N}$ is a finitely generated $R$-module. 
\itema Complete the proof of the Theorem.
\end{enumerate}

\solution{
\begin{enumerate}
\itema Yes; we just check the definition.
\itema We proceed by induction on top degree of $f\in I$. For $f$ with top degree less than $N$, we just take $g=f$ and $r_i=0$. For $f$ with top degree $t\geq N$, write $f= a X^t + \text{lower degree terms}$, and $a= \sum_i a_i \mathrm{LT}(f_i)$. Then $\sum_i a_i X^{t-n_i} f_i = a X^t + \text{lower degree terms}$, so $f' = f- \sum_i a_i X^{t-n_i} f_i \in I$ is of lower degree. We can then write $f'$ in the desired form by induction, and then the original $f$ as well.
\itema $I\cap R[X]_{<N}$ is an $R$-submodule of $R[X]_{<N}$, which is generated by $1,X,\dots,X^{N-1}$, whence finitely generated. Since $R$ is Noetherian, this submodule is also Noetherian.
\itema Fix an $R$-module generating set $g_1,\dots,g_s$ for $I\cap R[X]_{<N}$. We claim that $I=(f_1,\dots,f_n, g_1,\dots,g_s)$. By construction we have $\supseteq$. Then, given $f\in I$, we can write $f=\sum_i r_i f_i + g$ and $g=\sum_j a_j g_j$ with $a_j\in R$, so $f\in (f_1,\dots,f_n, g_1,\dots,g_s)$. Thus, $I$ is finitely generated.
\end{enumerate}

}


\itemB Proof of Hilbert Basis Theorem for $R\llbracket X\rrbracket$: How can you modify the Proof of Hilbert Basis Theorem for $R[X]$ to work in the power series case? Make it happen!

\solution{We use lowest degree terms instead. Define $\mathrm{LT}(f)$ to be the bottom coefficient of $f$. Proceeding similarly, we can show that if $f_1,\dots,f_n\in R\llbracket X\rrbracket$ are such that $\mathrm{LT}(f_1), \dots, \mathrm{LT}(f_n)$ generate $\mathrm{LT}(I)$, then and $f\in I$ can be written as $\sum_i r_i f_i + g$ with $g$ a \emph{polynomial} in $X$ of top degree less than $N$, and continue as in the polynomial case.
}



\itemB Prove the Lemma.

\solution{}

\itemB Noetherianity and module-finite inclusions: Let $R\subseteq S$ be module-finite.
\begin{enumerate}
\itemb Without using the Hilbert Basis Theorem, show that is $R$ is Noetherian, then $S$ is Noetherian.
\itemc \textsc{Eakin-Nagata Theorem:} Show that if $S$ is Noetherian, then $R$ is Noetherian.
\end{enumerate}
\end{enumerate}


\vfill





\end{document}
