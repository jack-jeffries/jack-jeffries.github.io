\documentclass[12pt]{amsart}


\usepackage{times}
\usepackage[margin=0.8in]{geometry}
\usepackage{amsmath,amssymb,multicol,graphicx,framed,ifthen,color,xcolor,stmaryrd,enumitem,colonequals,bbm}
\definecolor{chianti}{rgb}{0.6,0,0}
\definecolor{meretale}{rgb}{0,0,.6}
\definecolor{leaf}{rgb}{0,.35,0}
\newcommand{\Q}{\mathbb{Q}}
\newcommand{\N}{\mathbb{N}}
\newcommand{\Z}{\mathbb{Z}}
\newcommand{\R}{\mathbb{R}}
\newcommand{\C}{\mathbb{C}}
\newcommand{\e}{\varepsilon}
\newcommand{\m}{\mathfrak{m}}
\newcommand{\p}{\mathfrak{p}}
\newcommand{\ord}{\mathrm{ord}}
\newcommand{\gr}{\mathrm{gr}}
\newcommand{\1}{\mathbbm{1}}





\usepackage[outline]{contour}
\contourlength{.4pt}
\contournumber{10}
\newcommand{\Bold}[1]{\contour{black}{#1}}




\newcommand{\inv}{^{-1}}
\newcommand{\dabs}[1]{\left| #1 \right|}
\newcommand{\ds}{\displaystyle}
\newcommand{\solution}[1]{\ifthenelse {\equal{\displaysol}{1}} {\begin{framed}{\color{meretale}\noindent #1}\end{framed}} { \ }}
\newcommand{\showsol}[1]{\def\displaysol{#1}}
\newcommand{\rsa}{\rightsquigarrow}

\newcommand\itemA{\stepcounter{enumi}\item[{\Bold{(\theenumi)}}]}
\newcommand\itemB{\stepcounter{enumi}\item[(\theenumi)]}
\newcommand\itemC{\stepcounter{enumi}\item[{\it{(\theenumi)}}]}
\newcommand\itema{\stepcounter{enumii}\item[{\Bold{(\theenumii)}}]}
\newcommand\itemb{\stepcounter{enumii}\item[(\theenumii)]}
\newcommand\itemc{\stepcounter{enumii}\item[{\it{(\theenumii)}}]}
\newcommand\itemai{\stepcounter{enumiii}\item[{\Bold{(\theenumiii)}}]}
\newcommand\itembi{\stepcounter{enumiii}\item[(\theenumiii)]}
\newcommand\itemci{\stepcounter{enumiii}\item[{\it{(\theenumiii)}}]}
\newcommand\ceq{\colonequals}

\DeclareMathOperator{\res}{res}
\setlength\parindent{0pt}
%\usepackage{times}

%\addtolength{\textwidth}{100pt}
%\addtolength{\evensidemargin}{-45pt}
%\addtolength{\oddsidemargin}{-60pt}

\pagestyle{empty}
%\begin{document}\begin{itemize}

%\thispagestyle{empty}

\usepackage[hang,flushmargin]{footmisc}


\begin{document}
\showsol{0}
	
	\thispagestyle{empty}
	
	\section*{\S3.14: Rees Rings and the Artin-Rees Lemma}	

\begin{framed}

\noindent \textsc{Definition:} Let $R$ be a ring and $I$ be an ideal.
The \textbf{Rees ring} of $I$ is the $\N$-graded $R$-algebra
\[  R[IT] \ceq \bigoplus_{d\geq 0} I^d T^d = R \oplus I T \oplus I^2 T^2 \oplus \cdots \]
with multiplication determined by $(a T^d)(b T^e) = ab T^{d+e}$ for $a\in I^d$, $b\in I^e$ (and extended by the distributive law for nonhomogeneous elements). Here $I^n$ means the $n$th power of the ideal $I$ in $R$, and $T$ is an indeterminate. Equivalently, $R[IT]$ is the $R$-subalgebra of the polynomial ring $R[T]$ generated by $IT$, with $R[T]$ is given the standard grading $R[T]_d = R \cdot T^d$.

\

\noindent \textsc{Definition:} Let $R$ be a ring and $I$ be an ideal.
The \textbf{associated graded ring} of $I$ is the $\N$-graded ring
\[ \mathrm{gr}_I(R) \ceq \bigoplus_{d\geq 0} (I^d / I^{d+1}) T^d = R/I \oplus (I/I^2) T \oplus (I^2/I^3) T^2 \oplus \cdots \]
with multiplication determined by $(a+I^{d+1} T^d)(b + I^{e+1} T^e) = ab+I^{d+e+1} \, T^{d+e}$ for $a\in I^d$, $b\in I^e$ (and extended by the distributive law).
For an element $r\in R$, its \textbf{initial form} in $\mathrm{gr}_I(R)$ is
\[ r^* \ceq \begin{cases} (r+ I^{d+1})T^d &\text{if} \ r\in I^d \smallsetminus I^{d+1} \\ 0 &\text{if} \ r\in \bigcap_{n\geq 0} I^n.\end{cases}\]

\

\

\noindent \textsc{Artin-Rees Lemma:} Let $R$ be a Noetherian ring, $I$ an ideal of $R$, $M$ a finitely generated module, and $N\subseteq M$ a submodule. Then there is a constant\footnotemark\ $c\geq 0$ such that for all $n\geq c$, we have~${I^{n} M \cap N \subseteq I^{n-c} N}$.

 \end{framed}
 \footnotetext[1]{The constant $c$ depends on $I, M,$ and $N$ but works for all $n$.}
 

 
\begin{enumerate}

\itemA Warmup with Rees rings:
\begin{enumerate}
\itema Let $R$ be a ring and $I$ be an ideal. Show that if $I=(a_1,\dots,a_n)$, then  $R[IT] = R[a_1 T,\dots, a_n T]$.
\itema Let $K$ be a field, $R=K[X,Y]$ and $I=(X,Y)$. Find $K$-algebra generators for $R[IT]$, and find a relation on these generators.
%\itema Let $K$ be a field, $R=K[X,Y]/(XY)$ and $I=(x,y)$, where $x,y$ represent the classes of $X$ and~$Y$, respectively. Find $K$-algebra generators for $R[It]$, and find some relations on these generators.
\end{enumerate}

\solution{
\begin{enumerate}
\itema This follows from the Theorem we showed last time: given a (finite, though this isn't necessary) set of homogeneous elements that generate $R_+$ (in this case, $R[IT]_+$) as an ideal, these elements generate $R$ as an $R_0$-algebra.
\itema The elements $X,Y, XT, YT$ generate. A relation is $X(YT) - Y(XT)$, or $X_1 X_4- X_2 X_3$ in dummy variables. In fact, this is a defining set of relations.
%\itema The elements $X,Y, XT, YT$ still generate. Some relations are $XY, X(YT), Y(XT)$, and these do in fact generate.
\end{enumerate}
}

\itemA Warmup with associated graded rings:
\begin{enumerate}
\itema Convince yourself that the multiplication given in the definition of $\mathrm{gr}_I(R)$ is well-defined. After doing this, do \emph{not} use coset notation for elements of $\gr_I(R)$ and instead write a typical homogeneous element as something like $\overline{r} \,T^d$.
\itema Let $K$ be a field, $R=K[X,Y]$, and $\m=(X,Y)$. Show that $\mathrm{gr}_\m(R)_d \cong R_d$ as $K$-vector spaces, and construct a ring isomorphism $\mathrm{gr}_\m(R) \cong R$.
\itema For the same $R$, show that the map $R \to \mathrm{gr}_\m(R)$ given by $r\mapsto r^*$ is \emph{not} a ring homomorphism.
\itema Let $K$ be a field, $R=K\llbracket X,Y \rrbracket$,  and $\m=(X,Y)$. Show\footnote{Yes, the brackets changed. This is not a typo!} that $\mathrm{gr}_\m(R)\cong K[X,Y]$.
\itema What happens in (b) and (d) if we have $n$ variables instead of $2$?
\end{enumerate}

\solution{
\begin{enumerate}
\itema Let $a\in I^d$ and $b\in I^e$. Then given $a'\in I^{d+1}$ and $b'\in I^{e+1}$, we have $(a +a') (b + b') = ab + a'b+ab'+a'b' \in ab+ I^{d+e+1}$.
\itema Note that $\gr_I(R)_d$ is exactly the vector space $f T^d$ with $f\in R_d$. So ``ignoring'' $T$ is an isomorphism of vector spaces. One checks directly that it is compatible with multiplication by reducing to the case of homogeneous elements.
\itema For example, if $f=X-1$ and $g=1$, then $f^*=-1$, $g^*=1$, but $(f+g)^*=X$.
\itema Note that $\gr_I(R)_d$ is again just the vector space $f T^d$ with $f\in R_d$, and multiplication is the same as in the polynomial case.
\itema The same thing.
\end{enumerate}
}


\itemA Consider the special case of Artin-Rees where $M=R$, and $I=(f)$ and $N=(g)$. 
\begin{enumerate}
\itema What does Artin-Rees say in this setting? Express your answer in terms of ``divides''.
\itema Take $R=\Z$. Does $c=0$ ``work'' for every $f,g\in \Z$? Can you find a sequence of examples requiring arbitrarily large values of $c$?
\end{enumerate}

\solution{
\begin{enumerate}
\itema There is some $c$ such that $f^n | h$ and $g | h$ implies $(f^{n-c} g) | h$.
\itema Take $f=2$ and $g=2^m$. Then $2^n | h$ and $2^m | h$ implies $2^{\max\{m,n\}} | h$. Then $f^{n-c} g = 2^{m+n-c}$. To guarantee this to divide $h$, we must have $c\geq m$.
\end{enumerate}
}


\begin{samepage}
\itemB Proof of Artin-Rees: Let $R$ be a Noetherian ring, and $I$ be an ideal.
\begin{enumerate}
\itemb Explain why $R[IT]$ is a Noetherian ring.
\itemb Let $M=\sum_i R m_i$ be a finitely generated $R$-module. Set $\mathcal{M} \ceq \bigoplus_{n\geq 0} I^n M T^n$. Show that this is a graded $R[IT]$-module, and that $\mathcal{M} = \sum_i R[IT] \cdot m_i$, where in the last equality we consider $m_i$ as the element $m_i T^0\in \mathcal{M}_0$.
\itemb Given a submodule $N$ of $M$, set $\mathcal{N} \ceq \bigoplus_{n\geq 0} (I^n M \cap N) T^n \subseteq \mathcal{M}$. Show that $\mathcal{N}$ is a graded $R[IT]$-submodule of $\mathcal{M}$.
\itemb Show that there exist $n_1,\dots,n_k\in N$ and $c_1,\dots,c_k\geq 0$ such that $\mathcal{N}= \sum_j R[It]\cdot n_j T^{c_j}$.
\itemb Show that $c\ceq\max\{c_j\}$ satisfies the conclusion of the Artin-Rees Lemma.
\end{enumerate}
\end{samepage}

\solution{
\begin{enumerate}
\itemb Since $I$ is finitely generated, it is a finitely generated algebra over a Noetherian ring.
\itemb First, we check that this is an $R[IT]$-module. It is clearly an additive group. To check that it is closed under the $R[IT]$-action and that this yields a graded action, it suffices to check that $R[IT]_d \cdot \mathcal{M}_e \subseteq \mathcal{M}_{d+e}$. To see it, take $r T^d$ with $r\in I^d$ and $m T^e$ with $m\in I^e M$; then the action yields $rm T^{d+e}$ and $rm\in I^d(I^e M) = I^{d+e}M$, so $rm T^{d+e}\in \mathcal{M}_{d+e}$, as required. 

Clearly $m_i\in \mathcal{M}$, so $\sum_i R[IT] \cdot m_i \subseteq \mathcal{M}$. Now we check that this generates. It suffices to check that any homogeneous element can be generated by this generating set, so take some $m T^{d} \in \mathcal{M}_d$ with $m\in I^d M$. This means we can write $m=\sum_j a_j u_j$ with $a_j\in I^d$ and $u_j\in M$. Then we can write $u_j = \sum b_{ij} m_i$ for some $b_{ij}\in R$, yielding an expression $m=\sum_i c_i m_i$ with $c_i\in I^d$. Thus, $m=\sum_i (c_i t^d) m_i \in R[IT] \cdot m_i$. 

\itemb It suffices to check that $R[IT]_d \cdot \mathcal{N}_e \subseteq \mathcal{N}_{d+e}$. Take $r T^d$ with ${r\in I^d}$ and ${n t^e}$ with ${n\in (I^e M \cap N)}$. Then $r n\in I^d (I^e M \cap N)$, so $rn \in I^d I^e M = I^{d+e} M$ and $rn \in I^d N \subseteq  N$, and hence $rn\in I^{d+e} M \cap N$. Thus $(r t^d)(n t^e) \in \mathcal{N}_{d+e}$.

\itemb Since $R[IT]$ is Noetherian and $\mathcal{M}$ is finitely generated, so is $\mathcal{N}$. Since it is graded and finitely generated, it can be generated by finitely many homogeneous elements. The statement is just naming them.

\itemb Let $c=\max\{c_j\}$. Take $u\in I^n M \cap N$. Then $u T^n \in \mathcal{N}_n =  \sum_j R[IT]\cdot n_j T^{c_j}$. We can then express $u$ as a homogeneous linear combination of these generators, so ${uT^n = \sum_j (r_j T^{n-c_j}) (n_j T^{c_j})}$. Since $n-c_j  \geq n-c$, we have $r_j\in I^{n-c}$, and each $n_j\in N$, so $u=\sum_j r_j n_j \in I^{n-c} N$. Moving over the $c$, we obtain the statement.
\end{enumerate}}



\itemB Presentations of associated graded rings: Let $R$ be a ring and $I,J$ be ideals. Set $\mathrm{in}_I(J)$ to be the ideal of $\gr_I(R)$ generated by $\{a^* \ | \ a\in J\}$.
\begin{enumerate}
\itemb Show that $\mathrm{gr}_I(R/J) \cong \mathrm{gr}_I(R) / \mathrm{in}(J)$.
\itemb If $J=(f)$ is a principal ideal, show that $\mathrm{in}_I(J) = (f^*)$.
\itemb Is $\mathrm{in}_I((f_1,\dots,f_t))= (f_1^*,\dots,f_t^*)$ in general?
\itemb Compute $\displaystyle \mathrm{gr}_{(x,y,z)}\left(\frac{K \llbracket X,Y,Z \rrbracket}{(X^2+ XY + Y^3 + Z^7)}\right)$.
\end{enumerate}

\solution{}

\itemB Properties of associated graded rings: Let $R$ be a ring and $I$ be an ideal such that $\bigcap_{n\geq 0} I^n = 0$.
\begin{enumerate}
\itemb Show that if $\mathrm{gr}_I(R)$ is a domain, then so is $R$.
\itemb Show that if $\mathrm{gr}_I(R)$ is reduced, then so is $R$.
\itemb What about the converses of these statements?
\end{enumerate}

\solution{}


\itemB Show that for the ideal $I=(X,Y)^2$ in $R=K[X,Y]$, the Rees ring $R[IT]$ has defining relations of degree greater than one.

\solution{}

%\itemB Let $R$ be a ring and $I=(a_1,\dots,a_n)$ be an ideal. A ring of the form $S=R[IT]/(a_i T-1)$ is sometimes called a \emph{monoidal transform} of $R$. Show that $IS$ is a principal ideal of $S$.


%\solution{}

\vfill




\end{enumerate}

\end{document}
