\documentclass[12pt]{amsart}


\usepackage{times}
\usepackage[margin=1in]{geometry}
\usepackage{amsmath,amssymb,multicol,graphicx,framed,ifthen,color,xcolor,stmaryrd,enumitem,colonequals,bbm}
\usepackage[all]{xy}

\usepackage[outline]{contour}
\contourlength{.4pt}
\contournumber{10}
\newcommand{\Bold}[1]{\contour{black}{#1}}


\definecolor{chianti}{rgb}{0.6,0,0}
\definecolor{meretale}{rgb}{0,0,.6}
\definecolor{leaf}{rgb}{0,.35,0}
\newcommand{\Q}{\mathbb{Q}}
\newcommand{\N}{\mathbb{N}}
\newcommand{\Z}{\mathbb{Z}}
\newcommand{\R}{\mathbb{R}}
\newcommand{\C}{\mathbb{C}}
\newcommand{\e}{\varepsilon}
\newcommand{\m}{\mathfrak{m}}
\newcommand{\p}{\mathfrak{p}}
\newcommand{\q}{\mathfrak{q}}
\newcommand{\ord}{\mathrm{ord}}
\newcommand{\ann}{\mathrm{ann}}
\newcommand{\Min}{\mathrm{Min}}
\newcommand{\Max}{\mathrm{Max}}
\newcommand{\Spec}{\mathrm{Spec}}
\newcommand{\Ass}{\mathrm{Ass}}
\renewcommand{\1}{\mathbbm{1}}
\newcommand{\cZ}{\mathcal{Z}}

\newcommand{\inv}{^{-1}}
\newcommand{\dabs}[1]{\left| #1 \right|}
\newcommand{\ds}{\displaystyle}
\newcommand{\solution}[1]{\ifthenelse {\equal{\displaysol}{1}} {\begin{framed}{\color{meretale}\noindent #1}\end{framed}} { \ }}
\newcommand{\showsol}[1]{\def\displaysol{#1}}
\newcommand{\rsa}{\rightsquigarrow}

\newcommand\itemA{\stepcounter{enumi}\item[{\Bold{(\theenumi)}}]}
\newcommand\itemB{\stepcounter{enumi}\item[(\theenumi)]}
\newcommand\itemC{\stepcounter{enumi}\item[{\it{(\theenumi)}}]}
\newcommand\itema{\stepcounter{enumii}\item[{\Bold{(\theenumii)}}]}
\newcommand\itemb{\stepcounter{enumii}\item[(\theenumii)]}
\newcommand\itemc{\stepcounter{enumii}\item[{\it{(\theenumii)}}]}
\newcommand\itemai{\stepcounter{enumiii}\item[{\Bold{(\theenumiii)}}]}
\newcommand\itembi{\stepcounter{enumiii}\item[(\theenumiii)]}
\newcommand\itemci{\stepcounter{enumiii}\item[{\it{(\theenumiii)}}]}
\newcommand\ceq{\colonequals}

\DeclareMathOperator{\res}{res}
\setlength\parindent{0pt}
%\usepackage{times}

%\addtolength{\textwidth}{100pt}
%\addtolength{\evensidemargin}{-45pt}
%\addtolength{\oddsidemargin}{-60pt}

\pagestyle{empty}
%\begin{document}\begin{itemize}

%\thispagestyle{empty}

\usepackage[hang,flushmargin]{footmisc}


\begin{document}
\showsol{0}
	
	\thispagestyle{empty}
	
	\section*{\S6.27: Primary ideals}
	
	\begin{framed}

\noindent \textsc{Definition:} A proper ideal $I$ is \textbf{primary} if $rs\in I$ implies $r\in \sqrt{I}$ or $s\in I$. We say that $I$ is \textbf{$\p$-primary} if it is primary and $\sqrt{I}=\p$.

\

\noindent \textsc{Lemma:} Let $R$ be a Noetherian ring and $I$ an ideal. The following are equivalent:
\begin{enumerate}
\item[(i)] $I$ is primary;
\item[(ii)] Every zerodivisor on $R/I$ is nilpotent;
\item[(iii)] $\Ass_R(R/I)$ is a singleton.
\end{enumerate}

\


\noindent \textsc{Definition:} A \textbf{primary decomposition} of an ideal $I$ is an expression of the form
\[ I = Q_1 \cap \cdots Q_n\]
where each $Q_i$ is a primary ideal.

\


\noindent \textsc{Definition:} A proper ideal $I$ is \textbf{irreducible} if $I=J_1\cap J_2$ for some ideals $J_1,J_2$ implies $I=J_1$ or $I=J_2$.


\


\noindent \textsc{Theorem (Existence of primary decomposition):} Let $R$ be a Noetherian ring.
\begin{enumerate}
\item Every irreducible ideal $I$ is primary.
\item Every ideal can be written as a finite intersection of irreducible ideals.
\end{enumerate}
Hence, every ideal can be written as a finite intersection of primary ideals.
\end{framed}


\begin{enumerate}

\itemA Primary ideals
\begin{enumerate}
\itema Use the definition to show that a prime ideal is primary.
\itema Use the definition to show that the radical of a primary ideal is prime.
\itema  Use the definition to show that for the ideal $I=(X^2,XY)$ in $R=\Q[X,Y]$, $\sqrt{I}$ is prime but $I$ is not primary.
\itema  Use the definition and part (b) above to show that if $R$ is a UFD, then a proper principal ideal $(f)$ is primary if and only if it is not  generated\footnote{Note that if $(f)$ is not generated by a power of a prime element, then $f$ has nonassociate irreducible factors.} by a power of a prime element.
\itema Use the Lemma to show that if $\sqrt{I}=\m$ is a maximal ideal, then $I$ is $\m$-primary.
\end{enumerate}



\solution{
\begin{enumerate}
\itema A prime ideal is radical in particular, so if $Q$ is prime and $rs\in Q$ and $r\notin \sqrt{Q}=Q$, then $s\in Q$.
\itema Let $Q$ be primary. Suppose that $rs\in \sqrt{Q}$. Then for some $n$, $r^n s^n = (rs)^n\in Q$ so either $r^n\in \sqrt{Q}$ (whence $r\in \sqrt{Q}$) or $s^n\in Q$ (whence $s\in \sqrt{Q}$).
\itema We have computed $\sqrt{I}=(X)$ earlier, so $\sqrt{I}$ is prime. This ideal is not primary since $XY\in I$ but $X\notin I$ and $Y\notin \sqrt{I}$.
\itema Suppose that $(f)=(r^n)$ for some irreducible $r$. If $xy\in (f)$, then $r^n | (xy)$, so either $r|x$ (whence $x\in \sqrt{(f)}$) or $r^n | y$ (whence $y\in (f)$). Conversely, suppose that $f$ admits a factorization $f=gh$ with $g,h$ coprime. Then $gh\in (f)$, but $g\notin \sqrt{(f)}$ and $h\notin (f)$.
\itema If $\sqrt{I}=\m$, then $V(I)=\{\m\}$ and since $\varnothing \neq \Ass_R(R/I) \subseteq V(I)$, we must have $\Ass_R(R/I)=\{\m\}$.
\end{enumerate}
 }

\itemA Primary decompositions
\begin{enumerate}
\itema Let $n$ be an integer. Show that if $n=\pm p_1^{e_1} \cdots p_m^{e_m}$ is the prime factorization of $n$, then
\[ (n) = (p_1^{e_1}) \cap \cdots \cap (p_m^{e_m})\]
is a primary decomposition of $(n)$ in $\Z$.
\itema Let $R$ be a Noetherian ring and $I$ be a radical ideal. Give a recipe for a primary decomposition of $I$ in terms of other named things pertaining to $I$.
\end{enumerate}

\solution{
\begin{enumerate}
\itema The equality is clear, and each $(p_i^{m_i})$ is primary by above.
\itema $I = \bigcap_{\p \in \Min(I)} \p$.
\end{enumerate}
}


\itemA Prove\footnote{Hint: For (ii)$\Leftrightarrow$(iii), recall that the set elements of $R$ that are zerodivisors modulo $I$ is the union of the associated primes of $R/I$ and the set of elements that are nilpotent modulo $I$ is the intersection of minimal primes of $I$.} the Lemma.

\solution{The equivalence between (i) and (ii) is straightforward. For the (ii)$\Leftrightarrow$(iii), recall that the set elements of $R$ that are zerodivisors modulo $I$ is the union of the associated primes of $R/I$ and the set of elements that are nilpotent modulo $I$ is the intersection of minimal primes of $I$. Every minimal prime of $I$ is associated. Thus, if every zerodivisor is nilpotent, then there must be one associated prime (because the union of two distinct sets is always larger than the intersection. Conversely, if there is only one associated prime, the union is the intersection and (ii) holds.} 

\begin{samepage}
\itemA Proof of Existence of Primary Decompositions: 
\begin{enumerate}
\itema Prove\footnote{Imitate the proof of finiteness of minimal primes.} part (2) of the Theorem.
\itema Suppose that $xy\in Q$ with $x\notin Q$ and $y\notin \sqrt{Q}$. Explain why the there is some $n\geq 1$ such that $(Q:y^n) = (Q:y^{n+1})$.
\itema Show that $Q = (Q,x) \cap (Q,y^n)$ and deduce part (1) of the Theorem.
\end{enumerate}
\end{samepage}

\solution{
\begin{enumerate}
\itema Consider the collection of ideals that are not finite intersections of irreducible ideals. If one exists, by Noetherianity, there is a maximal element $I$. Such $I$ is necessarily reducible, so $I= J_1 \cap J_2$, with $J_1,J_2 \supsetneqq I$. By maximality, $J_1,J_2$ are finite intersections of irreducible ideals. Substituting in those expressions gives an expression for $I$ as a finite intersection of irreducible ideals. 
\itema For each $n$, we have $(Q:y^n) \subseteq (Q:y^{n+1})$ since $fy^{n}\in Q$ implies $fy^{n+1} = y fy^n \in Q$. Thus, these ideals form an ascending chain, which must stabilize.
\itema Clearly $Q  \subseteq (Q,x) \cap (Q,y^n)$. Write $f=q+ax = q' + by^n$ with $q,q'\in Q$. Then $yf = qy + axy \in Q$, and $yf = q'y+by^{n+1}$, so $by^{n+1}\in Q$. Thus $b\in (Q:y^{n+1})= (Q:y^n)$, so $by^n\in Q$, but then $f\in Q$. We have shown that if $Q$ is not primary, then it is reducible.
\end{enumerate}
}

\itemB More examples: Let $K$ be a field.
\begin{enumerate}
\item Show that $(X^2,XY,Y^2)\subseteq K[X,Y]$ is primary but not irreducible.
\item Show that $(X^2, XY, Y^3)$ is primary, but not a power of a prime.
\item Show that $(X^2,XY)^2 \subseteq  K[X^2,XY,Y^2]$ is a power of a prime but not primary.
\end{enumerate}

\solution{
\begin{enumerate}
\item The radical of $(X^2,XY,Y^2)$ is $(X,Y)$, which is maximal, so this is primary. However, $(X^2,XY,Y^2)=(X^2,Y)\cap (X,Y^2)$.
\item As above, the radical is $(X,Y)$. Thus, if it is a power of a prime, that must be $(X,Y)$, since the radical of a power of an ideal agree with the radical of the same ideal. Note that $(X,Y)^2 = (X^2,XY,Y^2) \supsetneqq (X^2, XY, Y^3) \supsetneqq (X,Y)^3$, so this cannot be a power of $(X,Y)$.
\item Show that $(X^2,XY)^2 \subseteq  K[X^2,XY,Y^2]$ is a power of a prime but not primary.
\end{enumerate}
}


\

\itemB Let $R$ be a Noetherian ring and $\p$ a prime ideal. Show that there is an order-preserving bijection
\[ \{ \text{$\p$-primary ideals of $R$} \} \leftrightarrow  \{ \text{ideals of $(R_\p, \p R_\p)$ with radical $\p R_\p$}\}.\]

\

\itemB Let $R$ be a Noetherian ring. Show that $I$ is irreducible if and only if it is primary (with radical $\p$) and $\displaystyle \frac{I R_\p : \p R_\p}{I R_\p}$ is a one-dimensional $R_\p/\p R_\p$-vectorspace. 


\end{enumerate}

\end{document}
