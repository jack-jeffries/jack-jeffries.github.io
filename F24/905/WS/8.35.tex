 \documentclass[12pt]{amsart}


\usepackage{times}
\usepackage[margin=.7in]{geometry}
\usepackage{amsmath,amssymb,multicol,graphicx,framed,ifthen,color,xcolor,stmaryrd,enumitem,colonequals,bbm}
\usepackage[all]{xy}

\usepackage[outline]{contour}
\contourlength{.4pt}
\contournumber{10}
\newcommand{\Bold}[1]{\contour{black}{#1}}


\definecolor{chianti}{rgb}{0.6,0,0}
\definecolor{meretale}{rgb}{0,0,.6}
\definecolor{leaf}{rgb}{0,.35,0}
\newcommand{\Q}{\mathbb{Q}}
\newcommand{\N}{\mathbb{N}}
\newcommand{\Z}{\mathbb{Z}}
\newcommand{\R}{\mathbb{R}}
\newcommand{\C}{\mathbb{C}}
\newcommand{\e}{\varepsilon}
\newcommand{\m}{\mathfrak{m}}
\newcommand{\p}{\mathfrak{p}}
\newcommand{\q}{\mathfrak{q}}
\newcommand{\ord}{\mathrm{ord}}
\newcommand{\ann}{\mathrm{ann}}
\newcommand{\hgt}{\mathrm{height}}
\newcommand{\Min}{\mathrm{Min}}
\newcommand{\Max}{\mathrm{Max}}
\newcommand{\Spec}{\mathrm{Spec}}
\newcommand{\Ass}{\mathrm{Ass}}
\renewcommand{\1}{\mathbbm{1}}
\newcommand{\cZ}{\mathcal{Z}}

\newcommand{\inv}{^{-1}}
\newcommand{\dabs}[1]{\left| #1 \right|}
\newcommand{\ds}{\displaystyle}
\newcommand{\solution}[1]{\ifthenelse {\equal{\displaysol}{1}} {\begin{framed}{\color{meretale}\noindent #1}\end{framed}} { \ }}
\newcommand{\showsol}[1]{\def\displaysol{#1}}
\newcommand{\rsa}{\rightsquigarrow}

\newcommand\itemA{\stepcounter{enumi}\item[{\Bold{(\theenumi)}}]}
\newcommand\itemB{\stepcounter{enumi}\item[(\theenumi)]}
\newcommand\itemC{\stepcounter{enumi}\item[{\it{(\theenumi)}}]}
\newcommand\itema{\stepcounter{enumii}\item[{\Bold{(\theenumii)}}]}
\newcommand\itemb{\stepcounter{enumii}\item[(\theenumii)]}
\newcommand\itemc{\stepcounter{enumii}\item[{\it{(\theenumii)}}]}
\newcommand\itemai{\stepcounter{enumiii}\item[{\Bold{(\theenumiii)}}]}
\newcommand\itembi{\stepcounter{enumiii}\item[(\theenumiii)]}
\newcommand\itemci{\stepcounter{enumiii}\item[{\it{(\theenumiii)}}]}
\newcommand\ceq{\colonequals}

\DeclareMathOperator{\res}{res}
\setlength\parindent{0pt}
%\usepackage{times}

%\addtolength{\textwidth}{100pt}
%\addtolength{\evensidemargin}{-45pt}
%\addtolength{\oddsidemargin}{-60pt}

\pagestyle{empty}
%\begin{document}\begin{itemize}

%\thispagestyle{empty}

\usepackage[hang,flushmargin]{footmisc}


\begin{document}
\showsol{0}
	
	\thispagestyle{empty}
	
	\section*{\S8.35: Artinian rings and modules}
	
	\begin{framed}
	
	\noindent	 \textsc{Definition:} A ring $R$ is \textbf{Artinian} if every descending chain of ideals $I_1 \supseteq I_2 \supseteq I_3 \supseteq \cdots$ eventually stabilizes: i.e., there is some $N$ such that $I_n=I_N$ for all $n\geq N$. A module is \textbf{Artinian} if every descending chain of submodules $N_1 \supseteq N_2 \supseteq N_3 \supseteq \cdots$ eventually stabilizes.
	
	\
	
	\noindent \textsc{Proposition:} Let $R$ be a ring and $M$ be a module.
	\begin{enumerate}
	\item $M$ is Artinian if and only if every nonempty family $\mathcal{S}$ of submodules of $M$ has a minimal element.
	\item If $N$ is a submodule of $M$, then $M$ is Artinian if and only if $N$ and $M/N$ are both Artinian.
	\end{enumerate}
	
\
	
		\noindent	\textsc{Theorem:} Let $R$ be a ring. The following are equivalent:
		\begin{enumerate}
		\item $R$ is Noetherian of dimension zero,
		\item $R$ is a finite product of Noetherian local rings of dimension zero,
		\item $R$ is a finite length $R$-module,
		\item $R$ is Artinian.
		\end{enumerate}
	
		\end{framed}


\begin{enumerate}
\itemA Jordan-H\"older review: Explain why a finite length module is Artinian.

\solution{Given such a chain, the length of each successive submodule is smaller, so any such chain can have length at most the length of $M$.
}


\itemA Proof of the Theorem, the useful part: Prove\footnote{Hint: In the setting of (i), note that $V(0)$ is a finite set of maximal ideals, and use a homework problem.} that (i)$\Rightarrow$(ii)$\Rightarrow$(iii)$\Rightarrow$(iv).

\solution{For (i)$\Rightarrow$(ii), if $R$ is Noetherian of dimension zero, then $V(0)$ consists of maximal ideals by the dimension assumption, and is finite since every such ideal is minimal, and Noetherian rings have finitely many minimal primes. Then, by a homework problem, since $V(0)$ is a finite set of maximal ideals, $R=R/0 \cong R/Q_1 \times \cdots \times R/Q_t$ where each $Q_t$ is primary to a maximal ideal, so each $R/Q_i$ is local of dimension zero (and Noetherian, since it is a quotient of a Noetherian ring).

For (ii)$\Rightarrow$(iii), we have that $R$ is a fintiely generated $R$-module whose support is a finite set of maximal ideals, so $R$ has finite length.

(iii)$\Rightarrow$(iv) follows from the previous problem.
}

\itemB Vector spaces:
\begin{enumerate} 
\itemb Let $K$ be a field and $V$ be a vector space. Show that $V$ is finite-dimensional if and only if $V$ is Noetherian if and only if $V$ is Artinian.
\itemb Let $(R,\m,k)$ be a local ring, and $M$ be an $R$-module such that $\m M=0$. Show that 
$M$ has finite length  if and only if $M$ is Noetherian  if and only if $M$ is Artinian.
\end{enumerate}

\solution{
\begin{enumerate} 
\itema Finite-dimensional is the same as finite length, and finite length implies Noetherian and Artinian in general. Conversely, in an infinite dimensional vector space, one can take a basis and construct infinite ascending or descending chains of subsets of the basis, and the spans form infinite ascending or descending chains of subspaces, so $V$ is neither Artinian nor Noetherian.
\itema One can identify such a module with a $k$-module and apply part (1).
\end{enumerate}
}

\itemB Proof of the Theorem, the fun part: Suppose that $R$ is Artinian.
\begin{enumerate}
\itemb First, we show $\dim(R)=0$: By way of contradiction, suppose there is nonmaximal prime~$\p$, so there is some nonzero nonunit $a\in R/\p$. Consider the descending chain of ideals 
\[ (a) \supseteq (a^2) \supseteq (a^3) \supseteq\cdots.\] 
to obtain a contradiction.
\itemb Second, we show that $\Max(R)$ is finite: By way of contradiction, if $\m_1,\m_2,\m_3,\dots$ are distinct maximal ideals, consider the descending chain of ideals
\[ \m_1  \supseteq \m_1 \cap \m_2  \supseteq \m_1 \cap \m_2 \cap \m_3\supseteq \cdots.\] 
\itemb Third, we show that $R$ is a finite product of Artinian local rings of dimension zero: Apply a homework problem.
\itemb Fourth, we show that an Artinian local ring $(R,\m,k)$ has finite length: Consider the chain
\[ \m \supseteq \m^2 \supseteq \m^3 \supseteq\cdots.\]
What do we deduce? Why do we \emph{not} immediate deduce that $\m^n=0$ for some $n$ from NAK?
\itemb Fourth continued: If $\m^n=\m^{n+1}$ and $\m^n\neq 0$, consider $\mathcal{S}=\{ \text{ideals} \ J \ | \ J\m^n \neq 0\}$. Explain why $\mathcal{S}$ has a minimal element $I$, and $I$ is principal. Now deduce that $\m^n=0$.
\itemb Fourth continueder: Explain why $\m^i / \m^{i+1}$ has finite length for each $i$. Deduce that $R$ has finite length.
\itemb  Complete the proof.
\end{enumerate}
\solution{
\begin{enumerate} 
\itema Finite-dimensional is the same as finite length, and finite length implies Noetherian and Artinian in general. Conversely, in an infinite dimensional vector space, one can take a basis and construct infinite ascending or descending chains of subsets of the basis, and the spans form infinite ascending or descending chains of subspaces, so $V$ is neither Artinian nor Noetherian.
\itema One can identify such a module with a $k$-module and apply part (1).
\end{enumerate}
}

\itemB Artinian Modules:

\begin{enumerate}
\item Let $K$ be a field. Show that the $K[X]$-module $K[X]_X / K[X]$ is Artinian but not finite length.
\item Show that an $R$-module $M$ has finite length if and only if it is Artinian and Noetherian.
\item Let $R$ be a Noetherian ring. Show that if $M$ is an Artinian module, then $\Ass_R(M)\subseteq \Max(R)$.
\item Let $R$ be a Noetherian $\N$-graded ring with $R_0=K$ a field. Show that if $M$ is an Artinian $\Z$-graded module, then there is some $n$ such that $M_{\geq n}=0$.
\item Let $R$ be a Noetherian ring. If $M$ is an Artinian module, must $\Ass_R(M)$ be finite?\end{enumerate}
\end{enumerate}
\end{document}
