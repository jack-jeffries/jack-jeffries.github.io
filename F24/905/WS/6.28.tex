\documentclass[12pt]{amsart}


\usepackage{times}
\usepackage[margin=1in]{geometry}
\usepackage{amsmath,amssymb,multicol,graphicx,framed,ifthen,color,xcolor,stmaryrd,enumitem,colonequals,bbm}
\usepackage[all]{xy}

\usepackage[outline]{contour}
\contourlength{.4pt}
\contournumber{10}
\newcommand{\Bold}[1]{\contour{black}{#1}}


\definecolor{chianti}{rgb}{0.6,0,0}
\definecolor{meretale}{rgb}{0,0,.6}
\definecolor{leaf}{rgb}{0,.35,0}
\newcommand{\Q}{\mathbb{Q}}
\newcommand{\N}{\mathbb{N}}
\newcommand{\Z}{\mathbb{Z}}
\newcommand{\R}{\mathbb{R}}
\newcommand{\C}{\mathbb{C}}
\newcommand{\e}{\varepsilon}
\newcommand{\m}{\mathfrak{m}}
\newcommand{\p}{\mathfrak{p}}
\newcommand{\q}{\mathfrak{q}}
\newcommand{\ord}{\mathrm{ord}}
\newcommand{\ann}{\mathrm{ann}}
\newcommand{\Min}{\mathrm{Min}}
\newcommand{\Max}{\mathrm{Max}}
\newcommand{\Spec}{\mathrm{Spec}}
\newcommand{\Ass}{\mathrm{Ass}}
\renewcommand{\1}{\mathbbm{1}}
\newcommand{\cZ}{\mathcal{Z}}

\newcommand{\inv}{^{-1}}
\newcommand{\dabs}[1]{\left| #1 \right|}
\newcommand{\ds}{\displaystyle}
\newcommand{\solution}[1]{\ifthenelse {\equal{\displaysol}{1}} {\begin{framed}{\color{meretale}\noindent #1}\end{framed}} { \ }}
\newcommand{\showsol}[1]{\def\displaysol{#1}}
\newcommand{\rsa}{\rightsquigarrow}

\newcommand\itemA{\stepcounter{enumi}\item[{\Bold{(\theenumi)}}]}
\newcommand\itemB{\stepcounter{enumi}\item[(\theenumi)]}
\newcommand\itemC{\stepcounter{enumi}\item[{\it{(\theenumi)}}]}
\newcommand\itema{\stepcounter{enumii}\item[{\Bold{(\theenumii)}}]}
\newcommand\itemb{\stepcounter{enumii}\item[(\theenumii)]}
\newcommand\itemc{\stepcounter{enumii}\item[{\it{(\theenumii)}}]}
\newcommand\itemai{\stepcounter{enumiii}\item[{\Bold{(\theenumiii)}}]}
\newcommand\itembi{\stepcounter{enumiii}\item[(\theenumiii)]}
\newcommand\itemci{\stepcounter{enumiii}\item[{\it{(\theenumiii)}}]}
\newcommand\ceq{\colonequals}

\DeclareMathOperator{\res}{res}
\setlength\parindent{0pt}
%\usepackage{times}

%\addtolength{\textwidth}{100pt}
%\addtolength{\evensidemargin}{-45pt}
%\addtolength{\oddsidemargin}{-60pt}

\pagestyle{empty}
%\begin{document}\begin{itemize}

%\thispagestyle{empty}

\usepackage[hang,flushmargin]{footmisc}


\begin{document}
\showsol{0}
	
	\thispagestyle{empty}
	
	\section*{\S6.28: Uniqueness of primary decompositions}
	
	\begin{framed}

\noindent \textsc{Definition:} A \textbf{minimal primary decomposition} of an ideal $I$ is a primary decomposition
\[ I = Q_1 \cap \cdots \cap Q_n\]
such that $Q_i \not\supseteq \bigcap_{j\neq i} Q_j$, and $\sqrt{Q_i} \neq \sqrt{Q_j}$ for $i\neq j$.

\

\noindent \textsc{Theorem (First uniqueness theorem for primary decomposition):} Let $R$ be a Noetherian ring and $I$ an ideal. Let \[I = Q_1 \cap \cdots \cap Q_n\] be a minimal primary decomposition of $I$. Then \[\{ \sqrt{Q_1},\dots,\sqrt{Q_n} \} = \Ass_R(R/I).\]
In particular, the set of primes occurring as the radicals of the primary components are uniquely determined.


\

\noindent \textsc{Theorem (Second uniqueness theorem for primary decomposition):} Let $R$ be a Noetherian ring and $I$ an ideal. Let \[I = Q_1 \cap \cdots \cap Q_n\] be a minimal primary decomposition of $I$. Suppose that $\p =\sqrt{Q_i}$ is a \emph{minimal} prime of $I$. Then $Q_i = I R_{\p} \cap R$.
In particular, the primary components corresponding to the minimal primes are uniquely determined.


\

\noindent \textsc{Lemma:} Let $I_1,\dots,I_t$ be ideals. Then
\begin{enumerate}
\item for any multiplicatively closed set $W$, $W^{-1}(I_1 \cap \cdots \cap I_t) = W^{-1} I_1 \cap \cdots \cap W^{-1}I_t$.
\item $\Ass_R\left(R/\bigcap_{i=1}^t I_i\right) \subseteq \bigcup_{i=1}^t \Ass_R(R/I_i)$.
\end{enumerate}
\end{framed}


\begin{enumerate}

\itemA Uniqueness theorems:  
\begin{enumerate}
\itema Let $K$ be a field, $R=K[X,Y]$ a polynomial ring, and $I=(X^2,XY)$. Verify\footnote{You can take for granted that in each case the intersection is $I$, but explain why the ideals are primary and the minimality hypotheses hold.} that $I = (X) \cap (X^2,Y) = (X) \cap (X^2,XY,Y^2)$ gives two different minimal primary decompositions of $I$.
\itema In the previous part, which aspects of the decomposition are the same, and which are different. Compare with the uniqueness theorems.
\itema Use the uniqueness theorems to explain why, for $n\in \Z$ with prime factorization $n=\pm p_1^{e_1} \cdots p_m^{e_m}$, the \emph{only}\footnote{We don't care about the order.} minimal primary decomposition of $(n)$ is
\[ (n) = (p_1^{e_1}) \cap \cdots \cap (p_m^{e_m}).\]
\end{enumerate}

\solution{
\begin{enumerate}
\itema $(X)$ is prime, hence primary. $(X^2,Y)$ and $ (X^2,XY,Y^2)$ both have radical $(X,Y)$, which is maximal, so they are primary. In each case we have different radicals and neither component contained in the other.
\itema In both cases the radicals of the primes are the same, and the $(X)$-component are the same.
\itema For any such decomposition, the prime ideals occurring are the same, since each prime is minimal, the the components are the same.
\end{enumerate}
}

\itemA Minimal primary decompositions: Let $R$ be a Noetherian ring.
\begin{enumerate}
\itema Use the Lemma to explain why a finite intersection of $\p$-primary ideals is $\p$-primary.
\itema Explain how to turn a general $I = Q_1 \cap \cdots \cap Q_m$  primary decomposition into a minimal primary decomposition.
\end{enumerate}

\solution{
\begin{enumerate}
\itema Because $\p$-primary is equivalent to $\Ass_R(R/I)=\{\p\}$.
\itema Intersect all of the $Q_i$'s with the same radical to get a decomposition satisfying the second condition. Then remove any component that is contained in the intersection of the others to satisfy the first condition.
\end{enumerate}

}


\itemA Proof of Second Uniqueness Theorem: 
\begin{enumerate}
\itema Use the definition of primary to show that if $Q$ is $\p$-primary, then $Q R_\p \cap R = Q$.
\itema Show\footnote{One possibility is to consider the support of $R/Q$.} that if $Q$ is $\q$-primary and $\q \not\subseteq \p$, then $Q R_\p = R_\p$.
\itema Let $R$ be Noetherian and $I=Q_1 \cap \cdots \cap Q_n$ be a minimal primary decomposition, and $\p = \sqrt{Q_i}$ a minimal prime of $I$. Use the Lemma to show that $I R_{\p} = Q_i R_\p$.
\itema Complete the proof.
\end{enumerate}
\solution{
\begin{enumerate}
\itema Clearly $Q\subseteq Q R_\p \cap R$. Let $r\in Q R_\p \cap R$, so there is some $q\in Q$ and $w \notin \p$ such that $\frac{q}{w}=\frac{r}{1}\in R_\p$. This means there is some $v\notin \p$ such that $v(q-rw)=0$ in $R$; i.e., $vwr=qv$, so in particular there is some $u\notin \p$ such that $ur\in Q$. By definition of primary, $r\in Q$.
\itema We have $\Supp(R/Q)=V(Q) = V(\q)$. If $\p \not\subseteq \q$, then $\p\noint V(\q)$, so $(R/Q)_\p=0$ and $R_\p = QR_\p$.
\itema We have $IR_\p = Q_1 R_\p \cap \cdots \cap Q_n R_\p$. By the previous part, each term on the right is all of $R_\q$ except $Q_i R_\p$.
\itema Follows from part (1).
\end{enumerate}
}

\itemB Proof of First Uniqueness Theorem: Let $R$ be Noetherian and $I=Q_1 \cap \cdots \cap Q_n$ be a minimal primary decomposition.
\begin{enumerate}
\itemb Use the Lemma to prove that $\Ass_R(R/I) \subseteq \{\sqrt{Q_1},\dots,\sqrt{Q_n}\}$.
\itemb Set $J_i = \bigcap_{j\neq i} Q_j$. Explain why it suffices to show that $\Ass_R(J_i/I) =\{ \sqrt{Q_i}\}$ to establish the other containment.
\itemb Let $\q$ be an associated prime of $J_i/I$ and $r\in R$ such that $\overline{r}\in J_i/I$ is a witness (and in particular, nonzero). Show that $Q_i \subseteq \q$ and deduce that $\sqrt{Q_i} \subseteq \q$.
\itemb Use the definition of primary to show that $\q \subseteq \sqrt{Q_i}$, and conclude the proof.
\end{enumerate}
\solution{
\begin{enumerate}
\itemb Yes, it is immediate from the lemma.
\itemb Because $J_i/I \subseteq R/I$ so $\Ass_R(J_i/I) \subseteq \Ass_R(R/I)$.
\itemb We have $Q_i r \subseteq Q_i \cap J_i \subseteq I$, so $Q_i \subseteq \ann_R(\overline{r}) = \q$. Since $\sqrt{Q_i}$ is the unique minimal prime of $Q_i$ and $\q$ is a prime containing $Q_i$, we have $\q \supseteq \sqrt{Q_i}$.
\itemb Let $q\in \q$, so $q r \in I \subseteq Q_i$. Since $\overline{r}\neq 0$, we have $r\notin Q_i$, so by definition of primary, $q\in \sqrt{Q_i}$. Thus $\q\subseteq \sqrt{Q_i}$. This shows that $\sqrt{Q_i}=\q$ is an associated prime of $J_i/I$ and hence of $R/I$.
\end{enumerate}
}

\itemB Prove the Lemma.

\

\itemB Let $R$ be a Noetherian ring, and $I$ be an ideal. Consider a collection of minimal primary decompositions of $I$:
\[ I = \q_{1,\alpha} \cap \cdots \cap \q_{s,\alpha}, \quad \alpha\in \Lambda\]
where, for each $\alpha$,  $\sqrt{\q_{i,\alpha}} = \p_i$.
\begin{enumerate}
	\item Suppose that $\p_j$ is not contained in any other associated prime of $I$, and let ${W=R\smallsetminus \bigcup_{i\neq j} \p_i}$. Find some minimal primary decompositions of $I(W^{-1} R) \cap R$.
	\item Show (by induction on $s$) that if we take components $\q_{1,\alpha_1} ,\dots, \q_{s,\alpha_s}$ from different primary decompositions of $I$, that we can put them together to get a primary decomposition of $I$; namely $I=\q_{1,\alpha_1} \cap \cdots \cap \q_{s,\alpha_s}$.
\end{enumerate}



\end{enumerate}

\end{document}
