\documentclass[12pt]{amsart}


\usepackage{times}
\usepackage[margin=1in]{geometry}
\usepackage{amsmath,amssymb,multicol,graphicx,framed,ifthen,color,xcolor,stmaryrd,enumitem,colonequals,bbm}


\usepackage[outline]{contour}
\contourlength{.4pt}
\contournumber{10}
\newcommand{\Bold}[1]{\contour{black}{#1}}


\definecolor{chianti}{rgb}{0.6,0,0}
\definecolor{meretale}{rgb}{0,0,.6}
\definecolor{leaf}{rgb}{0,.35,0}
\newcommand{\Q}{\mathbb{Q}}
\newcommand{\N}{\mathbb{N}}
\newcommand{\Z}{\mathbb{Z}}
\newcommand{\R}{\mathbb{R}}
\newcommand{\C}{\mathbb{C}}
\newcommand{\e}{\varepsilon}
\newcommand{\m}{\mathfrak{m}}
\newcommand{\p}{\mathfrak{p}}
\newcommand{\ord}{\mathrm{ord}}
\newcommand{\1}{\mathbbm{1}}

\newcommand{\inv}{^{-1}}
\newcommand{\dabs}[1]{\left| #1 \right|}
\newcommand{\ds}{\displaystyle}
\newcommand{\solution}[1]{\ifthenelse {\equal{\displaysol}{1}} {\begin{framed}{\color{meretale}\noindent #1}\end{framed}} { \ }}
\newcommand{\showsol}[1]{\def\displaysol{#1}}
\newcommand{\rsa}{\rightsquigarrow}

\newcommand\itemA{\stepcounter{enumi}\item[{\Bold{(\theenumi)}}]}
\newcommand\itemB{\stepcounter{enumi}\item[(\theenumi)]}
\newcommand\itemC{\stepcounter{enumi}\item[{\it{(\theenumi)}}]}
\newcommand\itema{\stepcounter{enumii}\item[{\Bold{(\theenumii)}}]}
\newcommand\itemb{\stepcounter{enumii}\item[(\theenumii)]}
\newcommand\itemc{\stepcounter{enumii}\item[{\it{(\theenumii)}}]}
\newcommand\itemai{\stepcounter{enumiii}\item[{\Bold{(\theenumiii)}}]}
\newcommand\itembi{\stepcounter{enumiii}\item[(\theenumiii)]}
\newcommand\itemci{\stepcounter{enumiii}\item[{\it{(\theenumiii)}}]}
\newcommand\ceq{\colonequals}

\DeclareMathOperator{\res}{res}
\setlength\parindent{0pt}
%\usepackage{times}

%\addtolength{\textwidth}{100pt}
%\addtolength{\evensidemargin}{-45pt}
%\addtolength{\oddsidemargin}{-60pt}

\pagestyle{empty}
%\begin{document}\begin{itemize}

%\thispagestyle{empty}

\usepackage[hang,flushmargin]{footmisc}


\begin{document}
\showsol{1}
	
	\thispagestyle{empty}
	
	\section*{\S4.15: Noether normalization and Zariski's Lemma}	

\begin{framed}

\noindent \textsc{Noether normalization:} Let $K$ be a field, and $R$ be a finitely-generated $K$-algebra. Then there exists a finite\footnotemark\,set of elements $f_1,\dots,f_m\in R$ that are algebraically independent over $K$ such that $K[f_1,\dots,f_m] \subseteq R$ is module-finite; equivalently, there is a module-finite injective $K$-algebra map from a polynomial ring ${K[X_1,\dots,X_m] \hookrightarrow R}$. Such a ring $S$ is called a \textbf{Noether normalization} for~$R$.

\

\noindent \textsc{Lemma:} Let $A$ be a ring, and $F \in R \ceq A[X_1,\dots,X_n]$ be a nonzero polynomial. Then there exists an $A$-algebra automorphism $\phi$ of $R$ such that $\phi(F)$, viewed as a polynomial in $X_n$ with coefficients in $A[X_1,\dots, X_{n-1}]$, has top degree term $a X_n^t$ for some $a\in A\smallsetminus 0$ and $t\geq 0$.
\begin{itemize}
\item If $A=K$ is an infinite field, one can take $\phi(X_n)=X_n$ and $\phi(X_i) = X_i +\lambda_i X_n$ for some $\lambda_1,\dots,\lambda_{n-1} \in K$.
\item In general, if the top degree of $F$ (with respect to the standard grading) is $D$, one can take $\phi(X_n) = X_n$ and $\phi(X_i) = X_i + X_n^{D^{n-i}}$ for $i<n$.
\end{itemize}

\


\noindent \textsc{Zariski's Lemma:} An algebra-finite extension of fields is module-finite.

\

\noindent \textsc{Useful variations on Noether normalization:}
\begin{itemize}
\item \textsc{NN for domains:} Let $A\subseteq R$ be an algebra-finite inclusion of domains\footnotemark. Then there exists $a\in A\smallsetminus 0$ and $f_1,\dots,f_m\in R[1/a]$ that are algebraically independent over $A[1/a]$ such that $A[1/a][f_1,\dots,f_m] \subseteq R[1/a]$ is module-finite.
\item \textsc{Graded NN:} Let $K$ be an infinite field, and $R$ be a standard graded $K$-algebra. Then there exist algebraically independent elements $L_1,\dots,L_m \in R_1$ such that $K[L_1,\dots,L_m] \subseteq R$ is module-finite.
\item \textsc{NN for power series:} Let $K$ be an infinite field, and $R=K\llbracket X_1,\dots,X_n \rrbracket / I$. Then there exists a module-finite injection $K\llbracket Y_1,\dots,Y_m \rrbracket \hookrightarrow R$ for some power series ring in $m$ variables.
\end{itemize}

 \end{framed}
 \footnotetext[1]{Possibly empty!}
\footnotetext[2]{The assumption that $R$ is a domain is actually not necessary, but can't quite state the general statement yet. We assume that $R$ is a domain so that there is fraction field of $R$ in which to take $R[1/a]$.}

 
\begin{enumerate}
\itemA Examples of Noether normalizations: Let $K$ be a field.
\begin{enumerate}
\itema Show that $K[x,y]$ is a Noether normalization of $\ds R= \frac{K[X,Y,Z]}{(X^3+Y^3+Z^3)}$, where $x,y$ are the classes of $X$ and $Y$ in $R$, respectively.
\itema Show that $K[x]$ is \emph{not} a Noether normalization of $\ds R=\frac{K[X,Y]}{(XY)}$. Then show that ${K[x+y] \subseteq R}$ \emph{is} a Noether normalization.
\itema Show that $K[X^4,Y^4]$ is a  Noether normalization for  $R=K[X^4,X^3Y,XY^3,Y^4]$.
\end{enumerate}

\solution{
\begin{enumerate}
\itema From the equation $z^3 +x^3+y^3 = 0$, we have $K[x,y]\subseteq R$ is integral, and since $z$ generates as an algebra, hence module-finite. We need to check that $x,y$ are algebraically independent in $R$. Suppose that $p(x,y)=0$ in $R$, so $p(X,Y) \in (X^3 + Y^3 + Z^3)$ in $K[X,Y,Z]$. By considering $K[X,Y,Z]=K[X,Y][Z]$ as polynomials in $Z$, the $Z$-degree of such a $p$, which forces $p=0$. Thus $x,y$ are algebraically independent.
\itema $y$ is not integral over $K[x]$: this would imply $Y^n + a_1(X) Y^{n-1} + \cdots a_n(X) = XY b(X,Y)$ in $K[X,Y]$, but no monomial from any term can cancel $Y^n$. Alternatively, if the inclusion is module-finite, go mod $x$ to get $K \subseteq K[X,Y]/(XY,X) = K[Y]$ module-finite, which it isn't.
\itema It is easy to check that $X^4, Y^4$ are algebraically independent, and $(X^3Y)^4 = (X^4)^3 Y^4$, $(XY^3)^4 = X^4 (Y^4)^3$ give integral dependence relations for the algebra generators.
\end{enumerate}}

\itemA Use Noether Normalization\footnote{and a suitable fact about integral extensions\dots} to prove Zariski's Lemma.

\solution{Let $K\subseteq L$ be an algebra-finite extension of fields. Take a NN of $L$: say ${K \subseteq K[\ell_1,\dots, \ell_t] \subseteq L}$, with $\ell_i$ algebraically independent and $R\ceq K[\ell_1,\dots, \ell_t] \subseteq L$ module-finite and a fortiori integral. From the Integral Extensions worksheet, since $L$ and $R$ are domains, the extension is integral, and $L$ is a field, we know that $R$ is a field. This means that $t=0$, so $K\subseteq L$ is module-finite.}

\itemA Proof of Noether Normalization (using the Lemma): Proceed by induction on the number of generators of $R$ as a $K$-algebra; write $R=K[r_1,\dots,r_n]$.
\begin{enumerate}
\itema Deal with the base case $n=0$.
\itema For the inductive step, first do the case that $r_1,\dots, r_n$ are algebraically~independent over~$K$.
\itema Let $\alpha:K[X_1,\dots,X_n] \to R$ be the $K$-algebra homomorphism such that $\alpha(X_i)=r_i$, and let $\phi$ be a $K$-algebra automorphism of $K[X_1,\dots,X_n]$. Let $r'_i = \alpha(\phi(X_i))$ for each $i$. Explain\footnote{Say $\alpha'$ is the $K$-algebra map given by $\alpha'(X_i)=r'_i$. Observe that $\alpha' = \alpha \circ \phi$. Why is this surjective?} why $R=K[r'_1,\dots,r'_n]$, and for any $K$-algebra relation $F$ on $r_1,\dots,r_n$, the polynomial $\phi^{-1}(F)$ is a $K$-algebra relation on $r'_1,\dots,r'_n$.
\itema Use the Lemma to find a $K$-subalgebra $R'$ of $R$ with $n-1$ generators such that the inclusion $R' \subseteq R$ is module-finite.
\itema Conclude the proof.
\end{enumerate}


\solution{
\begin{enumerate}
\itema This means that $R$ is a quotient of $K$, but $K$ is a field, so $R=K$; the identity map is module-finite.
\itema If we have an algebraically independent set of generators for $R$, then $R$ works: the identity map is module-finite.
\itema First we claim that $R=K[r_1',\dots,r_n']$: indeed, the map $\alpha' = \alpha\circ\phi$ is the $K$-algebra map that sends $X_i$ to $r'_i$, and since $\alpha$ and $\phi$ are surjective, $\alpha'$ is surjective, verifying the claim. The relations on the $r'_i$ are of the elements of the kernel of $\alpha'$; if $F$ is a relation on the originals, then $\alpha(F)=0$, so $\alpha'(\phi^{-1}(F))=0$ as well.
\itema Take a map $\phi$ as in the Lemma, and $n$ generators $r_1,\dots,r_n$. Set $r'_i = \phi^{-1}(r_i)$. By the previous part, these generate, and there is a relation on these that is monic in $X_n$, so $R' = K[r'_1,\dots,r'_{n-1}] \subseteq R$ is module-finite. 
\itema Apply IH to $R'$ to get $K[f_1,\dots,f_t] \subseteq R'$ with $f_i$ alg indep't and the inclusion module-finite. Then $K[f_1,\dots,f_t]$ is a Noether normalization.
\end{enumerate}

}

\itemB Proof of the ``general case'' of the Lemma: 
\begin{enumerate}
\item Where do ``base $D$ expansions'' fit in this picture?
\item Consider the automorphism $\phi$ from the general case of the Lemma. Show that for a monomial, we have $\phi(a X_1^{d_1} \cdots X_n^{d_n})$ is a polynomial with unique highest degree term $a X_n^{d_1 D^{n-1} + d_2 D^{n-2} + \cdots + d_n}$. 
\item Can two monomials $\mu,\nu$ in $F$, have $\phi(\mu)$ and $\phi(\nu)$ with the same highest degree term? 
\item Complete the proof.
\end{enumerate}

\solution{
}



\itemB Variations on NN.
\begin{enumerate}
\itemb Adapt the proof of NN to show Graded NN.
\itemb Adapt the proof of NN to show NN for domains.
\itemc Adapt the proof of NN to show NN for power series. %The Weierstrass Preparation Theorem from \S1.2 may be useful.
\end{enumerate}

\solution{}


\end{enumerate}
\vfill





\end{document}
