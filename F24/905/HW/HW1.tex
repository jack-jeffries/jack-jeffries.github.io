\documentclass[12pt]{amsart}


\usepackage{times}
\usepackage[margin=0.8in]{geometry}
\usepackage{amsmath,amssymb,multicol,graphicx,framed,ifthen,color,xcolor,stmaryrd,enumitem,colonequals,hyperref}
\definecolor{chianti}{rgb}{0.6,0,0}
\definecolor{meretale}{rgb}{0,0,.6}
\definecolor{leaf}{rgb}{0,.35,0}
\newcommand{\Q}{\mathbb{Q}}
\newcommand{\N}{\mathbb{N}}
\newcommand{\Z}{\mathbb{Z}}
\newcommand{\R}{\mathbb{R}}
\newcommand{\C}{\mathbb{C}}
\newcommand{\F}{\mathbb{F}}
\newcommand{\e}{\varepsilon}
\newcommand{\inv}{^{-1}}
\newcommand{\dabs}[1]{\left| #1 \right|}
\newcommand{\ds}{\displaystyle}
\newcommand{\solution}[1]{\ifthenelse {\equal{\displaysol}{1}} {\begin{framed}{\color{meretale}\noindent #1}\end{framed}} { \ }}
\newcommand{\showsol}[1]{\def\displaysol{#1}}
\newcommand{\rsa}{\rightsquigarrow}
\newcommand\itemA{\stepcounter{enumi}\item[{\bf{(\theenumi)}}]}
\newcommand\itemB{\stepcounter{enumi}\item[(\theenumi)]}
\newcommand\itemC{\stepcounter{enumi}\item[{\it{(\theenumi)}}]}
\newcommand\itema{\stepcounter{enumii}\item[{\bf{(\theenumii)}}]}
\newcommand\itemb{\stepcounter{enumii}\item[(\theenumii)]}
\newcommand\itemc{\stepcounter{enumii}\item[{\it{(\theenumii)}}]}
\newcommand\itemai{\stepcounter{enumiii}\item[{\bf{(\theenumiii)}}]}
\newcommand\itembi{\stepcounter{enumiii}\item[(\theenumiii)]}
\newcommand\itemci{\stepcounter{enumiii}\item[{\it{(\theenumiii)}}]}
\newcommand\ceq{\colonequals}
\DeclareMathOperator{\ord}{ord}
\renewcommand{\ceq}{\colonequals}

\DeclareMathOperator{\res}{res}
\setlength\parindent{0pt}
%\usepackage{times}

%\addtolength{\textwidth}{100pt}
%\addtolength{\evensidemargin}{-45pt}
%\addtolength{\oddsidemargin}{-60pt}

\pagestyle{empty}
%\begin{document}\begin{itemize}

%\thispagestyle{empty}




\begin{document}
\showsol{1}
	
	\thispagestyle{empty}
	
	\section*{Assignment \#1: Due Friday, September 6 at 7pm}
	
	This problem set is to be turned in on Canvas. You may reference any result or problem from our worksheets, unless it is the fact to be proven! You are encouraged to work with others, but you should understand everything you write. Please consult the class website for acceptable/unacceptable resources for the problem sets. You should use the techniques from this class and precursor classes to solve these problems, but not Commutative Algebra II or Homological Algebra. 
	
	\
	
	
	
	
\begin{enumerate}
\item Let $K$ be a field, and $S=K[X,Y]$ be a polynomial ring. Let $I=(X^2,XY)$ and $R=S/I$.
\begin{enumerate}
\item Show that any polynomial $f=\sum a_{i,j} X^i Y^j \in S$ is in $I$ if and only if every monomial of $f$ is divisible\footnote{i.e., if $a_{i,j}\neq 0$ then $X^2 | X^i Y^j$ or $XY | X^i Y^j$} by $X^2$ or $XY$.
\item Write out a $K$-vector space basis for $R$, and describe the multiplication on this basis.
\item Find, with justification,
\begin{itemize}
\item a nonzero nilpotent in $R$,
\item a zerodivisor in $R$ that is not nilpotent, and
\item a unit in $R$ that is not equivalent to a constant polynomial.
\end{itemize}
\end{enumerate}


\


\item Radicals:
\begin{enumerate}
\item Let $I$ be an ideal, and suppose that $\sqrt{I}$ is finitely generated. Show that there is some number~$n\geq 1$ such that $(\sqrt{I})^n \subseteq I$.
\item Consider $R=\mathcal{C}([0,1],\R)$, the ring of continuous functions from $[0,1]$ to $\R$. Show that for any $n\geq 1$, the function $x^{1/n}$ (i.e., the function $x\mapsto x^{1/n}$) is an element of the ideal $\sqrt{(x)}$, but $(x^{1/n})^m \notin (x)$ for any $m<n$. What does this say in conjunction with the previous part?
\end{enumerate}

\


\item Let $R$ be a domain with fraction field $F$. Let $f\in R$ be nonzero, and consider the $R$-subalgebra of $F$ generated by $1/f$; namely $R[1/f]$. Find, with justification, an $R$-algebra presentation for~$R[1/f]$.


\end{enumerate}

\


The remaining problems are to be solved with Macaulay2. Copy the inputs and outputs (e.g., from the right-hand pane on the web interface) you used to solved these problems.

\

\begin{enumerate}\setcounter{enumi}{3}


\item Find generating sets for the following ideals in the polynomial ring $\Q[W,X,Y,Z]$:
\begin{enumerate}
\item The radical of the ideal $(X^5-Y^4,X^7-Z^4,Y^7-Z^5)$.
\item The radical of the ideal generated by the entries of the \emph{square} of the matrix $\begin{bmatrix} W & X \\ Y& Z\end{bmatrix}$.
\item $(X^2+Y^2+Z^2) \cap (X,Y,Z)^3$.
\end{enumerate}

\

\item Find a presentation for each of the following algebras (where $U,V,X,Y,Z$ represent indeterminates):
\begin{enumerate}
\item $R=\F_{101}[X^4,X^3Y,X^2Y^2,XY^3,Y^4]\subseteq \F_{101}[X,Y]$.
\item  $R=\F_{101}[X^4,X^3Y,XY^3,Y^3]\subseteq \F_{101}[X,Y]$.
\item  $\displaystyle R=\Q[UX,UY,UZ,VX,VY,VZ]\subseteq \frac{\Q[U,V,X,Y,Z]}{(X^3+Y^3+Z^3)}$.
\end{enumerate}

\



\end{enumerate}


\end{document}
