\documentclass[12pt]{amsart}


\usepackage{times}
\usepackage[margin=0.7in]{geometry}
\usepackage{amsmath,amssymb,multicol,graphicx,framed,ifthen,color,xcolor,stmaryrd,enumitem,colonequals}
\definecolor{chianti}{rgb}{0.6,0,0}
\definecolor{meretale}{rgb}{0,0,.6}
\definecolor{leaf}{rgb}{0,.35,0}
\newcommand{\Q}{\mathbb{Q}}
\newcommand{\N}{\mathbb{N}}
\newcommand{\Z}{\mathbb{Z}}
\newcommand{\R}{\mathbb{R}}
\newcommand{\C}{\mathbb{C}}
\newcommand{\F}{\mathbb{F}}
\newcommand{\X}{\mathbf{X}}
\newcommand{\Y}{\mathbf{Y}}
\newcommand{\e}{\varepsilon}
\newcommand{\p}{\mathfrak{p}}
\newcommand{\q}{\mathfrak{q}}
\newcommand{\m}{\mathfrak{m}}
\newcommand{\cZ}{\mathcal{Z}}
\newcommand{\Spec}{\mathrm{Spec}}
\newcommand{\Supp}{\mathrm{Supp}}
\newcommand{\hgt}{\mathrm{height}}
\newcommand{\Min}{\mathrm{Min}}
\newcommand{\Max}{\mathrm{Max}}
\newcommand{\Ass}{\mathrm{Ass}}
\newcommand{\inv}{^{-1}}
\newcommand{\dabs}[1]{\left| #1 \right|}
\newcommand{\ds}{\displaystyle}
\newcommand{\solution}[1]{\ifthenelse {\equal{\displaysol}{1}} {\begin{framed}{\color{meretale}\noindent #1}\end{framed}} { \ }}
\newcommand{\showsol}[1]{\def\displaysol{#1}}
\newcommand{\rsa}{\rightsquigarrow}
\newcommand\itemA{\stepcounter{enumi}\item[{\bf{(\theenumi)}}]}
\newcommand\itemB{\stepcounter{enumi}\item[(\theenumi)]}
\newcommand\itemC{\stepcounter{enumi}\item[{\it{(\theenumi)}}]}
\newcommand\itema{\stepcounter{enumii}\item[{\bf{(\theenumii)}}]}
\newcommand\itemb{\stepcounter{enumii}\item[(\theenumii)]}
\newcommand\itemc{\stepcounter{enumii}\item[{\it{(\theenumii)}}]}
\newcommand\itemai{\stepcounter{enumiii}\item[{\bf{(\theenumiii)}}]}
\newcommand\itembi{\stepcounter{enumiii}\item[(\theenumiii)]}
\newcommand\itemci{\stepcounter{enumiii}\item[{\it{(\theenumiii)}}]}
\newcommand\ceq{\colonequals}
\DeclareMathOperator{\ord}{ord}
\renewcommand{\ceq}{\colonequals}

\DeclareMathOperator{\res}{res}
\setlength\parindent{0pt}
%\usepackage{times}

%\addtolength{\textwidth}{100pt}
%\addtolength{\evensidemargin}{-45pt}
%\addtolength{\oddsidemargin}{-60pt}

\pagestyle{empty}
%\begin{document}\begin{itemize}

%\thispagestyle{empty}




\begin{document}
\showsol{1}
	
	\thispagestyle{empty}
	
	\section*{Assignment \#7: Due Friday, December 13 at 7pm}
	
	This problem set is to be turned in by Canvas. You may reference any result or problem from our worksheets, unless it is the fact to be proven! You are encouraged to work with others, but you should understand everything you write. Please consult the class website for acceptable/unacceptable resources for the problem sets. You should use the techniques from this class and precursor classes to solve these problems, but not Commutative Algebra II or Homological Algebra.
	
	
	\
	
\begin{enumerate}

\item Let $R\subset S$ be an inclusion of rings such that $R$ is a direct summand of $S$. Show that the induced map on $\Spec$ is surjective.

\



\item Let $R$ be a ring, not necessarily Noetherian, and $S=R[X]$ a polynomial ring in one variable over~$R$.
\begin{enumerate} 
\item Show that for any prime ideal $\p$ in $R$, any chain of prime ideals of $S$ that all contract to $\p$ has length at most one.
\item Show that if $\dim(R)=d$, then $d+1 \leq \dim(S) \leq 2d+1$.
\item Let $R= \Q + T \,\R \llbracket T \rrbracket$, i.e., the subring of $R' = \R \llbracket T \rrbracket$ consisting of all power series whose constant term is rational. Verify that $R$ is a ring, that $R$ has dimension one, and\footnote{Hint: Let $\p$ be the prime ideal $T\C \llbracket T \rrbracket\subseteq R$ and let  $\alpha:R[X]\to R'$ be the $R$-algebra homomorphism given by $\alpha(X)=e$. Show that $\ker(\alpha) \subsetneqq \p S$.} that the dimension of $R[X]$ is three.
\end{enumerate}



\



\item Show that the $\Z$-module $\Z[1/2]/\Z$ is Artinian but not Noetherian.

\


\item Let $K$ be an algebraically closed field, $R=K[X_1,\dots,X_n]$, and 
\[S=R[Y_1,\dots,Y_n]=K[X_1,\dots,X_n,Y_1,\dots,Y_n]\] be polynomial rings. For $f\in R$, we will also write $f(\X)$ for $f$, and we write $f(\Y)$ for the element\footnote{To be pedantic, let $\phi:R\to S$ be the $K$-algebra homomorphism given by $\phi(X_i)=Y_i$; then $f(Y)=\phi(f)$.} of $S$ obtained by replacing $X$-variables with $Y$-variables in $f$. For an ideal $I\subseteq R$, we write $I(\Y) = \{ f(\Y) \ | \ f\in I\}$.

 Let $\p,\q$ be prime ideals in $R$.
 \begin{enumerate}
\item Let $\{F_1,\dots,F_a\}$ be a set elements of $R$ whose images generate a Noether normalization for $R/\p$, and $\{G_1,\dots,G_b\}$ be a set of elements of $R$ whose elements generate a Noether normalization for $R/\q$. Show that the images of $\{F_1(\X),\dots,F_a(\X),G_1(\Y),\dots,G_b(\Y)\}$ generate a Noether normalization\footnote{Hint: Given an equation $\sum c_{\alpha,\beta} F^\alpha G^\beta \in \p(\X) S + \q(\Y) S$, evaluate the $Y$ variables at $\beta\in Z(\q)$ to get an equation in $R$\dots} for $\ds\frac{S}{\p(\X) S + \q(\Y)S}$. Deduce that 
\[\dim\left(\frac{S}{\p(\X) S + \q(\Y)S}\right) = \dim(R/\p) + \dim(R/\q).\]
\item Show that \[\frac{S}{\p(\X) S + \q(\Y)S + (X_1-Y_1,\dots,X_n-Y_n)}\cong \frac{R}{\p+\q}.\] Deduce (this is the punchline) that $\hgt(\p+\q) \leq \hgt(\p) + \hgt(\q)$.
\item Let $\ds T=\frac{\C[X,Y,U,V]}{(XV-YU)}$, $\p=(X,U)$, and $\q=(Y,V)$. Show that $\hgt(\p)=\hgt(\q)=1$ but $\hgt(\p+\q)=3$.
\end{enumerate}




\end{enumerate}


\end{document}
