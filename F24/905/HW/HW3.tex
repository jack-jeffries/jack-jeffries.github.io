\documentclass[12pt]{amsart}


\usepackage{times}
\usepackage[margin=0.65in]{geometry}
\usepackage{amsmath,amssymb,multicol,graphicx,framed,ifthen,color,xcolor,stmaryrd,enumitem,colonequals,hyperref}
\definecolor{chianti}{rgb}{0.6,0,0}
\definecolor{meretale}{rgb}{0,0,.6}
\definecolor{leaf}{rgb}{0,.35,0}
\newcommand{\Q}{\mathbb{Q}}
\newcommand{\N}{\mathbb{N}}
\newcommand{\Z}{\mathbb{Z}}
\newcommand{\R}{\mathbb{R}}
\newcommand{\C}{\mathbb{C}}
\newcommand{\F}{\mathbb{F}}
\newcommand{\e}{\varepsilon}
\newcommand{\m}{\mathfrak{m}}
\newcommand{\inv}{^{-1}}
\newcommand{\dabs}[1]{\left| #1 \right|}
\newcommand{\ds}{\displaystyle}
\newcommand{\solution}[1]{\ifthenelse {\equal{\displaysol}{1}} {\begin{framed}{\color{meretale}\noindent #1}\end{framed}} { \ }}
\newcommand{\showsol}[1]{\def\displaysol{#1}}
\newcommand{\rsa}{\rightsquigarrow}
\newcommand\itemA{\stepcounter{enumi}\item[{\bf{(\theenumi)}}]}
\newcommand\itemB{\stepcounter{enumi}\item[(\theenumi)]}
\newcommand\itemC{\stepcounter{enumi}\item[{\it{(\theenumi)}}]}
\newcommand\itema{\stepcounter{enumii}\item[{\bf{(\theenumii)}}]}
\newcommand\itemb{\stepcounter{enumii}\item[(\theenumii)]}
\newcommand\itemc{\stepcounter{enumii}\item[{\it{(\theenumii)}}]}
\newcommand\itemai{\stepcounter{enumiii}\item[{\bf{(\theenumiii)}}]}
\newcommand\itembi{\stepcounter{enumiii}\item[(\theenumiii)]}
\newcommand\itemci{\stepcounter{enumiii}\item[{\it{(\theenumiii)}}]}
\newcommand\ceq{\colonequals}
\DeclareMathOperator{\ord}{ord}
\renewcommand{\ceq}{\colonequals}

\DeclareMathOperator{\res}{res}
\setlength\parindent{0pt}
%\usepackage{times}

%\addtolength{\textwidth}{100pt}
%\addtolength{\evensidemargin}{-45pt}
%\addtolength{\oddsidemargin}{-60pt}

\pagestyle{empty}
%\begin{document}\begin{itemize}

%\thispagestyle{empty}




\begin{document}
\showsol{1}
	
	\thispagestyle{empty}
	
	\section*{Assignment \#3: Due Friday, October 11 at 7pm}
	
	This problem set is to be turned in by Gradescope. You may reference any result or problem from our worksheets, unless it is the fact to be proven! You are encouraged to work with others, but you should understand everything you write. Please consult the class website for acceptable/unacceptable resources for the problem sets. You should use the techniques from this class and precursor classes to solve these problems, but not Commutative Algebra II or Homological Algebra. 
	
	\
	
	
	
	
\begin{enumerate}

\item Let $K$ be a field, and $R=K[X^2,X^3] \subseteq S=K[X]$. Show that for the ideal $I$ of $R$ generated by $X^2$, we have $I S \cap R \supsetneqq I$. Conclude that $R$ is not a direct summand of $S$.


\

\item Suppose that $R$ is a finitely-generated $\Z$-algebra, and that $\m$ is a maximal ideal of $R$. Show that $R/\m$ is a finite field.

\

\end{enumerate}

\textsc{Definition:} Let $R$ be an Noetherian $\N$-graded ring, with $R_0=K$ a field. The \textbf{Hilbert function} of $R$ is the function $H_R(t) = \dim_{K}(R_i)$, where $\dim_K$ denotes vector space dimension.

\

\begin{enumerate}
\setcounter{enumi}{2}


\item Let $K=\C$, and $R=K[X,Y,Z]$. In this problem, we will show that the radical of the ideal
\[ I= (X^5 - Y^4, X^7-Z^4, Y^7-Z^5) \quad \text{is} \quad J=(X^3-YZ, Y^3-X^2Z, Z^2-XY^2).\]
\begin{enumerate}
\item Show that $I\subseteq J$.
\item Show\footnote{Warning: $a^4=b^4 \not\Rightarrow a=b$.} that $Z(I)=\{ (\lambda^4,\lambda^5,\lambda^7) \ | \ \lambda \in \C\}$, and use the Nullstellensatz to show that $J \subseteq \sqrt{I}$.
\item Let $\phi: \C[X,Y,Z] \to \C[T]$ be the $K$-algebra homomorphism given by $\phi(X) = T^4$, $\phi(Y)=T^5$, and $\phi(Z)=T^7$. Show that $J\subseteq \ker(\phi)$.
\item Give $\C[X,Y,Z]$ the weighted grading so that $\deg(X)=4, \deg(Y)=5, \deg(Z)=7$, so that $\phi$ is a graded $\C$-algebra homomorphism. 
Let $R=\C[X,Y,Z]/J$ and $S=\C[T^4,T^5,T^7] = \mathrm{im}(\phi)$. Compute $H_S(t)$, show\footnote{Hint: Show that any element of $R$ is congruent modulo $J$ to some element of the form ${f(X) + g(X)Y + h(X) Z + k(X) Y^2}$. You could do this directly, or with a suitable application of Graded NAK.} that $H_R(t) \leq H_S(t)$ for all $t$, and deduce that $R\cong S$.
\item Deduce the conclusion.
\end{enumerate}

\




\item Let $U,\dots,Z$ be indeterminates over $\C$, and $\displaystyle R= \C \begin{bmatrix} U & V & W \\ X & Y & Z \end{bmatrix} \Big/ I$ where $I=I_2\left(\begin{bmatrix} U & V & W \\ X & Y & Z \end{bmatrix}\right)$. In this problem, we will show that $A=\C[u, v-x, w-y, z] \subseteq R$ is a Noether normalization, where $u,\dots,z$ denote the images of $U,\dots,Z$ in $R$.
\begin{enumerate}
\item Apply Graded NAK\footnote{One could instead show that the generators of $R$ are integral over $A$, but we will try out the power of Graded NAK.} to $R$ as a graded $A$-module to show that $A\subseteq R$ is module-finite.
\item Show that for any $\alpha,\beta,\gamma,\zeta \in \C$, there is a rank one $2\times 3$ matrix $\begin{bmatrix}  a & b & c \\ d & e & f\end{bmatrix}$ with $a,\dots,f\in \C$ such that $a =\alpha$, $b-d = \beta$, $c-e= \gamma$, $f=\zeta$.
\item Suppose that $F(u,v-x,w-y,z)$ is a $\C$-algebraic relation on $u,v-x,w-y,z$, so \\ ${F(U,V-X,W-Y,Z)\in I}$. Use the previous part to show that $F(\alpha,\beta,\gamma,\zeta)=0$ for all $\alpha,\beta,\gamma,\zeta \in \C$, and deduce\footnote{You can use without proof that any nonzero (multivariate) polynomial over an infinite field is a nonzero function.} that $u,v-x,w-y,z$ are algebraically independent.
\end{enumerate}


\

\end{enumerate}


\


The remaining problems are to be solved with Macaulay2. Copy or screenshot the inputs and outputs (e.g., from the right-hand pane on the web interface) you used to solved these problems.

\

\begin{enumerate}\setcounter{enumi}{5}


\item  Read the discussion of modules in Macaulay2 from \S1.1 of Grifo's Lecture Notes. Then let \\ $\displaystyle{R=\frac{\Q[U,V, W , X , Y , Z]}{(UY-VX,UZ-WX,VZ-WY)}}$ and find presentation matrices of the following:
\begin{itemize}
\item the ideal $(U,V,W)$,
\item the submodule of $R^2$ generated by $\begin{bmatrix} U \\ X \end{bmatrix}$, $\begin{bmatrix} V \\ Y \end{bmatrix}$, and $\begin{bmatrix} W \\ Z \end{bmatrix}$.
\item the submodule of $R^2$ generated by $\begin{bmatrix} U +V \\ W+X \end{bmatrix}$, $\begin{bmatrix} V+W \\ X+Y \end{bmatrix}$, and $\begin{bmatrix} W+X \\ Y+Z \end{bmatrix}$.
\end{itemize}
\end{enumerate}


\end{document}
