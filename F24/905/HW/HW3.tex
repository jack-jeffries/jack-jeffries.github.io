\documentclass[12pt]{amsart}


\usepackage{times}
\usepackage[margin=0.65in]{geometry}
\usepackage{amsmath,amssymb,multicol,graphicx,framed,ifthen,color,xcolor,stmaryrd,enumitem,colonequals}
\definecolor{chianti}{rgb}{0.6,0,0}
\definecolor{meretale}{rgb}{0,0,.6}
\definecolor{leaf}{rgb}{0,.35,0}
\newcommand{\Q}{\mathbb{Q}}
\newcommand{\N}{\mathbb{N}}
\newcommand{\Z}{\mathbb{Z}}
\newcommand{\R}{\mathbb{R}}
\newcommand{\C}{\mathbb{C}}
\newcommand{\F}{\mathbb{F}}
\newcommand{\e}{\varepsilon}
\newcommand{\m}{\mathfrak{m}}
\newcommand{\inv}{^{-1}}
\newcommand{\dabs}[1]{\left| #1 \right|}
\newcommand{\ds}{\displaystyle}
\newcommand{\solution}[1]{\ifthenelse {\equal{\displaysol}{1}} {\begin{framed}{\color{meretale}\noindent #1}\end{framed}} { \ }}
\newcommand{\showsol}[1]{\def\displaysol{#1}}
\newcommand{\rsa}{\rightsquigarrow}
\newcommand\itemA{\stepcounter{enumi}\item[{\bf{(\theenumi)}}]}
\newcommand\itemB{\stepcounter{enumi}\item[(\theenumi)]}
\newcommand\itemC{\stepcounter{enumi}\item[{\it{(\theenumi)}}]}
\newcommand\itema{\stepcounter{enumii}\item[{\bf{(\theenumii)}}]}
\newcommand\itemb{\stepcounter{enumii}\item[(\theenumii)]}
\newcommand\itemc{\stepcounter{enumii}\item[{\it{(\theenumii)}}]}
\newcommand\itemai{\stepcounter{enumiii}\item[{\bf{(\theenumiii)}}]}
\newcommand\itembi{\stepcounter{enumiii}\item[(\theenumiii)]}
\newcommand\itemci{\stepcounter{enumiii}\item[{\it{(\theenumiii)}}]}
\newcommand\ceq{\colonequals}
\DeclareMathOperator{\ord}{ord}
\renewcommand{\ceq}{\colonequals}

\DeclareMathOperator{\res}{res}
\setlength\parindent{0pt}
%\usepackage{times}

%\addtolength{\textwidth}{100pt}
%\addtolength{\evensidemargin}{-45pt}
%\addtolength{\oddsidemargin}{-60pt}

\pagestyle{empty}
%\begin{document}\begin{itemize}

%\thispagestyle{empty}




\begin{document}
\showsol{1}
	
	\thispagestyle{empty}
	
	\section*{Assignment \#3: Due Friday, October 11 at 7pm}
	
	This problem set is to be turned in by Canvas. You may reference any result or problem from our worksheets, unless it is the fact to be proven! You are encouraged to work with others, but you should understand everything you write. Please consult the class website for acceptable/unacceptable resources for the problem sets. You should use the techniques from this class and precursor classes to solve these problems, but not Commutative Algebra II or Homological Algebra.
	
	
	\
	
\begin{enumerate}



\item Let $K$ be a field, and $R=K[X^2,X^3] \subseteq S=K[X]$. Show that for the ideal $I$ of $R$ generated by $X^2$, we have $I S \cap R \supsetneqq I$. %Conclude that $R$ is not a direct summand of $S$.

\

\item Let $K$ be an infinite field, $m \geqslant 1$, and let $R = K[X_1,\dots, X_n]$ be a polynomial ring. Let $G= (K^\times)^m$ act on $R$ by (degree-preserving) $K$-algebra automorphisms as follows: 
\[ 
 (\lambda_1,\dots,\lambda_m) \cdot X_i = \lambda_1^{a_{1,i}} \cdots  \lambda_m^{a_{m,i}} X_i \qquad  {i=1,\dots,n} 
 \]
 for some $m \times n$ matrix of integers $A=[a_{i,j}]$. For shorthand, we may write $\underline{\lambda} \cdot X_i = \underline{\lambda}^{\underline{a_i}} X_i$ with $\underline{\lambda} = (\lambda_1,\dots,\lambda_m)$ and $\underline{a_i}$ is the $i$th column on $A$.
\begin{enumerate}
\item Show\footnote{You can use without proof that any nonzero (multivariate) polynomial over an infinite field is a nonzero function.} $R^G$ has a $K$-vector space basis given by the set of monomials $\underline{X}^{\underline{b} }= X_1^{b_1} \cdots X_n^{b_n}$ such that $A\underline{b}=0$.
\item Consider the polynomial ring $R$ with a (nonstandard) $\mathbb{Z}^m$-grading given by setting $\textrm{deg}(X_i) = \underline{a_i}$
for each $i$. Show that $R^G$ is the degree zero piece of $R$ under this grading.
\item Show that $R^G$ is a direct summand of $R$, and deduce that $R^G$ is a finitely generated $K$-algebra.
\item Let $R = K[X,Y,Z,W]$ and consider the $K$-algebra action of $G = K^\times$ on $R$ given by
\[\lambda \cdot X = \lambda X \qquad \lambda \cdot Y = \lambda Y \qquad \lambda \cdot Z = \lambda^{-1} Z \qquad \lambda \cdot W = \lambda^{-1} W.\]
Find a finite set of generators for $R^G$ as a $K$-algebra.
\end{enumerate}

\

\item Let $U,\dots,Z$ be indeterminates over $\C$, and $\displaystyle R= \C \begin{bmatrix} U & V & W \\ X & Y & Z \end{bmatrix} \Big/ I$ where $I=I_2\left(\begin{bmatrix} U & V & W \\ X & Y & Z \end{bmatrix}\right)$. In this problem, we will show that $A=\C[u, v-x, w-y, z] \subseteq R$ is a Noether normalization, where $u,\dots,z$ denote the images of $U,\dots,Z$ in $R$.
\begin{enumerate}
\item Apply Graded NAK\footnote{One could instead show that the generators of $R$ are integral over $A$, but we will try out the power of Graded NAK.} to $R$ as a graded $A$-module to show that $A\subseteq R$ is module-finite.
\item Show that for any $\alpha,\beta,\gamma,\zeta \in \C$, there is a rank at most one $2\times 3$ matrix $\begin{bmatrix}  a & b & c \\ d & e & f\end{bmatrix}$ with $a,\dots,f\in \C$ such that $a =\alpha$, $b-d = \beta$, $c-e= \gamma$, $f=\zeta$.
\item Suppose that $F(T_1,T_2,T_3,T_4)$ is a $\C$-algebraic relation on $u,v-x,w-y,z$. Use the previous part to show that $F(\alpha,\beta,\gamma,\zeta)=0$ for all $\alpha,\beta,\gamma,\zeta \in \C$, and deduce\footnotemark[1] that $u,v-x,w-y,z$ are algebraically independent.
\end{enumerate}

\


\item Suppose that $R$ is a finitely-generated $\Z$-algebra, and that $\m$ is a maximal ideal of $R$. Show that $R/\m$ is a finite field.





\end{enumerate}






\end{document}
