\documentclass[12pt]{amsart}


\usepackage{times}
\usepackage[margin=0.65in]{geometry}
\usepackage{amsmath,amssymb,multicol,graphicx,framed,ifthen,color,xcolor,stmaryrd,enumitem,colonequals,hyperref}
\definecolor{chianti}{rgb}{0.6,0,0}
\definecolor{meretale}{rgb}{0,0,.6}
\definecolor{leaf}{rgb}{0,.35,0}
\newcommand{\Q}{\mathbb{Q}}
\newcommand{\N}{\mathbb{N}}
\newcommand{\Z}{\mathbb{Z}}
\newcommand{\R}{\mathbb{R}}
\newcommand{\C}{\mathbb{C}}
\newcommand{\F}{\mathbb{F}}
\newcommand{\e}{\varepsilon}
\newcommand{\inv}{^{-1}}
\newcommand{\dabs}[1]{\left| #1 \right|}
\newcommand{\ds}{\displaystyle}
\newcommand{\solution}[1]{\ifthenelse {\equal{\displaysol}{1}} {\begin{framed}{\color{meretale}\noindent #1}\end{framed}} { \ }}
\newcommand{\showsol}[1]{\def\displaysol{#1}}
\newcommand{\rsa}{\rightsquigarrow}
\newcommand\itemA{\stepcounter{enumi}\item[{\bf{(\theenumi)}}]}
\newcommand\itemB{\stepcounter{enumi}\item[(\theenumi)]}
\newcommand\itemC{\stepcounter{enumi}\item[{\it{(\theenumi)}}]}
\newcommand\itema{\stepcounter{enumii}\item[{\bf{(\theenumii)}}]}
\newcommand\itemb{\stepcounter{enumii}\item[(\theenumii)]}
\newcommand\itemc{\stepcounter{enumii}\item[{\it{(\theenumii)}}]}
\newcommand\itemai{\stepcounter{enumiii}\item[{\bf{(\theenumiii)}}]}
\newcommand\itembi{\stepcounter{enumiii}\item[(\theenumiii)]}
\newcommand\itemci{\stepcounter{enumiii}\item[{\it{(\theenumiii)}}]}
\newcommand\ceq{\colonequals}
\DeclareMathOperator{\ord}{ord}
\renewcommand{\ceq}{\colonequals}

\DeclareMathOperator{\res}{res}
\setlength\parindent{0pt}
%\usepackage{times}

%\addtolength{\textwidth}{100pt}
%\addtolength{\evensidemargin}{-45pt}
%\addtolength{\oddsidemargin}{-60pt}

\pagestyle{empty}
%\begin{document}\begin{itemize}

%\thispagestyle{empty}




\begin{document}
\showsol{1}
	
	\thispagestyle{empty}
	
	\section*{Assignment \#9: Due Thursday, November 21 at midnight}
	
	This problem set is to be turned in on Canvas. You may reference any result or problem from our worksheets or lectures, unless it is the fact to be proven! You are encouraged to work with others, but you should understand everything you write. Please consult the class website for acceptable/unacceptable resources for the problem sets.
	
	\
	

\begin{enumerate}


\item For each of the following, either give an example of a function with the indicated property or explain why none exists.
\begin{enumerate}
\item A function $f:(-1,1)\to \R$ that is continuous on the open interval $(-1,1)$  with range $[2,3)$.
\item A function $f:(-1,1)\to \R$ that is continuous on the open interval $(-1,1)$ with range ${(2,3) \cup (4,5)}$.
\item A function $f:[-1,1]\to \R$ that is continuous on the closed interval $[-1,1]$ with range $[0,\infty)$.
\item A function $f:[-1,1]\to \R$ that is continuous on the closed interval $[-1,1]$ with range $[2,3)$.
\item A function $f:(-1,1)\to \R$ that achieves a minimum value but not a maximum value on $(-1,1)$.
\item A function $f:\R\to \R$ that is continuous on $\R$ with range $(0,\infty)$.
\end{enumerate}

\

\item Assume that $f$ is continuous on the closed interval $[0,1]$ and that $0\leq f(x) \leq 1$ for all $x\in [0,1]$. Show that there is some real number $c$ such that $0\leq c \leq 1$ and $f(c)=c$.

\


\item Prove\footnote{Hint: For existence, use the Intermediate Value Theorem. For uniqueness, it may be useful to use the difference of cubes factorization $y^3 - x^3  = (y-x)(y^2+xy+x^2)$ and apply completing the square to the latter factor.}  that every real number has a unique cube root.
\end{enumerate}

\

\begin{framed}
\noindent \textbf{Definition:} Let $f$ be a function and $r$ be a real number. We say that $f$ is \textbf{differentiable at $r$} if $f$ is defined at $r$ and the limit
\[ \lim_{x\to r} \frac{ f(x) - f(r) }{x-r}\]
exists. In this case, we call the limit \textbf{the derivative of $f$ at $r$} and write $f'(r)$ for this limit.
\end{framed}

\

\begin{enumerate}\setcounter{enumi}{3}
\item Use the definition and any theorems and previous examples of limits to show $f(x)= x^3$  is differentiable at every $r\in \R$ and that $f'(r) = 3r^2$.

\

\item Use the definition and any theorems and previous examples of limits to show $f(x)= \sqrt{x}$  is differentiable at every $r\in (0,\infty)$ and that $\ds f'(r) = \frac{1}{2\sqrt{r}}$.

\


\item Show that the function 
\[ f(x) = \begin{cases} x^2 & \text{if} \ x\in \Q \\ 0 & \text{if} \ x\notin \Q\end{cases} \]
is differentiable at $x=0$.


\end{enumerate}
\end{document}

























\end{document}