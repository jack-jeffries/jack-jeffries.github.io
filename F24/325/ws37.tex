\documentclass[12pt]{amsart}


\usepackage{times}
\usepackage[margin=.75in]{geometry}
\usepackage{amsmath,amssymb,multicol,graphicx,framed}
\usepackage[all]{xy}
\newcommand{\Q}{\mathbb{Q}}
\newcommand{\N}{\mathbb{N}}
\newcommand{\Z}{\mathbb{Z}}
\newcommand{\R}{\mathbb{R}}
\newcommand{\inv}{^{-1}}
\newcommand{\dabs}[1]{\left| #1 \right|}
\newcommand{\e}{\varepsilon}
\newcommand{\de}{\delta}
\newcommand{\ds}{\displaystyle}

\DeclareMathOperator{\res}{res}

\pagestyle{empty}




\begin{document}
	
	
	\thispagestyle{empty}
	
	
	\section*{Derivatives and Optimization \S4.2}
	

\begin{framed} 
 \noindent \textsc{Theorem 37.1:} Let $f$ be a function that is differentiable at $x=r$.
 \begin{enumerate}
 \item 
 If $f'(r) > 0$, then there is some $\delta>0$ such that 
 \begin{itemize}
 \item if $x\in (r,r+\delta)$ then $f(r) < f(x)$;
  \item if $x\in (r-\delta,r)$ then $f(x) < f(r)$.
  \end{itemize}
  
  \item  If $f'(r) < 0$, then there is some $\delta>0$ such that 
  \begin{itemize}
 \item if $x\in (r,r+\delta)$ then $f(r) > f(x)$;
 \item if $x\in (r-\delta,r)$ then $f(x) > f(r)$.
  \end{itemize}
  \end{enumerate}
  
  \
  
  \noindent \textsc{Corollary 37.2 (Derivatives and optimization):}
  Let $f$ be a function that is continuous on a closed interval $[a,b]$. If $f$ attains a maximum or minimum value on $[a,b]$ at $r\in (a,b)$, and $f$ is differentiable at $r$, then $f'(r)=0$.
 \end{framed}


\

 \begin{enumerate}
 
 \item Find the values of $x$ on $[0,2]$ at which the function $f(x) = x^3-x^2-2x$ achieves its minimum and maximum values. Justify your answer carefully using the results above.
 
 \ 
 
 \item Explain why the Corollary follows from the Theorem.
 
 \
 
 \item Give examples of continuous functions on $[0,2]$ such that
 \begin{enumerate}
 \item $f(x)$ attains its maximum at $x=0$;
 \item $g(x)$ attains its maximum at $x=2$;
 \item $h(x)$ attains its maximum at $x=1$ and $h$ is differentiable at $x=1$;
 \item $j(x)$ attains its maximum at $x=1$ and $j$ is not differentiable at $x=1$.
 \end{enumerate}
 
 \
 
 \item Prove part (1) of the Theorem:
 \begin{itemize}
 \item Consider the function $\ds h(x) = \frac{ f(x) - f(r) }{x-r}$. Apply the definition of limit to this function with $\e = f'(r)$. What does the definition give you?
 \item If $h(x) > 0$ and $x>r$, what can you say about $f(x) - f(r)$?
  \item If $h(x) > 0$ and $x<r$, what can you say about $f(x) - f(r)$?
  \end{itemize}


\

 \item Prove part (2) of the Theorem.
 
 \
 
 \item True or false: If $f'(7)>0$, then $f(7.0000001)>f(7)$.
 
 \
 
  \item True or false: If $f'(7)>0$, then there exists some $N\in \N$ such that for all natural numbers $n>N$,  $f\left(7+ \frac{1}{10^n}\right)>f(7)$.

 
\end{enumerate}




%\begin{framed}
% 
%  \noindent \textsc{Definition 37.3:} Let $f$ be a function. We say that a real number $r$ is a \textbf{local minimum} of $f$ if there exists some $\delta>0$ such that $f$ is defined on $(r-\delta,r+\delta)$ and $f$ achieves its minimum on the interval $(r-\delta,r+\delta)$ at the input value $x=r$. We define \textbf{local maximum} analogously.
%  
%  
%  \
%  
%  
%    \noindent \textsc{Corollary 37.4:} If $f$ attains a local maximum or local minimum at $x=r$, then either $f$ is not differentiable at $x=r$ or $f'(r)=0$. 
%    
%    \end{framed}
  
\end{document}

