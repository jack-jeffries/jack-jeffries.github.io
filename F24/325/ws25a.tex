\documentclass[12pt]{amsart}


\usepackage{times}
\usepackage[margin=.6in]{geometry}
\usepackage{amsmath,amssymb,multicol,graphicx,framed}
\usepackage[all]{xy}
\newcommand{\Q}{\mathbb{Q}}
\newcommand{\N}{\mathbb{N}}
\newcommand{\Z}{\mathbb{Z}}
\newcommand{\R}{\mathbb{R}}
\newcommand{\inv}{^{-1}}
\newcommand{\dabs}[1]{\left| #1 \right|}
\newcommand{\e}{\varepsilon}
\newcommand{\de}{\delta}
\newcommand{\ds}{\displaystyle}

\DeclareMathOperator{\res}{res}

\pagestyle{empty}




\begin{document}
	
	
	\thispagestyle{empty}
	
	\section*{Theorems about Limits \S 3.1}
	
	\begin{framed}

\noindent \textbf{Theorem 26.1:}  Let $f(x)$ be a function and let $a$ be a real number.
  Let $r > 0$ be a positive real number such that
  $f$ is defined
  at every point of $\{x \in \R \mid 0 < |x-a| < r\}$.
    Let $L$ be any real number. 

Then $\lim_{x \to a} f(x) = L$ if and only if for every sequence 
$\{x_n\}_{n=1}^\infty$ that converges to $a$ and satisfies $0 < |x_n - a| < r$ for all $n$, we have that the sequence $\{f(x_n)\}_{n=1}^\infty$ converges to $L$. 

\
	
	\noindent \textbf{Theorem 26.2. (Algebra and limits of functions):} Suppose $f$ and $g$ are two functions and that $a$ is a real number, and
	assume  that 
	$$
	\lim_{x \to a} f(x) = L  \ \text{and} \  \lim_{x \to a} g(x) = M
	$$
	for some real numbers $L$ and $M$. Then
	\begin{enumerate}
		\item $\lim_{x \to a} (f(x) + g(x)) = L  + M$.
		\item For any real number $c$, $\lim_{x \to a} (c \cdot f(x)) = c \cdot L$.
		\item $\lim_{x \to a} (f(x) \cdot g(x)) = L \cdot M$.
		\item If, in addition, we have that $M \ne 0$,
		then $\lim_{x \to a} (f(x)/g(x)) = L/M$.
	\end{enumerate}
	
	\
	
	\noindent \textbf{Theorem 26.3. (Squeeze Theorem for limits):}  Suppose $f$, $g$, and $h$ are three functions and $a$ is a real number. Suppose there is a positive real number $r > 0$
	such that 
	\begin{itemize}
		\item each of $f,g,h$ is defined on $\{x \in \R \mid 0 < |x-a| < r\}$,
		\item $f(x) \leq g(x) \leq h(x)$ for all
		$x$ such that $0 < |x-a| < r$, and
		\item	$\lim_{x \to a} f(x) = L = \lim_{x \to    a} h(x)$ for some number $L$.
	\end{itemize}  
	Then $\lim_{x \to a} g(x) = L$.
\end{framed}	
	



\begin{enumerate} 

\item Use Theorem 26.1 to show that $\ds \lim_{x\to 0} \sin\left(\frac{1}{x}\right)$ does not exist.\\
Suggestion: Let $f(x)= \sin(\frac{1}{x})$ and suppose $\lim_{x\to 0} f(x) = L$. Find sequences $\{x_n\}_{n=1}$  and $\{y_n\}_{n=1}$ such that
\begin{itemize}
\item  $\{x_n\}_{n=1}$ and $\{y_n\}_{n=1}$ both converge to $0$,
\item $f(x_n)=1$ for all $n$, and
\item $f(y_n)=-1$ for all $n$.
\end{itemize}
You can use any trig facts on the bottom of the page.

\

\item Use Theorem~26.2 plus a fact from last time\footnote{$\lim_{x\to a} \ mx+b = ma+b$. In particular, $\lim_{x\to a} \ x = a$ and $\lim_{x \to b} \ b = b$.}  to compute $\ds \lim_{x\to 2} \frac{3x^2 - x +2}{x+3}$.

\

\item Use Theorem~26.3 to show that $\displaystyle \lim_{x\to 0} x \sin\left(\frac{1}{x}\right)=0$. You can use any trig facts on the bottom of the page.

\


\item Use Theorem 26.1 to deduce Theorem~26.3 from our Squeeze Theorem for sequences.

\

\item Use Theorem 26.1 to deduce Theorem~26.2 part (1) from our Theorem on algebra and sequences.

\

\item Use Theorem 26.1 to deduce Theorem~26.2 part (4) from our Theorem on algebra and sequences.




\vfill 

{ \hrulefill
\begin{multicols}{2}
\footnotesize
\begin{itemize}
\item $-1 \leq \sin(x) \leq 1$ for all $x\in \R$
\item $\sin(x) = 1$ $\Longleftrightarrow$ $x\in  \frac{\pi}{2} + 2\pi \Z$ 
\item $\sin(x) = 0$ $\Longleftrightarrow$ $x\in \pi \Z$ 
\item $\sin(x) = -1$ $\Longleftrightarrow$ $x\in  \frac{-\pi}{2} + 2\pi \Z$
\item $\pi \notin \Q$
\item $\sin(x) = \sin(y)$ $\Longleftrightarrow$ $x-y\in 2\pi \Z$ or $x+y\in \pi + 2\pi \Z$
\end{itemize}
\end{multicols}
}
\end{enumerate}

\end{document}




