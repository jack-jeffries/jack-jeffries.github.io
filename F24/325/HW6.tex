\documentclass[12pt]{amsart}


\usepackage{times}
\usepackage[margin=0.8in]{geometry}
\usepackage{amsmath,amssymb,multicol,graphicx,framed,ifthen,color,xcolor,stmaryrd,enumitem,colonequals,hyperref}
\definecolor{chianti}{rgb}{0.6,0,0}
\definecolor{meretale}{rgb}{0,0,.6}
\definecolor{leaf}{rgb}{0,.35,0}
\newcommand{\Q}{\mathbb{Q}}
\newcommand{\N}{\mathbb{N}}
\newcommand{\Z}{\mathbb{Z}}
\newcommand{\R}{\mathbb{R}}
\newcommand{\C}{\mathbb{C}}
\newcommand{\F}{\mathbb{F}}
\newcommand{\e}{\varepsilon}
\newcommand{\inv}{^{-1}}
\newcommand{\dabs}[1]{\left| #1 \right|}
\newcommand{\ds}{\displaystyle}
\newcommand{\solution}[1]{\ifthenelse {\equal{\displaysol}{1}} {\begin{framed}{\color{meretale}\noindent #1}\end{framed}} { \ }}
\newcommand{\showsol}[1]{\def\displaysol{#1}}
\newcommand{\rsa}{\rightsquigarrow}
\newcommand\itemA{\stepcounter{enumi}\item[{\bf{(\theenumi)}}]}
\newcommand\itemB{\stepcounter{enumi}\item[(\theenumi)]}
\newcommand\itemC{\stepcounter{enumi}\item[{\it{(\theenumi)}}]}
\newcommand\itema{\stepcounter{enumii}\item[{\bf{(\theenumii)}}]}
\newcommand\itemb{\stepcounter{enumii}\item[(\theenumii)]}
\newcommand\itemc{\stepcounter{enumii}\item[{\it{(\theenumii)}}]}
\newcommand\itemai{\stepcounter{enumiii}\item[{\bf{(\theenumiii)}}]}
\newcommand\itembi{\stepcounter{enumiii}\item[(\theenumiii)]}
\newcommand\itemci{\stepcounter{enumiii}\item[{\it{(\theenumiii)}}]}
\newcommand\ceq{\colonequals}
\DeclareMathOperator{\ord}{ord}
\renewcommand{\ceq}{\colonequals}

\DeclareMathOperator{\res}{res}
\setlength\parindent{0pt}
%\usepackage{times}

%\addtolength{\textwidth}{100pt}
%\addtolength{\evensidemargin}{-45pt}
%\addtolength{\oddsidemargin}{-60pt}

\pagestyle{empty}
%\begin{document}\begin{itemize}

%\thispagestyle{empty}




\begin{document}
\showsol{1}
	
	\thispagestyle{empty}
	
	\section*{Assignment \#6: Due Tuesday, October 29at midnight}
	
	This problem set is to be turned in on Canvas. You may reference any result or problem from our worksheets or lectures, unless it is the fact to be proven! You are encouraged to work with others, but you should understand everything you write. Please consult the class website for acceptable/unacceptable resources for the problem sets.
	
	\
	
	

\begin{enumerate}

\item For each of the following, give an explicit example as indicated; no proofs are necessary.
\begin{enumerate}
\item A sequence that has a subsequence that converges to $1$, another subsequence that converges to $2$, and a third subsequence that converges to $3$. 
\item A sequence that has one sequence that is monotone and converges to $0$ and another subsequence that is monotone and diverges to $+\infty$.
\item A sequence of natural numbers such that for every natural number $n$ there is a subsequence converging to $n$. 
\end{enumerate}


\

\item Let $\{a_n\}_{n=1}^\infty$ be a sequence. Show that if $\{a_n\}_{n=1}^\infty$ diverges to $+\infty$, then every subsequence of $\{a_n\}_{n=1}^\infty$ diverges to $+\infty$ as well.

\


\item Let $\{a_n\}_{n=1}^\infty$ be a sequence and $L$ a real number. Show that $\{a_n\}_{n=1}^\infty$ converges to $L$ if and only if both of the subsequences $\{a_{2k}\}_{k=1}^\infty$ and $\{a_{2k+1}\}_{k=1}^\infty$ converge to $L$.

\end{enumerate}

\

\noindent \hrulefill

\

\begin{enumerate}\setcounter{enumi}{3}



\item Using just the $\e-\delta$ definition of limit, show that
\[ \lim_{x\to -1} \frac{x^2 - 6x - 7}{x+1} = -8.\]

\

\item Using just the $\e-\delta$ definition of limit, show that\footnote{Hint: You may want to use that $\displaystyle |\sqrt{x} - \sqrt{a} | = \frac{ |x-a| }{|\sqrt{x} + \sqrt{a} | }\leq\frac{ |x-a| }{|\sqrt{a} |}.$} for any $a>0$, $\lim_{x\to a} \sqrt{x} = \sqrt{a}$.




\end{enumerate}

\end{document}

























\end{document}