\documentclass[12pt]{amsart}


\usepackage{times}
\usepackage[margin=0.8in]{geometry}
\usepackage{amsmath,amssymb,multicol,graphicx,framed,ifthen,color,xcolor,stmaryrd,enumitem,colonequals,hyperref}
\definecolor{chianti}{rgb}{0.6,0,0}
\definecolor{meretale}{rgb}{0,0,.6}
\definecolor{leaf}{rgb}{0,.35,0}
\newcommand{\Q}{\mathbb{Q}}
\newcommand{\N}{\mathbb{N}}
\newcommand{\Z}{\mathbb{Z}}
\newcommand{\R}{\mathbb{R}}
\newcommand{\C}{\mathbb{C}}
\newcommand{\F}{\mathbb{F}}
\newcommand{\e}{\varepsilon}
\newcommand{\inv}{^{-1}}
\newcommand{\dabs}[1]{\left| #1 \right|}
\newcommand{\ds}{\displaystyle}
\newcommand{\solution}[1]{\ifthenelse {\equal{\displaysol}{1}} {\begin{framed}{\color{meretale}\noindent #1}\end{framed}} { \ }}
\newcommand{\showsol}[1]{\def\displaysol{#1}}
\newcommand{\rsa}{\rightsquigarrow}
\newcommand\itemA{\stepcounter{enumi}\item[{\bf{(\theenumi)}}]}
\newcommand\itemB{\stepcounter{enumi}\item[(\theenumi)]}
\newcommand\itemC{\stepcounter{enumi}\item[{\it{(\theenumi)}}]}
\newcommand\itema{\stepcounter{enumii}\item[{\bf{(\theenumii)}}]}
\newcommand\itemb{\stepcounter{enumii}\item[(\theenumii)]}
\newcommand\itemc{\stepcounter{enumii}\item[{\it{(\theenumii)}}]}
\newcommand\itemai{\stepcounter{enumiii}\item[{\bf{(\theenumiii)}}]}
\newcommand\itembi{\stepcounter{enumiii}\item[(\theenumiii)]}
\newcommand\itemci{\stepcounter{enumiii}\item[{\it{(\theenumiii)}}]}
\newcommand\ceq{\colonequals}
\DeclareMathOperator{\ord}{ord}
\renewcommand{\ceq}{\colonequals}

\DeclareMathOperator{\res}{res}
\setlength\parindent{0pt}
%\usepackage{times}

%\addtolength{\textwidth}{100pt}
%\addtolength{\evensidemargin}{-45pt}
%\addtolength{\oddsidemargin}{-60pt}

\pagestyle{empty}
%\begin{document}\begin{itemize}

%\thispagestyle{empty}




\begin{document}
\showsol{1}
	
	\thispagestyle{empty}
	
	\section*{Assignment \#4: Due Thursday, October 3 at midnight}
	
	This problem set is to be turned in on Canvas. You may reference any result or problem from our worksheets or lectures, unless it is the fact to be proven! You are encouraged to work with others, but you should understand everything you write. Please consult the class website for acceptable/unacceptable resources for the problem sets.
	
	\
	
	



\begin{enumerate}

\item Prove that the sequence $\{ \sqrt{n} \}_{n=1}^{\infty}$ diverges.

\

\item Assume that $\{a_n\}_{n=1}^\infty$ converges to zero, and that $a_n\geq 0$ for all natural numbers $n$. Show\footnote{Warning! It is \emph{not} true that $\sqrt{r} \leq r$ for all nonnegative numbers $r$.} that $\{\sqrt{a_n}\}_{n=1}^\infty$ converges to zero also. 

\

\item Prove that the sequence $\displaystyle \left\{ \frac{-3 n^2 + 4}{4 n^2 -n +3 }\right\}_{n=1}^\infty$ converges to $\displaystyle\frac{-3}{4}$. You should use our Theorem on Limits and Algebra, but explain carefully each step how you apply the Theorem.

\

\item Find, with justification, examples of sequences such that:
\begin{enumerate}
\item $\{a_n\}_{n=1}^\infty$ gets closer and closer to to $7$ (meaning $\{|a_n-7|\}_{n=1}^\infty$ is strictly decreasing) but $\{a_n\}_{n=1}^{\infty}$ does not converge to $7$.
\item Every term of $\{b_n\}_{n-1}^\infty$ is within $0.0000001$ of $\pi$ (meaning for all $n\in \N$, we have \\ ${|b_n-\pi|<0.0000001}$), but $\{b_n\}_{n=1}^{\infty}$ does not converge to $\pi$.
\item The first $1000$ terms of $\{c_n\}_{n=1}^\infty$ are all larger than $1000000$, but $\{c_n\}_{n=1}^\infty$ converges to $0$.
\end{enumerate}





\end{enumerate}

\end{document}

























\end{document}