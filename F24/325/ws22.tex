\documentclass[12pt]{amsart}


\usepackage{times}
\usepackage[margin=.9in]{geometry}
\usepackage{amsmath,amssymb,multicol,graphicx,framed}
\usepackage[all]{xy}
\newcommand{\Q}{\mathbb{Q}}
\newcommand{\N}{\mathbb{N}}
\newcommand{\Z}{\mathbb{Z}}
\newcommand{\R}{\mathbb{R}}
\newcommand{\inv}{^{-1}}
\newcommand{\dabs}[1]{\left| #1 \right|}
\newcommand{\e}{\varepsilon}
\newcommand{\de}{\delta}
\newcommand{\ds}{\displaystyle}

\DeclareMathOperator{\res}{res}

\pagestyle{empty}




\begin{document}
	
	\thispagestyle{empty}
	
	\section*{The $\e-\de$ game. \S 3.1}
	



\begin{enumerate}
\item Play the $\e-\de$ game. Rules:
\begin{itemize}
\item Player 0 starts by graphing a function $f$ (like a familiar one from calculus) and specifies an $x$-value $a$ and a $y$-value $L$ that (based on previous calculus knowledge) they think makes $\lim_{x\to a} f(x) = L$ \textbf{true}.\\
The graph should be large.
\item Player 1 choses an $\e$. This is how close we would like our function to be to $L$. Thus, $\e$ goes up and down from $L$. Draw horizontal dotted lines with $y$-values $L-\e$ and $L+\e$.\\
 The $\e$ should be large enough for people to see and have room to work in the picture.
\item Player 2 must find a $\de$ such that every $x \in (a-\de,a) \cup (a,a+\de)$ is 
\begin{itemize}
\item in the domain of $f$, and
\item has an output $f(x)$ within $(L-\e,L+\e)$.
\end{itemize}
Draw vertical dotted lines for the $x$-values $a-\de$ and $a+\de$.\\
Everyone in the team can assist player 2!
\end{itemize}

\

\item Repeat with another function.


\

\item Now play except player 0 graphs an $f$ with an $L$ that makes $\lim_{x\to a} f(x) = L$ \textbf{false}.

\

\item Now play except player 0 ``chooses'' $\ds f(x) = \frac{2x-2}{x-1}$, $a=1$, and $L=2$. Play with actual numbers $\e$ and $\de$.


\end{enumerate}


\end{document}
