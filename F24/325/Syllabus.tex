\documentclass{amsart}
\usepackage{pdfpages}
% \pagestyle{empty}
\usepackage{amscd}
\usepackage{amssymb}
%\usepackage[all, knot]{xy}



\usepackage[top=1.6in, bottom=1.6in, left=1in, right=1in]{geometry}
\usepackage{multicol}
%\setlength{\oddsidemargin}{-.10in}
%\setlength{\evensidemargin}{-.10in}
%\setlength{\textwidth}{5.5in}
%\setlength{\topmargin}{-.250in}
%\setlength{\headheight}{0in}
%\setlength{\headsep}{0in}
%\setlength{\topskip}{0in}
%\setlength{\textheight}{9.5in}
%\parindent = 0in
\font\bigbf = cmbx10 scaled \magstep1
\font\medbf = cmbx10 scaled \magstephalf
\begin{document}
 
%\magnification \magstep1
%\parindent = 0pt
%\nopagenumbers
%\voffset = -.5truein
%\vsize = 10 truein
%\baselineskip = 1.5 \baselineskip
%\font\bigbf = cmbx10 scaled \magstep1
%\font\medbf = cmbx10 scaled \magstephalf








\centerline{\bigbf Elementary Analysis: Math 325 -- Section 001}
\centerline{\bigbf Fall Semester 2024}
\centerline{\bigbf MWF 12:30pm -- 1:20pm }
\centerline{\bigbf Avery Hall 111}


\bigskip

\noindent
{\bf Instructor:}  Jack Jeffries


\noindent
{\bf Office:} 325 Avery Hall

\noindent
{\bf Email:} jack.jeffries@unl.edu

\noindent
{\bf Learning Assistant:}  TBA

\noindent
{\bf Office Hours:} Mondays 9:30am --11:00am and Wednesdays 1:30pm -- 3:00pm at Jack's office. Our learning assistant will also hold office hours, at time and location to be announced.




\noindent{\bf Textbook:} 
{\it Understanding Real Analysis} by Paul Zorn. This book is optional. You are encouraged to read through the book as a supplement to our work in class and homework. %My typed lecture notes will also be available on the course website.


\medskip


\noindent{\bf Course Description:} 
I view this course as having two central goals:

One goal is to develop your ability to 
read, write, and understand 
rigorous mathematical proofs and definitions. The only way to get better at writing good 
proofs is by doing so. There will be a heavy emphasis on
weekly problem sets in this course and most of the
problems
will require you to develop original, rigorous
proofs of mathematical statements. Writing good proofs is a
difficult skill to master and success in this course 
will require a sustained effort on your
part along with my help. 

The second main goal is to learn about
real analysis, which, at
least from our point of view, will mean developing rigorously the
material of calculus. We will start with a short list of axioms that
characterize the real numbers.  Using them, and the rules of logical
reasoning, we will prove many of the theorems you accepted on faith
when you took calculus.

The first course that I took in analysis as a student was one of the most inspiring classes I ever had. I hope to share the beauty of this subject with you.




\smallskip

\noindent{\bf Class engagement policy:} 
This is an in-person class. Class time will involve lecture, discussion, and groupwork, as well as quizzes. Attendance for this class is mandatory. If you are temporarily unable to attend in person for health reasons, you should let me know as soon as possible, and we will arrange for you to participate some other way while you cannot attend in person.



\noindent{\bf Grading policy:} Your grade will have four components: problem sets, quizzes, exams, and participation. 

The problem sets will be assigned and collected approximately once per week, except for exam weeks. I anticipate there being about eight problems sets total.
You are encouraged to work together on
the problem sets, but each of you will hand in 
your own solutions, written in your own words, and your work must
demonstrate a true understanding of the material. Never hand in
something that you do not completely understand. You can ask me about I ask that at the top of each assignment you list the students, if any, with whom you collaborated and any outside references you use. 
Please hand in all assignments on time.


We will have short quizzes in class on a semi-regular basis. Quiz times will be discussed in class. The participation grade is based on your participation in class. If you fulfill the class engagement policy then you will get the full score; I expect everyone to do so. This part of the grade is an excuse for me to give you easy points.

There will be two midterm exams and a final exam. \emph{Tentative} midterm dates are Friday October 11 and Friday, November 15, and the final exam will be \textbf{3:30 pm -- 5:30 pm, Tuesday, December 17}, in our standard classroom unless announced otherwise.

\newpage

\noindent The following table summarizes the grading scheme:



\bigskip

\noindent{\bf Components of your grade:} 
\medskip

\begin{tabular}{|l|l|}
\hline
Component & Value \\
\hline \hline
Problem Sets & 30\% \\
\hline
Quizzes & 20\% \\
\hline
Midterm Exams (two) &  11\% each \\
\hline
Final Exam & 22\% \\
\hline
Participation & 6\%\\
\hline
\end{tabular}


\


\noindent Letter grades will be based on the usual 10 point scale (90 cutoff between A-/B+, etc.); however, grade cutoffs may be lower (i.e., grades may be higher).

\

\noindent
{\bf Syllabus}: The plan is to cover the majority of the text, skipping some of the less important sections but covering the high points.
The main topics we will learn about are as follows:
\begin{itemize}
  \item 
    basic terminology and  notation, proof techniques, the axioms of the real numbers
    \item sequences, convergence of sequences, Bolzano-Weierstrass, Cauchy sequences
      \item continuity of functions,  value theorems,
\item formal definition of the derivative and its properties, Mean Value theorem
  \item (if time permits) Riemann integrals, Fundamental Theorems of Calculus \\ 
  \end{itemize}

\bigskip

\noindent {\bf Students With Disabilities:} Students with disabilities are encouraged to contact the instructor for a confidential discussion of their
individual needs for academic accommodation. It is the policy of the University of Nebraska-Lincoln to provide flexible and individualized accommodation to
students with documented disabilities that may affect their ability to fully participate in course activities or to meet course requirements. To receive
accommodation services, students must be registered with the Services for Students with Disabilities office, 132 Canfield Administration, 472-3787 voice or TTY,
{\tt http://www.unl.edu/ssd.}

\bigskip
\noindent {\bf Department Grading Policy:} Students who believe their academic evaluation has been prejudiced or capricious have recourse for appeals to (in
order) the instructor, the department vice chair, the department chair, the departmental appeals committee, and the college appeals committee. 



\bigskip
\noindent
{\bf Academic Honesty:}
Academic honesty is essential to the existence and integrity of an academic institution. The responsibility for maintaining that integrity is shared by all members of the academic community. The University's Student Code of Conduct addresses academic dishonesty. Students who commit acts of academic dishonesty are subject to disciplinary action and are granted due process and the right to appeal any decision.

\bigskip
\noindent
\textbf{UNL Course Policies and Resources}
Students are responsible for knowing the university policies and resources found on this page (https://go.unl.edu/coursepolicies):
 \begin{multicols}{2}
 \begin{itemize}
\item University-wide Attendance Policy
\item Academic Honesty Policy
\item Services for Students with Disabilities
\item Mental Health and Well-Being Resources
\item Final Exam Schedule 
\item Fifteenth Week Policy 
\item Emergency Procedures 
\item Diversity \& Inclusiveness 
\item Title IX Policy 
\item Other Relevant University-Wide Policies
\end{itemize} 
\end{multicols}

\vfill
\pagebreak
\smallskip
%\includepdf{UNLFace.pdf}



\end{document}


