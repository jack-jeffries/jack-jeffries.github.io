\documentclass[12pt]{amsart}


\usepackage{times}
\usepackage[margin=0.8in]{geometry}
\usepackage{amsmath,amssymb,multicol,graphicx,framed,ifthen,color,xcolor,stmaryrd,enumitem,colonequals,hyperref}
\definecolor{chianti}{rgb}{0.6,0,0}
\definecolor{meretale}{rgb}{0,0,.6}
\definecolor{leaf}{rgb}{0,.35,0}
\newcommand{\Q}{\mathbb{Q}}
\newcommand{\N}{\mathbb{N}}
\newcommand{\Z}{\mathbb{Z}}
\newcommand{\R}{\mathbb{R}}
\newcommand{\C}{\mathbb{C}}
\newcommand{\F}{\mathbb{F}}
\newcommand{\e}{\varepsilon}
\newcommand{\inv}{^{-1}}
\newcommand{\dabs}[1]{\left| #1 \right|}
\newcommand{\ds}{\displaystyle}
\newcommand{\solution}[1]{\ifthenelse {\equal{\displaysol}{1}} {\begin{framed}{\color{meretale}\noindent #1}\end{framed}} { \ }}
\newcommand{\showsol}[1]{\def\displaysol{#1}}
\newcommand{\rsa}{\rightsquigarrow}
\newcommand\itemA{\stepcounter{enumi}\item[{\bf{(\theenumi)}}]}
\newcommand\itemB{\stepcounter{enumi}\item[(\theenumi)]}
\newcommand\itemC{\stepcounter{enumi}\item[{\it{(\theenumi)}}]}
\newcommand\itema{\stepcounter{enumii}\item[{\bf{(\theenumii)}}]}
\newcommand\itemb{\stepcounter{enumii}\item[(\theenumii)]}
\newcommand\itemc{\stepcounter{enumii}\item[{\it{(\theenumii)}}]}
\newcommand\itemai{\stepcounter{enumiii}\item[{\bf{(\theenumiii)}}]}
\newcommand\itembi{\stepcounter{enumiii}\item[(\theenumiii)]}
\newcommand\itemci{\stepcounter{enumiii}\item[{\it{(\theenumiii)}}]}
\newcommand\ceq{\colonequals}
\DeclareMathOperator{\ord}{ord}
\renewcommand{\ceq}{\colonequals}

\DeclareMathOperator{\res}{res}
\setlength\parindent{0pt}
%\usepackage{times}

%\addtolength{\textwidth}{100pt}
%\addtolength{\evensidemargin}{-45pt}
%\addtolength{\oddsidemargin}{-60pt}

\pagestyle{empty}
%\begin{document}\begin{itemize}

%\thispagestyle{empty}




\begin{document}
\showsol{1}
	
	\thispagestyle{empty}
	
	\section*{Assignment \#1: Due Thursday, September 5 at 7pm}
	
	This problem set is to be turned in on Canvas. You may reference any result or problem from our worksheets or lectures, unless it is the fact to be proven! You are encouraged to work with others, but you should understand everything you write. Please consult the class website for acceptable/unacceptable resources for the problem sets.
	
	\
	
	



\begin{enumerate}
\item For each of the following sets, which of the properties listed in Proposition 1.2, do \emph{not} hold if one replaces $\mathbb{Q}$ with the indicated set? Give a brief explanation. 
\begin{enumerate}
\item The set of nonnegative integers $\{0, 1, 2, 3, \dots\}$.
\item The set of nonnegative rational numbers $\{q \in \Q \ | \ q \geq 0\}$.
\item The set of all integers $\Z = \{\dots,-2,-1,0,1,2,\dots\}$.
\end{enumerate}

\

\item Prove the following ``Cancellation of multiplication'' property: If $x, y,$ and $z$ are real numbers such that $xy = xz$ and $x\neq 0$, then $y = z$. Your proof should use nothing other than the axioms of the real numbers, just as the proof of Cancellation of Addition from class. (You will not need to use the completeness axiom).

\


\item Let $x$ and $y$ be real numbers. 
\begin{enumerate}
\item Prove that if $x^2$ is irrational, then $x$ is irrational.
\item\label{or} Prove that if $xy$ is irrational, then $x$ is irrational or $y$ is irrational.
\item Is the converse of (\ref{or}) true? Prove or disprove.
\end{enumerate}

\

\item Let $x$ be a real number. Use the axioms of $\R$ and facts we have proven in class to show that if there exists a real number y such that $xy = 1$, then $x\neq 0$.

\


\item\label{sqrt3} Prove that there is no rational number whose square is 3 by mimicking\footnote{This means many of the steps will be the same, but some details will be different. In particular, ``even'' and ``odd'' might not show up in your proof\dots} the proof of Theorem~1.1 from class.


%\item Prove\footnote{Hint: This will require a different idea than problem (\ref{sqrt3}). Instead, you can use without proof that if $x\leq 0$ and $y\leq 0$, then $xy\geq 0$, and that $1>0$, which we discussed in class.} that there is no real number whose square is $-1$.
\end{enumerate}

\end{document}

























\end{document}