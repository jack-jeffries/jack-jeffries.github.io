\documentclass[12pt]{amsart}


\usepackage{times}
\usepackage[margin=0.8in]{geometry}
\usepackage{amsmath,amssymb,multicol,graphicx,framed}
\newcommand{\Q}{\mathbb{Q}}
\newcommand{\N}{\mathbb{N}}
\newcommand{\Z}{\mathbb{Z}}
\newcommand{\R}{\mathbb{R}}
\newcommand{\inv}{^{-1}}
\newcommand{\dabs}[1]{\left| #1 \right|}

\DeclareMathOperator{\res}{res}

%\usepackage{times}

%\addtolength{\textwidth}{100pt}
%\addtolength{\evensidemargin}{-45pt}
%\addtolength{\oddsidemargin}{-60pt}

\pagestyle{empty}
%\begin{document}\begin{itemize}

%\thispagestyle{empty}




\begin{document}
	
	\thispagestyle{empty}
	
	\section*{Warming up with if-then statements and real numbers}
	
\subsection*{Making sense of if-then statements}

\

\begin{framed}
\begin{itemize}
\item The \emph{converse} of the statement  ``If $P$ then $Q$'' is the statement  ``If $Q$ then $P$''.
\item The \emph{contrapositive} of the statement  ``If $P$ then $Q$'' is the statement  ``If not $Q$ then not $P$''.
\item Any if then statement is equivalent to its contrapositive, but not necessarily to its converse!
\end{itemize}
\end{framed}




\begin{enumerate}
\item For each of the following statements, write its contrapositive and its converse. Is the original/contrapositive/converse true or false for real numbers $a,b$? Explain why (but don't prove). 
\begin{enumerate}
\item If $a$ is irrational, then $1/a$ is irrational.
\item If $a$ and $b$ are irrational, then $ab$ is irrational.
\item If $a>3$, then $a^2>9$.
\end{enumerate}
\end{enumerate}





\subsection*{Proving if-then statements}

\

\begin{framed}
\begin{itemize}
\item The general outline of a direct proof of ``If $P$ then $Q$'' goes
\begin{enumerate}
\item Assume $P$.
\item Do some stuff.
\item Conclude $Q$.
\end{enumerate}
\item Often it is easier to prove the contrapositive of an if-then statement than the original, especially when the conclusion is something negative. We sometimes call this an \emph{indirect proof} or a \emph{proof by contraposition}.
\end{itemize}

\end{framed}

\begin{enumerate}\setcounter{enumi}{1}
\item Consider the following proof of the claim ``For real numbers $x,y,z$, if $x+y=z+y$, then $x=z$'' from the list of axioms of $\R$. Match the parts of this proof with the general outline above. Which sentences are \emph{assumptions} and which sentences are \emph{assertions} (i.e., say something is true)? Is it clear \emph{just from reading each sentence on its own} whether it is an assumption or an assertion?
\begin{quote}
\emph{Proof.} Suppose that $x+y=z+y$. Then we can add $-y$ (which exists by Axiom 6) to get 
\[(x+y)+ (-y) = (z+y) + (-y).\]
This can be rewritten (by Axiom 3) as
\[x+ (y+ (-y) )= z+ (y + (-y)),\]
and hence (by Axiom 6) as
\[ x+ 0 = z+0,\]
which gives $x=z$ (by Axioms 4 and 2).\qed
\end{quote}



\item Consider the following purported proof of the true fact ``If $2x + 5 \geq 7$ then $x\geq 1$.'' Is this a good proof? Is it a correct proof?
\begin{quote}
\emph{Proof.}
\[ x \geq 1.\]
Multiply both sides by two.
\[ 2x \geq 2.\]
Add five to both sides.
\[2x+5 \geq 7. \qed\]
\end{quote}
\end{enumerate}


\subsection*{Proving if-then statements about real numbers}

\begin{enumerate}\setcounter{enumi}{3}


\item Let $x$ be a real number. Show that if $x^2$ is irrational, then $x$ is irrational.

\

\item Let $x$ and $y$ be real numbers. Use the axioms of $\R$ to prove\footnote{Hint: You may want to add something to both sides.} that if $x \geq y$ then $-x \leq -y$.


\

\item Let $x$ and $y$ be real numbers. Use the axioms of $\R$ to prove\footnote{Hint: You may want to add something to both sides.} that $x \geq y$ if and only if $-x \leq -y$.


\

\item Let $x,y$ be real numbers. Use the axioms of $\R$ and facts we have already proven\footnote{Be careful: are you using any facts that we have not already proven?} to prove that if $x\leq 0$ and $y\leq 0$, then $xy\geq 0$. 

\



\item Use\footnote{Hint: Try a proof by contradiction.} the axioms of $\R$ and facts we have already proven to prove that $1>0$.


\end{enumerate}


\end{document}
