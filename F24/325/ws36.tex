\documentclass[12pt]{amsart}


\usepackage{times}
\usepackage[margin=.75in]{geometry}
\usepackage{amsmath,amssymb,multicol,graphicx,framed}
\usepackage[all]{xy}
\newcommand{\Q}{\mathbb{Q}}
\newcommand{\N}{\mathbb{N}}
\newcommand{\Z}{\mathbb{Z}}
\newcommand{\R}{\mathbb{R}}
\newcommand{\inv}{^{-1}}
\newcommand{\dabs}[1]{\left| #1 \right|}
\newcommand{\e}{\varepsilon}
\newcommand{\de}{\delta}
\newcommand{\ds}{\displaystyle}

\DeclareMathOperator{\res}{res}

\pagestyle{empty}




\begin{document}
	
	
	\thispagestyle{empty}
	
	
	\section*{The Chain Rule \S4.1}
	
\begin{framed} 


 \noindent \textsc{Theorem 36.1 (Chain Rule):}   
  Suppose $g$ is differentiable at $s$ and $f$ is
  differentiable  at $g(s)$. Then $f \circ g$ is differentiable at $s$ and
$$
(f \circ g)'(s) = f'(g(s)) g'(s).
$$


 \end{framed}


\

\begin{enumerate}
\item Use the chain rule to compute the derivative of $\sqrt{3x^2 + 13}$.

\

\

\item Given that the derivative of $\sin(x)$ at $x=r$ is $\cos(r)$, and reusing any earlier computations, compute the derivative of the function
\[ j(x) = \begin{cases} x^2 \sin(\frac{1}{x}) & \ \text{if} \ x\neq 0 \\
0 & \ \text{if} \ x= 0 \end{cases}\]
at any real number $x=r$.


\

\


\item Proof of Chain Rule:

\begin{enumerate}
\item In the setting of the statement, set $r=g(s)$ and define
\[ d(y) = \begin{cases} \ds \frac{f(y) - f(r)}{y-r} & \text{if} \ y\neq r \\ f'(r) & \text{if} \ y=r.\end{cases}\]
Show that $d$ is continuous at $y=r$.
\item Show that if $x\neq s$, then
\[ \frac{f(g(x)) - f(g(s))}{x-s} = d(g(x)) \cdot \frac{g(x)-g(s)}{x-s}.\]
Note: You might have to consider separately the cases with $g(x)\neq r$ and $g(x)=r$.
\item Use the fact that a composition of continuous functions (at the suitable inputs) is continuous to compute $\lim\limits_{x\to s} d(g(x))$.
\item Compute $\lim\limits_{x\to s}$ on both sides of (b). Use your computation to deduce the Chain Rule.
\end{enumerate}
\end{enumerate}

\end{document}

