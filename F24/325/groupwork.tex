\documentclass[12pt]{amsart}


\usepackage{times}
\usepackage[margin=0.7in]{geometry}
\usepackage{amsmath,amssymb,multicol,graphicx,framed}
\usepackage[shortlabels]{enumitem}
\newcommand{\Q}{\mathbb{Q}}
\newcommand{\N}{\mathbb{N}}
\newcommand{\Z}{\mathbb{Z}}
\newcommand{\R}{\mathbb{R}}
\newcommand{\inv}{^{-1}}
\newcommand{\dabs}[1]{\left| #1 \right|}

\DeclareMathOperator{\res}{res}

%\usepackage{times}

%\addtolength{\textwidth}{100pt}
%\addtolength{\evensidemargin}{-45pt}
%\addtolength{\oddsidemargin}{-60pt}

\pagestyle{empty}
%\begin{document}\begin{itemize}

%\thispagestyle{empty}




\begin{document}
	
	\thispagestyle{empty}
	
	\section*{Expectations of Groupwork}
	
Here are the expectations for groupwork in this class:

\subsection*{Solving a problem means everyone in the group understands the solution}  If you do not understand your solution or are unsure of something, let your group know: they are probably missing something or could understand some detail better. Conversely, if someone in your group doesn't understand the solution, you should thank them for the opportunity to understand the problem better, as you may have missed something, or you might understand better by explaining your thoughts if you think you haven't.

\subsection*{Equal participation} Every class, I expect every person in each group to write a roughly equal amount, and every person to contribute ideas. Your participation not only benefits yourself, but all of your group members as well!


\subsection*{Share responsibility for making sure all voices are heard}
If you tend to have a lot to say, make sure you leave sufficient space to hear from others.  If you tend to stay quiet in group discussions, challenge yourself to contribute so that others can learn from you.

\subsection*{Understand that we are bound to make mistakes} 
As anyone does when approaching complex tasks or learning new skills.
In particular, you are invited to step outside your comfort zone!

\subsection*{Other thoughts}
\begin{itemize}
\item Respect others' rights to solve problems differently than you or to approach a problem differently from you.
\item When you disagree, challenge or criticize the idea, not the person.
\item Listen carefully to what others are saying even when you disagree with what  is  being  said.   Comments  that  you make (asking  for  clarification, sharing critiques, expanding on a point, etc.)  should reflect that you have paid attention to the speaker’s comments.
\item Be  courteous.  Don’t interrupt or engage in private conversations while others are speaking.   Be aware of messages you may be communicating with your body language.
\item Support your statements.  Use evidence and provide a rationale for your points.
\item Understand that there are different approaches to solving problems.  If you are uncertain  about someone else's approach, ask a question to explore areas of uncertainty.  Listen respectfully to  how and why the approach could work.
\item Take groupwork seriously.  Remember that your peers' learning partly depends upon your engagement.
%\item Be careful about how you use humor or irony in class.  Keep in mind that we don't all find the same things funny.
\item Make an effort to get to know other students.  Introduce yourself to students sitting near you.  Refer to classmates by name and make eye contact with other students.
\item Encourage respectful disagreement with one another and with the professor.
\item Be aware of different communication styles---the ways we communicate differently based on our backgrounds and current contexts---and look for ways to expand your communication toolkits.
\item Recognize that we are all still learning.  Be willing to change your perspective, and make space for others to do the same.
\end{itemize}

\end{document}
