\documentclass[12pt]{amsart}


\usepackage{times}
\usepackage[margin=.9in]{geometry}
\usepackage{amsmath,amssymb,multicol,graphicx,framed,enumitem}
\newcommand{\Q}{\mathbb{Q}}
\newcommand{\N}{\mathbb{N}}
\newcommand{\Z}{\mathbb{Z}}
\newcommand{\R}{\mathbb{R}}
\newcommand{\e}{\varepsilon}
\def\d{\delta}
\newcommand{\cts}{continuous\,}

\newcommand{\dabs}[1]{\left| #1 \right|}
\newcommand{\ds}{\displaystyle}
\usepackage[all, knot]{xy}
\usepackage{color,xcolor}
%\usepackage{times}

%\addtolength{\textwidth}{100pt}
%\addtolength{\evensidemargin}{-45pt}
%\addtolength{\oddsidemargin}{-60pt}

\pagestyle{empty}
%\begin{document}\begin{itemize}

%\thispagestyle{empty}




\begin{document}
	
	\thispagestyle{empty}
	
	\section*{Continuous functions}
	
	\begin{framed}
\noindent \textsc{From last time:}

\

\noindent \textbf{Definition:}  A function $f$ is \emph{continuous at $a$} provided: For any $\e>0$, there exists $\d>0$ such that if $|x-a|<\d$ then $f(x)$ is defined and $|f(x)-f(a)|<\e$.

\

\noindent \textbf{Theorem:} If $f$ is defined at $a$ then $f$ is \cts at $a$ if and only if $\ds\lim_{x\to a} f(x) = f(a)$.

\

\noindent \textbf{Theorem:} If $f$ and $g$ are both continuous at $a$, and $c$ is any constant, then
\begin{enumerate}
\item $f+g$ is \cts at $a$.
\item $cf$ is \cts at $a$.
\item $fg$ is \cts at $a$.
\item $f/g$ is \cts at $a$, provided $g(a)\neq 0$.
\end{enumerate}

\

\noindent \textbf{Theorem:} If $g$ is \cts at $a$ and $f$ is \cts at $g(a)$, then $f\circ g$ is \cts at $a$.


\end{framed}

\

\begin{enumerate}
\item Let \[f(x) = \begin{cases} 2x &\text{if} \ x\geq 1 \\ x+1 &\text{if} \ x<1 \end{cases}.\]
 Use the $\e-\delta$ definition to show that $f(x)$ is continuous at $1$.
 
 \
 
\item Let \[g(x) = \begin{cases} x &\text{if} \ x\in \Q \\ 0 &\text{if} \ x\notin \Q \end{cases}.\]
 Show that $g(x)$ is continuous at $0$ and is \emph{not} continuous at any other real number. You can use any theorems you like and anything relevant from the homework.

\

\item Let $h(x) = \sqrt{ x^2 +5}$. Show that $h$ is continuous at $a$ for every $a\in \R$.
 \end{enumerate}
 
 \
 
 \begin{framed}
\noindent It is tiresome to say ``continuous at $a$ for every $a\in \R$''. The following definition is then convenient.

\

\noindent \textbf{Definition 21.1:}  Let $S$ be an open interval of $\R$ of the form $S=(a,b)$, $S=(a,\infty)$, $S=(-\infty,a)$, or $S=(-\infty,\infty)=\R$. We say $f$ is {\em continuous on $S$} if $f$ is continuous at $a$ for all $a\in S$. 

 
 \end{framed}
 
 \
 
 \begin{enumerate}
 \setcounter{enumi}{3}
 \item Which of the following functions are continuous on $\R$?
\begin{multicols}{2} 
\begin{itemize}
 \item $f(x)=\sqrt{ x^2 +5}$.
 \item Every polynomial function.
  \item $f(x) = \sqrt{x}$.
  \item $f(x)=\frac{1}{x}$.
 \end{itemize}
 \end{multicols}
 
 \
 
 \
 
 \item Which of the following functions are continuous on $(0,\infty)$?
 \begin{multicols}{2} 
 \begin{itemize}
 \item $f(x)=\sqrt{ x^2 +5}$.
 \item Every polynomial function.
  \item $f(x) = \sqrt{x}$.
  \item $f(x)=\frac{1}{x}$.
 \end{itemize}
 \end{multicols}
 
 \
 
 \item Prove that $j(x) = x \sin(1/x)$ is continuous on $\R$. (You can use wiithout proof that $\sin(x)$ is continuous on $\R$).
 
 \
 
 \item Prove or disprove: If $f$ and $g$ are two functions, $a\in \R$, and  $f(a)=g(a)$ , then $f$ is continuous at $a$ if and only if $g$ is continuous at $a$.
 
 \

 
 \item Prove or disprove: If $f$ and $g$ are two functions, $a<b$, and  $f(x)=g(x)$ for all $x\in (a,b)$, then $f$ is continuous on $(a,b)$ if and only if $g$ is continuous on $(a,b)$.
   \end{enumerate}
 
 \
 
 \begin{framed} 
 \noindent The definition of continuous on a closed interval $[a,b]$ is actually a bit different: we shouldn't necessarily ask that $f$ be continuous at $a$, since to know that would have to use something about $f$ on input values outside of our interval!
 
 \
 
\noindent\textbf{Definition~21.2:}  Given a function $f(x)$ and real numbers $a < b$,
we say $f$ is {\em continuous on the closed interval $[a,b]$} provided 
\begin{enumerate}
\item for every $r \in (a,b)$, $f$ is  continuous at $r$ in the sense defined already,
\item for every $\e > 0$ there is a $\d > 0$ such that if $a \leq x < a+\d$, then ${|f(x)
  - f(a)| < \e}$.
\item for every $\e > 0$ there is a $\d > 0$ such that if $b -\d < x \leq b$, then ${|f(x)
  - f(b)| < \e}$.
\end{enumerate}
\end{framed}

\

 \begin{enumerate}
 \setcounter{enumi}{8}
 \item Explain why if $f$ is continuous at $x$ for every $x\in[a,b]$, then $f$ is continuous on the closed interval $[a,b]$. Conclude that every polynomial is continuous on every closed interval.


\

\item Show that the function $f(x) = \sqrt{1-x^2}$ is continuous on the closed interval $[-1,1]$:
\begin{itemize}
\item For showing condition (1), I recommend using a Theorem from last class. 
\item For condition (2), it may help to write $\sqrt{1-x^2}=\sqrt{1-x}\sqrt{1+x}$. %\footnote{You can ask me for a $\delta$ if you get stuck.}
\item Condition (3) is similar to condition (2) so you can just say ``Similar to (2)'' for that.
\end{itemize}
   \end{enumerate}








\end{document}
