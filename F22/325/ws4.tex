\documentclass[12pt]{amsart}


\usepackage{times}
\usepackage[margin=0.8in]{geometry}
\usepackage{amsmath,amssymb,multicol,graphicx,framed}
\newcommand{\Q}{\mathbb{Q}}
\newcommand{\N}{\mathbb{N}}
\newcommand{\Z}{\mathbb{Z}}
\newcommand{\R}{\mathbb{R}}
\newcommand{\e}{\varepsilon}
\newcommand{\inv}{^{-1}}
\newcommand{\dabs}[1]{\left| #1 \right|}
\newcommand{\ds}{\displaystyle}

\DeclareMathOperator{\res}{res}

%\usepackage{times}

%\addtolength{\textwidth}{100pt}
%\addtolength{\evensidemargin}{-45pt}
%\addtolength{\oddsidemargin}{-60pt}

\pagestyle{empty}
%\begin{document}\begin{itemize}

%\thispagestyle{empty}




\begin{document}
	
	\thispagestyle{empty}
	
	\section*{Convergence of sequences}
	
	

\begin{framed}
\noindent \textbf{Definition:} Let $\{a_n\}_{n=1}^\infty$ be an arbitrary sequence and $L$ a real number. We say $\{a_n\}_{n=1}^\infty$ {\em converges} to $L$ provided if for every real number $\e > 0$, there is a real number $N$ such that $|a_n - L| < \e$ for all natural numbers $n$ such that $n > N$.

\

\noindent To prove that a particular sequence $\{a_n\}_{n=1}^\infty$ converges to a particular real number $L$ directly from the definition:
\begin{itemize}
\item Let $\e>0$ be arbitrary.
\item Take $N=[\text{expression from scratchwork outside of the proof, maybe in terms of $\e$,} \\ \text{that makes } {|a_n-L|<\e} \text{ whenever } n>N]$.
\item Let $n>N$ be a natural number.
\item $\text{[Argument that } |a_n-L|<\e \text{ (that cannot refer to the previous scratchwork outside the proof)}]$
\item Thus $\{a_n\}_{n=1}^\infty$ converges to $L$.
\end{itemize}
\end{framed}

\

\begin{enumerate}
\item Let $c$ be a real number. Prove that the constant sequence $\{ c\}_{n=1}^\infty$ converges to $c$.

\



\item Prove that\footnote{By $\sqrt{n}$, we mean the positive number whose square is $n$. Such a number exists by a proof similar to the one that $\sqrt{2}$~exists.} the sequence $\ds \left\{ \frac{1}{\sqrt{n}} \right\}_{n=1}^\infty$ converges to $0$.

\



\item Let $\{a_n\}_{n=1}^\infty$ be a sequence. Suppose we know that $\{a_n\}_{n=1}^\infty$ converges to $1$. Prove that there is a natural number $n\in \N$ such that $a_n>0$.

\



\item Prove or disprove: The sequence $\ds \left\{\frac{n+1}{2n} \right\}_{n=1}^\infty$ converges to $0$.

\



\item Prove or disprove: The sequence\footnote{That is, the sequence 1,0,0,0,0,0,0,0,0,1,0,0,0,0,0,0,0,0,0,0,0,0,0,0,0,0,0,0,0,0,0,0,0,0,0,0,0,0,0,0,0,0,0,0,0,0,0,0,0,0,\\
0,0,0,0,0,0,0,0,0,0,0,0,0,0,0,0,0,0,0,0,0,0,0,0,0,0,0,0,0,0,0,0,0,0,0,0,0,0,0,0,0,0,0,0,0,0,0,0,0,1,0,0,0,\dots; the $1$'s get further and further apart.} $\{a_n\}_{n=1}^\infty$ where
\[ a_n = \begin{cases} 1 &\text{if } n = 10^m \text{ for some } m\in \N \\
0 &\text{otherwise}\end{cases}
\]
converges to $0$. 


\


\end{enumerate}
\begin{framed}
\noindent \textbf{Definition:} A sequence $\{a_n\}_{n=1}^\infty$ is \emph{convergent} if there is a real number $L$ such that $\{a_n\}_{n=1}^\infty$ converges to $L$. Otherwise, it is said to be \emph{divergent}.
\end{framed}

\

\begin{enumerate}\setcounter{enumi}{5}
\item In this problem, we will prove that the sequence $\{ (-1)^{n} \}_{n=1}^\infty$ is divergent.

\begin{itemize}
\item Proceed by contradiction and suppose it converges to $L$.
\item Apply the definition of ``converges to $L$'' with $\e = \frac{1}{2}$, so we get some $N$.
\item Take an odd integer $n$ bigger than $N$: what does this say about $L$?
\item  Take an even integer $n$ bigger than $N$: what does this say about $L$?
\item Conclude the proof.
\end{itemize}
\end{enumerate}


\end{document}
