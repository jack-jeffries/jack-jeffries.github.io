\documentclass[12pt]{amsart}


\usepackage{times}
\usepackage[margin=.9in]{geometry}
\usepackage{amsmath,amssymb,multicol,graphicx,framed,enumitem}
\newcommand{\Q}{\mathbb{Q}}
\newcommand{\N}{\mathbb{N}}
\newcommand{\Z}{\mathbb{Z}}
\newcommand{\R}{\mathbb{R}}
\newcommand{\e}{\varepsilon}
\def\d{\delta}
\newcommand{\cts}{continuous\,}

\newcommand{\dabs}[1]{\left| #1 \right|}
\newcommand{\ds}{\displaystyle}
\usepackage[all, knot]{xy}
\usepackage{color,xcolor}
%\usepackage{times}

%\addtolength{\textwidth}{100pt}
%\addtolength{\evensidemargin}{-45pt}
%\addtolength{\oddsidemargin}{-60pt}

\pagestyle{empty}
%\begin{document}\begin{itemize}

%\thispagestyle{empty}




\begin{document}
	
	\thispagestyle{empty}
	
	\section*{Intermediate Value Theorem}
	
\begin{framed} 
 \noindent The definition of continuous on a closed interval $[a,b]$ is actually a bit different: we shouldn't necessarily ask that $f$ be continuous at $a$, since to know that would have to use something about $f$ on input values outside of our interval!
 
 \
 
\noindent\textbf{Definition~22.1:}  Given a function $f(x)$ and real numbers $a < b$,
we say $f$ is {\em continuous on the closed interval $[a,b]$} provided 
\begin{enumerate}
\item for every $r \in (a,b)$, $f$ is  continuous at $r$ in the sense defined already,
\item for every $\e > 0$ there is a $\d > 0$ such that if $a \leq x < a+\d$, then $f(x)$ is defined and ${|f(x)
  - f(a)| < \e}$.
\item for every $\e > 0$ there is a $\d > 0$ such that if $b -\d < x \leq b$, then ${|f(x)
  - f(b)| < \e}$.
\end{enumerate}
\end{framed}

\

\begin{enumerate}
 \item Explain why if $f$ is continuous at $x$ for every $x\in[a,b]$, then $f$ is continuous on the closed interval $[a,b]$. In particular, if $f$ is continuous on any open interval containing $[a,b]$, then $f$ is continuous on $[a,b]$. Conclude that every polynomial is continuous on every closed interval.


\

\item Show that the function $f(x) = \sqrt{1-x^2}$ is continuous on the closed interval $[-1,1]$:
\begin{itemize}
\item For showing condition (1), I recommend using a Theorem about compositions of functions.
\item For conditions (2) and (3), show that $\delta = \min\{\e^2 /\sqrt{2} , 2\}$ works\footnote{Hint: Write $\sqrt{1-x^2} = \sqrt{1-x} \sqrt{1+x}$.}.
\end{itemize}
Is this function continuous on any open interval containing $[-1,1]$?
\end{enumerate}

\

\begin{framed} 
\noindent\textbf{Theorem~22.2. (Intermediate Value Theorem):}  Let $a<b$ and $f(x)$ be a function that is continuous on the closed interval $[a,b]$. If $y$ is any real number between $f(a)$ and $f(b)$, then there is some $c\in [a,b]$ such that $f(c)=y$. More precisely, if $f(a) \leq y \leq f(b)$ or $f(b) \leq y \leq f(a)$, then there is some $c\in [a,b]$ such that $f(c)=y$.
\end{framed}

\


\begin{enumerate}\setcounter{enumi}{2}
\item Draw a picture of this theorem as follows: 
\begin{itemize}
\item Mark some $a$ and $b$ on the $x$-axis. 
\item Graph a function $f$ that is continuous on $[a,b]$.
\item Mark $f(a)$ and $f(b)$ on the $y$-axis.
\item Pick some $y$ in between  $f(a)$ and $f(b)$, and make a horizontal line for this $y$-value.
\item Does it intersect the graph of $f$?
\end{itemize}
Repeat with at least one graph that is increasing, at least one graph that is decreasing, and at least one graph that is neither increasing nor decreasing.

\

\item  Give a counterexample to the statement of the Intermediate Value Theorem without the hypothesis that $f$ is continuous on $[a,b]$.

\

\item Prove or disprove: There is a real number $x\in [0,2]$ such that $x^3 - 3x = 1$.

\

\item Prove or disprove: There\footnote{Draw a graph of this function before you declare victory on this problem.} are at least two real numbers $x\in [0,2]$ such that $x^3 - 3x = -1$.

\

\item True or false: If $f(x)$ is continuous on $[a,b]$, and $y$ is \emph{not} in between $f(a)$ and $f(b)$, then there is no $c\in[a,b]$ such that $f(c) = y$.

\

\item \textbf{Proof of the Intermediate Value Theorem:}
\begin{enumerate}
\item Let's assume that $f(a) \leq f(b)$ to get started. Explain why the cases $y=f(a)$ and $y=f(b)$ are easy. Hence, we assume that $f(a)< y < f(b)$.
\item Let $S=\{ x\in [a,b] \ | \ f(r) < y \ \text{for all} \ a\leq r \leq x \}$. In short, $S$ is the set of $x$-values in the interval where the graph of $f$ hasn't crossed $y$ yet. Explain why $S$ has a supremum, and let $c= \sup(S)$.
\item Show that $c >a$. [ Hint: Apply part (2) of definition of continuous on $[a,b]$ with $\e = y-f(a)$, and show that $a$ is not an upper bound for $S$.]
\item The argument that $c<b$ is similar (so come back to it later if you want). Thus, $c\in (a,b)$, so we know that $f$ is continuous at $c$.
\item Suppose that $f(c)<y$, and obtain a contradiction. [ Hint: Apply continuous at $c$ with $\e= y-f(c)$, and show that $c$ is not an upper bound for $S$.]
\item Suppose that $f(c)>y$, and obtain a contradiction. [ Hint: Apply continuous at $c$ with $\e= f(c)-y$, and find a smaller upper bound for $S$.]
\item This concludes the case when $f(a) \leq f(b)$. If $f(a) \geq f(b)$, what can you say about $g(x) = -f(x)$? Can we apply the case we just did?
\end{enumerate}
\end{enumerate}

\

\noindent We say that a function is \emph{increasing} on an interval $I$ if for any $x,y\in I$, $x<y$ implies $f(x)<f(y)$. We say that a function is \emph{decreasing} on an interval $I$ if for any $x,y\in I$, $x<y$ implies $f(x)>f(y)$. We say that a function is \emph{monotone} on an interval $I$ if it is either increasing on $I$ or decreasing on $I$. We say that a function is \emph{one-to-one} on an interval $I$ if for any $x,y\in I$, $x\neq y$ implies $f(x)\neq f(y)$.

\

\begin{enumerate}\setcounter{enumi}{8}
\item Show that if $f$ is continuous and one-to one on an interval $(a,b)$, then $f$ is monotone.


   \end{enumerate}








\end{document}
