\documentclass[12pt]{amsart}


\usepackage{times}
\usepackage[margin=.8in]{geometry}
\usepackage{amsmath,amssymb,multicol,graphicx,framed,enumitem}
\newcommand{\Q}{\mathbb{Q}}
\newcommand{\N}{\mathbb{N}}
\newcommand{\Z}{\mathbb{Z}}
\newcommand{\R}{\mathbb{R}}
\newcommand{\e}{\varepsilon}
\newcommand{\inv}{^{-1}}
\newcommand{\dabs}[1]{\left| #1 \right|}
\newcommand{\ds}{\displaystyle}
\usepackage[all, knot]{xy}
\usepackage{color,xcolor}
%\usepackage{times}

%\addtolength{\textwidth}{100pt}
%\addtolength{\evensidemargin}{-45pt}
%\addtolength{\oddsidemargin}{-60pt}

\pagestyle{empty}
%\begin{document}\begin{itemize}

%\thispagestyle{empty}




\begin{document}
	
	\thispagestyle{empty}
	
	\section*{Limits}

\begin{enumerate}
\item Use the definition to show that $\lim_{x\to 2}  | 2x-4 | = 0$.
\end{enumerate}

\begin{framed}
\noindent From last time:

\

\noindent \textbf{Theorem 18.4:}  Let $f(x)$ be a function and let $a$ be a real number.
  Let $r > 0$ be a positive real number such that
  $f$ is defined
  at every point of $\{x \in \R \mid 0 < |x-a| < r\}$.
    Let $L$ be any real number. 

Then $\lim_{x \to a} f(x) = L$ if and only if for every sequence 
$\{x_n\}_{n=1}^\infty$ that converges to $a$ and satisfies $0 < |x_n - a| < r$ for all $n$, we have that the sequence $\{f(x_n)\}_{n=1}^\infty$ converges to $L$. 

\

\noindent \textbf{Corollary 18.5:}   Let $f$ be a function and $a$ and $L$ be real numbers. Suppose that the domain of $f$ is all of $\R$ or $\R \smallsetminus \{ a\}$. Then $\lim_{x \to a} f(x) = L$ if and only if for every sequence 
	$\{x_n\}_{n=1}^\infty$ that converges to $a$ such that $x_n\neq a$ for all $n$, we have that the sequence $\{f(x_n)\}_{n=1}^\infty$ converges to~$L$. 
	
	\
	
\noindent	From the homework:

\
	
	\noindent \textbf{Example (HW):} If $m,b,a\in \R$, then $\lim_{x\to a} \, mx+b = ma+b$. In particular, $\lim_{x\to a} \, b = b$ and $\lim_{x\to a} \, x = a$.
\end{framed}

\begin{enumerate} \setcounter{enumi}{1}

\item Use Corollary 18.5 to show that $\ds \lim_{x\to 0} \cos\left(\frac{1}{x}\right)$ does not exist.\\
Suggestion: Let $f(x)= \cos(\frac{1}{x})$ and suppose $\lim_{x\to 0} f(x) = L$. Find sequences $\{x_n\}_{n=1}$  and $\{y_n\}_{n=1}$ such that
\begin{itemize}
\item  $\{x_n\}_{n=1}$ and $\{y_n\}_{n=1}$ both converge to $0$,
\item $f(x_n)=1$ for all $n$, and
\item $f(y_n)=-1$ for all $n$.
\end{itemize}
What does this say about $L$?

\end{enumerate}


\begin{framed}
\noindent \textbf{Theorem 19.1:} Suppose $f$ and $g$ are two functions and that $a$ is a real number, and
	assume  that 
	$$
	\lim_{x \to a} f(x) = L  \ \text{and} \  \lim_{x \to a} g(x) = M
	$$
	for some real numbers $L$ and $M$. Then
	\begin{enumerate}
		\item $\lim_{x \to a} (f(x) + g(x)) = L  + M$.
		\item For any real number $c$, $\lim_{x \to a} (c \cdot f(x)) = c \cdot L$.
		\item $\lim_{x \to a} (f(x) \cdot g(x)) = L \cdot M$.
		\item If, in addition, we have that $M \ne 0$,
		then $\lim_{x \to a} (f(x)/g(x)) = L/M$.
	\end{enumerate}
\end{framed}

\begin{enumerate} \setcounter{enumi}{2}

\item Use Theorem~19.1 plus Example~HW to compute $\ds \lim_{x\to 2} \frac{3x^2 - x +2}{x+3}$.

\

\item Use Theorem 18.4 to deduce Theorem~19.1 part (1) from our Theorem~10.2 on algebra and sequences.

\end{enumerate}

\end{document}
