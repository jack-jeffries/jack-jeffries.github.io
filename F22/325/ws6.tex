\documentclass[12pt]{amsart}


\usepackage{times}
\usepackage[margin=1in]{geometry}
\usepackage{amsmath,amssymb,multicol,graphicx,framed,enumitem}
\newcommand{\Q}{\mathbb{Q}}
\newcommand{\N}{\mathbb{N}}
\newcommand{\Z}{\mathbb{Z}}
\newcommand{\R}{\mathbb{R}}
\newcommand{\e}{\varepsilon}
\newcommand{\inv}{^{-1}}
\newcommand{\dabs}[1]{\left| #1 \right|}
\newcommand{\ds}{\displaystyle}

\DeclareMathOperator{\res}{res}

%\usepackage{times}

%\addtolength{\textwidth}{100pt}
%\addtolength{\evensidemargin}{-45pt}
%\addtolength{\oddsidemargin}{-60pt}

\pagestyle{empty}
%\begin{document}\begin{itemize}

%\thispagestyle{empty}




\begin{document}
	
	\thispagestyle{empty}
	
	\section*{Theorems about convergence warmup}
	
\noindent Which of the following implications about sequences hold in general? Either mention a relevant theorem or give a counterexample.
	
\begin{multicols}{2}
\begin{enumerate}[label=(\alph*)]
\item monotone \ $\Longrightarrow$ \ convergent
\item increasing \ + \ convergent  \ $\Longrightarrow$ \ bounded
\item bounded \ + \ decreasing  \ $\Longrightarrow$ \ convergent
\item convergent  \ $\Longrightarrow$ \ monotone
\item convergent \ $\Longrightarrow$ \ bounded
\item bounded \ $\Longrightarrow$ \ convergent
\end{enumerate}
\end{multicols}

	
	\section*{Divergence to $\pm \infty$}
	
\noindent It is sometimes useful to distinguish between sequences like $\{(-1)^n\}_{n=1}^\infty$
that diverge because they ``oscillate'', and sequences like $
\{n\}_{n=1}^\infty$
that diverge because they ``head toward infinity''.

\


\begin{enumerate}[label=(\Roman*)]

\item In intuitive language, a sequence converges to $L$ if no matter how close we want or sequence to be to $L$, all values past some point are at least that close. Intuitively, a sequence \emph{diverges to $+\infty$} if no matter how large we want our sequence to be, all values past some point are at least that large. Write a precise definition for a sequence to diverge to $+\infty$.

\

\item Write a precise definition for a sequence to diverge to $-\infty$.

\end{enumerate}

\

\noindent \hrulefill

CHECK ANSWERS TO I \& II BEFORE CONTINUING

\noindent \hrulefill

\

\begin{enumerate}

%\item Carefully write the logical negation of ``$\{ a_n\}_{n=1}^\infty$ diverges to $+\infty$'' in simplified form.

%\

\item Use the definition to prove that the sequence $\{ \sqrt{n} \}_{n=1}^\infty$ diverges to $+\infty$.

\

%\item Use the definition to prove that the sequence $\{ (-1)^n\}_{n=1}^\infty$ does not diverge to $+ \infty$.

%\

\item\label{nba} Prove that if a sequence $\{ a_n\}_{n=1}^\infty$ diverges to $+\infty$ then it is not bounded above. 

\

\item Use (\ref{nba}) to show that if a sequence diverges to $+\infty$ then it diverges.

\

\item Disprove the following: If a sequence is not bounded above, then it diverges to $+\infty$.

\

\item Disprove the following: If a sequence diverges to $+\infty$ then it is increasing.


\end{enumerate}


\end{document}
