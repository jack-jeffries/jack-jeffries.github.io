\documentclass[12pt]{amsart}


\usepackage{times}
\usepackage[margin=0.8in]{geometry}
\usepackage{amsmath,amssymb,multicol,graphicx,framed}
\newcommand{\Q}{\mathbb{Q}}
\newcommand{\N}{\mathbb{N}}
\newcommand{\Z}{\mathbb{Z}}
\newcommand{\R}{\mathbb{R}}
\newcommand{\inv}{^{-1}}
\newcommand{\dabs}[1]{\left| #1 \right|}

\DeclareMathOperator{\res}{res}

%\usepackage{times}

%\addtolength{\textwidth}{100pt}
%\addtolength{\evensidemargin}{-45pt}
%\addtolength{\oddsidemargin}{-60pt}

\pagestyle{empty}
%\begin{document}\begin{itemize}

%\thispagestyle{empty}




\begin{document}
	
	\thispagestyle{empty}
	
	\section*{Upper bounds and the completeness axiom}
	
	

\begin{framed}
Let $S$ be a set of real numbers. 
\begin{itemize}
\item A number $b$ is an \emph{upper bound} for $S$ provided for all $x\in S$ we have $b\geq x$. 
\item The set $S$ is \emph{bounded above} provided there exists at least one upper bound for $S$.
\item A number $m$ is the \emph{maximum} of $S$ provided
\begin{enumerate}
\item $m\in S$, and
\item $m$ is an upper bound of $S$.
\end{enumerate}
\item A number $\ell$ is a \emph{supremum} of $S$ provided
\begin{enumerate}
\item $\ell$ is an upper bound of $S$, and
\item for any upper bound $b$ for $S$, we have $\ell \leq b$.
\end{enumerate}
\end{itemize}
\end{framed}

\

\begin{enumerate}
\item Write, in simplified form, the negation of the statement ``$b$ is an upper bound for $S$''.

\

\item Write, in simplified form, the negation of the statement ``$S$ is bounded above''.

\

\item Let $S$ be a set of real numbers and suppose that $\ell=\sup(S)$. 
\begin{enumerate}
\item If $x > \ell$, what is the most concrete thing you can say about $x$ and $S$?
\item If $x < \ell$, what is the most concrete thing\footnote{Hint: Use one of the previous problems.} you can say about $x$ and $S$?
\end{enumerate}

\

\item Let $S$ be a set of real numbers, and set\footnote{For example, if $S= \{-1,1,2\}$, then $T=\{-2,2,4\}$.} $T=\{ 2s \ | \ s\in S\}$. Prove\footnote{First, before all else, this is an if then statement: start by assuming the ``if'' part. We now need to show the ``then'' part, which is about the existence of an upper bound. Use your assumption about $S$ to find an upper bound for $T$ (and prove that it is indeed an upper bound for $T$).} that if $S$ is bounded above, then $T$ is bounded above.


\end{enumerate}



\end{document}
