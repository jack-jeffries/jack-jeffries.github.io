\documentclass[12pt]{amsart}


\usepackage{times}
\usepackage[margin=0.8in]{geometry}
\usepackage{amsmath,amssymb,multicol,graphicx,framed}
\newcommand{\Q}{\mathbb{Q}}
\newcommand{\N}{\mathbb{N}}
\newcommand{\Z}{\mathbb{Z}}
\newcommand{\R}{\mathbb{R}}
\newcommand{\inv}{^{-1}}
\newcommand{\dabs}[1]{\left| #1 \right|}

\DeclareMathOperator{\res}{res}

%\usepackage{times}

%\addtolength{\textwidth}{100pt}
%\addtolength{\evensidemargin}{-45pt}
%\addtolength{\oddsidemargin}{-60pt}

\pagestyle{empty}
%\begin{document}\begin{itemize}

%\thispagestyle{empty}




\begin{document}
	
	\thispagestyle{empty}
	
	\section*{Warming up with proof techniques and real numbers}
	
	
	
	\
\subsection*{Making sense of if then statements and quantifier statements}

\

\begin{framed}
\begin{itemize}
\item The \emph{converse} of the statement  ``If $P$ then $Q$'' is the statement  ``If $Q$ then $P$''.
\item The \emph{contrapositive} of the statement  ``If $P$ then $Q$'' is the statement  ``If not $Q$ then not $P$''.
\item Any if then statement is equivalent to its contrapositive, but not necessarily to its converse!
\end{itemize}
\end{framed}

\

\begin{enumerate}
\item For each of the following statements, write its contrapositive and its converse. Is the original/contrapositive/converse true or false for real numbers $a,b$? Explain why (but don't prove). 
\begin{enumerate}
\item If $a$ is irrational, then $1/a$ is irrational.
\item If $a$ and $b$ are irrational, then $ab$ is irrational.
\item If $a>3$, then $a^2>9$.
\end{enumerate}
\end{enumerate}

\

\begin{framed}
\begin{itemize}
\item The symbol for ``for all'' is $\forall$ and the symbol for there exists is $\exists$.
\item The negation of ``For all $x\in S$, $P$'' is ``There exists $x\in S$ such that $\mathrm{not} \, P$''.
\item The negation of ``There exists $x\in S$ such that $P$'' is ``For all $x\in S$, $\mathrm{not} \, P$''.
\end{itemize}
\end{framed}

\


\begin{enumerate}\setcounter{enumi}{1}
\item Rewrite each statement with symbols in place of quantifiers, and write its negation. Is the original statement true or false? Explain why (but don't prove them).
\begin{enumerate}
\item There exists $x\in \Q$ such that $x^2 = 2$.
\item For all $x\in \R$,  $x^2 >0$.
\item For all $x\in \R$ such that\footnote{In a statement of the form ``For all $x\in S$ such that $Q$, $P$'', the ``such that $Q$'' part is part of the hypothesis: it is restricting the set $S$ that we are ``alling''' over.} $x\neq 0$,  $x^2 >0$.
\item For all $x\in \R$, there exists $y\in \R$ such that $x<y$.
\item There exists $x\in \R$ such that for all $y\in \R$, $x<y$.
%\item There exists $x\in (0,\infty)$ such that\footnote{$(0,\infty)=\{z\in \R \ | \ 0 < z\}$, just like in calculus.} for all $y\in (0,\infty)$, $xy=1$.
%\item For all $x\in (0,\infty)$, there exists $y\in (0,\infty)$ such that $xy=1$.
\end{enumerate}
\end{enumerate}

\newpage

\subsection*{Proving if then statements and quantifier statements}


\

\begin{framed}
\begin{itemize}
\item The general outline of a direct proof of ``If $P$ then $Q$'' goes
\begin{enumerate}
\item Assume $P$.
\item Do some stuff.
\item Conclude $Q$.
\end{enumerate}
\item Often it is easier to prove the contrapositive of an if then statement than the original, especially when the negation of the hypothesis or conclusion is something negative.
\item The general outline of a proof of ``For all $x\in S$, $P$'' goes
\begin{enumerate}
\item Let $x\in S$ be arbitrary.
\item Do some stuff.
\item Conclude that $P$ holds for $x$.
\end{enumerate}
\item To prove a there exists statement, you just need to give an example. To prove ``There exists $x\in S$ such that $P$'' directly:
\begin{enumerate}
\item Consider $x=$[some specific element of $S$].
\item Do some stuff.
\item Conclude that $P$ holds for $x$.
\end{enumerate}


\end{itemize}

\end{framed}

\


\begin{enumerate}\setcounter{enumi}{2}
\item Let $x$ and $y$ be real numbers. Use the axioms of $\R$ to prove\footnote{Hint: You may want to add something to both sides.} that $x \geq y$ if and only if $-y\geq -x$.

\

\item Let $x$ be a real number. Show that if $x^2$ is irrational, then $x$ is irrational.

\

\item\label{xy=1} Let $x$ be a real number. Use the axioms of $\R$ and facts we have proven in class to show that if there exists a real number $y$ such that $xy=1$, then $x\neq 0$.

\

\item Prove that\footnote{Hint: Use (\ref{xy=1}).} for all $x\in \R$ such that $x\neq 0$, we have $x^2\neq 0$.

\

\item Prove that there exists some $x\in \R$ such that for every $y\in \R$, $xy=x$.

\

\item Prove\footnote{You can ``work out of order here'' and \emph{use} (\ref{1>0}) now.} that (2d) is true and (2e) is false. 

\

\item Let $S\subseteq \R$ be a set of real numbers. Apply your results above to prove that if for every $x\in S$, $x^2$ is irrational, then for every $y\in S$, $y$ is irrational.

\

\item\label{1>0} Prove that $1>0$.


\end{enumerate}


\end{document}
