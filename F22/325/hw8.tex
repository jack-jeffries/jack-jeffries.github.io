
\documentclass{amsart}
\usepackage{amstext,amsfonts,amssymb,amscd,amsbsy,amsmath,framed}
\usepackage{geometry}[margin=1in]
\pagestyle{empty}
%\parindent = 0in


\def\sol#1{{\bf Solution: } #1}
%\def\sol#1{}
\def\bl{\vskip .1in}
\def\star{${}^*$}
\def\cS{\mathcal S}
\def\cT{\mathcal T}
\def\cB{\mathcal B}
\def\e{\varepsilon}
\def\R{\mathbb R}
\def\N{\mathbb N}
\def\Q{\mathbb Q}
\def\Z{\mathbb Z}
\def\ds{\displaystyle}

\def\cC{\mathcal C}

\def\e{\epsilon}
\def\d{\delta}

\begin{document}



\begin{center}
{\large\bfseries
Math 325-002 --- Problem Set \#8 \\
Due: Thursday, November 3 by 7 pm, on Canvas}
\end{center}





{\bf Instructions:} You are encouraged to work together on these
problems, but each student should hand in their own final draft,
written in a way that indicates their individual understanding of
the solutions. Never submit something for grading
that you do not completely understand. 

If you do work with others, I ask that you write something along the
top like ``I collaborated with Steven Smale on problems 1 and 3''.
If you use a reference, indicate so clearly in your solutions. 
In short, be intellectually
honest at all times.

Please write neatly, using complete sentences and correct
punctuation. Label the problems clearly. 


\begin{framed}To prove that $\ds \lim_{x\to a} f(x) = L$ by the definition we:
\begin{itemize}
\item Let $\e>0$.
\item Take $\d =$ [some number computed from scratch work that makes $f(x)$ defined and $|f(x)-L|<\e$ true for all $x$ satisfying $0<|x-a|<\d$.]
\item Let $x$ be a real number such that $0< |x-a|< \d$.
\item \, [Argument that $f(x)$ is defined and $|f(x)-L|<\e$ true for this $x$.]
\item Thus $\lim_{x\to a} f(x) = L$.
\end{itemize}
\end{framed}


\begin{enumerate}

\item Prove that $\ds \lim_{x\to 1} \frac{2x^2 - 2}{x-1} = 4$.

\

\item Let $a,m,$ and $b$ be real numbers. Prove\footnote{Hint: You might want to separate the cases withe $m=0$ and $m\neq 0$.} that $\ds \lim_{x\to a} \, mx+b = ma+b$.

\

\item Prove\footnote{Hint: To obtain a contradiction, suppose that $L\neq M$. Take $\e= \frac{|L-M|}{3}$ in the definition. Compare with the argument that a sequence converges to at most one value.} that limits, if they exist, are unique. That is, show that if $\ds \lim_{x\to a} f(x) = L$ and $\ds \lim_{x\to a} f(x) = M$ are both true, then $L=M$.

\end{enumerate}

\end{document}

























\end{document}