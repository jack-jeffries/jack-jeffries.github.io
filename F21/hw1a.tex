\documentclass{amsart}[12pt]
\usepackage{graphicx}
\usepackage{comment}
\usepackage{amscd,mathabx}
\usepackage{amssymb,setspace}
\usepackage{latexsym,amsfonts,amssymb,amsthm,amsmath,amscd,stmaryrd,mathrsfs}
\usepackage[all, knot]{xy}
\usepackage[top=1in, bottom=1in, left=1in, right=1in]{geometry}
\xyoption{all}
\xyoption{arc}
\usepackage{hyperref}


%\usepackage[notcite,notref]{showkeys}
 
%\CompileMatricesx
\newcommand{\edit}[1]{\marginpar{\footnotesize{#1}}}
%\newcommand{\edit}[1]{}
\newcommand{\rperf}[2]{\operatorname{RPerf}(#1 \into #2)}



\newcommand{\vectwo}[2]{\begin{bmatrix} #1 \\ #2 \end{bmatrix}}

\newcommand{\vecfour}[4]{\begin{bmatrix} #1 \\ #2 \\ #3 \\ #4 \end{bmatrix}}

\newcommand{\Cat}[1]{\left<\left< \text{#1} \right>\right>}


\def\htpy{\simeq_{\mathrm{htpc}}}
\def\tor{\text{ or }}
\def\fg{finitely generated~}

\def\Ass{\operatorname{Ass}}
\def\ann{\operatorname{ann}}
\def\sign{\operatorname{sign}}

\def\ob{{\mathfrak{ob}} }
\def\BiAdd{\operatorname{BiAdd}}
\def\BiLin{\operatorname{BiLin}}

\def\Syl{\operatorname{Syl}}
\def\span{\operatorname{span}}

\def\sdp{\rtimes}
\def\cL{\mathcal L}
\def\cR{\mathcal R}



\def\ay{??}
\def\Aut{\operatorname{Aut}}
\def\End{\operatorname{End}}
\def\Mat{\operatorname{Mat}}


\def\a{\alpha}



\def\etale{\'etale~}
\def\tW{\tilde{W}}
\def\tH{\tilde{H}}
\def\tC{\tilde{C}}
\def\tS{\tilde{S}}
\def\tX{\tilde{X}}
\def\tZ{\tilde{Z}}
\def\HBM{H^{\text{BM}}}
\def\tHBM{\tilde{H}^{\text{BM}}}
\def\Hc{H_{\text{c}}}
\def\Hs{H_{\text{sing}}}
\def\cHs{{\mathcal H}_{\text{sing}}}
\def\sing{{\text{sing}}}
\def\Hms{H^{\text{sing}}}
\def\Hm{\Hms}
\def\tHms{\tilde{H}^{\text{sing}}}
\def\Grass{\operatorname{Grass}}
\def\image{\operatorname{im}}
\def\im{\image}
\def\ker{\operatorname{ker}}
\def\cone{\operatorname{cone}}
\newcommand{\Hom}{\mathrm{Hom}}


\def\ku{ku}
\def\bbu{\bf bu}
\def\KR{K{\mathbb R}}

\def\CW{\underline{CW}}
\def\cP{\mathcal P}
\def\cE{\mathcal E}
\def\cL{\mathcal L}
\def\cJ{\mathcal J}
\def\cJmor{\cJ^\mor}
\def\ctJ{\tilde{\mathcal J}}
\def\tPhi{\tilde{\Phi}}
\def\cA{\mathcal A}
\def\cB{\mathcal B}
\def\cC{\mathcal C}
\def\cZ{\mathcal Z}
\def\cD{\mathcal D}
\def\cF{\mathcal F}
\def\cG{\mathcal G}
\def\cO{\mathcal O}
\def\cI{\mathcal I}
\def\cS{\mathcal S}
\def\cT{\mathcal T}
\def\cM{\mathcal M}
\def\cN{\mathcal N}
\def\cMpc{{\mathcal M}_{pc}}
\def\cMpctf{{\mathcal M}_{pctf}}
\def\L{\Lambda}

\def\sA{\mathscr A}
\def\sB{\mathscr B}
\def\sC{\mathscr C}
\def\sZ{\mathscr  Z}
\def\sD{\mathscr  D}
\def\sF{\mathscr  F}
\def\sG{\mathscr G}
\def\sO{\mathscr  O}
\def\sI{\mathscr I}
\def\sS{\mathscr S}
\def\sT{\mathscr  T}
\def\sM{\mathscr M}
\def\sN{\mathscr N}



\def\Ext{\operatorname{Ext}}
 \def\ext{\operatorname{ext}}



\def\ov#1{{\overline{#1}}}

\def\vecthree#1#2#3{\begin{bmatrix} #1 \\ #2 \\ #3 \end{bmatrix}}

\def\tOmega{\tilde{\Omega}}
\def\tDelta{\tilde{\Delta}}
\def\tSigma{\tilde{\Sigma}}
\def\tsigma{\tilde{\sigma}}


\def\d{\delta}
\def\td{\tilde{\delta}}

\def\e{\epsilon}
\def\nsg{\unlhd}
\def\pnsg{\lhd}

\newcommand{\tensor}{\otimes}
\newcommand{\homotopic}{\simeq}
\newcommand{\homeq}{\cong}
\newcommand{\iso}{\approx}

\DeclareMathOperator{\ho}{Ho}
\DeclareMathOperator*{\colim}{colim}


\newcommand{\Q}{\mathbb{Q}}
\renewcommand{\H}{\mathbb{H}}

\newcommand{\bP}{\mathbb{P}}
\newcommand{\bM}{\mathbb{M}}
\newcommand{\A}{\mathbb{A}}
\newcommand{\bH}{{\mathbb{H}}}
\newcommand{\G}{\mathbb{G}}
\newcommand{\bR}{{\mathbb{R}}}
\newcommand{\bL}{{\mathbb{L}}}
\newcommand{\R}{{\mathbb{R}}}
\newcommand{\F}{\mathbb{F}}
\newcommand{\E}{\mathbb{E}}
\newcommand{\bF}{\mathbb{F}}
\newcommand{\bE}{\mathbb{E}}
\newcommand{\bK}{\mathbb{K}}


\newcommand{\bD}{\mathbb{D}}
\newcommand{\bS}{\mathbb{S}}

\newcommand{\bN}{\mathbb{N}}


\newcommand{\bG}{\mathbb{G}}

\newcommand{\C}{\mathbb{C}}
\newcommand{\Z}{\mathbb{Z}}
\newcommand{\N}{\mathbb{N}}

\newcommand{\M}{\mathcal{M}}
\newcommand{\W}{\mathcal{W}}



\newcommand{\itilde}{\tilde{\imath}}
\newcommand{\jtilde}{\tilde{\jmath}}
\newcommand{\ihat}{\hat{\imath}}
\newcommand{\jhat}{\hat{\jmath}}

\newcommand{\fc}{{\mathfrak c}}
\newcommand{\fp}{{\mathfrak p}}
\newcommand{\fm}{{\mathfrak m}}
\newcommand{\fn}{{\mathfrak n}}
\newcommand{\fq}{{\mathfrak q}}

\newcommand{\op}{\mathrm{op}}
\newcommand{\dual}{\vee}

\newcommand{\DEF}[1]{\emph{#1}\index{#1}}
\newcommand{\Def}[1]{#1 \index{#1}}


% The following causes equations to be numbered within sections
\numberwithin{equation}{section}


\theoremstyle{plain} %% This is the default, anyway
\newtheorem{thm}[equation]{Theorem}
\newtheorem{thmdef}[equation]{TheoremDefinition}
\newtheorem{introthm}{Theorem}
\newtheorem{introcor}[introthm]{Corollary}
\newtheorem*{introthm*}{Theorem}
\newtheorem{question}{Question}
\newtheorem{cor}[equation]{Corollary}
\newtheorem{por}[equation]{Porism}
\newtheorem{lem}[equation]{Lemma}
\newtheorem{lemminition}[equation]{Lemminition}
\newtheorem{prop}[equation]{Proposition}
\newtheorem{porism}[equation]{Porism}

\newtheorem{conj}[equation]{Conjecture}
\newtheorem{quest}[equation]{Question}

\theoremstyle{definition}
\newtheorem{defn}[equation]{Definition}
\newtheorem{chunk}[equation]{}
\newtheorem{ex}[equation]{Example}

\newtheorem{exer}[equation]{Optional Exercise}

\theoremstyle{remark}
\newtheorem{rem}[equation]{Remark}

\newtheorem{notation}[equation]{Notation}
\newtheorem{terminology}[equation]{Terminology}



\renewcommand{\sec}[1]{\section{#1}}
\newcommand{\ssec}[1]{\subsection{#1}}
\newcommand{\sssec}[1]{\subsubsection{#1}}

\newcommand{\br}[1]{\lbrace \, #1 \, \rbrace}
\newcommand{\li}{ < \infty}
\newcommand{\quis}{\simeq}
\newcommand{\xra}[1]{\xrightarrow{#1}}
\newcommand{\xla}[1]{\xleftarrow{#1}}
\newcommand{\xlra}[1]{\overset{#1}{\longleftrightarrow}}

\newcommand{\xroa}[1]{\overset{#1}{\twoheadrightarrow}}
\newcommand{\xria}[1]{\overset{#1}{\hookrightarrow}}
\newcommand{\ps}[1]{\mathbb{P}_{#1}^{\text{c}-1}}




\def\and{{ \text{ and } }}
\def\oor{{ \text{ or } }}

\def\Perm{\operatorname{Perm}}
\newcommand{\Ss}{\mathbb{S}}

\def\Op{\operatorname{Op}}
\def\res{\operatorname{res}}
\def\ind{\operatorname{ind}}

\def\sign{{\mathrm{sign}}}
\def\naive{{\mathrm{naive}}}
\def\l{\lambda}


\def\ov#1{\overline{#1}}
\def\cV{{\mathcal V}}
%%%-------------------------------------------------------------------
%%%-------------------------------------------------------------------

\newcommand{\chara}{\operatorname{char}}
\newcommand{\Kos}{\operatorname{Kos}}
\newcommand{\opp}{\operatorname{opp}}
\newcommand{\perf}{\operatorname{perf}}

\newcommand{\Fun}{\operatorname{Fun}}
\newcommand{\GL}{\operatorname{GL}}
\newcommand{\SL}{\operatorname{SL}}
\def\o{\omega}
\def\oo{\overline{\omega}}

\def\cont{\operatorname{cont}}
\def\te{\tilde{e}}
\def\gcd{\operatorname{gcd}}

\def\stab{\operatorname{stab}}

\def\va{\underline{a}}

\def\ua{\underline{a}}
\def\ub{\underline{b}}


\newcommand{\Ob}{\mathrm{Ob}}
\newcommand{\Set}{\mathbf{Set}}
\newcommand{\Grp}{\mathbf{Grp}}
\newcommand{\Ab}{\mathbf{Ab}}
\newcommand{\Sgrp}{\mathbf{Sgrp}}
\newcommand{\Ring}{\mathbf{Ring}}
\newcommand{\Fld}{\mathbf{Fld}}
\newcommand{\cRing}{\mathbf{cRing}}
\newcommand{\Mod}[1]{#1-\mathbf{Mod}}
\newcommand{\vs}[1]{#1-\mathbf{vect}}
\newcommand{\Vs}[1]{#1-\mathbf{Vect}}
\newcommand{\vsp}[1]{#1-\mathbf{vect}^+}
\newcommand{\Top}{\mathbf{Top}}
\newcommand{\Setp}{\mathbf{Set}_*}
\newcommand{\Alg}[1]{#1-\mathbf{Alg}}
\newcommand{\cAlg}[1]{#1-\mathbf{cAlg}}
\newcommand{\PO}{\mathbf{PO}}
\newcommand{\Cont}{\mathrm{Cont}}
\newcommand{\MaT}[1]{\mathbf{Mat}_{#1}}

%%%-------------------------------------------------------------------
%%%-------------------------------------------------------------------
%%%-------------------------------------------------------------------
%%%-------------------------------------------------------------------
%%%-------------------------------------------------------------------

\makeindex
\title{Assignment \#1}


\begin{document}
\onehalfspacing

\maketitle




\begin{enumerate}
\item 
\begin{enumerate}
\item Show that if $R$ is a ring and $\alpha:M\to N$ is a morphism in $\Mod{R}$, then $\alpha$ is monic if and only if it is injective, and $\alpha$ is epic if and only if it is surjective.%\footnote{Hint: You may want to consider $\mathrm{ker}(\alpha)$ and $N/\mathrm{im}(\alpha)$.}
\item Show that the map $\Z \xra{\cdot 2} \Z$ in $\Mod{\Z}$ has no left inverse (even though it is injective) and that the quotient map $\Z \twoheadrightarrow \Z/2\Z$ in $\Mod{\Z}$ has no right inverse (even though it is surjective).
\end{enumerate}

\


\item
\begin{enumerate}
\item An abelian group $M$ is \emph{divisible} if for every $m\in M$ and nonzero $n\in \Z$, there is some $m'\in M$ such that $m=n m'$. Let $\mathbf{DAb}$ be the full subcategory of $\Ab$ consisting of all divisible abelian groups. Show that the quotient map $\Q \to \Q/\Z$ is monic in $\mathbf{DAb}$ (even though it isn't injective).
\item Show that the inclusion map $\Z \hookrightarrow \Q$ is epic in $\Ring$ (even though it isn't surjective).
\end{enumerate}

\


%\item Recall that a set of elements $S$ in a left module $M$ forms a \emph{free basis} if every element $m\in M$ can be written in a unique way as a linear combination of elements of $S$ and a left module is \emph{free} if it has a free basis. Show that a left $R$-module is free if and only if it is isomorphic to a coproduct of copies of the ring $R$ as a left $R$-module.

%\

\item \begin{enumerate}
\item Find a pair of objects in $\Fld$ with no product.
\item Find a pair of objects in $\Fld$ with no coproduct.
\item[(c*)] If $K$ and $L$ are fields of characteristic zero, do $K$ and $L$ admit a product/coproduct in $\Fld$?
\end{enumerate}


\

\item Let $N$ be a left $R$-module, and $\{M_\lambda\}_{\lambda\in \Lambda}$ be a family of submodules of $N$.
We say that $N$ is the \emph{internal direct sum} of $\{M_\lambda\}$ if the canonical map $\bigoplus_{\lambda\in\Lambda} M_{\lambda} \to N$ is an isomorphism. Show that $N$ is the internal direct sum of $\{M_\lambda\}$ if and only if
\begin{itemize}
\item $N$ is generated by $\bigcup_{\lambda\in \Lambda} M_\lambda$, and
\item for every finite subset $\lambda_0,\lambda_1,\dots,\lambda_t$ of (at least two distinct) elements of $\Lambda$,
\[ M_{\lambda_0} \cap (M_{\lambda_1} + \cdots + M_{\lambda_t}) = 0.\]
\end{itemize}
\end{enumerate}


\noindent A \emph{covariant functor} $F$ between two categories $\sC$ and $\sD$ is a rule that assigns
to each object $A$ of $\sC$ an object $F(A)$ of $\sD$, and to each morphism $A\xra{\alpha}B$ of $\sC$ a morphism $F(A) \xra{F(\alpha)} F(B)$ of $\sD$ such that for every object $A$ of $\sC$, $F(1_A) = 1_{F(A)}$, and $F(\alpha\circ \beta) = F(\alpha)\circ F(\beta)$.

\


\begin{enumerate}\setcounter{enumi}{4}
\item 
Suppose that $\sC$ and $\sD$ are subcategories of $\Set$ (e.g., $\Set,\Sgrp,\Grp,\Ab,\Ring,\Mod{R},\Top$) and $F:\sC\to \sD$ is a covariant functor.
\begin{enumerate}
\item Show that if $\alpha$ has a left inverse, then $F(\alpha)$ is injective (as a function).
\item Show that if $\alpha$ has a right inverse, then $F(\alpha)$ is surjective (as a function).
\item Use part (a) to show\footnote{Hint: You might consider symmetric groups.} that there is no covariant functor $F:\Grp\to \Grp$ that, on objects, maps a group to its center.
\end{enumerate}

\
\item
\begin{enumerate}
\item Show\footnote{Hint: You may want to use the fact that the collection of finite sequences with values in a countable set is a countable set.} that in $\Mod{\Z}$, the objects $\coprod_{n\in \N} \Z$ and $\prod_{n\in \N} \Z$ are not isomorphic.
\item[(b*)] Show\footnote{Hint: Suppose so. Show that there is a countable free submodule $T$ such that $\coprod_{n\in \N} \Z \subseteq T \subseteq \prod_{n\in \N} \Z$ and that the quotient $ \big(\prod_{n\in \N} \Z\big) / T$ is free. Then find a nonzero element in $ \big(\prod_{n\in \N} \Z\big)/T$ that is divisible by infinitely many integers.} that $\prod_{n\in \N} \Z$ is not a free module.
\end{enumerate}
\end{enumerate}


%\item \begin{enumerate}
%\item Compute the direct limit of the following directed system in $\Mod{\Z}$:
%\[ \Z/100\Z \xra{\cdot 2} \Z/100\Z \xra{\cdot 2} \Z/100\Z \xra{\cdot 2} \Z/100\Z \xra{\cdot 2} \Z/100\Z \rightarrow \cdots.\]
%\item Compute the direct limit of the following directed system in $\Mod{\Z}$:
%\[ \Z \xra{\cdot 2} \Z \xra{\cdot 2} \Z \xra{\cdot 2} \Z \xra{\cdot 2} \Z \rightarrow \cdots.\]
%\item[(c*)] Suppose that $\Lambda$ is a directed poset with a maximal element $\omega$, and suppose that $\{X_\lambda\}$ is a directed system on $\Lambda$ in a category $\sC$. Prove that $\varinjlim_{\lambda\in \Lambda} X_\lambda = X_\omega$.
%\item[(d*)] Suppose that $\Lambda$ is a directed poset, and that $\Gamma$ is a subposet of $\Gamma$ with the property that for every $\lambda\in \Lambda$, there is some $\gamma\in\Gamma$ with $\gamma\geq \lambda$. Show that if $\{X_\lambda\}$ is a directed system on $\Lambda$ in a category $\sC$, then there is an isomorphism $ \varinjlim_{\lambda\in \Lambda} X_\lambda \cong \varinjlim_{\gamma\in \Gamma} X_\gamma$.
%\end{enumerate}

%\
%
%
%\item[(6*)] Let $P$ be the product of a countable number of copies of $\Z$: $P=\prod_{n\in \N} \Z$. In this problem, we will show that $P$ is \emph{not} a free $\Z$-module (in contrast to the fact that an arbitrary coproduct of $R$'s is a free $R$-module). To obtain a contradiction, suppose that $S\subset P$ is a free basis.
%\begin{enumerate}
%\item Explain why $S$ is uncountable, and the submodule $F=\bigoplus_{n\in \N} \Z \subseteq P$ is countable.
%\item Show that there is a module $T$ such that
%\begin{itemize}
%\item $F\subseteq T \subseteq P$
%\item $T$ is free with countable basis
%\item $P/T$ is free
%\end{itemize}
%\item Show that there is a sequence $(a_n)\in P\smallsetminus T$ such that for every $n$, $a_{n+1}=b_n a_n$ for some $b_n\in \Z_{>1}$.
%\item Conclude the proof.
%\end{enumerate}
%\end{enumerate}


\end{document}







  
 


