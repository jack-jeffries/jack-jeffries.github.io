\documentclass{amsart}
\usepackage{amstext,amsfonts,amssymb,amscd,amsbsy,amsmath}
\usepackage{geometry}[margin=.7in]
\pagestyle{empty}
%\parindent = 0in


\def\sol#1{{\bf Solution: } #1}
%\def\sol#1{}
\def\bl{\vskip .1in}
\def\star{${}^*$}
\def\cS{\mathcal S}
\def\cT{\mathcal T}
\def\cB{\mathcal B}

\def\R{\mathbb R}
\def\N{\mathbb N}
\def\Q{\mathbb Q}
\def\Z{\mathbb Z}

\def\cC{\mathcal C}

\def\e{\epsilon}
\def\d{\delta}

\begin{document}



\begin{center}
{\large\bfseries
Math 325-002 --- Problem Set \#7\\
Due: Friday, October 22 by 5 pm}
\end{center}





{\bf Instructions:} You are encouraged to work together on these
problems, but each student should hand in their own final draft,
written in a way that indicates their individual understanding of
the solutions. Never submit something for grading
that you do not completely understand. 

Please write neatly, using complete sentences and correct
punctuation. Label the problems clearly. 






\begin{enumerate}


	\item Define a sequence $\{a_n\}_{n=1}^\infty$ recursively by $a_1 = 2$ and $a_n = \frac{a_{n-1}}{2} + \frac{1}{a_{n-1}}$ for all $n \geq 2$. 
	
	\begin{enumerate}
		
		\item Prove $a_n > 0$ for all $n \in \N$ by induction on $n$. 
		
		\item Prove\footnote{Hint: Write $a_n-2$ in terms of $a_{n-1}$, and factor the expression.} $a_n^2 \geq 2$ for all $n \in \N$. 
		
		\item Prove\footnote{Hint: Consider $a_n - a_{n-1}$ and use (b).} the sequence is decreasing.
		
		\item Since the sequence is decreasing and bounded below, it
		necessarily converges. 
		Determine what the sequence converges to.\footnote{Hint: Use that
		$$
		\lim_{n \to \infty} a_n = \frac{\lim_{n \to \infty} a_{n-1}}{2} +
		\frac{1}{\lim_{n \to \infty} a_{n-1}}
		$$
		so that if we set $L = \lim_{n \to \infty} a_n$ then we have $L =
		\frac{L}2 + \frac{1}{L}$.}
		
		\
		
			\end{enumerate}
		
			\item For each of the following, give an explicit example as indicated. No proofs are necessary.
	
	\begin{enumerate}
		
		\item A sequence that has a subsequence that converge to $1$, another subsequence that converges to $2$, and a third subsequence that converges to $3$. 
		
		\item A sequence that has one subsequence that is monotone and converges to $0$ and another subsequence that is monotone and diverges to $+ \infty$.
		
		\item A sequence of natural numbers such that for each $j \in \N$, it has a subsequence that converges to $j$. (Feel free to just describe the pattern -- no formulas needed.
		As a hint, recall that the constant sequence $j$ converges to $j$.)
		
	\end{enumerate}
	
	\
	
	\item Prove that if $\{a_n\}_{n=1}^{\infty}$ is a sequence that diverges to $\infty$, then every subsequence of $\{a_n\}_{n=1}^{\infty}$ diverges to $\infty$.
	
	\
	
	\item Prove that for every real number $x$, there is a sequence of irrational numbers that converges to $x$.
	
	\
	
		\item  Let $\{a_n\}_{n=1}^\infty$ be any sequence and $L$ any real
	number. 
	Prove that if $\{a_n\}_{n=1}^\infty$ does not converge to $L$,  then
	there exists an $\e > 0$ and a subsequence
	$\{a_{n_k}\}_{k=1}^\infty$ such that $|a_{n_k} - L| \geq \e$ for all $k$.

	
\

\item Determine whether each of the following sequences is Cauchy, and prove your answer just using the definition of Cauchy (not any theorems).
	\begin{enumerate}
		
	%	\item $\left\{   \frac{(-1)^n}{n}  \right\}_{n=1}^\infty$
		
			\item $\left\{ \sqrt{n}  \right\}_{n=1}^\infty$
			
			
		\item $\left\{  \frac{n}{n+1}   \right\}_{n=1}^\infty$
		
	
		
	\end{enumerate}


\end{enumerate}

\end{document}

























\end{document}