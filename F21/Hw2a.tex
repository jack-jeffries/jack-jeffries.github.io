\documentclass{amsart}[12pt]
\usepackage{graphicx}
\usepackage{comment}
\usepackage{amscd,mathabx}
\usepackage{amssymb,setspace}
\usepackage{latexsym,amsfonts,amssymb,amsthm,amsmath,amscd,stmaryrd,mathrsfs}
\usepackage[all, knot]{xy}
\usepackage[top=1in, bottom=.9in, left=1in, right=1in]{geometry}
\xyoption{all}
\xyoption{arc}
%\usepackage{hyperref}


%\usepackage[notcite,notref]{showkeys}
 
%\CompileMatricesx
\newcommand{\edit}[1]{\marginpar{\footnotesize{#1}}}
%\newcommand{\edit}[1]{}
\newcommand{\rperf}[2]{\operatorname{RPerf}(#1 \into #2)}



\newcommand{\vectwo}[2]{\begin{bmatrix} #1 \\ #2 \end{bmatrix}}

\newcommand{\vecfour}[4]{\begin{bmatrix} #1 \\ #2 \\ #3 \\ #4 \end{bmatrix}}

\newcommand{\Cat}[1]{\left<\left< \text{#1} \right>\right>}


\def\htpy{\simeq_{\mathrm{htpc}}}
\def\tor{\text{ or }}
\def\fg{finitely generated~}

\def\Ass{\operatorname{Ass}}
\def\ann{\operatorname{ann}}
\def\sign{\operatorname{sign}}

\def\ob{{\mathfrak{ob}} }
\def\BiAdd{\operatorname{BiAdd}}
\def\BiLin{\operatorname{BiLin}}

\def\Syl{\operatorname{Syl}}
\def\span{\operatorname{span}}

\def\sdp{\rtimes}
\def\cL{\mathcal L}
\def\cR{\mathcal R}



\def\ay{??}
\def\Aut{\operatorname{Aut}}
\def\End{\operatorname{End}}
\def\Mat{\operatorname{Mat}}


\def\a{\alpha}



\def\etale{\'etale~}
\def\tW{\tilde{W}}
\def\tH{\tilde{H}}
\def\tC{\tilde{C}}
\def\tS{\tilde{S}}
\def\tX{\tilde{X}}
\def\tZ{\tilde{Z}}
\def\HBM{H^{\text{BM}}}
\def\tHBM{\tilde{H}^{\text{BM}}}
\def\Hc{H_{\text{c}}}
\def\Hs{H_{\text{sing}}}
\def\cHs{{\mathcal H}_{\text{sing}}}
\def\sing{{\text{sing}}}
\def\Hms{H^{\text{sing}}}
\def\Hm{\Hms}
\def\tHms{\tilde{H}^{\text{sing}}}
\def\Grass{\operatorname{Grass}}
\def\image{\operatorname{im}}
\def\im{\image}
\def\ker{\operatorname{ker}}
\def\cone{\operatorname{cone}}
\newcommand{\Hom}{\mathrm{Hom}}


\def\ku{ku}
\def\bbu{\bf bu}
\def\KR{K{\mathbb R}}

\def\CW{\underline{CW}}
\def\cP{\mathcal P}
\def\cE{\mathcal E}
\def\cL{\mathcal L}
\def\cJ{\mathcal J}
\def\cJmor{\cJ^\mor}
\def\ctJ{\tilde{\mathcal J}}
\def\tPhi{\tilde{\Phi}}
\def\cA{\mathcal A}
\def\cB{\mathcal B}
\def\cC{\mathcal C}
\def\cZ{\mathcal Z}
\def\cD{\mathcal D}
\def\cF{\mathcal F}
\def\cG{\mathcal G}
\def\cO{\mathcal O}
\def\cI{\mathcal I}
\def\cS{\mathcal S}
\def\cT{\mathcal T}
\def\cM{\mathcal M}
\def\cN{\mathcal N}
\def\cMpc{{\mathcal M}_{pc}}
\def\cMpctf{{\mathcal M}_{pctf}}
\def\L{\Lambda}

\def\sA{\mathscr A}
\def\sB{\mathscr B}
\def\sC{\mathscr C}
\def\sZ{\mathscr  Z}
\def\sD{\mathscr  D}
\def\sF{\mathscr  F}
\def\sG{\mathscr G}
\def\sO{\mathscr  O}
\def\sI{\mathscr I}
\def\sS{\mathscr S}
\def\sT{\mathscr  T}
\def\sM{\mathscr M}
\def\sN{\mathscr N}



\def\Ext{\operatorname{Ext}}
 \def\ext{\operatorname{ext}}



\def\ov#1{{\overline{#1}}}

\def\vecthree#1#2#3{\begin{bmatrix} #1 \\ #2 \\ #3 \end{bmatrix}}

\def\tOmega{\tilde{\Omega}}
\def\tDelta{\tilde{\Delta}}
\def\tSigma{\tilde{\Sigma}}
\def\tsigma{\tilde{\sigma}}


\def\d{\delta}
\def\td{\tilde{\delta}}

\def\e{\epsilon}
\def\nsg{\unlhd}
\def\pnsg{\lhd}

\newcommand{\tensor}{\otimes}
\newcommand{\homotopic}{\simeq}
\newcommand{\homeq}{\cong}
\newcommand{\iso}{\approx}

\DeclareMathOperator{\ho}{Ho}
\DeclareMathOperator*{\colim}{colim}


\newcommand{\Q}{\mathbb{Q}}
\renewcommand{\H}{\mathbb{H}}

\newcommand{\bP}{\mathbb{P}}
\newcommand{\bM}{\mathbb{M}}
\newcommand{\A}{\mathbb{A}}
\newcommand{\bH}{{\mathbb{H}}}
\newcommand{\G}{\mathbb{G}}
\newcommand{\bR}{{\mathbb{R}}}
\newcommand{\bL}{{\mathbb{L}}}
\newcommand{\R}{{\mathbb{R}}}
\newcommand{\F}{\mathbb{F}}
\newcommand{\E}{\mathbb{E}}
\newcommand{\bF}{\mathbb{F}}
\newcommand{\bE}{\mathbb{E}}
\newcommand{\bK}{\mathbb{K}}


\newcommand{\bD}{\mathbb{D}}
\newcommand{\bS}{\mathbb{S}}

\newcommand{\bN}{\mathbb{N}}


\newcommand{\bG}{\mathbb{G}}

\newcommand{\C}{\mathbb{C}}
\newcommand{\Z}{\mathbb{Z}}
\newcommand{\N}{\mathbb{N}}

\newcommand{\M}{\mathcal{M}}
\newcommand{\W}{\mathcal{W}}



\newcommand{\itilde}{\tilde{\imath}}
\newcommand{\jtilde}{\tilde{\jmath}}
\newcommand{\ihat}{\hat{\imath}}
\newcommand{\jhat}{\hat{\jmath}}

\newcommand{\fc}{{\mathfrak c}}
\newcommand{\fp}{{\mathfrak p}}
\newcommand{\fm}{{\mathfrak m}}
\newcommand{\fn}{{\mathfrak n}}
\newcommand{\fq}{{\mathfrak q}}

\newcommand{\op}{\mathrm{op}}
\newcommand{\dual}{\vee}

\newcommand{\DEF}[1]{\emph{#1}\index{#1}}
\newcommand{\Def}[1]{#1 \index{#1}}


% The following causes equations to be numbered within sections
\numberwithin{equation}{section}


\theoremstyle{plain} %% This is the default, anyway
\newtheorem{thm}[equation]{Theorem}
\newtheorem{thmdef}[equation]{TheoremDefinition}
\newtheorem{introthm}{Theorem}
\newtheorem{introcor}[introthm]{Corollary}
\newtheorem*{introthm*}{Theorem}
\newtheorem{question}{Question}
\newtheorem{cor}[equation]{Corollary}
\newtheorem{por}[equation]{Porism}
\newtheorem{lem}[equation]{Lemma}
\newtheorem{lemminition}[equation]{Lemminition}
\newtheorem{prop}[equation]{Proposition}
\newtheorem{porism}[equation]{Porism}

\newtheorem{conj}[equation]{Conjecture}
\newtheorem{quest}[equation]{Question}

\theoremstyle{definition}
\newtheorem{defn}[equation]{Definition}
\newtheorem{chunk}[equation]{}
\newtheorem{ex}[equation]{Example}

\newtheorem{exer}[equation]{Optional Exercise}

\theoremstyle{remark}
\newtheorem{rem}[equation]{Remark}

\newtheorem{notation}[equation]{Notation}
\newtheorem{terminology}[equation]{Terminology}



\renewcommand{\sec}[1]{\section{#1}}
\newcommand{\ssec}[1]{\subsection{#1}}
\newcommand{\sssec}[1]{\subsubsection{#1}}

\newcommand{\br}[1]{\lbrace \, #1 \, \rbrace}
\newcommand{\li}{ < \infty}
\newcommand{\quis}{\simeq}
\newcommand{\xra}[1]{\xrightarrow{#1}}
\newcommand{\xla}[1]{\xleftarrow{#1}}
\newcommand{\xlra}[1]{\overset{#1}{\longleftrightarrow}}

\newcommand{\xroa}[1]{\overset{#1}{\twoheadrightarrow}}
\newcommand{\xria}[1]{\overset{#1}{\hookrightarrow}}
\newcommand{\ps}[1]{\mathbb{P}_{#1}^{\text{c}-1}}




\def\and{{ \text{ and } }}
\def\oor{{ \text{ or } }}

\def\Perm{\operatorname{Perm}}
\newcommand{\Ss}{\mathbb{S}}

\def\Op{\operatorname{Op}}
\def\res{\operatorname{res}}
\def\ind{\operatorname{ind}}

\def\sign{{\mathrm{sign}}}
\def\naive{{\mathrm{naive}}}
\def\l{\lambda}


\def\ov#1{\overline{#1}}
\def\cV{{\mathcal V}}
%%%-------------------------------------------------------------------
%%%-------------------------------------------------------------------

\newcommand{\chara}{\operatorname{char}}
\newcommand{\Kos}{\operatorname{Kos}}
\newcommand{\opp}{\operatorname{opp}}
\newcommand{\perf}{\operatorname{perf}}

\newcommand{\Fun}{\operatorname{Fun}}
\newcommand{\GL}{\operatorname{GL}}
\newcommand{\SL}{\operatorname{SL}}
\def\o{\omega}
\def\oo{\overline{\omega}}

\def\cont{\operatorname{cont}}
\def\te{\tilde{e}}
\def\gcd{\operatorname{gcd}}

\def\stab{\operatorname{stab}}

\def\va{\underline{a}}

\def\ua{\underline{a}}
\def\ub{\underline{b}}


\newcommand{\Ob}{\mathrm{Ob}}
\newcommand{\Set}{\mathbf{Set}}
\newcommand{\Grp}{\mathbf{Grp}}
\newcommand{\Ab}{\mathbf{Ab}}
\newcommand{\Sgrp}{\mathbf{Sgrp}}
\newcommand{\Ring}{\mathbf{Ring}}
\newcommand{\Fld}{\mathbf{Fld}}
\newcommand{\cRing}{\mathbf{cRing}}
\newcommand{\Mod}[1]{#1-\mathbf{Mod}}
\newcommand{\vs}[1]{#1-\mathbf{vect}}
\newcommand{\Vs}[1]{#1-\mathbf{Vect}}
\newcommand{\vsp}[1]{#1-\mathbf{vect}^+}
\newcommand{\Top}{\mathbf{Top}}
\newcommand{\Setp}{\mathbf{Set}_*}
\newcommand{\Alg}[1]{#1-\mathbf{Alg}}
\newcommand{\cAlg}[1]{#1-\mathbf{cAlg}}
\newcommand{\PO}{\mathbf{PO}}
\newcommand{\Cont}{\mathrm{Cont}}
\newcommand{\MaT}[1]{\mathbf{Mat}_{#1}}

%%%-------------------------------------------------------------------
%%%-------------------------------------------------------------------
%%%-------------------------------------------------------------------
%%%-------------------------------------------------------------------
%%%-------------------------------------------------------------------

\makeindex
\title{Assignment \#2}


\begin{document}
\onehalfspacing

\maketitle




\begin{enumerate}

\item Opposites: Let $R$ be a ring.
\begin{enumerate}
\item Prove that there is an isomorphism\footnote{Hint: Your map should involve transposes.} $M_n(R^\mathrm{op}) \cong M_n(R)^{\mathrm{op}}$.
\item Prove that there is an isomorphism $\End_R(R) \cong R^{\mathrm{op}}$.
\end{enumerate}

\

\item A module is \emph{finitely generated} if it has a finite generating set, and \emph{finitely presented} if it has a finite generating set for which the module of relations is finitely generated. Let 
\[0 \to M' \xra{i} M \xra{p} M'' \to 0\] be a short exact sequence of $R$-modules.
\begin{enumerate}
\item Show that if $M'$ and $M''$ are finitely generated, then $M$ is finitely generated.
\item[(b*)] Show that if $M'$ and $M''$ are finitely presented, then $M$ is finitely presented. 
\end{enumerate}

\


\item Fix a field $K$. The collection of pairs $(V,W)$ where $W\subseteq V$ are vector spaces forms a category $\sC$, where the morphisms from $(V,W)\to(V',W')$ are linear transformations $\phi:V\to V'$ such that $\phi(W)\subseteq W'$. There are covariant functors $F,G:\sC \to \Vs{K}$ given by
\[\begin{aligned} &F(V,W)= V  \quad	&F(\phi)&=\phi \\
&G(V,W) = W\oplus V/W 	\quad	 &G(\phi)&= \phi|_W \oplus \overline{\phi}\end{aligned}\]
where $\overline{\phi}:V/W\to V'/W'$ is the induced map $\overline{\phi}(v+W)=\phi(v) + W'$ on the quotient spaces.
\begin{enumerate}
\item Show that for every $(V,W)\in \Ob(\sC)$, there is an isomorphism of vector spaces $F(V) \cong G(V)$.
\item Let $W=K\oplus \{0\} \subseteq V= K^2$, and take $\phi:K^2 \to K^2$ to be the map given by the matrix $\begin{bmatrix} 1 & 1 \\ 0 & 1\end{bmatrix}$. Check that $\phi$ is a morphism in $\sC$, and compute $F(\phi)$ and $G(\phi)$.
\item Show that there is no natural isomorphism\footnote{Moral: Every short exact sequence of vector spaces splits, but \emph{not} naturally!} $\eta: F\Rightarrow G$.
\end{enumerate}

\

\item A covariant functor $F:\Mod{R} \to \Mod{S}$ is \emph{additive} if for every $M,N\in \Mod{R}$, the map
\[\xymatrix@R=.2em{ \Hom_R(M,N) \ar[r] & \Hom_S(F(M),F(N)) \\ f \ar@{|->}[r] & F(f) }\]
is a homomorphism of abelian groups. Show that if $F$ is an additive covariant functor, and \[ 0 \to M' \xra{i} M \xra{p} M'' \to 0\]
is a split exact sequence, then
\[ 0 \to F(M') \xra{F(i)} F(M) \xra{F(p)} F(M'') \to 0\]
is exact\footnote{Moral: Functors (additive or not) between module categories don't always preserve short exact sequences, but (at least additive functors) always preserve \emph{split} exact sequences.}.

\

\item The localization functor:

 Let $R$ be a commutative ring. A subset $S$ of $R$ is \emph{multiplicatively closed} if $1 \in S$ and $s,t \in S \Rightarrow st \in S$. Define a new ring $S^{-1}R$ as follows:
\[S^{-1}R =\left \{\frac{r}{s} \mid  r \in R, s \in S \right \}/\sim\]
where $\sim$ is the equivalence relation  $\frac{r}{s} \sim \frac{r'}{s'}$ if and only if $t(rs' - r's) = 0$ for some $t \in S$.
This\footnote{This generalizes the construction of the fraction field of a domain $R$, where $S=R\smallsetminus \{0\}$ gives $S^{-1}R={\rm Frac}(R)$.} set is a ring (a fact you need not check) with respect to the operations
\[
\frac{r}{s}+\frac{r'}{s'}=\frac{rs'+r's}{ss'} \qquad
\frac{r}{s}\cdot\frac{r'}{s'}=\frac{rr'}{ss'}.
\]
For an $R$-module $M$ define  
\[S^{-1}M =\left \{\frac{m}{s} \mid  m \in M, s \in S \right \}/\sim\]
where $\sim$ is the equivalence relation  $\frac{m}{s} \sim \frac{m'}{s'}$ if and only if $t(ms' - m's) = 0$ for some $t \in S$.
Then $S^{-1}M$ is an $S^{-1}R$-module (a fact you need not check) via the operations
\[
\frac{m}{s}+\frac{m'}{s'}=\frac{ms'+m's}{ss'} \qquad
\frac{r}{s}\cdot\frac{m}{s'}=\frac{rm}{ss'}.
\]
\begin{enumerate}
\item Show that there is a functor $S^{-1}:\Mod{R}\to \Mod{{S^{-1}R}}$ that on objects maps $M\mapsto S^{-1}M$ and on morphisms maps $f\mapsto S^{-1}f$ where $(S^{-1}f)(\frac{m}{s})=\frac{f(m)}{s}$.
\item A covariant functor $\Mod{R} \to \Mod{S^{-1}R}$ is \emph{exact} if it is additive and takes short exact sequences to short exact sequences. Show that the localization functor from (a) is exact.
\end{enumerate}

\


\item[(6*)]
\begin{enumerate}
\item We only defined a notion of natural transformation/isomorphism for $F,G$ both covariant or $F,G$ both contravariant. Come up with a definition of natural transformation/isomorphism for $F$ covariant and $G$ contravariant.
\item Show that with this definition, for a field $K$, the functors $1_{\vs{K}},(-)^*:\vs{K}\to \vs{K}$ are still not naturally isomorphic.
\item Let $K-\mathbf{inn}$ where 
\begin{itemize}
\item objects are finite dimensional $K$-vector spaces equipped with a nondegenerate\footnote{That is, for every $v\in V\smallsetminus\{0\}$, there is some $v'\in V$ such that $\langle v, v' \rangle_V\neq 0$.} symmetric bilinear form $\langle - , - \rangle_V: V\times V \to K$, and the 
\item morphisms are linear maps $\phi:V\to W$ such that $\langle v, v' \rangle_V = \langle \phi(v) , \phi(v')\rangle_W$.
\end{itemize}
 Show that the functors $F,G:K-\mathbf{inn}\to \vs{K}$ given by
\[\begin{aligned} &F(V)= V  \quad	&F(\phi)&=\phi \\
&G(V) = V^* 	\quad	 &G(\phi)&= \phi^*\end{aligned}\]
 are naturally isomorphic.
\end{enumerate}
\end{enumerate}



\end{document}







  
 


