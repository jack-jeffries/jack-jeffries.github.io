\documentclass{amsart}
\usepackage{amstext,amsfonts,amssymb,amscd,amsbsy,amsmath}
\usepackage{geometry}[margin=1in]
\pagestyle{empty}
%\parindent = 0in


\def\sol#1{{\bf Solution: } #1}
%\def\sol#1{}
\def\bl{\vskip .1in}
\def\star{${}^*$}
\def\cS{\mathcal S}
\def\cT{\mathcal T}
\def\cB{\mathcal B}

\def\R{\mathbb R}
\def\N{\mathbb N}
\def\Q{\mathbb Q}
\def\Z{\mathbb Z}

\def\cC{\mathcal C}

\def\e{\epsilon}
\def\d{\delta}

\begin{document}



\begin{center}
{\large\bfseries
Math 325-002 --- Problem Set \#3 \\
Due: Wednesday, September 15 by 5 pm}
\end{center}





{\bf Instructions:} You are encouraged to work together on these
problems, but each student should hand in their own final draft,
written in a way that indicates their individual understanding of
the solutions. Never submit something for grading
that you do not completely understand. 

If you do work with others, I ask that you write something along the
top like ``I collaborated with Steven Smale on problems 1 and 3''.
If you use a reference, indicate so clearly in your solutions. 
In short, be intellectually
honest at all times.

Please write neatly, using complete sentences and correct
punctuation. Label the problems clearly. 






\begin{enumerate}
\item Prove\footnote{Tip: You will need to use the following fact, proven in lecture: If $x$ is any real number, then there is a natural number $n$ such that $n > x$.}
 that if $\e$ is any real number such that $\e > 0$, then there 
exists  a natural number $n$ such that $0 < \frac{1}{n} < \e$.

\

\item Let $S$ be the set $\{1 - \frac{1}{n} \mid n \in \N\}$. Prove\footnote{Tip: First show $1$ is an upper bound, and then use a proof by contradiction. That is, assume $b$ is an upper bound of $S$ 
such that $b < 1$ and  proceed to derive a contradiction. The statement proven in the previous exercise might be useful.}
 that $1$ is the supremum (aka least upper bound) of $S$.


\


\item Let $r$ be any real number. Consider the set
$$
S_r = \{q \in \Q \mid q < r\}.
$$
In words, $S_r$ is the set of those {\em rational} numbers that are
strictly less
than $r$. Prove that the supremum of $S_r$ is $r$. 

\

\item Read Section 1.7 of the text, and do problems 4 and 7 on pages 65--66.

 %\item 
%Given any two real numbers $x$ and $y$, $\max \{x, y\}$ refers to the larger of the two numbers $x$ and
%$y$; that is, $\max \{x,y\}$ is $x$ if $x \geq y$ and otherwise it is~$y$.
%Similarly, $\min \{x, y\}$ refers to the smaller of the two numbers $x$ and
%$y$; that is, $\min\{x,y\}$ is $x$ if $x \leq y$ and otherwise it is~$y$.



\end{enumerate}

\end{document}

























\end{document}