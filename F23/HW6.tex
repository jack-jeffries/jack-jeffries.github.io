\documentclass{amsart}
\usepackage{amstext,amsfonts,amssymb,amscd}
\usepackage[margin=1.2in]{geometry}
%\usepackage[showframe=false,headheight=1cm,margin=1in,bottom=1in]{geometry}
%\pagestyle{empty}
%\parindent = 0in


\def\sol#1{{\bf Solution: } #1}
%\def\sol#1{}
\def\bl{\vskip .1in}
\def\star{${}^*$}
\def\cS{\mathcal S}
\def\cT{\mathcal T}
\def\cB{\mathcal B}

\def\R{\mathbb R}
\def\N{\mathbb N}
\def\Q{\mathbb Q}
\def\Z{\mathbb Z}

\def\cC{\mathcal C}

\def\e{\epsilon}
\def\d{\delta}

\begin{document}


\begin{center}
{\large\bfseries
Math 445 --- Problem Set \#6 \\
Due: Friday, October ?? by 7 pm, on Canvas}
\end{center}





{\bf Instructions:} You are encouraged to work together on these
problems, but each student should hand in their own final draft,
written in a way that indicates their individual understanding of
the solutions. Never submit something for grading
that you do not completely understand. If you do work with others, I ask that you write something along the
top like ``I collaborated with Steven Smale on problems 1 and 3''.
If you use a reference, indicate so clearly in your solutions. 
In short, be intellectually
honest at all times. Please write neatly, using complete sentences and correct
punctuation. Label the problems clearly. 






\begin{enumerate}

\item Use the methods from class to give a formula\footnote{As in class, in terms of coefficients powers of some $a+b\sqrt{D}$.} for all solutions of the Pell's equation
\[ x^2 - 13 y^2 = 1.\]

\

\item Closed formulas for solutions to Pell's equations.
\begin{enumerate}
\item Explain why the $k$th positive solution $(x_k,y_k)$ of the Pell's equation $x^2 - 2y^2=1$ satisfies the equation
\[ \begin{bmatrix} x_k \\ y_k \end{bmatrix} = \begin{bmatrix} 3 & 4 \\ 2 & 3\end{bmatrix}^k \begin{bmatrix} 1 \\ 0\end{bmatrix}.\]
\item Diagonalize the matrix $\begin{bmatrix} 3 & 4 \\ 2 & 3\end{bmatrix}$ and use this to give a closed expression for $(x_k,y_k)$ in terms of $k$. Your formulas should be in terms of particular linear combinations of powers of two numbers.
\item Use\footnote{Recall that $\lfloor x \rfloor$ denotes the greatest integer $n$ such that $n\leq x$ and $\lceil x \rceil$ denotes the smallest integer $n$ such that $n\geq x$.} your formulas from the previous part to show that 
\[ x_k =\left\lceil \frac{(3+\sqrt{2})^k}{2} \right\rceil \qquad\text{and}\qquad y_k =\left\lfloor \frac{(3+\sqrt{2})^k}{2\sqrt{2}} \right\rfloor.\]
Use this to quickly write down the first seven positive solutions to the Pell's equation $x^2 - 2y^2=1$.
\item Repeat the steps above with the appropriate numbers for the Pell's equation $x^2 - 5y^2=1$.
\end{enumerate}


\

\item Not solving $x^2 - D y^2 = -1$: Let $D>1$ be a positive integer that is not a perfect square.
\begin{enumerate}
\item Show that if $D\equiv 0 \pmod{4}$ or $D\equiv 3 \pmod{4}$, then the equation $x^2 + Dy^2 = -1$ has no integer solutions.
\item Show that if $q\equiv 3 \pmod{4}$ is prime and $q\,|\,D$, then the equation $x^2 + Dy^2 = -1$ has no integer solutions.
\end{enumerate}


\

\item Solving $x^2 - D y^2 = -1$: Let $D>1$ be a positive integer that is not a perfect square.
\begin{enumerate}
\item Show that if $(c,d)$ is a positive integer solution to $x^2 - D y^2 = -1$, then $\frac{e}{f}$ is a convergent in the continued fraction expansion of $\sqrt{D}$.
\item Show that if $(c,d)$ is a positive integer solution to $x^2 - D y^2 = -1$, $(a,b)$ is a positive integer solution to $x^2 - D y^2 = 1$, and
\[ e + f\sqrt{D} = (a+b\sqrt{D})(c+d\sqrt{D}),\]
then $(e,f)$ is another positive integer solution to $x^2 - D y^2 = -1$.
\item Describe infinitely many solutions to the equation $x^2 - 13y^2 = -1$.
\end{enumerate}




\end{enumerate}

\noindent  \hrulefill

\noindent The remaining problem is only required for Math 845 students, though all are encouraged to think about it.

\

\begin{enumerate}\setcounter{enumi}{4}
\item Let $D$ be a positive integer that is not a perfect square. Suppose that $x^2 - D y^2 = -1$ has a solution, and let $(c,d)$ be the smallest positive integer solution. Let $(a,b)$ be the smallest integer solution to the Pell's equation $x^2 - D y^2 = 1$. Show that $(c+d\sqrt{D})^2 = a+b\sqrt{D}$, and use this to describe all solutions to $x^2 - D y^2 = -1$ in terms of $c$ and $d$.
\end{enumerate}

\end{document}

























\end{document}