\documentclass[12pt]{amsart}


\usepackage{times}
\usepackage[margin=0.7in]{geometry}
\usepackage{amsmath,amssymb,multicol,graphicx,framed,ifthen,color,xcolor,enumerate,colonequals}
\definecolor{chianti}{rgb}{0.6,0,0}
\definecolor{meretale}{rgb}{0,0,.6}
\definecolor{leaf}{rgb}{0,.35,0}
\newcommand{\blue}{\color {blue}}
\newcommand{\red}{\color {red}}
\newcommand{\olive}{\color {olive}}
\newcommand{\violet}{\color {violet}}
\newcommand{\orange}{\color{orange}}
\newcommand{\Q}{\mathbb{Q}}
\newcommand{\N}{\mathbb{N}}
\newcommand{\Z}{\mathbb{Z}}
\newcommand{\R}{\mathbb{R}}
\newcommand{\e}{\varepsilon}
\newcommand{\inv}{^{-1}}
\newcommand{\dabs}[1]{\left| #1 \right|}
\newcommand{\ds}{\displaystyle}
\newcommand{\solution}[1]{\ifthenelse {\equal{\displaysol}{1}} {\begin{framed}{\color{meretale}\noindent #1}\end{framed}} { \ }}
\newcommand{\showsol}[1]{\def\displaysol{#1}}
\newcommand{\rsa}{\rightsquigarrow}
\newcommand{\ceq}{\colonequals}

\DeclareMathOperator{\res}{res}

\numberwithin{equation}{section}


\theoremstyle{plain} %% This is the default, anyway
\newtheorem{thm}	[section]	{Theorem}
\newtheorem{cor}	[section]	{Corollary}
\newtheorem{lem}	[section]	{Lemma}
\newtheorem{prop}	[section]	{Proposition}
\newtheorem{defn}	[section]	{Definition}
\newtheorem{axiom}	[section]	{Axiom}
\newtheorem{conj}	[section]	{Conjecture}
\newtheorem{quest}	[section]	{Question}

%\usepackage{times}

%\addtolength{\textwidth}{100pt}
%\addtolength{\evensidemargin}{-45pt}
%\addtolength{\oddsidemargin}{-60pt}

\pagestyle{empty}
%\begin{document}\begin{itemize}

%\thispagestyle{empty}




\begin{document}
\showsol{0}
	
	\thispagestyle{empty}
	
	\section*{Worksheet \#1}
	
	\begin{defn} A triple $(a,b,c)$ of natural numbers is a \textbf{Pythagoran triple} if they form the side lengths of a right triangle, where $c$ is the length of the hypotenuse.
\end{defn}

\begin{thm}[Fundamental Theorem of Arithmetic] Every natural number $n\geq 1$ can be written as a product of prime numbers:
\[ n = p_1^{e_1} p_2^{e_2} \cdots p_k^{e_k}.\]
This expression is unique up to reordering. \qed
\end{thm}

\begin{defn} We call the number $e_i$ the \textbf{multiplicity} of the prime $p_i$ in the prime factorization of \[n=p_1^{e_1} p_2^{e_2} \cdots p_k^{e_k}.\]\end{defn}

	\begin{defn}  Let $m,n$ be integers and $K\geq 1$ be a natural number. We say that \textbf{$m$ is congruent to $n$ modulo $K$}, written as $m \equiv n \pmod K$, if $m-n$ is a multiple of $K$.\end{defn}


\begin{thm} Let $n$ be an integer and $K\geq 1$ a natural number. Then $n$ is congruent to exactly one nonnnegative integer between $0$ and $K-1$: this number is the ``remainder'' when you divide $n$ by~$K$. \qed
\end{thm}

\begin{prop} Let $m,m',n,n'$ and $K$ be natural numbers. Suppose that 
\[ m \equiv m' \pmod K \quad \text{and} \quad n\equiv n' \pmod K.\]
Then
\[ \pushQED{\qed} m+n \equiv m'+n' \pmod K \quad \text{ and} \quad mn \equiv m'n' \pmod K .\qedhere\popQED\]\end{prop}
	
\begin{defn}  A triple $(a,b,c)$ of natural numbers is a \textbf{primitive Pythagoran triple (PPT)} if ${a^2+b^2=c^2}$, and there is no common factor of $a,b,c$ greater than $1$; equivalently, $a,b,c$ have no common prime factor.
\end{defn}

\begin{thm} The set of primitive Pythagorean triples $(a,b,c)$ with $a$ odd is given by the formula
\[ a=st, \quad b=\frac{s^2-t^2}{2} , \quad c=\frac{s^2+t^2}{2},\]
where $s>t\geq 1$ are odd integers with no common factors.\end{thm}

\begin{thm} The set of points on the unit circle $x^2+y^2=1$ with positive rational coordinates is given by the formula
\[ (x,y) = \left( \frac{ 2v}{v^2+1}, \frac{v^2-1}{v^2+1} \right)\]
where $v$ ranges through rational numbers greater than one. \end{thm}



\section*{Worksheet \#2}


\begin{defn} The \textbf{greatest common divisor} of two integers $a$ and $b$, denoted $\gcd(a,b)$, is the largest integer that divides $a$ and $b$.\end{defn}


\begin{defn} Two integers $a$ and $b$ are \textbf{coprime} if $\gcd(a,b)=1$.\end{defn}


\begin{thm}
The Euclidean algorithm terminates and outputs the correct value of $\gcd(a,b)$.
\end{thm}


\begin{defn}  An expression of the form $ra+sb$ with $r,s\in \Z$ is a \textbf{linear combination} of $a$ and $b$.\end{defn}

\begin{cor} If $a,b$ are integers, then $\gcd(a,b)$ can be realized as a linear combination of $a$ and $b$. Concretely, we can use the Euclidean algorithm to do this.
\end{cor}

\begin{thm} Let $a,b,c$ be integers. The equation 
\[ ax+by=c\]
has an integer solution if and only if $c$ is divisible by $d\ceq \gcd(a,b)$. If this is the case, there are infinitely many solutions. If $(x_0,y_0)$ is a one particular solution, then the general solution is of the form
\[ x = x_0 - (b/d)n ,\quad y= y_0+(a/d) n\]
as $n$ ranges through all integers.
\end{thm}

\section*{Problem Set \#1}

\begin{lem} Lat $a,b,c$ be integers. If $a$ and $b$ are coprime, and $a$ divides $bc$, then $a$ divides $bc$.
\end{lem}

\section*{Worksheet \#3}

\begin{defn} A \textbf{congruence class} modulo $K$ is a set of the form
\[ [a] \ceq \{ n\in \Z \ | \ n\equiv a \pmod{K}\}\]
for some $a\in \Z$.
\end{defn} 

\begin{defn} A \textbf{representative} for a congruence class is an element of the congruence class.
\end{defn}

\begin{prop} Given $K>0$, the set of integers $\Z$ is the disjoint union of $K$ congruence classes:
\[  \Z = [0] \sqcup [1] \sqcup \cdots \sqcup [K-1]. \]
\end{prop}

\begin{defn} The ring $\Z_K$ is the set of congruence classes modulo $K$:
\[\{ [0] , [1], \ldots, [K-1]\}\] equipped with the operations
\[ [a] + [b] = [a+b] \quad\text{and}\quad [a][b] = [ab].\]
\end{defn}


\begin{defn} We say that a number $a$ is a \textbf{unit modulo $K$}  if there is an integer solution $x$ to $ax\equiv 1 \pmod{K}$, and we say that such a number $x$ is an \textbf{inverse modulo $K$} to $a$.
\end{defn}

\begin{defn} We say that a congruence class $[a]$ is a \textbf{unit in $\Z_K$} if there is a congruence class $x \in \Z_K$ such that $[a] x = [1]$, and we say that such a class $x$ is an \textbf{inverse} to $[a]$ in $\Z_K$.
\end{defn}

\begin{thm}
 Let $a$ and $n$ be integers, with $n$ positive. Then $a$ is a unit modulo $n$ if and only if $a$ and $n$ are coprime.
 \end{thm}
 
 \begin{thm}[Chinese Remainder Theorem]
 Given $m_1,\dots,m_k >0$ integers such that $m_i$ and $m_j$ are coprime for each $i\neq j$, and $a_1,\dots,a_k\in \Z$, the system of congruences
\[ \begin{cases} \begin{aligned} 
x &\equiv a_1 \pmod{m_1}  \\
x &\equiv a_2 \pmod{m_2}  \\
&\ \ \vdots \qquad \qquad \vdots\\
x &\equiv a_k \pmod{m_k}
\end{aligned}\end{cases}\]
has a solution $x\in \Z$. Moreover, the set of solutions forms a unique congruence class modulo $m_1 m_2 \cdots m_k$.
\end{thm}

\section*{Problem Set \#2}

\begin{lem} Lat $a,b,c$ be integers. If $a$ and $b$ are coprime, $a$ divides $c$, and $b$ divides $c$, then $a$ divides $bc$.
\end{lem}

\begin{defn}  Given integers $a_1,\dots,a_m$, the \textbf{greatest common divisor} of $a_1,\dots,a_m$ is the largest integer that divides all of them. 
\end{defn}

\begin{thm} Let $a,b,n$ be integers, with $n>0$. Then $[a] x = [b]$ has a solution $x$ in $\Z_n$ if and only if $\gcd(a,n)$ divides $b$. In this case, the number of distinct solutions is exactly $\gcd(a,n)$.
\end{thm}

\section*{Worksheet \#4}

\begin{defn} A \textbf{group} is a set $G$ equipped with a product operation \[ G\times G \to G \qquad (g,h)\mapsto gh \] and an \textbf{identity} element $1\in G$ such that
\begin{itemize}
\item the product is associative: $(gh)k = g(hk)$ for all $g,h,k\in G$,
\item $g1 = 1g = g$ for all $g\in G$, and
\item for every $g\in G$, there is an inverse element $g^{-1}\in G$ such that $gg^{-1} = g^{-1} g = 1$.
\end{itemize}
\end{defn}


\begin{defn} A group is \textbf{abelian} if the product is commutative: $gh=hg$ for all $g,h\in G$.
\end{defn}

\begin{defn} A \textbf{finite group} is a group $G$ that is a finite set.
\end{defn}


\begin{defn}  Let $G$ be a group and $g\in G$. The \textbf{order} of $g$ is the smallest positive integer $n$ such that $g^n =e$, if some such $n$ exists, and $\infty$ if no such integer exists.
\end{defn}

\begin{thm}[Lagrange's Theorem]  Let $G$ be a finite group and $g\in G$. Then the order of $g$ is finite and divides the cardinality of the group $G$.
\end{thm}


\begin{thm}[Fermat's Little Theorem] Let $p$ be a prime number and $a$ an integer. If $p$ does not divide $a$, then \[ a^{p-1} \equiv 1 \pmod p.\] 
\end{thm}

\begin{defn} Let $n$ be a positive integer. We define $\varphi(n)$ to be the number of elements of $\Z_n^\times$. We call this \textbf{Euler's phi function}.
\end{defn}

\begin{prop} Euler's phi function satisfies the following properties.
\begin{enumerate}
\item If $p$ is a prime and $n$ is a positive integer, then $\varphi(p^n) = p^{n-1}(p-1)$.
\item If $m,n$ are coprime positive integers, then $\varphi(mn)=\varphi(m)\varphi(n)$.
\end{enumerate}
\end{prop}

\begin{thm}[Euler's Theorem] Let $a,n$ be coprime integers, with $n$ positive. Then
\[ a^{\varphi(n)} \equiv 1 \pmod n.\]
\end{thm}


\section*{Worksheet \#5}


\begin{prop} Let $p$ be a prime. Let $p(x)$ be a polynomial of degree $d$ with coefficients in $\Z_p$. Then $p(x)$ has at most $d$ roots in $\Z_p$.\qed
\end{prop}

\begin{lem} If $G$ is a group, $g\in G$, and $n$ a positive integer such that $g^n=1$, then the order of $g$ divides $n$.
\end{lem}


\begin{defn} Let $n$ be a positive integer. An element $x \in \Z_n^\times$ is a \textbf{primitive root} if the order of $x$ in $\Z_n^\times$ equals $\phi(n)$ (the cardinality of $\Z_n^\times$).
\end{defn}

\begin{thm} Let $p$ be a prime number. Then there exists a primitive root in $\Z_p^\times$.
\end{thm}

\begin{defn} If $[a]$ is a primitive root in $\Z_p$, the function
\[ \Z_p^\times \to \Z_{p-1} \qquad [b]\mapsto [m] \ \text{such that} \ [b]=[a]^m\]
is called the \textbf{discrete logarithm} or \textbf{index} of $\Z_p^\times$ with base $[a]$.
\end{defn}

\begin{lem} Let $p$ be a prime and $[a]$ a primitive root in $\Z_p$. The corresponding discrete logarithm function $I: \Z_p^\times \to \Z_{p-1}$ satisfies the property
\[ I( xy) = I(x) + I(y) \quad \text{and} \quad I(x^n) = [n] I(x)\]
for $x,y\in \Z_p^\times$ and $n\in \N$.
\end{lem}


\begin{prop} Let $n$ be a positive integer. Then $\displaystyle \sum_{d \, |\, n} \varphi(d) = n.$
\end{prop}

\begin{thm} Let $p$ be a prime. Suppose that there are $n$ distinct solutions to $x^n=1$ in $\Z_p$. Then $\Z_p^\times$ has exactly $\varphi(n)$ elements of order $n$.
\end{thm}



\end{document}
