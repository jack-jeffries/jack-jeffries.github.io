\documentclass{amsart}
\usepackage{amstext,amsfonts,amssymb,amscd}
\usepackage[margin=1.2in]{geometry}
%\usepackage[showframe=false,headheight=1cm,margin=1in,bottom=1in]{geometry}
%\pagestyle{empty}
%\parindent = 0in


\def\sol#1{{\bf Solution: } #1}
%\def\sol#1{}
\def\bl{\vskip .1in}
\def\star{${}^*$}
\def\cS{\mathcal S}
\def\cT{\mathcal T}
\def\cB{\mathcal B}

\def\R{\mathbb R}
\def\N{\mathbb N}
\def\Q{\mathbb Q}
\def\Z{\mathbb Z}

\def\cC{\mathcal C}

\def\e{\epsilon}
\def\d{\delta}

\begin{document}


\begin{center}
{\large\bfseries
Math 445 --- Problem Set \#5 \\
Due: Friday, October 20 by 7 pm, on Canvas}
\end{center}





{\bf Instructions:} You are encouraged to work together on these
problems, but each student should hand in their own final draft,
written in a way that indicates their individual understanding of
the solutions. Never submit something for grading
that you do not completely understand. If you do work with others, I ask that you write something along the
top like ``I collaborated with Steven Smale on problems 1 and 3''.
If you use a reference, indicate so clearly in your solutions. 
In short, be intellectually
honest at all times. Please write neatly, using complete sentences and correct
punctuation. Label the problems clearly. 






\begin{enumerate}

\item The continued fraction expansion of Euler's constant $e$ is given by
\[ e = [2;1,2,1,1,4,1,1,6,1,1,8, \dots].\]
Use this and results from class to find a rational approximation of $e$ that is accurate to four digits (beyond the decimal place) without using any other knowledge about the number $e$.

\


\item Find the real number with continued fraction expansion 
\[[1; 2,3,2,3,2,3,\dots]\quad \text{ (and repeats forever like so).}\]

\

\item Let $d\geq  2$ be a positive integer. \begin{enumerate}
\item Show that the continued fraction expansion of $\sqrt{d^2+1}$ is 
\[\sqrt{d^2+1}=  [d; 2d, 2d, 2d, 2d, 2d, 2d \dots]\quad \text{ (and repeats forever like so).}\]

\item Show that the continued fraction expansion of $\sqrt{d^2-1}$ is 
\[\sqrt{d^2-1} = [d-1; 1,  2d-2, 1, 2d-2, 1, 2d-2, \dots] \quad \text{ (and repeats forever like so).}\]
\item Apply the previous parts to give continued fraction expansions for $\sqrt{101}$ and $\sqrt{63}$.
\end{enumerate}


\


\item In this problem, we will prove the following theorem, which basically says that the convergents are the \emph{best} approximations of a rational number. 

\noindent \textsc{Theorem:} Let $r$ be a real number, $C_k= \frac{p_k}{q_k}$ be the $k$-th convergent of $r$, and $\frac{p}{q}\neq r$ be a rational number, with $q>0$. If $q \leq q_k$, then  $\displaystyle \left| r-\frac{p}{q} \right| \geq \left| r-\frac{p_k}{q_k} \right| $.

\begin{enumerate} 
\item Set $u=(-1)^k (q_k p -  p_k q)$ and $v=(-1)^k( p_{k+1} q - q_{k+1} p)$. Show that $p_k u + p_{k+1} v = p$ and $q_k u + q_{k+1} v = q$.
\item Show\footnote{Use the Proposition from class to show that $p_{k+1}, q_{k+1}$ are coprime, and that $u=0$ implies $q_{k+1} | q$.} that $u,v\neq 0$, and that\footnote{Use the second equation from part (a).} $u$ and $v$ have opposite signs.
\item Show that $q_k r - p_k$ and $q_{k+1} r - p_{k+1}$ have opposite signs.
\item Show that $|q r - p| = |u (q_k r - p_k) + v(q_{k+1} r - p_{k+1})| \geq |q_k r - p_k|$ and conclude the proof.
\end{enumerate}

\end{enumerate}

%\noindent  \hrulefill

%\noindent The remaining problem is only required for Math 845 students, though all are encouraged to think about them.

%\

%\begin{enumerate}\setcounter{enumi}{5}






%\end{enumerate}

\end{document}

























\end{document}