\documentclass{amsart}
\usepackage{pdfpages}
% \pagestyle{empty}
\usepackage{amscd}
\usepackage{amssymb}
%\usepackage[all, knot]{xy}
\usepackage{multicol}



\usepackage[margin=1in]{geometry}

%\setlength{\oddsidemargin}{-.10in}
%\setlength{\evensidemargin}{-.10in}
%\setlength{\textwidth}{5.5in}
%\setlength{\topmargin}{-.250in}
%\setlength{\headheight}{0in}
%\setlength{\headsep}{0in}
%\setlength{\topskip}{0in}
%\setlength{\textheight}{9.5in}
%\parindent = 0in
\font\bigbf = cmbx10 scaled \magstep1
\font\medbf = cmbx10 scaled \magstephalf
\begin{document}
 
%\magnification \magstep1
%\parindent = 0pt
%\nopagenumbers
%\voffset = -.5truein
%\vsize = 10 truein
%\baselineskip = 1.5 \baselineskip
%\font\bigbf = cmbx10 scaled \magstep1
%\font\medbf = cmbx10 scaled \magstephalf








\centerline{\bigbf Number Theory: Math 445/845 Fall Semester 2023}
\centerline{\bigbf TR 9:30am–10:45am Nebraska Hall W106}


\bigskip


\begin{multicols}{2}
\noindent
{\bf Instructor:}  Jack Jeffries

\noindent
{\bf Office:} 325 Avery Hall

\noindent
{\bf email:} jack.jeffries@unl.edu

\noindent
{\bf Office Hours:} to be announced
\end{multicols}




\noindent{\bf Textbook:} 
There is no required textbook for the course. I will post lecture notes from the class on the course website. Recommended texts covering similar material are
\emph{A friendly introduction to number theory} by Joseph Silverman and
\emph{Elementary number theory and its applications} by Kenneth Rosen.


\medskip


\noindent{\bf Course content:} 
In this class, we will cover the fundamentals of number theory. Topics include pythagorean triples, primes and primality tests, systems of linear congruences and the Chinese Remainder Theorem, quadratic reciprocity, sums of squares, Pell's equation, continued fractions, and approximation of real numbers by rational numbers. We will explore this material through a combination of lecture, in-class group work, and exercise sets. This content will build on Math 310 topics like the division algorithm, Euclidean algorithm, congruences, rings, and proof by induction.


\smallskip

\noindent{\bf Class engagement policy:} 
This is an in-person class and attendance is required. Class time will involve lecture, discussion, groupwork, and presentations. The expectation for the class is that you will participate in person as health and quarantine circumstances allow. If you are unable to attend in person, you should let me know, and we will arrange for you to participate some other way while you cannot attend in person.

\medskip

\noindent{\bf Grading policy:} Your grade has four components: problem sets, exams, presentation, and participation. 

\begin{itemize}
\item Problem sets will be assigned and collected approximately once per week or every other week. I anticipate there being about seven problems sets total.
You are encouraged to work together on the problem sets, but each of you will hand in your own solutions, written in your own words, and your work must demonstrate a true understanding of the material. Never hand in something that you do not completely understand.

\item There will be two midterm exams and a final exam. The final exam will be \textbf{10:00--noon
Wednesday, Dec. 13} in our usual classroom. For the midterms, we will try to find a common times outside of class to take the exam.
%We will have short quizzes in class on a semi-regular basis. Quiz times will be discussed in class. 

\item The participation grade is based on your participation in class. If you fulfill the class engagement policy then you will get the full score; I expect everyone to do so. This part of the grade is an excuse for me to reward you for working hard on groupwork in class.

\item Each student will, with a partner, write a short ($\sim$5 page) paper and give an in-class presentation ($\sim$20 min) on a topic supplementary to the class material. Around the fifth week, topics will be assigned and discussed. Presentations will take place on Nov 30 and Dec 5.
\end{itemize}
The following table summarizes the grading scheme:


\begin{center}

\begin{tabular}{|l|l|}
\hline
Component & Value \\
\hline \hline
Problem Sets & 30\% \\
\hline
Presentation & 15\% \\
\hline
Midterm Exams (two) &  15\% each \\
\hline
Final Exam & 20\% \\
\hline
Participation & 5\%\\
\hline
\end{tabular}

\end{center}
\


\noindent Letter grades will be based on the usual 10 point scale (90 cutoff between A-/B+, etc.); however, grade cutoffs may be lower (i.e., grades may be higher).

\medskip

\noindent {\bf Math 445 vs 845:} This is a cross-listed undergraduate (445) and graduate (845) course. Students in 845 will have additional problems on problem sets and exams that go into further depth. 

\medskip


\noindent {\bf Students needing accommodations:} Students with a documented need for accommodations should contact me as soon as possible to set this up; if you have a need for an accommodation without documentation, contact the SSD office as soon as possible as well.

\medskip

\noindent {\bf Standard syllabus policies:} All of the University's standard syllabus policies apply in this course. You can find them at:
\tt{http://go.unl.edu/coursepolicies}.

\vfill
\pagebreak
\smallskip
%\includepdf{UNLFace.pdf}



\end{document}


