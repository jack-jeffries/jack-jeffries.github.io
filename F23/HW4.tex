\documentclass{amsart}
\usepackage{amstext,amsfonts,amssymb,amscd}
\usepackage[margin=1.2in]{geometry}
%\usepackage[showframe=false,headheight=1cm,margin=1in,bottom=1in]{geometry}
%\pagestyle{empty}
%\parindent = 0in


\def\sol#1{{\bf Solution: } #1}
%\def\sol#1{}
\def\bl{\vskip .1in}
\def\star{${}^*$}
\def\cS{\mathcal S}
\def\cT{\mathcal T}
\def\cB{\mathcal B}

\def\R{\mathbb R}
\def\N{\mathbb N}
\def\Q{\mathbb Q}
\def\Z{\mathbb Z}

\def\cC{\mathcal C}

\def\e{\epsilon}
\def\d{\delta}

\begin{document}


\begin{center}
{\large\bfseries
Math 445 --- Problem Set \#4 \\
Due: Friday, September 29 by 7 pm, on Canvas}
\end{center}





{\bf Instructions:} You are encouraged to work together on these
problems, but each student should hand in their own final draft,
written in a way that indicates their individual understanding of
the solutions. Never submit something for grading
that you do not completely understand. If you do work with others, I ask that you write something along the
top like ``I collaborated with Steven Smale on problems 1 and 3''.
If you use a reference, indicate so clearly in your solutions. 
In short, be intellectually
honest at all times. Please write neatly, using complete sentences and correct
punctuation. Label the problems clearly. 






\begin{enumerate}

\item Use quadratic reciprocity and its variants to determine if each of the following is a square modulo $257$ (which is prime):
\begin{itemize}
\item $-2$
\item $59$
\item $53$
\end{itemize}


\

\item The number $p=892,371,481=1+8\cdot 3\cdot 5\cdot 7 \cdot 11 \cdot 13 \cdot 17 \cdot 19 \cdot 23$ is prime. (You do not need to check this.) Show that $\left(\frac{n}{p}\right) = 1$ for $0< n < 29$. Deduce that there is no primitive root $[n]$ in $\Z_p$ with $0<n<29$.

\


\item Show that if $p$ is an odd prime, then $5$ is a square modulo $p$ if and only if $p\equiv \pm 1 \pmod{5}$.

\

\item Use Gauss' Lemma to prove that if $p\equiv 7 \pmod 8$, then $2$ is a quadratic residue modulo $p$. (This is the $p\equiv -1\pmod 8$ case of QR part 2.)

\

\item Explicit square roots modulo some primes:
\begin{enumerate}
\item Show that\footnote{Hint: Use Euler's criterion} if $p\equiv 3 \pmod{4}$ and $a$ is a quadratic residue modulo $p$, then $a^{(p+1)/4}$ is a square root of $a$ modulo $p$.
\item Show that if $p\equiv 5 \pmod{8}$ and $a$ is a quadratic residue modulo $p$, then either $a^{(p+3)/8}$ or $(2a) (4a)^{(p-5)/8}$ is a square root of $a$ modulo $p$.
\item Use parts (a) and (b) to find square roots of $[13]_{23}$ and $[6]_{29}$.
\end{enumerate}


\end{enumerate}

\noindent  \hrulefill

\noindent The remaining problem is only required for Math 845 students, though all are encouraged to think about them.

\

\begin{enumerate}\setcounter{enumi}{5}

\item The $n$th \textbf{Fermat number} is given by $F_n = 2^{2^n} +1$. The first four Fermat numbers are prime; Fermat thought $F_5=2^{2^5} +1=4294967297$ was too, but about a hundred years later, Euler factored it as a product of two primes $641 \cdot  6700417$. In this problem, we will prove \textbf{P\'epin's test}: For $n>0$, $F_n$ is prime if and only if $\displaystyle 3^{\frac{F_n-1}{2}} \equiv -1 \pmod{F_n}$.
\begin{enumerate}
\item Show\footnote{Hint: Apply Euler's criterion and QR.}  that if $F_n$ is prime, then $3^{\frac{F_n-1}{2}} \equiv -1 \pmod{F_n}$.
\item Show\footnote{Hint: Let $p$ be a prime factor of $F_n$, which necessarily is odd. Show that the order of $[3]$ in $\Z_p^\times$ is exactly $F_n-1$.} that if $3^{\frac{F_n-1}{2}}  \equiv -1 \pmod{F_n}$ then $F_n$ is prime.
\item Use P\'epin's test to verify that $F_3$ is prime.
\end{enumerate}






\end{enumerate}

\end{document}

























\end{document}