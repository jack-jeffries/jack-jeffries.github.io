\documentclass{amsart}
\usepackage{amstext,amsfonts,amssymb,amscd,amsbsy,amsmath}
\usepackage{geometry}[margin=1in]
\pagestyle{empty}
%\parindent = 0in


\def\sol#1{{\bf Solution: } #1}
%\def\sol#1{}
\def\bl{\vskip .1in}
\def\star{${}^*$}
\def\cS{\mathcal S}
\def\cT{\mathcal T}
\def\cB{\mathcal B}

\def\R{\mathbb R}
\def\N{\mathbb N}
\def\Q{\mathbb Q}
\def\Z{\mathbb Z}

\def\cC{\mathcal C}

\def\e{\epsilon}
\def\d{\delta}

\begin{document}



\begin{center}
{\large\bfseries
Math 445 --- Problem Set \#1 \\
Due: Friday, September 1 by 7 pm, on Canvas}
\end{center}





{\bf Instructions:} You are encouraged to work together on these
problems, but each student should hand in their own final draft,
written in a way that indicates their individual understanding of
the solutions. Never submit something for grading
that you do not completely understand. 

If you do work with others, I ask that you write something along the
top like ``I collaborated with Steven Smale on problems 1 and 3''.
If you use a reference, indicate so clearly in your solutions. 
In short, be intellectually
honest at all times.

Please write neatly, using complete sentences and correct
punctuation. Label the problems clearly. 






\begin{enumerate}

\item Which of the following are true?
\begin{enumerate}
\item $10 \equiv 45 \pmod 5$
\item $19 \equiv 2 \pmod{12}$
\item $150974 \equiv 6 \pmod 8$.
\end{enumerate}


\


\item Let $m,m',n,n',K$ be integers with $K>0$. Prove that if
\[ m \equiv m' \pmod K \ \ \text{and}\  \ n \equiv n' \pmod K\]
then
\[ m+n \equiv m'+n' \pmod K \ \ \text{and} \ \ mn \equiv m'n'\pmod K.\]

\

%\item Find an integer $N$ such that
%\[ 1! + 2! + 3! +4! + 5! + 6! + 7! + \cdots + 100! \equiv N \pmod 5.\]
%Explain.

%\



\item Divisibility tests and congruences:
\begin{enumerate}
\item Show that any natural number is congruent modulo $4$ to the two digit number (in base ten) that corresponds to its last two digits. Use this to show that a number is divisible by $4$ if and only if its last ``two digit part'' is divisible by $4$.
\item Show that any natural number is congruent modulo $8$ to the three digit number (in base ten) that corresponds to its last three digits. Use\footnote{The step from the first sentence to the second sentence is similar to that in part (a); once you are convinced of this, you can just say this instead of repeating the argument.} this to show that a number is divisible by $8$ if and only if its ``last three digit part'' is divisible by $8$.
\item Show\footnote{Hint: Start by showing that $10^k \equiv 1 \pmod 3$ for any $k$.} that any natural number is congruent modulo $3$ to the sum of its digits. Use this to show that a number is divisible by $3$ if and only the sum of its digits is divisible by $3$.
\item Show that any natural number is congruent modulo $9$ to the sum of its digits. Use this to show that a number is divisible by $9$ if and only the sum of its digits is divisible by~$9$.
\item Show that any natural number is congruent modulo $11$ to the alternating sum of its digits, i.e.
\[ \text{1s digit} - \text{10s digit} + \text{100s digit} \pm \cdots.\]
Use this to show that a number is divisible by $11$ if and only the alternating sum of its digits is divisible by $11$.
\end{enumerate}

\newpage


\item The number $150974$ is a sum of three squares:
\[ 362^2 + 141^2 +7^2 =150974.\]
In this problem we will show that $150975$ is \emph{not} a sum of three squares; i.e., there are no integers $a,b,c$ such that
\[ a^2 + b^2 + c^2 = 150975.\]
\begin{enumerate}
\item Show that if $a$ is odd, then $a^2\equiv 1 \pmod 8$.
\item Show\footnote{Hint: Every even number is congreunt to $0$ mod $4$ or to $2$ mod $4$.} that if $a$ is even, then either $a^2 \equiv 0 \pmod 8$ or $a^2 \equiv 4 \pmod 8$.
\item Show that if $n=a^2+b^2+c^2$, then $n \equiv 7 \pmod 8$ is impossible.
\item Conclude that $150975$ is {not} a sum of three squares.
\end{enumerate}

\

\item  Let $a,b,c$ be integers. Use prime factorization to show that if $a$ and $b$ have no common prime factor and $a$ divides $bc$, then $a$ divides $c$.

\end{enumerate}

\noindent  \hrulefill

\noindent The remaining problems are only required for Math 845 students, though all are encouraged to think about them.

\

\begin{enumerate}\setcounter{enumi}{5}



\item Recall that the Fibonacci sequence is given by the formula
\[ f_{n+2} = f_{n+1} + f_n, \ f_0=f_1=1.\]
For which $n$ is $f_n$ a multiple of $2$? A multiple of $4$? A multiple of $5$?

\

\item Find a formula for all of the rational points $(x,y)$ on the hyperbola $x^2-2y^2=1$. 

\end{enumerate}

\end{document}

























\end{document}