\documentclass{amsart}
\usepackage{amstext,amsfonts,amssymb,amscd,amsbsy,amsmath,framed}
\usepackage{geometry}[margin=1in]
\pagestyle{empty}
%\parindent = 0in


\def\sol#1{{\bf Solution: } #1}
%\def\sol#1{}
\def\bl{\vskip .1in}
\def\star{${}^*$}
\def\cS{\mathcal S}
\def\cT{\mathcal T}
\def\cB{\mathcal B}

\def\R{\mathbb R}
\def\N{\mathbb N}
\def\Q{\mathbb Q}
\def\Z{\mathbb Z}

\def\cC{\mathcal C}

\def\e{\epsilon}
\def\d{\delta}

\begin{document}



\begin{center}
{\large\bfseries
Math 445 --- Problem Set \#2 \\
Due: Friday, September 8 by 7 pm, on Canvas}
\end{center}





{\bf Instructions:} You are encouraged to work together on these
problems, but each student should hand in their own final draft,
written in a way that indicates their individual understanding of
the solutions. Never submit something for grading
that you do not completely understand. 

If you do work with others, I ask that you write something along the
top like ``I collaborated with Steven Smale on problems 1 and 3''.
If you use a reference, indicate so clearly in your solutions. 
In short, be intellectually
honest at all times.

Please write neatly, using complete sentences and correct
punctuation. Label the problems clearly. 






\begin{enumerate}

\item Let $a,b,c$ be integers. Show that if $a$ and $b$ are coprime, $a$ divides $c$, and $b$ divides $c$, then $ab$ divides $c$.


\begin{framed}
We can write $am+bn=1$ for some $m,n\in \Z$ by the coprime hypothesis. Write $c=ak=b\ell$ for some $k,\ell\in \Z$. Then $k = k(am+bn) = (am)k+bkn= b\ell m + bkn = bt$ for $t=\ell m + kn$ so $c=abt$. (You can also argue using prime factorization.)
\end{framed}

\item Find all solutions to the equation $x^2 + [4]x = [5]$ in $\Z_8$ by trial and error (plugging in all possible values). Use this to find all integer solutions to $x^2+ 4x \equiv 5 \pmod 8$.

\begin{framed}
Plugging in $x= [0], [1], \dots, [7]$ into the left hand side, we get $[5]$ for $x=[1],[3],[5],[7]$.
\end{framed}

\item  Given integers $a_1,\dots,a_m$, the \textbf{greatest common divisor} of $a_1,\dots,a_m$ is the largest integer that divides all of them. 
\begin{enumerate}
%\item Show that $\gcd(a,b,c) = \gcd(a,\gcd(b,c))$.
\item Compute $\gcd(12,18,42)$.
\item Prove or disprove: If $\gcd(a,b,c) = 1$, then some pair of the numbers $a,b,c$ is coprime.
\end{enumerate}

\begin{framed}
\begin{enumerate}
\item Taking prime factorizations, $12 = 2^2 \cdot 3$, $18 = 2 \cdot 3^2$, $42 = 2 \cdot 3 \cdot 7$. Thus $2$ is a common divisor, and no larger number can be, so it is the GCD. 
\item This is false: for example, we can take $a=6$, $b=10$, $c=15$.
\end{enumerate}
\end{framed}

\item \emph{Use the methods we have developed in class} to solve the following:
\begin{enumerate}
\item Find all integer pairs $(x,y)$ such that $275x-126y=9$.
\item Find the inverse of $[126]$ in $\Z_{275}$.
\item Find the smallest positive integer $x$ such that 
\[ x\equiv 7 \pmod{126} \quad\text{and}\quad x\equiv 8 \pmod{275}.\]
\end{enumerate}


\begin{framed}
\begin{enumerate}
\item To see if there is a solution, and to find a particular solution if so, we start by using the Euclidean algorithm to find the GCD of $275$ and $126$.
\[\begin{aligned} 
 275 &= 2 \cdot 126 + 23 \\
 126 &= 5 \cdot 23 + 11 \\
 23 &= 2 \cdot 11 + 1
\end{aligned} \]
so the GCD is one, and
\[\begin{aligned} 
 23 &= 1\cdot 275 -  2 \cdot 126 \\
 11 &= 1 \cdot 126 - 5 \cdot 23 = -5 \cdot 275 + 11 \cdot 126 \\
1 &=  1 \cdot 23 - 2 \cdot 11 = 11 \cdot 275 - 24 \cdot 126
\end{aligned} \]
so 
\[ 9 = (9\cdot 11) \cdot 275 - (9\cdot 24) \cdot 126\]
yielding particular solution $(x,y) = (99,-216)$.
Then the general solution is of the form
\[ (x,y) = (99-126n,-216+275n) \quad n\in \Z.\]
\item From the equation $1  = 11 \cdot 275 - 24 \cdot 126$, an evident inverse is $[-24]$.
While we're at it, an inverse for $275$ modulo $126$ is $11$.
\item For a particular solution, we use the formula $x = 7*126*(-24) + 8*275*11 = 3032$. Every solution is of the form $3032+ 126*275n$ for $n\in \Z$. Since $0\leq 3032< 34650=126*275$, we must have the smallest positive solution.
\end{enumerate}
\end{framed}

\item  Solving linear equations in $\Z_n$: Let $a,b,n$ be integers, with $n>0$.
\begin{enumerate}
\item Show that $[a] x = [b]$ has a solution $x$ in $\Z_n$ if and only if $\gcd(a,n)$ divides $b$.
\item Show that if $[a] x = [b]$ has a solution $x$ in $\Z_n$, then there are exactly $\gcd(a,n)$ distinct solutions.
\item Solve the equation $[20] [x] + [17] = [29]$ in $\Z_{36}$.
\end{enumerate}

\begin{framed}
\begin{enumerate}
\item We have that $x=[k]$ is a solution to $[a] x = [b]$ if and only if $ak \equiv b \pmod{n}$. This is equivalent to $ak - b = n \ell$ for some $\ell\in \Z$, which we can rewrite as $ak + (-n)\ell = b$. From our theorem on linear diophantine equations, there exist $k,\ell$ that solve this if and only if $\gcd(a,n)$ divides $b$.
\item Set $d=\gcd(a,n)$. Suppose that $ak \equiv b \pmod{n}$ has a solution. As above, $k$ is a solution if and only there is some $\ell\in \Z$ such that $ak + (-n)\ell = b$. The general solution is of the form $(k,\ell) = (k_0 + n/d w, \ell_0 - a/d w)$ for some particular solution $(k_0,\ell_0)$ and $w\in \Z$. We claim that the integers of the form $k_0 + n/d w$ for $w\in \Z$ form exactly $d$ congruence classes modulo $n$, namely $[k_0], [k_0 + n/d] ,\dots, [k_0 + (d-1)\frac{n}{d}]$. Indeed, we can write $w = v d + u$ with $0\leq u < d$, and so 
\[ k_0+ w n/d = k_0 + (v d + u) n/d = k_0 + u n/d + vn \equiv k_0 + u n/d \pmod{n},\]
showing that each such integer is in one of these congruence classes. A similar argument shows that these classes are distinct. Thus, there are exactly $d$ solutions.
\item First, rewrite as $[20][x] = [12]$. As above, we rewrite as $20 x +36 y = 12$. 
We use the Euclidean algorithm to find the GCD of $20$ and $36$ and linear combination
\[ 2 \cdot 20 - 1 \cdot 36 = 4.\]
Multiplying by $3$ gives a particular solution:
\[ 6 \cdot 20 - 3 \cdot 36 = 12,\]
and for the general solution we have
\[ (x,y) = (6 + 9n, -3 - 5n) , \quad n\in \Z.\]
Then, following the proof above, we get the four solutions
\[ [6], [6+9]=[15], [6+18]=[24], [6+27] = [33].\]
\end{enumerate}
\end{framed}


\end{enumerate}

\noindent  \hrulefill

\noindent The remaining problems are only required for Math 845 students, though all are encouraged to think about them.

\

\begin{enumerate}\setcounter{enumi}{5}



\item Solve the equation $8x + 25y + 15z = 19$ over $\Z$.

\begin{framed} First, take the change of variables $x=u-3y$, so $u=x-3y$:
\[ 8(u-3y) +  25y +15 z = 19\]
\[ 8u + y +15 z = 19.\]
Then we can express $y$ in terms of $u,z$:
\[ y= 19-8u -15 z\]
\[ (u,y,z) = (u, 19-8u -15 z, z).\]
Then we rewrite in $x,y,z$-coordinates:
\[ (x,y,z) = (u-3y, 19-8u -15 z, z) = (-57 + 23 u +45 z , 19-8u -15 z, z), \quad u,z\in \Z.\]
\end{framed}

\end{enumerate}

\end{document}

























\end{document}