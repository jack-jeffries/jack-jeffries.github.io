\documentclass[12pt]{amsart}


\usepackage{times}
\usepackage[margin=1in]{geometry}
\usepackage{amsmath,amssymb,multicol,graphicx,framed,ifthen,color,xcolor,enumerate,colonequals}
\definecolor{chianti}{rgb}{0.6,0,0}
\definecolor{meretale}{rgb}{0,0,.6}
\definecolor{leaf}{rgb}{0,.35,0}
\newcommand{\blue}{\color {blue}}
\newcommand{\red}{\color {red}}
\newcommand{\olive}{\color {olive}}
\newcommand{\violet}{\color {violet}}
\newcommand{\orange}{\color{orange}}
\newcommand{\Q}{\mathbb{Q}}
\newcommand{\N}{\mathbb{N}}
\newcommand{\Z}{\mathbb{Z}}
\newcommand{\R}{\mathbb{R}}
\newcommand{\C}{\mathbb{C}}
\newcommand{\e}{\varepsilon}
\newcommand{\inv}{^{-1}}
\newcommand{\dabs}[1]{\left| #1 \right|}
\newcommand{\ds}{\displaystyle}
\newcommand{\solution}[1]{\ifthenelse {\equal{\displaysol}{1}} {\begin{framed}{\color{meretale}\noindent #1}\end{framed}} { \ }}
\newcommand{\showsol}[1]{\def\displaysol{#1}}
\newcommand{\rsa}{\rightsquigarrow}
\newcommand{\ceq}{\colonequals}

\DeclareMathOperator{\res}{res}

%\usepackage{times}

%\addtolength{\textwidth}{100pt}
%\addtolength{\evensidemargin}{-45pt}
%\addtolength{\oddsidemargin}{-60pt}

\pagestyle{empty}
%\begin{document}\begin{itemize}

%\thispagestyle{empty}




\begin{document}
\showsol{0}
	
	\thispagestyle{empty}
	
	\section*{Quadratic residues}
	
	
\begin{framed}
\noindent \textsc{Definition:} We say that an element $x\in \Z_n$ is a \textbf{square} or a \textbf{quadratic residue} if there is some $y\in \Z_n$ such that $y^2=x$, and in this case, we call $y$ a \textbf{square root} of $x$. %Likewise, we say that a integer $a$ is a \textbf{square} or a \textbf{quadratic residue} modulo $n$ if there is some integer $b$ such that $a^2 \equiv b \pmod n$ and call $b$ a \textbf{square root} of $a$ modulo $n$.
\end{framed}

\begin{enumerate}
\item Let $n$ be an odd positive integer. Suppose that $[a]$ is a unit in $\Z_n$. Show that\footnote{Hint: Complete the square!} the solutions $x$ to the equation $ [a] x^2 + [b] x + [c] = [0]$ in $\Z_n$ are exactly the elements of the form
\[ x=\frac{-[b] + u}{[2a]} \quad \text{such that} \ u \ \text{is a square root of} \ [b^2-4ac]. \]


\solution{Since we assumed $[a]$ is a unit, we can rewrite as $x^2 + \frac{[b]}{[a]} x + \frac{[c]}{[a]} = [0]$. Since $n$ is odd, $[2]$ is a unit too, so we can complete the square:
\[\begin{aligned}  [0] &= x^2 + \frac{[b]}{[a]} x + \frac{[c]}{[a]} \\
&= x^2 + [2] \frac{[b]}{[2a]} x + \left(\frac{[b]}{[2a]} \right)^2 -  \left(\frac{[b]}{[2a]} \right)^2 + \frac{[c]}{[a]} \\
&= \left(x+  \frac{[b]}{[2a]} \right)^2  +  \frac{[4ac- b^2]}{[4a^2]},
\end{aligned}\]
so
\[ \left(\frac{[2a] x+ [b]}{[2a]} \right)^2 = \frac{[b^2-4ac]}{[4a^2]}.\]
Thus, $x$ is a solution if and only if $[2a]x + [b]$ is a square root of $[b^2-4ac]$. Rearranging slightly gives the form above.
}



\item Let $p$ be an odd prime and $x \in \Z_p^\times$. Show that if $x$ is a quadratic residue, then $x$ has exactly two square roots $y\neq y'$, and for these roots, $y'=-y$.

\solution{If $y^2 - x = 0$ has a solution, it has at most two since this is a polynomial of degree two over a field. If $y$ is a solution, then $y'=-y$ is too.}

\item\label{indexres} Let $p$ be a prime number and $g$ be a primitive root of $\Z_p$. Show that $[n]\in \Z_p^\times$ is a quadratic residue if and only if the index of $[n]$ with respect to $g$ is even.

\solution{Write $[n] = g^k$, so the index is $k$. If $k=2\ell$ is even, then $[n] = g^k = g^{2\ell} = (g^{\ell})^2$, so $[n]$ is a quadratic residue. Conversely, if $[n]= [m]^2$, write $[m] = g^\ell$, so $[n]=[m]^2 = g^{2 \ell}$, which is even. (Note that even and odd are well-defined in $\Z_{p-1}$ for $p$ odd, since any two representatives differ by a multiple of two.)}

\end{enumerate}
\begin{framed}
\noindent\textsc{Definition:} Let $p$ be an odd prime. For $r\in \Z$ not a multiple of $p$ we define the \textbf{Legendre symbol}  of $r$ with respect to $p$ as 
\[ \left( \frac{r}{p} \right) = \begin{cases} 1 & \text{if} \ [r] \ \text{is a square in} \ \Z_p, \\ -1 & \text{if} \ [r] \ \text{is a not square in} \ \Z_p. \end{cases}\]

\

\noindent\textsc{Theorem (Euler's criterion):} For $p$ an odd prime and $r\in \Z$ not a multiple of $p$, we have
\[\displaystyle \left( \frac{r}{p} \right) \equiv r^{(p-1)/2} \pmod p.\]

\

\noindent\textsc{Theorem (Quadratic Reciprocity part $-1$):} If $p$ is odd, then
\[ \left( \frac{-1}{p} \right) = \begin{cases} 1 &\text{if} \ p\equiv 1 \pmod 4\\ -1 &\text{if} \ p\equiv 3 \pmod 4\end{cases}.\]



\


\noindent \textsc{Proposition:} Let $p$ be an odd prime and $a,b$ integers not divisible by $p$. Then
%\begin{multicols}{2}
\begin{enumerate}
\item $\displaystyle a\equiv b \pmod{p}$ implies that $\displaystyle \left(\frac{a}{p}\right) = \left(\frac{b}{p}\right)$.

\vspace{2mm}

\item $\displaystyle\left(\frac{ab}{p}\right) = \left(\frac{a}{p}\right) \left(\frac{b}{p}\right)$.

\vspace{2mm}

\item $\displaystyle\left(\frac{a^2}{p}\right) = 1$.
\end{enumerate}
%\end{multicols}

\end{framed}

\

\

\begin{enumerate}\setcounter{enumi}{3}
\item \begin{enumerate} 
\item Without using the Proposition above, explain why $\displaystyle\left(\frac{4}{p}\right) = 1$ for $p$ an odd prime. Now explain why part (3) of the Proposition above is true in general.
\item Use the Proposition above to explain the following: If $a,b$ are not squares modulo $p$, then $ab$ is a square modulo $p$.
\item Use\footnote{You might find it convenient to write $168 = 4 \cdot 42$.} the Proposition and Corollary above to determine how many solutions $x$ to \[{[3]x^2 + [12] x - [2] = [0]}\] there are in $\Z_{43}$.
\end{enumerate}

\solution{\begin{enumerate}
\item $[4]=[2]^2$; $[a^2] = [a]^2$.
\item We have $\left( \frac{a}{p} \right) = \left( \frac{b}{p} \right) = -1$, so $\left( \frac{ab}{p} \right) = \left( \frac{a}{p} \right) \left( \frac{b}{p} \right) = (-1)^2 = 1$.
\item Using the quadratic formula, we need to determine whether $[12^2 - 4 \cdot 3 \cdot -2] = [168]$ is a square in $\Z_{43}$. By the hint, we have $168 = 4 \cdot 42$, so \[\left( \frac{168}{43} \right) = \left( \frac{4}{43} \right) \left( \frac{42}{43} \right) = 1 \left( \frac{-1}{43} \right) = 1 \cdot -1 = -1.\]
We conclude that there are no solutions.
\end{enumerate}
}

\item Use problem \#3 to prove Euler's criterion.

\solution{Let $g=[a]$ be a primitive root and write $[r] = g^k$ for some $k$. 

If $[r]$ is a residue, then $k=2\ell$ is even, and $r^{(p-1)/2} \equiv a^{2\ell (p-1)/2 } \equiv a^{\ell (p-1)} \equiv 1 \pmod{p}$ by FLT.

If $[r]$ is not a residue, then $k=2\ell+1$ is odd, and $r^{(p-1)/2} \equiv a^{(2\ell+1) (p-1)/2 } \equiv a^{\ell (p-1) + (p-1)/2} \equiv a^{(p-1)/2} \pmod{p}$ by FLT. We know that $(a^{(p-1)/2})^2 \equiv a^{p-1} \equiv 1 \pmod{p}$ again by FLT, so $a^{(p-1)/2} \equiv \pm 1 \pmod{p}$.
But, by definition of primitive root, $a^{(p-1)/2} \not\equiv 1 \pmod{p}$, so $a^{(p-1)/2} \equiv -1 \pmod{p}$.}

\item Prove the proposition above.

\solution{We already did part (3). Part (1) is clear since the value of $\left( \frac{a}{p} \right)$ only depends on the congruence class of $a$ modulo $p$. For (2), take a primitive root $g=[r]$ and write $a\equiv r^k, b\equiv r^\ell$. Then, by Euler's criterion,
\[ \left(\frac{a}{p}\right) \left(\frac{b}{p}\right) \equiv r^{k \frac{p-1}{2}} r^{\ell \frac{p-1}{2}} \equiv r^{(k +\ell)\frac{p-1}{2}}\equiv \left(\frac{ab}{p}\right) \pmod{p}.\]
}

\item Use Euler's criterion to prove QR part $-1$ above.

\solution{If $p\equiv 1 \pmod{4}$, write $p=4k+1$; then $(-1)^{\frac{p-1}{2}} \equiv (-1)^{2k} \equiv 1$, so $-1$ is a residue by Euler's criterion. If $p\equiv 3 \pmod{4}$, write $p=4k+3$; then $(-1)^{\frac{p-1}{2}} \equiv (-1)^{2k+1} \equiv -1$, so $-1$ is not a residue by Euler's criterion.}

\item When $n$ is not a prime\dots
\begin{enumerate}
\item Does the conclusion of \#4(b) hold if $n$ is replaced by a general positive integer~$n$ instead of a prime $p$?
\item Suppose that $n=pq$ for primes $p\neq q$. Show that $a$ is a quadratic residue modulo~$n$ if and only if $a$ is a quadratic residue modulo~$p$ and a quadratic residue modulo~$q$.
\end{enumerate}


\end{enumerate}









\end{document}
