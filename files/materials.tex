\documentclass{beamer}
\usetheme{AnnArbor}
\setbeamertemplate{page number in head/foot}{}
\usecolortheme{beaver}
\usefonttheme{structuresmallcapsserif}
\usepackage{ragged2e}
\usepackage{etoolbox}
\usepackage{ wasysym }
\usepackage{graphicx}
\usepackage{tikz}


\apptocmd{\frame}{}{\justifying}{}


\usepackage{amsmath,amssymb,enumerate,amsthm,amsfonts,comment,stmaryrd,multicol}
\usepackage{color,xcolor}
\usepackage[all]{xy}


\setbeamercolor{postit}{fg=black,bg=lightgray}

%\newtheorem{theorem}{Theorem}
\newtheorem{question}{Question}
%\theoremstyle{example}
\newtheorem{counterexample}{Counterexample}

\newtheorem{conjecture}{Conjecture}

\newcommand{\m}{\mathfrak{m}}
\newcommand{\CC}{\mathbb{C}}
\newcommand{\red}[1]{{\color {red} #1}}     
\newcommand{\blue}[1]{{\color {blue} #1}}
\newcommand{\orange}[1]{{\color {orange} #1}}
\newcommand{\purple}[1]{{\color {purple} #1}}
\newcommand{\cyan}[1]{{\color {cyan} #1}}     
\newcommand{\magenta}[1]{{\color {magenta} #1}}
\newcommand{\olive}[1]{{\color {olive} #1}}
\newcommand{\violet}[1]{{\color {violet} #1}}
\newcommand{\green}[1]{{\color {green} #1}}
\newcommand{\gray}[1]{{\color {gray} #1}}
\newcommand{\ZZ}{\mathbb{Z}}

\newcommand{\ZN}{\violet{Zariski--Nagata theorem}}
\newcommand{\symbolicPower}{\green{symbolic power}}
\newcommand{\symbolicPowers}{\green{symbolic powers}}
\newcommand{\SymbolicPowers}{\green{Symbolic powers}}
\newcommand{\differentialOperators}{\olive{differential operators}}
\newcommand{\fundamentalGroups}{\red{fundamental groups}}
\newcommand{\pDers}{\cyan{p-derivations}}

\newcommand{\lr}[1]{{\langle {#1} \rangle}}
%Command for differential powers linear over a subring
\newcommand{\dif}[2]{^{\lr{#1}_{#2}}}
%Command for p-derivation powers
\newcommand{\derp}[1]{^{\lr{#1}_{\delta}}}
%Command for mixed differential powers
\newcommand{\difM}[1]{^{\lr{#1}_{\mathrm{mix}}}}


\DeclareMathOperator{\Ass}{Ass}
\DeclareMathOperator{\Ext}{Ext}
\DeclareMathOperator{\HH}{H}
\DeclareMathOperator{\Hom}{Hom}


\begin{document}
\title[Materials for applications]{Intro to materials for academic job/postdoc applications} 
\author{Jack Jeffries}
\date{PASCA 2022}

%\frame{\frametitle{Table of contents}\tableofcontents} 

\frame{\titlepage}

\section{Introduction}

\begin{frame}\frametitle{Different job materials}
\begin{enumerate}
	\item Letters of recommendation
	\item CV
	\item Research Statement
	\item Teaching Statement
	\item Website
	\item Others
\end{enumerate}
\end{frame}

\section{Letters of recommendation}

\begin{frame}\frametitle{Letters of recommendation}

This is probably the most important part of your application!


\


Things to consider in choosing letter writers:
\begin{itemize}
	\item They should be able to describe your work and its importance in detail.
	\item It is good to have some writers who are well-known by other mathematicians / other mathematicians in your research area.\footnote{But it's useless to have a famous person who can't say anything about you!}
	\item If you are applying in Europe or the States, you want to have a letter writer who can compare you to people in Europe or the States.
	\item If you are applying in the States, you want to have someone who can comment on your ability to teach in English.
\end{itemize}

\end{frame}

\section{Letters of recommendation}

\begin{frame}\frametitle{Letters of recommendation}

How to choose letter writers?

\

Beyond the points mentioned before, you should ask your advisor's advice on whom to have write for you. 


\


People who may write for you are often busy people! If you are asking someone to write for you for the first time, I recommend asking at least \emph{six weeks} before the deadline. Even if you do this, there is no guarantee that they will agree to write for you!

\end{frame}

\section{CV}

\begin{frame}\frametitle{What goes in a CV?}

\begin{itemize}
	\item Professional history
	\item Educational history
	\item Contact information
	\item Papers
	\item Grants/Fellowships
	\item Teaching experience
	\item Service/organization
	\item Talks/posters
	\item Conferences attended
	\item Skills/languages
\end{itemize}

\end{frame}

\begin{frame}\frametitle{What goes in a CV?}

\begin{itemize}
	\item Professional history
	\begin{itemize}
		\item Where you have worked (math-related jobs: postdocs, teaching assistantship, course assistant, etc.)
	\end{itemize}
	\item Educational history
	\item Contact information
	\item Papers
	\item Grants/Fellowships
	\item Teaching experience
	\item Service/organization
	\item Talks/posters
	\item Conferences attended
	\item Skills/languages
\end{itemize}

\end{frame}

\begin{frame}\frametitle{What goes in a CV?}

\begin{itemize}
	\item Professional history
	\item Educational history
	\begin{itemize}
			\item Where you studied (licensatura, masters, Ph.D)
	\end{itemize}
	\item Contact information
	\item Papers
	\item Grants/Fellowships
	\item Teaching experience
	\item Service/organization
	\item Talks/posters
	\item Conferences attended
	\item Skills/languages
\end{itemize}

\end{frame}


\begin{frame}\frametitle{What goes in a CV?}

\begin{itemize}
	\item Professional history
	\item Educational history
	\item Contact information
	\begin{itemize}
		\item Mailing address, email, website
	\end{itemize}
	\item Papers
	\item Grants/Fellowships
	\item Teaching experience
	\item Service/organization
	\item Talks/posters
	\item Conferences attended
	\item Skills/languages
\end{itemize}

\end{frame}

\begin{frame}\frametitle{What goes in a CV?}

\begin{itemize}
	\item Professional history
	\item Educational history
	\item Contact information
	\item Papers	
	\begin{itemize}
		\item List accepted/submitted papers\footnote{Personal pet peeve: ``in preparation'' papers are NOT papers}, licensatura thesis, masters thesis, Ph.D. thesis
	\end{itemize}
	\item Grants/Fellowships
	\item Teaching experience
	\item Service/organization
	\item Talks/posters
	\item Conferences attended
	\item Skills/languages
\end{itemize}

\end{frame}

\begin{frame}\frametitle{What goes in a CV?}

\begin{itemize}
	\item Professional history
	\item Educational history
	\item Contact information
	\item Papers
	\item Grants/Fellowships
	\begin{itemize}
	\item Where you studied (licensatura, masters, Ph.D)
\end{itemize}
	\item Teaching experience
	\item Service/organization
	\item Talks/posters
	\item Conferences attended
	\item Skills/languages
\end{itemize}

\end{frame}

\begin{frame}\frametitle{What goes in a CV?}

\begin{itemize}
	\item Professional history
	\item Educational history
	\item Contact information
	\item Papers
	\item Grants/Fellowships
	\item Teaching experience
	\item Service/organization
	\begin{itemize}
		\item Papers refereed, conferences/seminars/math events organized
	\end{itemize}
	\item Talks/posters
	\item Conferences attended
	\item Skills/languages
\end{itemize}

\end{frame}

\begin{frame}\frametitle{What goes in a CV?}

\begin{itemize}
	\item Professional history
	\item Educational history
	\item Contact information
	\item Papers
	\item Grants/Fellowships
	\item Teaching experience
	\item Service/organization
	\item Talks/posters
	\begin{itemize}
	\item For students, you can include talks at your institution. Later on, only include talks outside of your institution.
\end{itemize}
	\item Conferences attended
	\item Skills/languages
\end{itemize}

\end{frame}


\begin{frame}\frametitle{What goes in a CV?}

\begin{itemize}
	\item Professional history
	\item Educational history
	\item Contact information
	\item Papers
	\item Grants/Fellowships
	\item Teaching experience
	\item Service/organization
	\item Talks/posters
	\item Conferences attended
	\begin{itemize}
		\item Do not include after you graduate and/or you have enough talks, papers, etc.
	\end{itemize}
	\item Skills/languages
\end{itemize}

\end{frame}

\begin{frame}\frametitle{What goes in a CV?}

\begin{itemize}
	\item Professional history
	\item Educational history
	\item Contact information
	\item Papers
	\item Grants/Fellowships
	\item Teaching experience
	\item Service/organization
	\item Talks/posters
	\item Conferences attended
	\item Skills/languages
		\begin{itemize}
		\item Languages, programming languages, LaTeX if applicable
		\item Do not include after you graduate and/or you have enough talks, papers, etc.
	\end{itemize}
\end{itemize}

\end{frame}

\section{Research Statement}

\begin{frame}\frametitle{Research Statement}

\begin{itemize}
	\item Target length: 4--5 pages (not including references)
	\item The first page or so should be accessible to people not familiar with your research area.
	\item You should emphasize your results (ideally in \textbf{Theorem} format).
	\item You should explain the backstory and context of your results; why are they important?
	\item Include lots of references.
\end{itemize}

\end{frame}

\section{Teaching Statement}

\begin{frame}\frametitle{Teaching Statement}

\begin{itemize}
	\item Target length: 1--2 pages
	\item Usually doesn't have references
	\item Describe your past teaching experience
	\item Include particular things in class that you did as a teacher that you think were effective
	\item I don't recommend making strong philosophical claims about teaching that somebody could disagree with.
\end{itemize}

\end{frame}

\section{Website}

\begin{frame}\frametitle{Website}

You (yes, you, no matter your career stage) should make an academic website.

\

People might meet you at a conference or workshop and want to learn more about what you are working on; if you have a website, they can do this more easily.

\

Contact your department's IT support or use an independent web host to help you get started.

\

If you don't know HTML, you can copy and modify someone's source code to make your own HTML file, in particular, another mathematician's.

\end{frame}

\begin{frame}\frametitle{Website}

Things to put on your website:

\begin{itemize}
	\item Basic info: Name, institution, job title, email address
	\item Research information: Research interests, papers
	\item Teaching information: Classes taught
	\item CV
	\item A recognizable photo (if you are comfortable with having your image online)
\end{itemize}

\end{frame}




\end{document}
