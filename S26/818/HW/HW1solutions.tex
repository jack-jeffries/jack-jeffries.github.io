\documentclass[11pt]{article}
\usepackage[margin=1in]{geometry}
\usepackage{amsmath,amsfonts,amssymb,amsthm,enumerate}
\usepackage[]{graphicx}
\usepackage{color,subfigure}
\definecolor{scarlet}{rgb}{0.81,0,0}
\usepackage{multicol}
\usepackage{float}
\usepackage[all]{xypic}
\usepackage[colorlinks=true,citecolor=scarlet,linkcolor=scarlet]{hyperref}
\usepackage{colonequals}

\usepackage{fancyhdr, lastpage}
\pagestyle{fancy}
\fancyfoot[C]{{\thepage} of \pageref{LastPage}}



\DeclareMathOperator{\mSpec}{mSpec}
\DeclareMathOperator{\Spec}{Spec}
\DeclareMathOperator{\Ass}{Ass}
\DeclareMathOperator{\Supp}{Supp}
\DeclareMathOperator{\height}{height}
\DeclareMathOperator{\Hom}{Hom}
\DeclareMathOperator{\ann}{ann}
\DeclareMathOperator{\End}{End}
\DeclareMathOperator{\coker}{coker}
%\DeclareMathOperator{\ker}{ker}
\DeclareMathOperator{\rank}{rank}
\DeclareMathOperator{\im}{im}
\DeclareMathOperator{\M}{M}
\DeclareMathOperator{\Tor}{Tor}
\DeclareMathOperator{\id}{id}
\DeclareMathOperator{\ch}{char}
\DeclareMathOperator{\Aut}{Aut}

%\DeclareMathOperator{\dim}{dim}

\DeclareMathOperator{\lcm}{lcm}

\def\ra{\rightarrow}
\newcommand{\m}{\mathfrak{m}}
\newcommand{\C}{\mathbb{C}}
\newcommand{\Q}{\mathbb{Q}}
\newcommand{\Z}{\mathbb{Z}}
\newcommand{\ZZ}{\mathbb{Z}}
\newcommand{\R}{\mathbb{R}}
\newcommand{\N}{\mathbb{N}}
\newcommand{\ov}[1]{\overline{#1}}
\newcommand{\norm}{\trianglelefteq}

\def\ov#1{\overline{#1}}


\title{}
\date{\vspace{-0.5in}}

\makeatletter
\g@addto@macro\@floatboxreset\centering
\makeatother

\theoremstyle{definition}
\newtheorem{problem}{Problem}


\begin{document}

\thispagestyle{fancy}
\pagestyle{fancy}
\rhead{UNL $\mid$ Spring 2026}
\lhead{Introduction to Modern Algebra II}

\vspace{3em}

\begin{center}
	{\LARGE Problem Set 1 \\}
	Due Thursday, January 22
\end{center}

\

\noindent
{\bf Instructions:}
You are encouraged to work together on these problems, but each student should hand in their own final draft, written in a way that indicates their individual understanding of the solutions. Never submit something for grading that you do not completely understand. You cannot use any resources besides me, your classmates, and our course notes.


I will post the .tex code for these problems for you to use if you wish to type your homework. If you prefer not to type, please  {\em write neatly}. As a matter of good proof writing style, please use complete sentences and correct grammar. You may use any result stated or proven in class or in a homework problem, provided you reference it appropriately by either stating the result or stating its name (e.g. the definition of ring or Lagrange's Theorem). Please do not refer to theorems by their number in the course notes, as that can change.


\smallskip

\begin{problem} Refresh your knowledge of Gauss' Lemma, and read the short Section 10.2 from the Math 817 lecture notes on Eisenstein's Criterion. Then
 \begin{enumerate}
\item[(a)] Prove that the polynomial $x^2 + y^2 -1$ is irreducible in $\Q[x,y]$.

\begin{proof}
Recall that $\Q[x,y]=\Q[x][y]$. Consider the prime ideal $P=(x-1) \subseteq \Q[x]$; this is prime since $x-1$ is irreducible and $\Q[x]$ is a UFD. Note that $x^2-1=(x-1)(x+1) \in P$ since $x-1 \ | \ x^2-1$ but $x^2-1\notin P^2$ since $(x-1)^2 \ | \ x^2-1$. Since $0\in P$, and the leading coefficient of $x^2 + y^2 -1$ as a polynomial in $y$ is $1$, the hypotheses of Eisenstein's criterion apply, so this polynomial is irreducible.
\end{proof}

  \item[(b)] Prove\footnote{Hint: Consider this polynomial in $\Z[x]$ and go modulo $2$.} that $5x^4 + 7x^3 + 11x^2 +6x + 1$ is irreducible in $\Q[x]$.
  
  \begin{proof}
By Gauss' Lemma, it suffices to show that $f(x)=5x^4 + 7x^3 + 11x^2 +6x + 1$ is irreducible in $\Z[x]$. Suppose that $f=gh$, with $g,h\in \Z[x]$; we want to show that either $g$ or $h$ is a unit. Consider the surjective homomorphism $\pi: \Z[x] \to \Z/2[x]$ given by reducing the coefficients modulo $2$. We have
\[ \pi(f) = x^4 + x^3 + x^2 +1 \o


\end{proof}

  \item[(c)] Let $p$ be a prime number. Prove\footnote{Hint: Consider $f(x+1)$. You can use without proof that the binomial theorem holds in any commutative ring~$R$: for any $A,B\in R$ and positive integer $n$, \[ (A+B)^n = \sum_{i=0}^n \binom{n}{i} A^i B^{n-i}, \qquad \text{where} \ \binom{n}{i} = \frac{n!}{i!(n-i)!} \in \Z_{\geq 0}.\]}
  that the polynomial $f(x) = x^{p-1} + x^{p-2} + \cdots  + x + 1$ is irreducible in $\Q[x]$.
\end{enumerate}
\end{problem}

\


\begin{problem} Let $R$ be a ring and let $M$ be a left $R$-module. The \textbf{annihilator} of $M$ is 
\[ \mathrm{ann}_R(M) \colonequals \{r\in R \ | \ rm=0 \ \text{for all} \ m\in M\}.\] Show that $\mathrm{ann}_R(M)$ is a two-sided ideal of $R$.
\end{problem}

\

\begin{problem} Let $R$ be a ring and $M$ be a left $R$-module.
\begin{enumerate}
\item[(a)] Let $I$ be a left ideal, and define 
\[ IM\colonequals \{ \sum_i a_i m_i \ | \ a_i \in I, m_i\in M\}.\] Show that $IM$ is a submodule of $M$.
\item[(b)] Show that if $I$ is a two-sided ideal and $IM=0$, then $M$ is a left $R/I$-module by the rule $(r+I) m \colonequals rm$.
\item[(c)] Show that if $I$ is a two-sided ideal, then $M/IM$ is an $R/I$-module.
\end{enumerate}
\end{problem}


\newpage

\begin{problem} Let $R$ be a commutative ring, $M$ a module, and $I$ an ideal.
\begin{enumerate}
\item[(a)] Prove that ${}_{I} M:=\{ m\in M \ | \ am=0 \ \text{for all} \ a\in I\}$ is a submodule of $M$.
\item[(b)] Prove that $\Hom_R(R/I,M)\cong {}_{I} M$.
\end{enumerate}
\end{problem}



\end{document}