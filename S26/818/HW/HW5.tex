\documentclass[11pt]{article}
\usepackage[margin=1in]{geometry}
\usepackage{amsmath,amsfonts,amssymb,amsthm,enumerate,bbm}
\usepackage[]{graphicx}
\usepackage{color,subfigure}
\definecolor{scarlet}{rgb}{0.81,0,0}
\usepackage{multicol}
\usepackage{float}
\usepackage[all]{xypic}
\usepackage[colorlinks=true,citecolor=scarlet,linkcolor=scarlet]{hyperref}
\usepackage{colonequals}

\usepackage{fancyhdr, lastpage}
\pagestyle{fancy}
\fancyfoot[C]{{\thepage} of \pageref{LastPage}}



\DeclareMathOperator{\mSpec}{mSpec}
\DeclareMathOperator{\Spec}{Spec}
\DeclareMathOperator{\Ass}{Ass}
\DeclareMathOperator{\Supp}{Supp}
\DeclareMathOperator{\height}{height}
\DeclareMathOperator{\Hom}{Hom}
\DeclareMathOperator{\ann}{ann}
\DeclareMathOperator{\End}{End}
\DeclareMathOperator{\coker}{coker}
%\DeclareMathOperator{\ker}{ker}
\DeclareMathOperator{\rank}{rank}
\DeclareMathOperator{\im}{im}
\DeclareMathOperator{\M}{M}
\DeclareMathOperator{\Tor}{Tor}
\DeclareMathOperator{\id}{id}
\DeclareMathOperator{\ch}{char}
\DeclareMathOperator{\Aut}{Aut}

%\DeclareMathOperator{\dim}{dim}

\DeclareMathOperator{\lcm}{lcm}

\def\ra{\rightarrow}
\newcommand{\m}{\mathfrak{m}}
\newcommand{\C}{\mathbb{C}}
\newcommand{\Q}{\mathbb{Q}}
\newcommand{\Z}{\mathbb{Z}}
\newcommand{\ZZ}{\mathbb{Z}}
\newcommand{\R}{\mathbb{R}}
\newcommand{\N}{\mathbb{N}}
\newcommand{\ov}[1]{\overline{#1}}
\newcommand{\norm}{\trianglelefteq}

\def\ov#1{\overline{#1}}


\title{}
\date{\vspace{-0.5in}}

\makeatletter
\g@addto@macro\@floatboxreset\centering
\makeatother

\theoremstyle{definition}
\newtheorem{problem}{Problem}


\begin{document}

\thispagestyle{fancy}
\pagestyle{fancy}
\rhead{UNL $\mid$ Spring 2026}
\lhead{Introduction to Modern Algebra II}

\vspace{3em}

\begin{center}
	{\LARGE Problem Set 5 \\}
	Due Thursday, February 19
\end{center}

\

\noindent
{\bf Instructions:}
You are encouraged to work together on these problems, but each student should hand in their own final draft, written in a way that indicates their individual understanding of the solutions. Never submit something for grading that you do not completely understand. You cannot use any resources besides me, your classmates, and our course notes.


I will post the .tex code for these problems for you to use if you wish to type your homework. If you prefer not to type, please  {\em write neatly}. As a matter of good proof writing style, please use complete sentences and correct grammar. You may use any result stated or proven in class or in a homework problem, provided you reference it appropriately by either stating the result or stating its name (e.g. the definition of ring or Lagrange's Theorem). Please do not refer to theorems by their number in the course notes, as that can change.


\smallskip

\begin{problem} Let $R$ be a commutative ring. Let $D=[d_{ij}] \in \mathrm{Mat}_{m\times n}(R)$. Suppose that $d_{11},\dots,d_{tt}\neq 0$ for some $t\leq \min\{m,n\}$, and that every other entry of $D$ is zero. Let $M$ be the module presented by $D$.
\begin{enumerate}[(a)]
\item Show that $M$ is isomorphic to
\[ R^{m-t} \oplus R/(d_{11}) \oplus \cdots \oplus R/(d_{tt}).\]
 \item Suppose that $d_{11} \ | \ \cdots \ | \ d_{tt}$. Give a formula for the annihilator of $M$ in terms of $m,n,t$ and the entries $d_{ii}$.
 \end{enumerate}
 \end{problem}
 
\

\begin{problem}
Let 
\[ A = \begin{bmatrix}75&-51&5&-26\\
-112&74&8&36\\
-36&24&0&12\end{bmatrix} \in \mathrm{Mat}_{3\times 4}(\Z).\]
Let $t_A$ be the $\Z$-module homomorphism $v\mapsto Av$.
 The Smith Normal Form is given by $A=PDQ$ where
\[ P= \begin{bmatrix}
1&1&-2\\
0&1&3\\
0&0&1
\end{bmatrix}, \ D=\begin{bmatrix}
1&0&0&0\\
0&2&0&0\\
0&0&12&0
\end{bmatrix}, \ \text{and} \ Q=\begin{bmatrix}
7&-5&-3&-2\\
-2&1&4&0\\
-3&2&0&1\\
0&0&1&0
\end{bmatrix}.\]
The inverses of $P$ and $Q$ are given by
\[ P^{-1} = \begin{bmatrix}
1&-1&5\\
0&1&-3\\
0&0&1
\end{bmatrix} \  \text{and} \  Q^{-1} = \begin{bmatrix} -1&-1&-2&1\\
-2&-1&-4&-2\\
0&0&0&1\\
1&-1&3&7
\end{bmatrix}.\]
\end{problem}
\begin{enumerate}[(a)]
\item Find the simplest representative, up to isomorphism, of the module presented by $A$.
\item Find a basis for $\mathrm{im}(t_A) \subseteq \Z^3$.
\item Find a basis for $\mathrm{ker}(t_A) \subseteq \Z^4$.
\end{enumerate}

\newpage

\textbf{The next two problems have been postponed to the next assignment.}


\begin{problem} Let $R$ be a domain. An $R$-module $M$ is \textbf{torsionfree} if for $r\in R$ and $m\in M$, we have $rm=0$ implies $r=0$ or $m=0$.
\begin{enumerate}[(a)]
\item Show that if $R$ is a PID and $M$ is a finitely generated torsionfree module, them $M$ is free.
\item Give an example of a torsionfree module $M$ over a PID $R$ such that $M$ is not free.
\item Give an example of a finitely generated torsionfree module $M$ over a domain $R$ such that $M$ is not free.
\end{enumerate}
\end{problem}

\


\begin{problem}
Let $K$ be a field, and $G$ be a finite subgroup of $K^\times$. Show\footnote{Hint: Consider the elementary divisors of $G$, and let $e$ be the largest one. Show that every element of $G$ is a root of the polynomial $x^e-1 \in K[x]$.}\footnote{Note that as a special case of this, $(\Z/p)^\times$ is cyclic.} that $G$ is cyclic.
\end{problem}

\end{document}