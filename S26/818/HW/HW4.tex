\documentclass[11pt]{article}
\usepackage[margin=1in]{geometry}
\usepackage{amsmath,amsfonts,amssymb,amsthm,enumerate,bbm}
\usepackage[]{graphicx}
\usepackage{color,subfigure}
\definecolor{scarlet}{rgb}{0.81,0,0}
\usepackage{multicol}
\usepackage{float}
\usepackage[all]{xypic}
\usepackage[colorlinks=true,citecolor=scarlet,linkcolor=scarlet]{hyperref}
\usepackage{colonequals}

\usepackage{fancyhdr, lastpage}
\pagestyle{fancy}
\fancyfoot[C]{{\thepage} of \pageref{LastPage}}



\DeclareMathOperator{\mSpec}{mSpec}
\DeclareMathOperator{\Spec}{Spec}
\DeclareMathOperator{\Ass}{Ass}
\DeclareMathOperator{\Supp}{Supp}
\DeclareMathOperator{\height}{height}
\DeclareMathOperator{\Hom}{Hom}
\DeclareMathOperator{\ann}{ann}
\DeclareMathOperator{\End}{End}
\DeclareMathOperator{\coker}{coker}
%\DeclareMathOperator{\ker}{ker}
\DeclareMathOperator{\rank}{rank}
\DeclareMathOperator{\im}{im}
\DeclareMathOperator{\M}{M}
\DeclareMathOperator{\Tor}{Tor}
\DeclareMathOperator{\id}{id}
\DeclareMathOperator{\ch}{char}
\DeclareMathOperator{\Aut}{Aut}

%\DeclareMathOperator{\dim}{dim}

\DeclareMathOperator{\lcm}{lcm}

\def\ra{\rightarrow}
\newcommand{\m}{\mathfrak{m}}
\newcommand{\C}{\mathbb{C}}
\newcommand{\Q}{\mathbb{Q}}
\newcommand{\Z}{\mathbb{Z}}
\newcommand{\ZZ}{\mathbb{Z}}
\newcommand{\R}{\mathbb{R}}
\newcommand{\N}{\mathbb{N}}
\newcommand{\ov}[1]{\overline{#1}}
\newcommand{\norm}{\trianglelefteq}

\def\ov#1{\overline{#1}}


\title{}
\date{\vspace{-0.5in}}

\makeatletter
\g@addto@macro\@floatboxreset\centering
\makeatother

\theoremstyle{definition}
\newtheorem{problem}{Problem}


\begin{document}

\thispagestyle{fancy}
\pagestyle{fancy}
\rhead{UNL $\mid$ Spring 2026}
\lhead{Introduction to Modern Algebra II}

\vspace{3em}

\begin{center}
	{\LARGE Problem Set 4 \\}
	Due Thursday, February 11
\end{center}

\

\noindent
{\bf Instructions:}
You are encouraged to work together on these problems, but each student should hand in their own final draft, written in a way that indicates their individual understanding of the solutions. Never submit something for grading that you do not completely understand. You cannot use any resources besides me, your classmates, and our course notes.


I will post the .tex code for these problems for you to use if you wish to type your homework. If you prefer not to type, please  {\em write neatly}. As a matter of good proof writing style, please use complete sentences and correct grammar. You may use any result stated or proven in class or in a homework problem, provided you reference it appropriately by either stating the result or stating its name (e.g. the definition of ring or Lagrange's Theorem). Please do not refer to theorems by their number in the course notes, as that can change.


\smallskip

\begin{problem} Let $R=\ZZ/3[X]^2$ and  $t_A:R^2 \to R^2$ be the linear transformation given by 
\[ t_A\left( \begin{bmatrix} p \\ q \end{bmatrix}\right) = A \begin{bmatrix} p \\ q \end{bmatrix}, \qquad \text{where} \ A= \begin{bmatrix} [2] & x+[1] \\ x & [1] \end{bmatrix}.\]
\end{problem}
\begin{enumerate}[(a)]
\item Given an sequence of elementary row and column operations that transforms $A$ to the matrix $A'=\begin{bmatrix} [1] & 0 \\ 0 & ?\end{bmatrix}$.
\item Using your transformations from (a), find bases $B, C$ for $\ZZ/3[X]^2$ such that $[t_A]_B^C = A'$.
\end{enumerate}

\

\begin{problem}
Let $R=\ZZ[X]$ and let $f:R^2 \to R$ be the linear transformation $f\left( \begin{bmatrix} p \\ q\end{bmatrix}\right) = 2p+Xq$. Show that there do not exist basis $B$ for $R^2$ and $C$ for $R$ and $r\in R$ for which $[f]_B^C = \begin{bmatrix} r & 0 \end{bmatrix}$.
\end{problem}


\


\begin{problem}
	Rank in terms of factoring.
	Rank is largest nonvanishing minor.
	Stable kernel problem
\end{problem}


\

\begin{problem} Let $R$ be a commutative ring. For a matrix $M$ with entries in $R$, a $t\times t$ minor of $M$ is a determinant $\det(M')$ of a matrix $M'$ obtained from selecting $t$ rows and $t$ columns of $M$.
We write $I_t(M)$ for the ideal generated by all $t\times t$ minors of $M$.
\begin{enumerate}[(a)]
\item Let $A$ be an $m\times k$ matrix and $B$ be a $k\times n$ matrix. Let $t\leq \min\{m,k,n\}$. Show that $I_t(AB) \subseteq I_t(A)$.
\item In the same setting as part (a), show that $I_t(AB) \subseteq I_t(B)$.
\item Let $A$ be an $m\times n$ matrix. Let $U$ be an invertible $m\times m$ matrix and $V$ be an invertible $n\times n$ matrix. Let $t\leq \min\{m,n\}$. Show that $I_t(A) = I_t(UAV)$.
\end{enumerate}
\end{problem}

\end{document}