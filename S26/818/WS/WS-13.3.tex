\documentclass[12pt]{amsart}


\usepackage{times}
\usepackage[margin=.9in]{geometry}
\usepackage{paralist,amsmath,amssymb,multicol,graphicx,framed,ifthen,color,xcolor,stmaryrd,enumitem,colonequals}
\usepackage[outline]{contour}
\contourlength{.4pt}
\contournumber{10}
\newcommand{\Bold}[1]{\contour{black}{#1}}

\definecolor{chianti}{rgb}{0.6,0,0}
\definecolor{meretale}{rgb}{0,0,.6}
\definecolor{leaf}{rgb}{0,.35,0}
\newcommand{\Q}{\mathbb{Q}}
\newcommand{\N}{\mathbb{N}}
\newcommand{\Z}{\mathbb{Z}}
\newcommand{\R}{\mathbb{R}}
\newcommand{\C}{\mathbb{C}}
\newcommand{\e}{\varepsilon}
\newcommand{\inv}{^{-1}}
\newcommand{\dabs}[1]{\left| #1 \right|}
\newcommand{\ds}{\displaystyle}
\newcommand{\solution}[1]{\ifthenelse {\equal{\displaysol}{1}} {\begin{framed}{\color{meretale}\noindent #1}\end{framed}} { \ }}
\newcommand{\solutione}[1]{\ifthenelse {\equal{\displaysol}{1}} {\begin{framed}{\color{leaf}This solution is embargoed.}\end{framed}} { \ }}
\newcommand{\showsol}[1]{\def\displaysol{#1}}
\newcommand{\nosolfill}{\ifthenelse {\equal{\displaysol}{1}} {} {\vfill}}
\newcommand{\nosolpage}{\ifthenelse {\equal{\displaysol}{1}} {} {\newpage}}
\newcommand{\nosolonly}[1]{\ifthenelse {\equal{\displaysol}{1}} {} {{#1}}}

\newcommand{\rsa}{\rightsquigarrow}


\newcommand\itemA{\stepcounter{enumi}\item[{\Bold{(\theenumi)}}]}
\newcommand\itemB{\stepcounter{enumi}\item[(\theenumi)]}
\newcommand\itemC{\stepcounter{enumi}\item[{\it{(\theenumi)}}]}
\newcommand\itema{\stepcounter{enumii}\item[{\Bold{(\theenumii)}}]}
\newcommand\itemb{\stepcounter{enumii}\item[(\theenumii)]}
\newcommand\itemc{\stepcounter{enumii}\item[{\it{(\theenumii)}}]}
\newcommand\itemai{\stepcounter{enumiii}\item[{\Bold{(\theenumiii)}}]}
\newcommand\itembi{\stepcounter{enumiii}\item[(\theenumiii)]}
\newcommand\itemci{\stepcounter{enumiii}\item[{\it{(\theenumiii)}}]}
\newcommand\ceq{\colonequals}


\DeclareMathOperator{\ord}{ord}

\DeclareMathOperator{\res}{res}
\setlength\parindent{0pt}
%\usepackage{times}

%\addtolength{\textwidth}{100pt}
%\addtolength{\evensidemargin}{-45pt}
%\addtolength{\oddsidemargin}{-60pt}

\pagestyle{empty}
%\begin{document}\begin{itemize}

%\thispagestyle{empty}




\begin{document}
\showsol{0}
	
	\thispagestyle{empty}
	
	\section*{Smith Normal Form}
	
\begin{framed}
\textsc{Theorem (Smith Normal Form):} Let $R$ be a PID. Let $A\in \mathrm{Mat}_{m\times n}(R)$.
\begin{enumerate}
\item[(i)] There exist invertible matrices $P,Q$ such that
\begin{itemize}
\item $PAQ=D$ is diagonal, meaning $d_{ij}=0$ whenever $i\neq j$, and
\item $d_{11} \, | \, d_{22} \, | \, \cdots \, | \, d_{tt}$, where $d_{tt}$ is the last nonzero diagonal entry.
\end{itemize}
\item[(ii)]  The elements $d_{ii}$ are unique up to associate, meaning that if $D'=[d'_{ij}]$ is another diagonal matrix as in (i), then for each $d'_{ii}$ is a unit times $d_{ii}$.
\item[(iii)]  If $R$ is a Euclidean domain, then $P,Q$ can be taken as products of elementary matrices.
\end{enumerate}

\

\textsc{Structure Theorem for Finitely Generated Modules over PIDs (Invariant Factor Form):} Let $R$ be a PID. Let $M$ be a finitely generated $R$-module. Then there exist $r,t\geq 0$ and $a_1, \dots, a_t\in R$ such that
\begin{itemize}
\item $M\cong R^r \oplus R/(a_1) \oplus R/(a_2) \oplus \cdots  \oplus R/(a_t)$, and
\item $a_1 \, | \, a_2 \, | \, \cdots \, | \, a_t$.
\end{itemize}
Moreover, $r,t$ are uniquely determined, and each $a_i$ is uniquely determined up to associates.   
 \end{framed}
 
 \

\begin{enumerate}
\itemA Use the \textsc{Smith Normal Form Theorem} and a homework problem to deduce the existence part of the \textsc{Structure Theorem for Finitely Generated Modules over PIDs (Invariant Factor Form)}.

\solution{From a Lemma in class, we know that $A$ and $PAQ=D$ present isomorphic modules. Then using the homework problem, we get that $D$ presents a module of the form we seek.}

\itemA Remember/state the \textsc{Structure Theorem for Finitely Generated Abelian Groups (Invariant Factor Form)}, and deduce it from the \textsc{Structure Theorem for Finitely Generated Modules over PIDs (Invariant Factor Form)}.

\solution{\textsc{Structure Theorem for Finitely Generated Abelian Groups (Invariant Factor Form):} Let $M$ be a finitely generated abelian group. Then there exist $r,t\geq 0$ and $a_1, \dots, a_t>0$ such that
\begin{itemize}
\item $M\cong \Z^r \oplus \Z/(a_1) \oplus \Z/(a_2) \oplus \cdots  \oplus \Z/(a_t)$, and
\item $a_1 \, | \, a_2 \, | \, \cdots \, | \, a_t$.
\end{itemize}
Moreover, $r,t$ are uniquely determined.


This is a special case of the PID theorem since every abelian groups are the same thing as $\Z$-modules, $\Z$ is a PID, and unique up to associate in $\Z$ is same thing as unique up to sign, and since we chose positive numbers, this is actually unique.}


\itemB Let $R$ be a Euclidean domain. Use the \textsc{Smith Normal Form Theorem} to deduce\footnote{Hint: Suppose that $D$ is diagonal and invertible. What can you say about the diagonal entries of $D$?} that any invertible matrix over $R$ is a product of elementary matrices.

\solution{Let $A$ be invertible and write $PAQ=D$ following the theorem. Note that $D$ is invertible and diagonal. We claim that $D$ must be a square matrix with unit diagonal entries. Such a matrix is invertible, and one can check directly that if any entry is not a unit, then $D$ is not surjective. We can choose $D$ to be the identity matrix by using some elementary row operations. Now $A=P^{-1} I Q^{-1} = P^{-1} Q^{-1}$, and $P^{-1}$ and $Q^{-1}$ are products of elementary matrices, since the inverse of an elementary matrix is an elementary matrix.}


\itemB Proof of the uniqueness part of the \textsc{Structure Theorem for Finitely Generated Modules over PIDs (Invariant Factor Form)}: Suppose that 
\[ R^m \oplus R/(d_1) \oplus \cdots \oplus R/(d_n) \cong R^{m'} \oplus R/(d'_1) \oplus \cdots \oplus R/(d'_{n'})\]
and $d_1 \,|\, \cdots \,|\, d_n$ and also $d'_1\,|\, \cdots \,|\, d'_{n'}$ with $n\geq n'$. We proceed by induction on $n$.
\begin{enumerate}
\itemb Deal with the base case $n=0$ (so $n'=0$).
\itemb Suppose that $n>0$. Let $\phi$ be and isomorphism from left to right, and ${m= (0, 0, \dots, 1+(d_n))}$ in the left-hand side. Show that $\mathrm{ann}_R(\phi(m)) = (d_n)$.
\itemb Show that $n'>0$ and that $d_n \,|\, \ d'_n$.
\itemb Show that $d_n$ and $d'_n$ are associates.
\itemb Complete the induction step and the proof.
\end{enumerate}
\end{enumerate}

\nosolfill
\nosolpage


\begin{framed}
\textsc{Structure Theorem for Finitely Generated Modules over PIDs (Elementary Divisor Form):} Let $R$ be a PID. Let $M$ be a finitely generated $R$-module. Then there exist $r,s\geq 0$ and prime elements $p_1, \dots, p_s\in R$ such that
 $M\cong R^r \oplus R/(p_1^{e_1}) \oplus \cdots  \oplus R/(p_s^{e_s})$.
Moreover, the number $r$ is uniquely determined and the list $p_1^{e_1},\dots,p_s^{e_s}$ is unique up to reordering and associates.


\

\textsc{CRT (from 817 HW):} Let $R$ be a commutative ring, and $I,J$ ideals such that $I+J=R$. Then $R/IJ \cong R/I \times R/J$ as rings, and hence also as $R$-modules.
 \end{framed}

\begin{enumerate}
\setcounter{enumi}{4}
\itemA Converting between forms:
\begin{enumerate}
\item[$\star$] To convert a cyclic module $R/(a)$ to elementary divisor form, write $f=p_1^{e_1}\cdots p_s^{e_s}$ as a product of prime powers, and use CRT to get
\[ R/a \cong R/(p_1^{e_1}) \oplus \cdots \oplus R/(p_s^{e_s}).\]
\itema Convert the $\R[x]$-module
\[ \R[x]^2 \oplus \R[x]/(x-1) \oplus \R[x]/(x^2-1) \oplus \R[x]/((x-1) (x^2-1))\]
to elementary divisor form.
\item[$\star$] To convert a module from elementary divisor form to invariant factor form, 
\begin{enumerate} 
\item[$-$] For each distinct prime $p_j$ occurring, take the largest power $E_j$ it has in an elementary divisor, and combine and combine $\bigoplus_j R/ p_j^{E_j} \cong R/(p_1^{E_1} \cdots p_\ell^{E_\ell})$ via CRT. If there's more than one copy of  $R/p_j^{E_j}$, just take one of the copies and leave the rest.
\item[$-$] Repeat with the remaining factors.
\end{enumerate}
\itema Convert $\R[x] / (x) \oplus \R[x] / (x^2) \oplus (\R[x]/(x-3))^{\oplus 2} \oplus \R[x] / ((x-7)^3)$ to invariant factor form.
\end{enumerate}

\solution{
\begin{enumerate}
\itema $\R[x]^2 \oplus (\R[x]/(x-1))^{\oplus 2} \oplus \R[x]/((x-1)^2) \oplus \R[x]/((x+1)^2)$
\itema $\R[x]/(x(x-3)) \oplus  \R[x]/( (x-3) (x^2)  (x-7)^3)$
\end{enumerate}}


\end{enumerate}



%\itemB Uniqueness of Smith Normal Form:
%\begin{enumerate}
%\itemb Let $D$ be a diagonal matrix and $d_{11} \ | \ d_{22} \ | \ \cdots \ | \ d_{tt}$. Show that $I_r(D) = ( d_{11} \cdots d_{rr} )$.
%\itemb Prove the uniqueness of the Smith Normal Form.
%\end{enumerate} 
%\end{enumerate} 
%
%\
%


%
%\textsc{Definition:} Let $R$ be a domain and $M$ be an $R$-module. We say that $M$ is \textbf{torsionfree} if for $r\in R$ and $m\in M$, we have $rm=0$ implies $r=0$ or $m=0$.
%
% 
% \
% 
\nosolonly{
 \begin{enumerate}\setcounter{enumi}{5}
 \itemB Let $R$ be a PID.
 \begin{enumerate}
 \itemb Show that any finitely generated torsionfree $R$-module is free.
 \itemb Show that any submodule of a finitely generated free $R$-module is free.
 \itemb Prove or disprove: any torsionfree $R$-module is free.
 \itemc Prove or disprove: any submodule of a free $R$-module is free.
 \end{enumerate}
 \end{enumerate}
}


\end{document}
