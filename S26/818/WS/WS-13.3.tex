\documentclass[12pt]{amsart}


\usepackage{times}
\usepackage[margin=1in]{geometry}
\usepackage{paralist,amsmath,amssymb,multicol,graphicx,framed,ifthen,color,xcolor,stmaryrd,enumitem,colonequals}
\usepackage[outline]{contour}
\contourlength{.4pt}
\contournumber{10}
\newcommand{\Bold}[1]{\contour{black}{#1}}

\definecolor{chianti}{rgb}{0.6,0,0}
\definecolor{meretale}{rgb}{0,0,.6}
\definecolor{leaf}{rgb}{0,.35,0}
\newcommand{\Q}{\mathbb{Q}}
\newcommand{\N}{\mathbb{N}}
\newcommand{\Z}{\mathbb{Z}}
\newcommand{\R}{\mathbb{R}}
\newcommand{\C}{\mathbb{C}}
\newcommand{\e}{\varepsilon}
\newcommand{\inv}{^{-1}}
\newcommand{\dabs}[1]{\left| #1 \right|}
\newcommand{\ds}{\displaystyle}
\newcommand{\solution}[1]{\ifthenelse {\equal{\displaysol}{1}} {\begin{framed}{\color{meretale}\noindent #1}\end{framed}} { \ }}
\newcommand{\solutione}[1]{\ifthenelse {\equal{\displaysol}{1}} {\begin{framed}{\color{leaf}This solution is embargoed.}\end{framed}} { \ }}
\newcommand{\showsol}[1]{\def\displaysol{#1}}

\newcommand{\rsa}{\rightsquigarrow}


\newcommand\itemA{\stepcounter{enumi}\item[{\Bold{(\theenumi)}}]}
\newcommand\itemB{\stepcounter{enumi}\item[(\theenumi)]}
\newcommand\itemC{\stepcounter{enumi}\item[{\it{(\theenumi)}}]}
\newcommand\itema{\stepcounter{enumii}\item[{\Bold{(\theenumii)}}]}
\newcommand\itemb{\stepcounter{enumii}\item[(\theenumii)]}
\newcommand\itemc{\stepcounter{enumii}\item[{\it{(\theenumii)}}]}
\newcommand\itemai{\stepcounter{enumiii}\item[{\Bold{(\theenumiii)}}]}
\newcommand\itembi{\stepcounter{enumiii}\item[(\theenumiii)]}
\newcommand\itemci{\stepcounter{enumiii}\item[{\it{(\theenumiii)}}]}
\newcommand\ceq{\colonequals}


\DeclareMathOperator{\ord}{ord}

\DeclareMathOperator{\res}{res}
\setlength\parindent{0pt}
%\usepackage{times}

%\addtolength{\textwidth}{100pt}
%\addtolength{\evensidemargin}{-45pt}
%\addtolength{\oddsidemargin}{-60pt}

\pagestyle{empty}
%\begin{document}\begin{itemize}

%\thispagestyle{empty}




\begin{document}
\showsol{0}
	
	\thispagestyle{empty}
	
	\section*{Smith Normal Form}
	
\begin{framed}
\textsc{Theorem (Smith Normal Form):} Let $R$ be a PID. Let $A\in \mathrm{Mat}_{m\times n}(R)$.
\begin{enumerate}
\item[(i)] There exist invertible matrices $P,Q$ such that
\begin{itemize}
\item $PAQ=D$ is diagonal, meaning $d_{ij}=0$ whenever $i\neq j$, and
\item $d_{11} \, | \, d_{22} \, | \, \cdots \, | \, d_{tt}$, where $d_{tt}$ is the last nonzero diagonal entry.
\end{itemize}
\item[(ii)]  The elements $d_{ii}$ are unique up to associate, meaning that if $D'=[d'_{ij}]$ is another diagonal matrix as in (i), then for each $d'_{ii}$ is a unit times $d_{ii}$.
\item[(iii)]  If $R$ is a Euclidean domain, then $P,Q$ can be taken as products of elementary matrices.
\end{enumerate}

\

\textsc{Structure Theorem for Finitely Generated Modules over PIDs (Invariant Factor Form):} Let $R$ be a PID. Let $M$ be a finitely generated $R$-module. Then there exist $r,t\geq 0$ and $a_1, \dots, a_t\in R$ such that
\begin{itemize}
\item $M\cong R^r \oplus R/(a_1) \oplus R/(a_2) \oplus \cdots  \oplus R/(a_t)$, and
\item $a_1 \, | \, a_2 \, | \, \cdots \, | \, a_t$.
\end{itemize}
Moreover, $r,t$ are uniquely determined, and each $a_i$ is uniquely determined up to associates.   
 \end{framed}
 
 \

\begin{enumerate}
\itemA Let $R$ be a commutative ring, and $D\in \mathrm{Mat}_{m\times n}(R)$ be a diagonal matrix with nonzero diagonal entries $d_{11},\dots,d_{tt}$. Prove that the module presented by $D$ is
\[  R/(d_{11}) \oplus R/(d_{22}) \oplus \cdots \oplus R/(d_{tt}) \oplus R^{m-t}.\]

\solution{Let $M$ be the module presented by $D$ and $N$ be the module given above. Consider the map from $\phi: R^m \to N$ given by
\[ \phi(e_i) = \begin{cases} ([0],\dots,[1]_\text{$i$th position}, \dots, [0], (0,\dots,0)) &\text{if} \ i\leq t \\
([0], \cdots, [0], (0,\dots, 1_\text{$(i-t)$th position}, \dots ,0)) &\text{if} \ i> t. \end{cases}\]
A unique such map exists by UMP for free modules. The map $\phi$ is surjective, since any element of $N$ is an $R$-linear combination of the images $\phi(e_i)$ of the generators. We claim that $\ker(\phi) = \mathrm{im}(t_D)$.
Indeed, 
\[ \begin{aligned} (r_1,\dots,r_m) \in \ker(\phi) &\Longleftrightarrow ([r_1]_{(d_{11})},\dots, [r_t]_{(d_{tt})}, (r_{t+1},\dots,r_m) ) = 0 \\
& \Longleftrightarrow r_1 \in (d_{11}) , \dots,r_t\in (d_{tt}) , r_{t+1} = \cdots = r_m = 0\\
& \Longleftrightarrow r_1 = d_{11} s_1, \dots,r_t = d_{tt} s_t , \text{ some } s_i\in R, \text{ and } r_{t+1} = \cdots = r_m = 0\\
& \Longleftrightarrow (r_1,\dots,r_m) = t_D(s_1,\dots,s_t, s_{t+1} ,\dots, s_{n}) , \text{ some } s_i\in R\\
& \Longleftrightarrow (r_1,\dots,r_m) \in \mathrm{im}(t_D).\end{aligned}\]
}


\itemA Use the Smith Normal Form Theorem to deduce the Structure Theorem for Finitely Generated Modules over PIDs (Invariant Factor Form).

\solution{From a Lemma in class, we know that $A$ and $PAQ=D$ present isomorphic modules. Then using the previous exercise, we get that $D$ presents a module of the form we seek.}

\itemA State the Structure Theorem for Finitely Generated Abelian Groups (Invariant Factor Form), and deduce it from the PID Theorem.

\solution{\textsc{Structure Theorem for Finitely Generated Abelian Groups (Invariant Factor Form):} Let $M$ be a finitely generated abelian group. Then there exist $r,t\geq 0$ and $a_1, \dots, a_t>0$ such that
\begin{itemize}
\item $M\cong \Z^r \oplus \Z/(a_1) \oplus \Z/(a_2) \oplus \cdots  \oplus \Z/(a_t)$, and
\item $a_1 \, | \, a_2 \, | \, \cdots \, | \, a_t$.
\end{itemize}
Moreover, $r,t$ are uniquely determined.


This is a special case of the PID theorem since every abelian groups are the same thing as $\Z$-modules, $\Z$ is a PID, and unique up to associate in $\Z$ is same thing as unique up to sign, and since we chose positive numbers, this is actually unique.}


\itemA Let $R$ be a Euclidean domain. Use the Smith Normal Form Theorem to deduce that any invertible matrix over $R$ is a product of elementary matrices.

\solution{Let $A$ be invertible and write $PAQ=D$ following the theorem. Note that $D$ is invertible and diagonal. We claim that $D$ must be a square matrix with unit diagonal entries. Such a matrix is invertible, and one can check directly that if any entry is not a unit, then $D$ is not surjective. We can choose $D$ to be the identity matrix by using some elementary row operations. Now $A=P^{-1} I Q^{-1} = P^{-1} Q^{-1}$, and $P^{-1}$ and $Q^{-1}$ are products of elementary matrices, since the inverse of an elementary matrix is an elementary matrix.}
\end{enumerate}

\begin{framed}
\textsc{Definition:} Let $R$ be a domain and $M$ be an $R$-module. We say that $M$ is \textbf{torsionfree} if for $r\in R$ and $m\in M$, we have $rm=0$ implies $r=0$ or $m=0$.
 \end{framed}
 
 \
 
 \begin{enumerate}\setcounter{enumi}{4}
 \itemB Let $R$ be a PID.
 \begin{enumerate}
 \itemb Show that any finitely generated torsionfree $R$-module is free.
 \itemb Show that any submodule of a finitely generated free $R$-module is free.
 \itemb Prove or disprove: any torsionfree $R$-module is free.
 \itemc Prove or disprove: any submodule of a free $R$-module is free.
 \end{enumerate}
 \end{enumerate}
 
 \newpage


\begin{framed}
\textsc{Structure Theorem for Finitely Generated Modules over PIDs (Elementary Divisor Form):} Let $R$ be a PID. Let $M$ be a finitely generated $R$-module. Then there exist $r,s\geq 0$ and prime elements $p_1, \dots, p_s\in R$ such that
 $M\cong R^r \oplus R/(p_1^{e_1}) \oplus \cdots  \oplus R/(p_s^{e_s})$, and
Moreover, the list $p_1^{e_1},\dots,p_s^{e_s}$ is unique up to reordering and associates.
 \end{framed}

\begin{enumerate}
\setcounter{enumi}{5}
\itemA Converting between forms:
\begin{enumerate}
\item[$\star$] To convert a cyclic module $R/(a)$ to elementary divisor form, write $f=p_1^{e_1}\cdots p_s^{e_s}$ as a product of prime powers, and use CRT get
\[ R/a \cong R/{p_1^{e_1}} \oplus \cdots \oplus R/{p_s^{e_s}}.\]
\itema Convert the $\R[x]$-module
\[ \R[x]^2 \oplus \R[x]/(x-1) \oplus \R[x]/(x^2-1) \oplus \R[x]/((x-1) (x^2-1))\]
to elementary divisor form.
\item[$\star$] To convert a module from elementary divisor form to invariant factor form, 
\begin{enumerate} 
\item[$\bullet$] For each distinct prime $p_j$ occurring, take the largest power $E_j$ it has in an elementary divisor, and combine and combine $\bigoplus_j R/ p_j^{E_j} \cong R/(p_1^{E_1} \cdots p_\ell^{E_\ell})$ via CRT. If there's more than one copy of  $R/p_j^{E_j}$, just take one of the copies and leave the rest.
\item[$\bullet$] Repeat with the remaining factors.
\end{enumerate}
\itema Convert $\R[x] / (x) \oplus \R[x] / (x^2) \oplus (\R[x]/(x-3))^{\oplus 2} \oplus \R[x] / (x-7)^3$ to invariant factor form.
\end{enumerate}

\solution{
\begin{enumerate}
\itema $\R[x]^2 \oplus (\R/(x-1))^{\oplus 2} \oplus \R/((x-1)^2) \oplus \R/((x+1)^2)$
\itema $\R[x]/(x(x-3)) \oplus  \R[x]/( (x-3) (x^2)  (x-7)^3)$
\end{enumerate}}

\itemB Uniqueness of Smith Normal Form:
\begin{enumerate}
\itemb Let $D$ be a diagonal matrix and $d_{11} \ | \ d_{22} \ | \ \cdots \ | \ d_{tt}$. Show that $I_r(D) = ( d_{11} \cdots d_{rr} )$.
\itemb Prove the uniqueness of the Smith Normal Form.
\end{enumerate} 
\end{enumerate} 

\




\end{document}
