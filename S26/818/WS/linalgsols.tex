\documentclass[12pt]{amsart}


\usepackage{times}
\usepackage[margin=.9in]{geometry}
\usepackage{paralist,amsmath,amssymb,multicol,graphicx,framed,ifthen,color,xcolor,stmaryrd,colonequals}
\usepackage[shortlabels]{enumitem}
\usepackage[all]{xy}
\usepackage[outline]{contour}
\contourlength{.4pt}
\contournumber{10}
\newcommand{\Bold}[1]{\contour{black}{#1}}

\definecolor{chianti}{rgb}{0.6,0,0}
\definecolor{meretale}{rgb}{0,0,.6}
\definecolor{leaf}{rgb}{0,.35,0}
\newcommand{\Q}{\mathbb{Q}}
\newcommand{\N}{\mathbb{N}}
\newcommand{\Z}{\mathbb{Z}}
\newcommand{\R}{\mathbb{R}}
\newcommand{\C}{\mathbb{C}}
\newcommand{\e}{\varepsilon}
\newcommand{\inv}{^{-1}}
\newcommand{\dabs}[1]{\left| #1 \right|}
\newcommand{\ds}{\displaystyle}
\newcommand{\solution}[1]{\ifthenelse {\equal{\displaysol}{1}} {\begin{framed}{\color{meretale}\noindent #1}\end{framed}} { \ }}
\newcommand{\solutione}[1]{\ifthenelse {\equal{\displaysol}{1}} {\begin{framed}{\color{leaf}This solution is embargoed.}\end{framed}} { \ }}
\newcommand{\showsol}[1]{\def\displaysol{#1}}

\newcommand{\rsa}{\rightsquigarrow}

\newcommand{\T}{{\color{leaf}TRUE}}
\newcommand{\F}{{\color{chianti}FALSE}}

\newcommand{\E}{{\color{meretale}EUC}}

\newcommand{\Tt}{{\color{orange}TRUE}}
\newcommand{\FE}{{\color{orange}F}/{\color{meretale}EUC} }

\newcommand\itemA{\stepcounter{enumi}\item[{\Bold{(\theenumi)}}]}
\newcommand\itemB{\stepcounter{enumi}\item[(\theenumi)]}
\newcommand\itemC{\stepcounter{enumi}\item[{\it{(\theenumi)}}]}
\newcommand\itema{\stepcounter{enumii}\item[{\Bold{(\theenumii)}}]}
\newcommand\itemb{\stepcounter{enumii}\item[(\theenumii)]}
\newcommand\itemc{\stepcounter{enumii}\item[{\it{(\theenumii)}}]}
\newcommand\itemai{\stepcounter{enumiii}\item[{\Bold{(\theenumiii)}}]}
\newcommand\itembi{\stepcounter{enumiii}\item[(\theenumiii)]}
\newcommand\itemci{\stepcounter{enumiii}\item[{\it{(\theenumiii)}}]}
\newcommand\ceq{\colonequals}
\newcommand\blank[1]{\underline{\phantom{ #1 }}}
\newcommand\Ar{$\Longrightarrow$ }
\newcommand\Arr{$\Longleftrightarrow$ }

\DeclareMathOperator{\ord}{ord}

\DeclareMathOperator{\res}{res}
\setlength\parindent{0pt}
%\usepackage{times}

%\addtolength{\textwidth}{100pt}
%\addtolength{\evensidemargin}{-45pt}
%\addtolength{\oddsidemargin}{-60pt}

\pagestyle{empty}
%\begin{document}\begin{itemize}

%\thispagestyle{empty}




\begin{document}
\showsol{0}
	
	\thispagestyle{empty}
	
	\section*{Linear algebra review}

	
	\begin{framed}Determine whether each statement is true when $R$ is a field, is a PID, or is an arbitrary commutative ring (CR). Throughout, $M, N$ are $R$-modules and $A, B$ are (finite) matrices. We write $M\subseteq N$ to mean $M$ is an $R$-submodule of $N$.\end{framed}
	
	\
	
	\begin{tabular}{l|c|c|c|}
	& Field & PID & CR \\ 
	\hline
(1) Every $M$ is free. &\T&\F&\F\\ 
	\hline
(2) If $M$ is torsionfree, then $M$ is free.&\T&\F&\F\\ 
	\hline
(3) If $M$ is finitely generated and torsionfree, then $M$ is free.&\T&\T&\F\\ 
	\hline
(4) If $M$ is finitely generated and free and $N\subseteq M$, then $N$ is free. &\T&\T&\F\\ 
	\hline
(5) If $M$ is finitely generated and $N\subseteq M$, then $N$ is finitely~generated.&\T&\T&\F\\ 
	\hline
(6) If $M$ is finitely generated, $N\subseteq M$, and $N\cong M$, then $N=M$.&\T&\F&\F\\ 
	\hline
(7) If $M$ is finitely generated and $f:M\to M$ is injective, then $f$ is an iso.&\T&\F&\F\\ 
	\hline
(8) If $M$ is finitely generated and $f:M\to M$ is surjective, then $f$ is an iso.&\T&\T&\Tt\\
	\hline
(9) If $M$ is a free module and $N\subseteq M$, then $N$ is free.&\T&\Tt&\F\\
	\hline

(10) For any matrix $A$ there are invertible $P,Q$ s.t.~$PAQ$ is diagonal.&\T&\T&\F\\
	\hline
(11) Any matrix $A$ can be turned into a diagonal matrix with EROs and ECOs.&\T&\FE&\F\\
	\hline
(12) Any invertible matrix $A$ can be turned into a diagonal matrix with EROs.&\T&\FE&\F\\
	\hline
(13) Any matrix $A$ can be turned into a diagonal matrix with EROs.&\F&\F&\F\\
	\hline
(14) An $n\times n$ matrix $A$ is invertible if and only if $\det(A)\neq 0$.&\T&\F&\F\\
	\hline
(15) An $n\times n$ matrix $A$ is invertible if and only if the columns of $A$ are LI.&\T&\F&\F\\
	\hline
(16) An $n\times n$ matrix $A$ is invertible if and only if $\exists B$ with $AB=I_n$.&\T&\T&\Tt\\
	\hline
(17) An $n\times n$ matrix $A$ is invertible if and only if $\exists B$ with $BA=I_n$.&\T&\T&\Tt\\
	\hline
(18) If $M$ is free and $S\subseteq M$ is LI, then $S$ is a subset of a basis.&\T&\F&\F\\
	\hline
(19) If $M$ is free and $S\subseteq M$ generates $M$, then $S$ contains a basis.&\T&\F&\F \\
	\hline
(20) If $M\cong N$ are free then $\mathrm{rank}(M) =\mathrm{rank}(N)$. &\T&\T&\T \\
	\hline
(21) If $M\twoheadrightarrow N$ are free then $\mathrm{rank}(M)\geq \mathrm{rank}(N)$. &\T&\T&\T \\
	\hline
(22) If $M\subseteq N$ are free then $\mathrm{rank}(M)\leq \mathrm{rank}(N)$. &\T&\T&\Tt\\
	\hline
(23) If $M\subseteq N$ and $N$ can be generated by $n$ elements then so can $M$. &\T&\T&\F\\
	\hline
(24) If $M\subseteq N$ and $N$ can be generated by $n$ elements then so can $N/M$. &\T&\T&\T\\
	\hline
(25) If $M\subseteq N$ and both $M$ and $N/M$ are free, then $N$ is free.&\T&\T&\T\\
	\hline
	\end{tabular}

	
	\footnotetext{Warning: There are a few tricky boxes here, but we can resolve almost all of them with techniques and examples we have considered in class.}
\end{document}
