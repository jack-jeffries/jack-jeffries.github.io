\documentclass[12pt]{amsart}


\usepackage{times}
\usepackage[margin=.9in]{geometry}
\usepackage{paralist,amsmath,amssymb,multicol,graphicx,framed,ifthen,color,xcolor,stmaryrd,enumitem,colonequals}
\usepackage[outline]{contour}
\contourlength{.4pt}
\contournumber{10}
\newcommand{\Bold}[1]{\contour{black}{#1}}

\definecolor{chianti}{rgb}{0.6,0,0}
\definecolor{meretale}{rgb}{0,0,.6}
\definecolor{leaf}{rgb}{0,.35,0}
\newcommand{\Q}{\mathbb{Q}}
\newcommand{\N}{\mathbb{N}}
\newcommand{\Z}{\mathbb{Z}}
\newcommand{\R}{\mathbb{R}}
\newcommand{\C}{\mathbb{C}}
\newcommand{\e}{\varepsilon}
\newcommand{\inv}{^{-1}}
\newcommand{\dabs}[1]{\left| #1 \right|}
\newcommand{\ds}{\displaystyle}
\newcommand{\solution}[1]{\ifthenelse {\equal{\displaysol}{1}} {\begin{framed}{\color{meretale}\noindent #1}\end{framed}} { \ }}
\newcommand{\solutione}[1]{\ifthenelse {\equal{\displaysol}{1}} {\begin{framed}{\color{leaf}This solution is embargoed.}\end{framed}} { \ }}
\newcommand{\showsol}[1]{\def\displaysol{#1}}
\newcommand{\nosolfill}{\ifthenelse {\equal{\displaysol}{1}} {} {\vfill}}
\newcommand{\nosolpage}{\ifthenelse {\equal{\displaysol}{1}} {} {\newpage}}
\newcommand{\nosolonly}[1]{\ifthenelse {\equal{\displaysol}{1}} {} {{#1}}}

\newcommand{\rsa}{\rightsquigarrow}


\newcommand\itemA{\stepcounter{enumi}\item[{\Bold{(\theenumi)}}]}
\newcommand\itemB{\stepcounter{enumi}\item[(\theenumi)]}
\newcommand\itemC{\stepcounter{enumi}\item[{\it{(\theenumi)}}]}
\newcommand\itema{\stepcounter{enumii}\item[{\Bold{(\theenumii)}}]}
\newcommand\itemb{\stepcounter{enumii}\item[(\theenumii)]}
\newcommand\itemc{\stepcounter{enumii}\item[{\it{(\theenumii)}}]}
\newcommand\itemai{\stepcounter{enumiii}\item[{\Bold{(\theenumiii)}}]}
\newcommand\itembi{\stepcounter{enumiii}\item[(\theenumiii)]}
\newcommand\itemci{\stepcounter{enumiii}\item[{\it{(\theenumiii)}}]}
\newcommand\ceq{\colonequals}


\DeclareMathOperator{\ord}{ord}

\DeclareMathOperator{\res}{res}
\setlength\parindent{0pt}
%\usepackage{times}

%\addtolength{\textwidth}{100pt}
%\addtolength{\evensidemargin}{-45pt}
%\addtolength{\oddsidemargin}{-60pt}

\pagestyle{empty}
%\begin{document}\begin{itemize}

%\thispagestyle{empty}




\begin{document}
\showsol{0}
	
	\thispagestyle{empty}
	
	\section*{Rational canonical form}
	\begin{framed}
	\textsc{Definition:} Let $F$ be a field. For a monic polynomial \[ f(x) = x^n + a_{n-1} x^{n-1} + \cdots + a_1 x + a_0 \in F[x],\] the \textbf{companion matrix} of $f$ is 
	\[ C(f) = \begin{bmatrix}
0 & \cdots & 0 & -a_0 \\
&&& -a_1 \\
& I_{n-1}&& \vdots\\
&&& -a_{n-1} \\
\end{bmatrix}.\]

\

	\textsc{Rational Canonical Form:} Let $F$ be a field. Let $V$ be a finite dimensional vector space, and $\phi:V\to V$ a linear transformation. Then there is a basis $B$ for $V$ such that
	\[ [\phi]_B^B = \begin{bmatrix} C(g_1) & 0 & \cdots & 0 \\ 0 & C(g_2) & \cdots & 0 \\ \vdots & \vdots & \ddots & \vdots \\ 0 & 0 & \cdots & C(g_s)\end{bmatrix}\]
	where $g_1 | \cdots | g_s$ are the invariant factors\footnotemark[1] of $\phi$. This matrix is called the \textbf{rational canonical form} of $\phi$, or $\mathrm{RCF}(\phi)$.
	
	\
	
	\textsc{Theorem:} Let $F$ be a field and $A,B\in \mathrm{Mat}_{n\times n}(F)$. The following are equivalent:
	\begin{enumerate}
	\item $A$ and $B$ are similar matrices.
	\item $A$ and $B$ have the same rational canonical form.
	\item $A$ and $B$ have the same invariant factors.
	\end{enumerate}
	\end{framed}
	\footnotetext[1]{The \textbf{invariant factors} of $\phi$ are the (monic) invariant factors of the {$F[x]$-module}~$V_\phi$.}
	
	\begin{enumerate}
	\itemA Computing some RCFs:
	\begin{enumerate}
	\itema Let 
	
	\end{enumerate} 
	
	\
	
	\itemA Classifying matrices up to similarity:	
	
	\itemB Uniqueness of rational canonical form:


	
	\end{enumerate}

\end{document}
