\documentclass[12pt]{amsart}


\usepackage{times}
\usepackage[margin=.58in]{geometry}
\usepackage{paralist,amsmath,amssymb,multicol,graphicx,framed,ifthen,color,xcolor,stmaryrd,enumitem,colonequals}
\usepackage[outline]{contour}
\contourlength{.4pt}
\contournumber{10}
\newcommand{\Bold}[1]{\contour{black}{#1}}

\definecolor{chianti}{rgb}{0.6,0,0}
\definecolor{meretale}{rgb}{0,0,.6}
\definecolor{leaf}{rgb}{0,.35,0}
\newcommand{\Q}{\mathbb{Q}}
\newcommand{\N}{\mathbb{N}}
\newcommand{\Z}{\mathbb{Z}}
\newcommand{\R}{\mathbb{R}}
\newcommand{\C}{\mathbb{C}}
\newcommand{\e}{\varepsilon}
\newcommand{\inv}{^{-1}}
\newcommand{\dabs}[1]{\left| #1 \right|}
\newcommand{\ds}{\displaystyle}
\newcommand{\solution}[1]{\ifthenelse {\equal{\displaysol}{1}} {\begin{framed}{\color{meretale}\noindent #1}\end{framed}} { \ }}
\newcommand{\solutione}[1]{\ifthenelse {\equal{\displaysol}{1}} {\begin{framed}{\color{leaf}This solution is embargoed.}\end{framed}} { \ }}
\newcommand{\showsol}[1]{\def\displaysol{#1}}

\newcommand{\rsa}{\rightsquigarrow}


\newcommand\itemA{\stepcounter{enumi}\item[{\Bold{(\theenumi)}}]}
\newcommand\itemB{\stepcounter{enumi}\item[(\theenumi)]}
\newcommand\itemC{\stepcounter{enumi}\item[{\it{(\theenumi)}}]}
\newcommand\itema{\stepcounter{enumii}\item[{\Bold{(\theenumii)}}]}
\newcommand\itemb{\stepcounter{enumii}\item[(\theenumii)]}
\newcommand\itemc{\stepcounter{enumii}\item[{\it{(\theenumii)}}]}
\newcommand\itemai{\stepcounter{enumiii}\item[{\Bold{(\theenumiii)}}]}
\newcommand\itembi{\stepcounter{enumiii}\item[(\theenumiii)]}
\newcommand\itemci{\stepcounter{enumiii}\item[{\it{(\theenumiii)}}]}
\newcommand\ceq{\colonequals}


\DeclareMathOperator{\ord}{ord}

\DeclareMathOperator{\res}{res}
\setlength\parindent{0pt}
%\usepackage{times}

%\addtolength{\textwidth}{100pt}
%\addtolength{\evensidemargin}{-45pt}
%\addtolength{\oddsidemargin}{-60pt}

\pagestyle{empty}
%\begin{document}\begin{itemize}

%\thispagestyle{empty}




\begin{document}
\showsol{0}
	
	\thispagestyle{empty}
	
	\section*{Matrices and homomorphisms between free modules}
	
\begin{framed}
\textsc{Definition:} Let $R$ be a commutative ring. Let $V$ be a free $R$-module with ordered basis ${\mathcal{B}=\{b_1,\dots,b_n\}}$ and let $W$ be a free $R$-module with ordered basis ${\mathcal{C}=\{c_1,\dots,c_m\}}$. 
Given an \mbox{$R$-module} homomorphism $T\colon V\to W$, for each $j=1,\dots n$, write 
\[\tag{$\clubsuit$} T(b_j) = r_{1,j} c_1 + \cdots + r_{m,j} c_m \]
for some elements $r_{i,j}\in R$. The matrix
\[ [T]_{\mathcal{B}}^{\mathcal{C}} = \begin{bmatrix}  
r_{1,1} & r_{1,2} & \cdots & r_{1,n} \\
r_{2,1} & r_{2,2} & \cdots & r_{2,n} \\
\vdots & \vdots & \ddots & \vdots \\
r_{m,1} & r_{m,2} & \cdots & r_{m,n} \end{bmatrix}\]
is the \textbf{matrix representing $T$ in the bases $\mathcal{B}$ and $\mathcal{C}$.}
 \end{framed}

\begin{enumerate}
\itemA Warming up with the definition:
\begin{enumerate}
\itema If $R$ is a field $F$, translate everything\footnote{You can do this aloud instead of rewriting everything.} in the definition into linear algebra terms.
\itema Use the equation ($\clubsuit$) to explain as concretely as possible what the $j$-th column of $[T]_{\mathcal{B}}^{\mathcal{C}}$ means in terms of $T$, $\mathcal{B}$, and $\mathcal{C}$.
\itema Explain why the entries $r_{i,j}$ are well-defined.
\itema \emph{Just using your answer for part \textbf{(b)} and not looking at the formula}, describe the dimensions of the matrix $[T]_{\mathcal{B}}^{\mathcal{C}}$ in terms of the rank of $V$ and the rank of $W$.
\itema Let $V$ be the $\mathbb{R}$-vector space of polynomials in $\mathbb{R}[x]$ of degree at most $3$ along with the zero polynomial. The derivative map $\frac{d}{dx}$ is a linear transformation from $V$ to $V$. Choose a nice basis $\mathcal{B}$ for $V$ and compute the matrix $[\frac{d}{dx}]_{\mathcal{B}}^{\mathcal{B}}$.
\vspace{-5mm}
\itema Find another\footnote{You might have to reorder or change $\mathcal{B}$ if you are unlucky.}  basis $\mathcal{C}$ for $V$ such that  $[\frac{d}{dx}]_{\mathcal{B}}^{\mathcal{C}} = {\footnotesize \begin{bmatrix} 1 & 0 & 0 & 0 \\ 0 & 1 & 0 & 0 \\ 0 & 0 & 1 & 0 \\ 0 & 0 & 0 & 0 \end{bmatrix}}$.
\end{enumerate}

\solution{
\begin{enumerate}
\itema Let $F$ be a field. Let $V$ be an $F$-vector space with ordered basis ${\mathcal{B}=\{b_1,\dots,b_n\}}$ and let $W$ be a free $R$-module with ordered basis ${\mathcal{C}=\{c_1,\dots,c_m\}}$. 
Given an $F$-linear transformation $T\colon V\to W$ (the rest is the same).
\itema The $j$-th column of $[T]_{\mathcal{B}}^{\mathcal{C}}$ is the expression of the image of the $j$th basis vector in $\mathcal{B}$ as a linear combination of $\mathcal{C}$.
\itema This is the uniqueness of expression in terms of a basis applied to $\mathcal{C}$.
\itema The number of columns equals the rank of $V$, since there is one column for each basis vector in $\mathcal{B}$. The number of rows is the number of entries in a column which is the rank of $W$, since in each column we have one coefficient for each element of $\mathcal{C}$.
\itema One possibility is $\mathcal{B}=\{x^3,x^2,x,1\}$, and the matrix is
\[ [\frac{d}{dx}]_{\mathcal{B}}^{\mathcal{B}} = \begin{bmatrix} 0 & 0 & 0 & 0 \\ 3 & 0 & 0 & 0 \\ 0 & 2 & 0 & 0 \\ 0 & 0 & 1 & 0\end{bmatrix}.\]
\itema Take $\mathcal{C}=\{3x^2,2x, 1, x^3\}$.
\end{enumerate}
}

\itemB Show that if $\mathcal{E}=\{e_1,\dots,e_n\}$ is the standard basis on $R^n$ and $\mathcal{E}'=\{e_1,\dots,e_m\}$ is the standard basis on $R^m$, then $T(v) =  [T]_{\mathcal{E}}^{\mathcal{E}'} \cdot v$, where the RHS is usual matrix-times-vector multiplication.
\end{enumerate}

%
%
%\begin{framed}
%\textsc{Definition:} Let $R$ be a commutative ring. Let $V$ be a free $R$-module with ordered basis ${\mathcal{B}=\{b_1,\dots,b_n\}}$. For any $v\in V$, let $[v]_{\mathcal{B}}$ denote the vector $(r_1,\dots,r_n)\in R^n$, where ${v=r_1 b_1 + \cdots + r_n b_n}$.
%
%\ 
%
%\textsc{Lemma:} Let $R$ be a commutative ring. Let $V$ be a free $R$-module with ordered basis ${\mathcal{B}=\{b_1,\dots,b_n\}}$ and let $W$ be a free $R$-module with ordered basis ${\mathcal{C}=\{c_1,\dots,c_m\}}$.  Given an \mbox{$R$-module} homomorphism $T\colon V\to W$, for any $v\in V$ we have
%\[ [T(v)]_{\mathcal{C}} = [T]_{\mathcal{B}}^{\mathcal{C}} [v]_{\mathcal{B}}, \quad \text{where the RHS is usual matrix-times-vector multiplication.}\]
%\end{framed}



\begin{framed}
\textsc{Proposition:} Let $R$ be a commutative ring. Let $V$ be a free $R$-module with ordered basis ${\mathcal{B}=\{b_1,\dots,b_n\}}$ and let $W$ be a free $R$-module with ordered basis ${\mathcal{C}=\{c_1,\dots,c_m\}}$. Then the map
\[ \begin{array}{ccc} \mathrm{Hom}_R(V,W) &\longrightarrow &\mathrm{Mat}_{m\times n}(R) \\
T &\longmapsto & [T]_{\mathcal{B}}^{\mathcal{C}}\end{array}\]
is bijective. Moreover, this is an isomorphism of $R$-modules. 

When $V=W$ and ${\mathcal{B}}={\mathcal{C}}$, the same map
\[ \begin{array}{ccc} \mathrm{End}_R(V) &\longrightarrow &\mathrm{Mat}_{n\times n}(R) \\
T &\longmapsto & [T]_{\mathcal{B}}^{\mathcal{B}}\end{array}\]
is an isomorphism of rings.
 \end{framed}


\begin{enumerate}
\setcounter{enumi}{2}

\itemA  Prove that the map $T\mapsto [T]_{\mathcal{B}}^{\mathcal{C}}$ in the Proposition is bijective.

\solution{ To see that this is surjective, we use the UMP for free modules: Given a matrix $A=[a_{i,j}]$, there is an $R$-module homomorphism $\phi:V\to W$ such that $\phi(b_j) = \sum_i a_{i,j} b_i$, and by definition, $[\phi]_{\mathcal{B}}^{\mathcal{C}} = A$.

To see that this is injective, we use the UMP for free modules: Suppose that $[\phi]_{\mathcal{B}}^{\mathcal{C}}=[\psi]_{\mathcal{B}}^{\mathcal{C}} = A$. Then $\phi(b_j) = \sum_i a_{i,j} b_i = \psi(b_j)$ for all $j$. Then the uniqueness part of the UMP says that $\phi=\psi$. This shows that the map is injective.
}

\itemB Suppose that $V$ is a free module with a countably infinite basis $\mathcal{B}= \{b_1,b_2,b_3,\dots\}$, and $W$ is free with a countably infinite basis $\mathcal{C} = \{c_1,c_2,c_3,\dots\}$. What is the analogue of the Proposition in this case?

\

\itemB Prove the Proposition.

\end{enumerate}











\end{document}
