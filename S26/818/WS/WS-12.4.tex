\documentclass[12pt]{amsart}


\usepackage{times}
\usepackage[margin=.8in]{geometry}
\usepackage{paralist,amsmath,amssymb,multicol,graphicx,framed,ifthen,color,xcolor,stmaryrd,enumitem,colonequals}
\usepackage[outline]{contour}
\contourlength{.4pt}
\contournumber{10}
\newcommand{\Bold}[1]{\contour{black}{#1}}

\definecolor{chianti}{rgb}{0.6,0,0}
\definecolor{meretale}{rgb}{0,0,.6}
\definecolor{leaf}{rgb}{0,.35,0}
\newcommand{\Q}{\mathbb{Q}}
\newcommand{\N}{\mathbb{N}}
\newcommand{\Z}{\mathbb{Z}}
\newcommand{\R}{\mathbb{R}}
\newcommand{\C}{\mathbb{C}}
\newcommand{\e}{\varepsilon}
\newcommand{\inv}{^{-1}}
\newcommand{\dabs}[1]{\left| #1 \right|}
\newcommand{\ds}{\displaystyle}
\newcommand{\solution}[1]{\ifthenelse {\equal{\displaysol}{1}} {\begin{framed}{\color{meretale}\noindent #1}\end{framed}} { \ }}
\newcommand{\solutione}[1]{\ifthenelse {\equal{\displaysol}{1}} {\begin{framed}{\color{leaf}This solution is embargoed.}\end{framed}} { \ }}
\newcommand{\showsol}[1]{\def\displaysol{#1}}

\newcommand{\rsa}{\rightsquigarrow}


\newcommand\itemA{\stepcounter{enumi}\item[{\Bold{(\theenumi)}}]}
\newcommand\itemB{\stepcounter{enumi}\item[(\theenumi)]}
\newcommand\itemC{\stepcounter{enumi}\item[{\it{(\theenumi)}}]}
\newcommand\itema{\stepcounter{enumii}\item[{\Bold{(\theenumii)}}]}
\newcommand\itemb{\stepcounter{enumii}\item[(\theenumii)]}
\newcommand\itemc{\stepcounter{enumii}\item[{\it{(\theenumii)}}]}
\newcommand\itemai{\stepcounter{enumiii}\item[{\Bold{(\theenumiii)}}]}
\newcommand\itembi{\stepcounter{enumiii}\item[(\theenumiii)]}
\newcommand\itemci{\stepcounter{enumiii}\item[{\it{(\theenumiii)}}]}
\newcommand\ceq{\colonequals}


\DeclareMathOperator{\ord}{ord}

\DeclareMathOperator{\res}{res}
\setlength\parindent{0pt}
%\usepackage{times}

%\addtolength{\textwidth}{100pt}
%\addtolength{\evensidemargin}{-45pt}
%\addtolength{\oddsidemargin}{-60pt}

\pagestyle{empty}
%\begin{document}\begin{itemize}

%\thispagestyle{empty}




\begin{document}
\showsol{0}
	
	\thispagestyle{empty}
	
	\section*{Determinants}
	
\begin{framed}
\textsc{Definition:} Let $R$ be a commutative ring, and $A\in \mathrm{Mat}_{n\times n}(R)$. The determinant of $M$ is 
\[ \det(A) = \sum_{\sigma \in S_n} \mathrm{sgn}(\sigma) \prod_{i=1}^n x_{i, \sigma(i)}.\]

\


\textsc{Theorem 1:} Identify $\mathrm{Mat}_{n\times n}(R)$ with $\underbrace{R^n \times \cdots \times R^n}_{n \, \text{times}}$ by considering a matrix as an $n$-tuple of columns. The determinant is the unique function 
\[ \det\colon \underbrace{R^n \times \cdots \times R^n}_{n \text{times}}\to R\] that satisfies the following three properties:
\begin{itemize}
\item  $\det$ is \textbf{multilinear}, meaning 
\[\begin{aligned}  \det(v_1,\dots,v_{i-1}, \boldsymbol{v + w},v_{i+1},\dots,v_n) &=  \det(v_1,\dots,v_{i-1}, \boldsymbol{v},v_{i+1},\dots,v_n) \\&  + \det(v_1,\dots,v_{i-1}, \boldsymbol{w},v_{i+1},\dots,v_n) \\ \det(v_1,\dots,v_{i-1}, \boldsymbol{rv},v_{i+1},\dots,v_n) &= \boldsymbol{r} \det(v_1,\dots,v_{i-1}, \boldsymbol{v},v_{i+1},\dots,v_n) \end{aligned} \]
\item $\det$ is \textbf{alternating}, meaning
\[ \det(v_1,\dots,v_n) = 0 \qquad \text{if} \ v_i=v_j \ \text{for some} \ i\neq j.\]
\item $\det(e_1,\dots,e_n) = 1$.
\end{itemize}

\


\textsc{Lemma:} Let $F\colon \underbrace{R^n \times \cdots \times R^n}_{n \text{times}}\to R$ be an alternating multilinear function. Then \[ F(v_{\sigma(1)},\dots,v_{\sigma(n)}) = \mathrm{sgn}(\sigma) F(v_1,\dots,v_n).\]
 \end{framed}

\begin{enumerate}
\itemA 

\end{enumerate}




\begin{framed}
\textsc{Theorem 2:} Let $R$ be a commutative ring and $A,B\in \mathrm{Mat}_{n\times n}(R)$. Then \[ \det(AB) = \det(A) \det(B).\]

\

\textsc{Proposition:} Let $R$ be a commutative ring. Let $A$ be a square matrix, and $B$ be a matrix obtained from $A$ by an elementary column operation.
\begin{itemize}
\item For the operation ``add $r\in R$ times column $i$ to column $j$'' we have $\det(B) = \det(A)$.
\item For the operation ``multiply column $i$ by $u\in R^\times$'' we have  $\det(B) = u \det(A)$.
\item For the operation ``swap column $i$ and column $j$'' we have  $\det(B) = -\det(A)$.
\end{itemize}
 \end{framed}


\begin{enumerate}
\setcounter{enumi}{2}

\itemA  Prove the Proposition.

\itemA Proof of Theorem in the case $R=F$ is a field:
\begin{enumerate}
\itema Use the Proposition (and the fact that over a field every invertible matrix is a product of elementary matrices) that if $A$ is invertible, then $\det(A)\neq 0$.
\itema Show\footnote{Hint: First show that the columns of $A$ are linearly dependent, and express some $v_i$ as a linear combination of the others.} that if $A$ is not invertible, then $\det(A)=0$. 
\itema 
\end{enumerate}

\end{enumerate}









\end{document}
