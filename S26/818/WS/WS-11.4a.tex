\documentclass[12pt]{amsart}


\usepackage{times}
\usepackage[margin=0.65in]{geometry}
\usepackage{paralist,amsmath,amssymb,multicol,graphicx,framed,ifthen,color,xcolor,stmaryrd,enumitem,colonequals}
\usepackage[outline]{contour}
\contourlength{.4pt}
\contournumber{10}
\newcommand{\Bold}[1]{\contour{black}{#1}}

\definecolor{chianti}{rgb}{0.6,0,0}
\definecolor{meretale}{rgb}{0,0,.6}
\definecolor{leaf}{rgb}{0,.35,0}
\newcommand{\Q}{\mathbb{Q}}
\newcommand{\N}{\mathbb{N}}
\newcommand{\Z}{\mathbb{Z}}
\newcommand{\R}{\mathbb{R}}
\newcommand{\C}{\mathbb{C}}
\newcommand{\e}{\varepsilon}
\newcommand{\inv}{^{-1}}
\newcommand{\dabs}[1]{\left| #1 \right|}
\newcommand{\ds}{\displaystyle}
\newcommand{\solution}[1]{\ifthenelse {\equal{\displaysol}{1}} {\begin{framed}{\color{meretale}\noindent #1}\end{framed}} { \ }}
\newcommand{\solutione}[1]{\ifthenelse {\equal{\displaysol}{1}} {\begin{framed}{\color{leaf}This solution is embargoed.}\end{framed}} { \ }}
\newcommand{\showsol}[1]{\def\displaysol{#1}}

\newcommand{\rsa}{\rightsquigarrow}


\newcommand\itemA{\stepcounter{enumi}\item[{\Bold{(\theenumi)}}]}
\newcommand\itemB{\stepcounter{enumi}\item[(\theenumi)]}
\newcommand\itemC{\stepcounter{enumi}\item[{\it{(\theenumi)}}]}
\newcommand\itema{\stepcounter{enumii}\item[{\Bold{(\theenumii)}}]}
\newcommand\itemb{\stepcounter{enumii}\item[(\theenumii)]}
\newcommand\itemc{\stepcounter{enumii}\item[{\it{(\theenumii)}}]}
\newcommand\itemai{\stepcounter{enumiii}\item[{\Bold{(\theenumiii)}}]}
\newcommand\itembi{\stepcounter{enumiii}\item[(\theenumiii)]}
\newcommand\itemci{\stepcounter{enumiii}\item[{\it{(\theenumiii)}}]}
\newcommand\ceq{\colonequals}


\DeclareMathOperator{\ord}{ord}

\DeclareMathOperator{\res}{res}
\setlength\parindent{0pt}
%\usepackage{times}

%\addtolength{\textwidth}{100pt}
%\addtolength{\evensidemargin}{-45pt}
%\addtolength{\oddsidemargin}{-60pt}

\pagestyle{empty}
%\begin{document}\begin{itemize}

%\thispagestyle{empty}




\begin{document}
\showsol{0}
	
	\thispagestyle{empty}
	
	\section*{Generators, Linear dependence, and Bases}
	
	

\begin{framed}
\textsc{Definition:} Let $R$ be a ring and $M$ be a (left) $R$-module. A linear combination of finitely many elements $m_1,\dots,m_n$ of $M$ is an element of the form $r_1 m_1 + \cdots + r_n m_n \in M$ for some $r_1,\dots,r_n\in R$. %Note that infinite linear combinations don't make sense in general.

\

\textsc{Definition:} Let $R$ be a ring and $M$ be a (left) $R$-module. Let $A$ be a subset of $M$. The submodule of $M$ \textbf{generated by} $A$ is the submodule $RA$ of $M$ given by the three equivalent following descriptions:
\begin{itemize}
\item $RA$ is the unique smallest $R$-submodule of $M$ containing $A$.
\item $\displaystyle RA = \bigcap N_\lambda$, where $N_\lambda$ ranges over all submodules of $M$ containing $A$.
\item $RA = \{ r_1 m_1 + \cdots + r_t m_t \ | \ r_i\in R, m_i\in A\}$, the set of linear combinations of elements of $A$.
\end{itemize}

\

\textsc{Definition:} Let $R$ be a ring and $M$ be a (left) $R$-module. Let $A$ be a subset of $M$.
\begin{itemize}
\item We say that $A$ \textbf{generates} $M$ if $RA=M$.
\item We say that $A$ is \textbf{linearly independent} if for $m_1,\dots,m_t\in A$ distinct and any $r_1,\dots,r_t\in R$,
\[ r_1 m_1 + \cdots + r_t m_t = 0 \quad \text{implies} \quad r_1=\cdots=r_t = 0.\]
\item We say that $A$ is a \textbf{basis} of $M$ if $A$ is linearly independent and generates $M$.
\item We say that $M$ is \textbf{free} if there exists a basis $A$ for $M$.
\end{itemize}
\end{framed}
\footnotetext[1]{This includes $0$ as the ``empty sum''.}

\medskip

\begin{enumerate}
\itemA Let $R=\Z$ and consider the $R$-module $M=\Z/n$ for some $n>1$.
\begin{enumerate}
\itema Explain why any nonempty subset of $M$ is \emph{not} linearly independent.
\itema Explain why $M$ is \emph{not} a free module.
\itema An $R$-module is \textbf{cyclic} if it is generated by a single element. Show that $M$ is cyclic.
\itema Does every generating set of $M$ consist of a single element?
\end{enumerate}

\

\itemA Let $R$ be a commutative ring. Let $R[x]$ be a polynomial ring over $R$.
\begin{enumerate}
\itema Explain why $\{1,x,x^2,x^3,\dots\}$ is a basis for $R[x]$ as an $R$-module.
\itema Give an example of a set that is $R$-linearly independent in $R[x]$ that is not a basis.
\itema Give an example of a set that generates  $R[x]$ that is not a basis.
\itema Give a different example of a basis for $R[x]$.
\end{enumerate}

\

\itemA Show that an $R$-module $M$ is cyclic if and only if $M\cong R/I$ for some left ideal $I$.

\




\itemB Let $R=\Z[x]$ and $I$ be the ideal $(2,x)$, considered as an $R$-module.
\begin{enumerate}
\itemb Explain\footnote[2]{Reuse something from 817!} why $I$ is not cyclic.
\itemb Show that $I$ is not free.
\itemb Give an example of a pair of modules $N\subseteq M$ where $N$ requires more generators than $M$.
\itemb Give an example of a pair of modules $N\subseteq M$ where $M$ is free and $N$ is not.
\end{enumerate}

\

\itemB We say that an $R$-module $M$ is \textbf{simple} if the only submodules of $M$ are $0$ and $M$. Let $R$ be a commutative ring. Show that $M$ is simple if and only if $M\cong R/\mathfrak{m}$ for some maximal ideal $\mathfrak{m}$ of $R$.

\

\itemB Let $R$ be a commutative ring, and $x,y$ be two indeterminates.
\begin{enumerate}
\itemb Show that $R[x,y]$ is a free $R[x]$-module, and find a basis.
\itemb Show that $R[x,y]$ is a free $R$ module, and find a basis.
\end{enumerate}
\end{enumerate}













\end{document}
