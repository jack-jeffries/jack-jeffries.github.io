\documentclass[12pt]{amsart}


\usepackage{times}
\usepackage[margin=0.8in]{geometry}
\usepackage{paralist,amsmath,amssymb,multicol,graphicx,framed,ifthen,color,xcolor,stmaryrd,enumitem,colonequals}
\usepackage[outline]{contour}
\contourlength{.4pt}
\contournumber{10}
\newcommand{\Bold}[1]{\contour{black}{#1}}

\definecolor{chianti}{rgb}{0.6,0,0}
\definecolor{meretale}{rgb}{0,0,.6}
\definecolor{leaf}{rgb}{0,.35,0}
\newcommand{\Q}{\mathbb{Q}}
\newcommand{\N}{\mathbb{N}}
\newcommand{\Z}{\mathbb{Z}}
\newcommand{\R}{\mathbb{R}}
\newcommand{\C}{\mathbb{C}}
\newcommand{\e}{\varepsilon}
\newcommand{\inv}{^{-1}}
\newcommand{\dabs}[1]{\left| #1 \right|}
\newcommand{\ds}{\displaystyle}
\newcommand{\solution}[1]{\ifthenelse {\equal{\displaysol}{1}} {\begin{framed}{\color{meretale}\noindent #1}\end{framed}} { \ }}
\newcommand{\solutione}[1]{\ifthenelse {\equal{\displaysol}{1}} {\begin{framed}{\color{leaf}This solution is embargoed.}\end{framed}} { \ }}
\newcommand{\showsol}[1]{\def\displaysol{#1}}

\newcommand{\rsa}{\rightsquigarrow}


\newcommand\itemA{\stepcounter{enumi}\item[{\Bold{(\theenumi)}}]}
\newcommand\itemB{\stepcounter{enumi}\item[(\theenumi)]}
\newcommand\itemC{\stepcounter{enumi}\item[{\it{(\theenumi)}}]}
\newcommand\itema{\stepcounter{enumii}\item[{\Bold{(\theenumii)}}]}
\newcommand\itemb{\stepcounter{enumii}\item[(\theenumii)]}
\newcommand\itemc{\stepcounter{enumii}\item[{\it{(\theenumii)}}]}
\newcommand\itemai{\stepcounter{enumiii}\item[{\Bold{(\theenumiii)}}]}
\newcommand\itembi{\stepcounter{enumiii}\item[(\theenumiii)]}
\newcommand\itemci{\stepcounter{enumiii}\item[{\it{(\theenumiii)}}]}
\newcommand\ceq{\colonequals}


\DeclareMathOperator{\ord}{ord}

\DeclareMathOperator{\res}{res}
\setlength\parindent{0pt}
%\usepackage{times}

%\addtolength{\textwidth}{100pt}
%\addtolength{\evensidemargin}{-45pt}
%\addtolength{\oddsidemargin}{-60pt}

\pagestyle{empty}
%\begin{document}\begin{itemize}

%\thispagestyle{empty}




\begin{document}
\showsol{0}
	
	\thispagestyle{empty}
	
	\section*{Free modules}
	
	

\begin{framed}

Bases and unique expression

UMP for free modules

\end{framed}

\begin{enumerate}
\itemA 

\end{enumerate}



\begin{framed}
\textsc{Theorem 2:} Let $R$ be a ring. Let $F$ be a free module with a basis $B$, and $F'$ be a free module with a basis $B'$.
\begin{enumerate}
\item If $|B| = |B'|$ (meaning there is a set bijection between $B$ and $B'$), then $F\cong F'$.
\item Let $R$ be a commutative ring. If $F\cong F'$, then $|B|=|B'|$.
\end{enumerate}

\

\textsc{Definition:} Let $R$ be a commutative ring, and $F$ be a free module. The \textbf{rank} of $F$ is the size of a basis $B$ of $F$. 
\end{framed}



\begin{enumerate}
\itemA Theorem 2:
\begin{enumerate}
\itema What about the Definition above needs justification? Use Theorem 2 to justify it.
\itema Prove part (1) of Theorem 2. (We will prove part (2) later as a consequence of the same result in the special case of vector spaces.)
\end{enumerate}

\

\itemB Let $R = M_{\infty}(\R)$ be the ring of countably infinite matrices with real entries:
\[ M_{\infty}(\R) =\{  [a_{ij}]_{\substack{i=1,2,3,\dots \\ j=1,2,3,\dots}} \ | \ a_{ij}\neq 0 \ \text{for at most finitely many pairs} \ (i,j) \}\]
with usual matrix addition and multiplication; you do not have to prove that this is a ring. Prove\footnote{Hint: Consider the map sending a matrix $[a_{ij}]$ to the pair of matrices $([a_{i,2j-1}],[a_{i,2j}])$ reconstituted from its odd columns and its even columns.} that $R^1 \cong R^2$ as $R$-modules. What does this say about Theorem 2?

\end{enumerate}








\end{document}
