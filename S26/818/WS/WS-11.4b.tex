\documentclass[12pt]{amsart}


\usepackage{times}
\usepackage[margin=0.75in]{geometry}
\usepackage{paralist,amsmath,amssymb,multicol,graphicx,framed,ifthen,color,xcolor,stmaryrd,enumitem,colonequals}
\usepackage[outline]{contour}
\contourlength{.4pt}
\contournumber{10}
\newcommand{\Bold}[1]{\contour{black}{#1}}

\definecolor{chianti}{rgb}{0.6,0,0}
\definecolor{meretale}{rgb}{0,0,.6}
\definecolor{leaf}{rgb}{0,.35,0}
\newcommand{\Q}{\mathbb{Q}}
\newcommand{\N}{\mathbb{N}}
\newcommand{\Z}{\mathbb{Z}}
\newcommand{\R}{\mathbb{R}}
\newcommand{\C}{\mathbb{C}}
\newcommand{\e}{\varepsilon}
\newcommand{\inv}{^{-1}}
\newcommand{\dabs}[1]{\left| #1 \right|}
\newcommand{\ds}{\displaystyle}
\newcommand{\solution}[1]{\ifthenelse {\equal{\displaysol}{1}} {\begin{framed}{\color{meretale}\noindent #1}\end{framed}} { \ }}
\newcommand{\solutione}[1]{\ifthenelse {\equal{\displaysol}{1}} {\begin{framed}{\color{leaf}This solution is embargoed.}\end{framed}} { \ }}
\newcommand{\showsol}[1]{\def\displaysol{#1}}

\newcommand{\rsa}{\rightsquigarrow}


\newcommand\itemA{\stepcounter{enumi}\item[{\Bold{(\theenumi)}}]}
\newcommand\itemB{\stepcounter{enumi}\item[(\theenumi)]}
\newcommand\itemC{\stepcounter{enumi}\item[{\it{(\theenumi)}}]}
\newcommand\itema{\stepcounter{enumii}\item[{\Bold{(\theenumii)}}]}
\newcommand\itemb{\stepcounter{enumii}\item[(\theenumii)]}
\newcommand\itemc{\stepcounter{enumii}\item[{\it{(\theenumii)}}]}
\newcommand\itemai{\stepcounter{enumiii}\item[{\Bold{(\theenumiii)}}]}
\newcommand\itembi{\stepcounter{enumiii}\item[(\theenumiii)]}
\newcommand\itemci{\stepcounter{enumiii}\item[{\it{(\theenumiii)}}]}
\newcommand\ceq{\colonequals}


\DeclareMathOperator{\ord}{ord}

\DeclareMathOperator{\res}{res}
\setlength\parindent{0pt}
%\usepackage{times}

%\addtolength{\textwidth}{100pt}
%\addtolength{\evensidemargin}{-45pt}
%\addtolength{\oddsidemargin}{-60pt}

\pagestyle{empty}
%\begin{document}\begin{itemize}

%\thispagestyle{empty}




\begin{document}
\showsol{0}
	
	\thispagestyle{empty}
	
	\section*{Free modules}
	
	

\begin{framed}
\textsc{Proposition:} Let $R$ be a ring and $F$ be a free module with basis $B$. Then every element of $f\in F$ admits a unique expression as a linear combination\footnotemark[1] of elements of $B$.

\

\textsc{Universal mapping property for free modules:} Let $R$ be a ring and $F$ be a free module with basis $B$. Let $N$ be an arbitrary $R$-module. Then for any function $j:B\to N$, there is a unique $R$-module homomorphism $h:F\to N$ such that $h(b) = j(b)$ for all $b\in B$.
\end{framed}
\footnotetext[1]{Recall that a linear combination of $B$ is a sum of the form $r_1 b_1 + \cdots + r_n b_n$ for some finite list of elements $b_1,\dots,b_n\in B$ and $r_1,\dots,r_n\in R$.}

\begin{enumerate}
\itemA Let $R$ be a ring and $n\in \Z_{>0}$. The \textbf{standard free module of rank $\boldsymbol{n}$} and its \textbf{standard basis} are, respectively,
\[ {\small R^n = \left\{ \begin{bmatrix} r_1\\ r_2 \\ \vdots \\ r_n \end{bmatrix} \ \Big| \ r_i\in R \right\} \quad \text{and} \quad \text{the set with elements} \  e_1 = \begin{bmatrix} 1 \\ 0 \\  \vdots \\ 0 \end{bmatrix}, e_2 = \begin{bmatrix} 0 \\ 1 \\  \vdots \\ 0 \end{bmatrix}, \dots, e_n =  \begin{bmatrix} 0 \\ 0 \\  \vdots \\ 1 \end{bmatrix}.}\]
We also write elements in the form $(r_1,\dots, r_n)$.
\begin{enumerate}
\itema Let $R=\Z[x]$ and $M=R^3$. Give the unique expression of $v= (2x + 3, 1, x^4)$ as a linear combination of the standard basis. 
\itema Let $R=\Z[x]$, and $M=R^3$, and $N=\Z/5[x]$. Let $h:M\to N$ be the unique $R$-linear map such that $h(e_1) =[2]$, $h(e_2) =[0]$, and $h(e_3)=x$. Compute $h(v)$.
\end{enumerate}

\

\itemA Proving things.
\begin{enumerate}
\itema Prove the Proposition above.
\itema Prove the UMP for free modules above.
\end{enumerate}




\end{enumerate}



\begin{framed}
\textsc{Theorem:} Let $R$ be a ring. Let $F$ be a free module with a basis $B$, and $F'$ be a free module with a basis $B'$.
\begin{enumerate}
\item If $|B| = |B'|$, meaning there is a set bijection between $B$ and $B'$, then $F\cong F'$.
\item Let $R$ be a commutative ring. If $F\cong F'$, then $|B|=|B'|$.
\end{enumerate}

\

\textsc{Definition:} Let $R$ be a commutative ring, and $F$ be a free module. The \textbf{rank} of $F$ is the size of a basis $B$ of $F$. 
\end{framed}



\begin{enumerate}
\setcounter{enumi}{2}
\itemA Rank:
\begin{enumerate}
\itema What about the Definition above needs justification? Use the Theorem to justify it.
\itema Prove part (1) of Theorem 2. (We will prove part (2) later as a consequence of the same result in the special case of vector spaces.)
\end{enumerate}

\

\itemB Let $A = M_{\infty}(\R)$ be the ring of countably infinite matrices with real entries:
\[ M_{\infty}(\R) =\{  [a_{ij}]_{\substack{i=1,2,3,\dots \\ j=1,2,3,\dots}} \ | \ a_{ij}\neq 0 \ \text{for at most finitely many pairs} \ (i,j) \}\]
with usual matrix addition and multiplication; you do not have to prove that this is a ring. Prove\footnote[2]{Hint: Consider the map sending a matrix $[a_{ij}]$ to the pair of matrices $([a_{i,2j-1}],[a_{i,2j}])$ reconstituted from its odd columns and its even columns.} that $A^1 \cong A^2$ as $A$-modules. What does this say about the Theorem?

\end{enumerate}








\end{document}
