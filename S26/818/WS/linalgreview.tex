\documentclass[12pt]{amsart}


\usepackage{times}
\usepackage[margin=.9in]{geometry}
\usepackage{paralist,amsmath,amssymb,multicol,graphicx,framed,ifthen,color,xcolor,stmaryrd,colonequals}
\usepackage[shortlabels]{enumitem}
\usepackage[all]{xy}
\usepackage[outline]{contour}
\contourlength{.4pt}
\contournumber{10}
\newcommand{\Bold}[1]{\contour{black}{#1}}

\definecolor{chianti}{rgb}{0.6,0,0}
\definecolor{meretale}{rgb}{0,0,.6}
\definecolor{leaf}{rgb}{0,.35,0}
\newcommand{\Q}{\mathbb{Q}}
\newcommand{\N}{\mathbb{N}}
\newcommand{\Z}{\mathbb{Z}}
\newcommand{\R}{\mathbb{R}}
\newcommand{\C}{\mathbb{C}}
\newcommand{\e}{\varepsilon}
\newcommand{\inv}{^{-1}}
\newcommand{\dabs}[1]{\left| #1 \right|}
\newcommand{\ds}{\displaystyle}
\newcommand{\solution}[1]{\ifthenelse {\equal{\displaysol}{1}} {\begin{framed}{\color{meretale}\noindent #1}\end{framed}} { \ }}
\newcommand{\solutione}[1]{\ifthenelse {\equal{\displaysol}{1}} {\begin{framed}{\color{leaf}This solution is embargoed.}\end{framed}} { \ }}
\newcommand{\showsol}[1]{\def\displaysol{#1}}

\newcommand{\rsa}{\rightsquigarrow}


\newcommand\itemA{\stepcounter{enumi}\item[{\Bold{(\theenumi)}}]}
\newcommand\itemB{\stepcounter{enumi}\item[(\theenumi)]}
\newcommand\itemC{\stepcounter{enumi}\item[{\it{(\theenumi)}}]}
\newcommand\itema{\stepcounter{enumii}\item[{\Bold{(\theenumii)}}]}
\newcommand\itemb{\stepcounter{enumii}\item[(\theenumii)]}
\newcommand\itemc{\stepcounter{enumii}\item[{\it{(\theenumii)}}]}
\newcommand\itemai{\stepcounter{enumiii}\item[{\Bold{(\theenumiii)}}]}
\newcommand\itembi{\stepcounter{enumiii}\item[(\theenumiii)]}
\newcommand\itemci{\stepcounter{enumiii}\item[{\it{(\theenumiii)}}]}
\newcommand\ceq{\colonequals}
\newcommand\blank[1]{\underline{\phantom{ #1 }}}
\newcommand\Ar{$\Longrightarrow$ }
\newcommand\Arr{$\Longleftrightarrow$ }

\DeclareMathOperator{\ord}{ord}

\DeclareMathOperator{\res}{res}
\setlength\parindent{0pt}
%\usepackage{times}

%\addtolength{\textwidth}{100pt}
%\addtolength{\evensidemargin}{-45pt}
%\addtolength{\oddsidemargin}{-60pt}

\pagestyle{empty}
%\begin{document}\begin{itemize}

%\thispagestyle{empty}




\begin{document}
\showsol{0}
	
	\thispagestyle{empty}
	
	\section*{Linear algebra review}

	
	\begin{framed}Determine whether each statement is true when $R$ is a field, is a PID, or is an arbitrary commutative ring (CR). Throughout, $M, N$ are $R$-module and $A, B$ are matrices.\end{framed}
	
	\
	
	\begin{tabular}{l|c|c|c|}
	& Field & PID & CR \\ 
(1) Every $M$ is free. &&&\\ 
(2) If $\mathrm{ann}_R(M)=0$, then $M$ is free.&&&\\ 
(3) If $M$ is finitely generated and $\mathrm{ann}_R(M)=0$, then $M$ is free.&&&\\ 
(4) If $M$ is finitely generated and free and $N\subseteq M$, then $N$ is free. &&&\\ 
(5) If $M$ is fin.~generated and $N\subseteq M$, then $N$ is fin.~generated.&&&\\ 
(6) If $M$ is fin.~generated, $N\subseteq M$, and $N\cong M$, then $N=M$.&&&\\ 
(7) If $M$ is fin.~generated and $f:M\to M$ is injective, then $f$ is an iso.&&&\\ 
(8) If $M$ is fin.~generated and $f:M\to M$ is surjective, then $f$ is an iso.&&&\\ 
(9) For any matrix $A$ there are invertible $P,Q$ s.t.~$PAQ$ is diagonal.&&&\\
(10) Any matrix $A$ can be turned into a diagonal matrix with EROs and ECOs.&&&\\
(11) Any invertible matrix $A$ can be turned into a diagonal matrix with EROs.&&&\\
(12) Any matrix $A$ can be turned into a diagonal matrix with EROs.&&&\\
(13) An $n\times n$ matrix $A$ is invertible if and only if $\det(A)\neq 0$.&&&\\
(14) An $n\times n$ matrix $A$ is invertible if and only if the columns of $A$ are LI.&&&\\
(15) An $n\times n$ matrix $A$ is invertible if and only if $\exists B$ with $AB=I_n$.&&&\\
(16) An $n\times n$ matrix $A$ is invertible if and only if $\exists B$ with $BA=I_n$.&&&\\
(17) If $M$ is free and $S\subseteq M$ is LI, then $S$ is a subset of a basis.&&&\\
(18) If $M$ is free and $S\subseteq M$ generates $M$, then $S$ contains a basis.&&&\\
(19) If $M\cong N$ are free then $\mathrm{rank}(M) =\mathrm{rank}(N)$. &&&\\
(20) If $M\twoheadrightarrow N$ are free then $\mathrm{rank}(M)\geq \mathrm{rank}(N)$. &&&\\
%(17) If $M\subseteq N$ are free then $\mathrm{rank}(M)\leq \mathrm{rank}(N)$. &&&\\
(21) If $M\subseteq N$ and $N$ can be generated by $n$ elements then so can $M$. &&&\\
(22) If $M\subseteq N$ and $N$ can be generated by $n$ elements then so can $N/M$. &&&\\
(23) $I_t(AB) \subseteq I_t(A) \cap I_t(B)$.
	\end{tabular}
	
	
\end{document}
