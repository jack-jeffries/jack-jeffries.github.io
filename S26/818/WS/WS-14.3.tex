\documentclass[12pt]{amsart}


\usepackage{times}
\usepackage[margin=.5in]{geometry}
\usepackage{paralist,amsmath,amssymb,multicol,graphicx,framed,ifthen,color,xcolor,stmaryrd,enumitem,colonequals}
\usepackage[outline]{contour}
\contourlength{.4pt}
\contournumber{10}
\newcommand{\Bold}[1]{\contour{black}{#1}}

\definecolor{chianti}{rgb}{0.6,0,0}
\definecolor{meretale}{rgb}{0,0,.6}
\definecolor{leaf}{rgb}{0,.35,0}
\newcommand{\Q}{\mathbb{Q}}
\newcommand{\N}{\mathbb{N}}
\newcommand{\Z}{\mathbb{Z}}
\newcommand{\R}{\mathbb{R}}
\newcommand{\C}{\mathbb{C}}
\newcommand{\e}{\varepsilon}
\newcommand{\inv}{^{-1}}
\newcommand{\dabs}[1]{\left| #1 \right|}
\newcommand{\ds}{\displaystyle}
\newcommand{\solution}[1]{\ifthenelse {\equal{\displaysol}{1}} {\begin{framed}{\color{meretale}\noindent #1}\end{framed}} { \ }}
\newcommand{\solutione}[1]{\ifthenelse {\equal{\displaysol}{1}} {\begin{framed}{\color{leaf}This solution is embargoed.}\end{framed}} { \ }}
\newcommand{\showsol}[1]{\def\displaysol{#1}}
\newcommand{\nosolfill}{\ifthenelse {\equal{\displaysol}{1}} {} {\vfill}}
\newcommand{\nosolpage}{\ifthenelse {\equal{\displaysol}{1}} {} {\newpage}}
\newcommand{\nosolonly}[1]{\ifthenelse {\equal{\displaysol}{1}} {} {{#1}}}

\newcommand{\rsa}{\rightsquigarrow}


\newcommand\itemA{\stepcounter{enumi}\item[{\Bold{(\theenumi)}}]}
\newcommand\itemB{\stepcounter{enumi}\item[(\theenumi)]}
\newcommand\itemC{\stepcounter{enumi}\item[{\it{(\theenumi)}}]}
\newcommand\itema{\stepcounter{enumii}\item[{\Bold{(\theenumii)}}]}
\newcommand\itemb{\stepcounter{enumii}\item[(\theenumii)]}
\newcommand\itemc{\stepcounter{enumii}\item[{\it{(\theenumii)}}]}
\newcommand\itemai{\stepcounter{enumiii}\item[{\Bold{(\theenumiii)}}]}
\newcommand\itembi{\stepcounter{enumiii}\item[(\theenumiii)]}
\newcommand\itemci{\stepcounter{enumiii}\item[{\it{(\theenumiii)}}]}
\newcommand\ceq{\colonequals}


\DeclareMathOperator{\ord}{ord}

\DeclareMathOperator{\res}{res}
\setlength\parindent{0pt}
%\usepackage{times}

%\addtolength{\textwidth}{100pt}
%\addtolength{\evensidemargin}{-45pt}
%\addtolength{\oddsidemargin}{-60pt}

\pagestyle{empty}
%\begin{document}\begin{itemize}

%\thispagestyle{empty}




\begin{document}
\showsol{0}
	
	\thispagestyle{empty}
	
	\section*{Cayley-Hamilton and classifying matrices up to similarity}
	\begin{framed}
	\textsc{Definition:} Let $F$ be a field, $V$ an $F$-vector space of dimension $n$, and $\phi:V\to V$ a linear transformation.
	\begin{itemize}
	\item The \textbf{characteristic polynomial} of $\phi$ is the polynomial $c_\phi(x) := \det(xI_n  - [\phi]_B^B)$ for some/any basis $B$ of $V$.
	\item The \textbf{minimal polynomial} of $\phi$ is the monic generator $m_\phi(x)$ of the ideal $\mathrm{ann}_{F[x]}(V_\phi)$. Equivalently, $m_\phi(x)$ is the monic polynomial of smallest degree such that $m_\phi(\phi)=0$.
\end{itemize}
We write $c_A(x):=c_{t_A}(x)$ and $m_A(x):=m_{t_A}(x)$ for a matrix $A$.

\

\textsc{Proposition:} Let $F$ be a field, $V$ an $F$-vector space of dimension $n$, and $\phi:V\to V$ a linear transformation. Let $g_1 \ |\  \cdots\  |\ g_k$ be the invariant factors of $\phi$.
\begin{enumerate}
\item $m_\phi(x)=g_k$.
\item $c_\phi(x)=g_1 \cdots g_k$.
\item $\deg(g_1) + \cdots + \deg(g_k) = n$.
\end{enumerate}

\medskip

\textsc{Corollary (Cayley-Hamilton):} $m_\phi(x) \  | \ c_\phi(x)$.

\

\textsc{Theorem:} Let $F$ be a field, $V$ an $F$-vector space of dimension $n$, and $\phi:V\to V$ a linear transformation. For $\lambda\in F$, \  $\lambda \ \text{is an eigenvalue of $\phi$} \  \Longleftrightarrow \ m_\phi(\lambda) = 0 \  \Longleftrightarrow \ c_\phi(\lambda) = 0.$
	\end{framed}

	
	\begin{enumerate}
	\itemA Let $A={\small\begin{bmatrix} -11 & -4 & -2 \\ 18 & 7 & 3 \\ 18 & 6 & 4\end{bmatrix}}\in \mathrm{Mat}_3(\Q)$. The SNF of $xI-A\in \mathrm{Mat}_3(\Q[x])$ is ${\small\begin{bmatrix} 1 & 0 & 0 \\ 0& x-1  & 0 \\ 0 & 0 & x^2+x-2\end{bmatrix}}$. Use the results above to answer the following:
	\begin{enumerate}
	\itema What is the minimal polynomial of $A$?
	\itema What is the characteristic polynomial of $A$?
	\itema What is the RCF of $A$?
	\itema What are the eigenvalues of $A$?
	\end{enumerate}
	

\
	
	
	\itemA Let $F$ be a field, and $A\in \mathrm{Mat}_{n}(F)$ be a nilpotent matrix, meaning that $A^t=0$ for some $t\geq 1$. 
	\begin{enumerate}
	\itema Prove that $A^n=0$.
	\itema If $n=4$, what are the possible lists of invariant factors?
	\itema For $n=4$, give a complete and nonredudant list of representatives of similarity classes of nilpotent matrices.
	\end{enumerate}
	
	\
	
	\itemA Let $f(x)=(x^2-1)(x^4-1) \in \mathbb{Q}[x]$.
	\begin{enumerate}
	\itema If $A\in \mathrm{Mat}_6(\Q)$ has characteristic polynomial $c_A(x) = f(x)$, then what are the possible lists of invariant factors of $A$?
	\itema Give a complete and nonredudant list of representatives of similarity classes of rational matrices with characteristic polynomial $f$.
	\end{enumerate}


\

\itemA Let $F$ be a field.
\begin{enumerate}
	\itema Let A and B be two $3 \times 3$ matrices with entries in $F$. Prove A and B are similar if and only if they have the same characteristic polynomial and the same minimum polynomial.
\itema Show, by way of an example with justification, that the statement in part (a) would become false if $3 \times 3$ were replaced by $4 \times 4$.
\end{enumerate}

\

\itemA Prove\footnote{Hint: You did most of the work for (1) in a homework problem.} the Proposition and the Theorem.

\

\itemA Give a complete and nonredundant list of representatives for the conjugacy classes of $\mathrm{GL}_3(\Z/2)$.




	
	\end{enumerate}

\end{document}
