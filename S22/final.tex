\documentclass{amsart}[12pt]
\usepackage{graphicx}
\usepackage{comment}
\usepackage{amscd,mathabx}
\usepackage{amssymb,setspace}
\usepackage{latexsym,amsfonts,amssymb,amsthm,amsmath,amscd,stmaryrd,mathrsfs}
\usepackage[all, knot]{xy}
\usepackage[top=1in, bottom=1in, left=1in, right=1in]{geometry}
\xyoption{all}
\xyoption{arc}
%\usepackage{hyperref}


%\usepackage[notcite,notref]{showkeys}
 
%\CompileMatricesx
\newcommand{\edit}[1]{\marginpar{\footnotesize{#1}}}
%\newcommand{\edit}[1]{}
\newcommand{\rperf}[2]{\operatorname{RPerf}(#1 \into #2)}



\newcommand{\vectwo}[2]{\begin{bmatrix} #1 \\ #2 \end{bmatrix}}

\newcommand{\vecfour}[4]{\begin{bmatrix} #1 \\ #2 \\ #3 \\ #4 \end{bmatrix}}

\newcommand{\Cat}[1]{\left<\left< \text{#1} \right>\right>}


\def\htpy{\simeq_{\mathrm{htpc}}}
\def\tor{\text{ or }}
\def\fg{finitely generated~}

\def\Ass{\operatorname{Ass}}
\def\ann{\operatorname{ann}}
\def\sign{\operatorname{sign}}

\def\ob{{\mathfrak{ob}} }
\def\BiAdd{\operatorname{BiAdd}}
\def\BiLin{\operatorname{BiLin}}

\def\Syl{\operatorname{Syl}}
\def\span{\operatorname{span}}

\def\sdp{\rtimes}
\def\cL{\mathcal L}
\def\cR{\mathcal R}



\def\ay{??}
\def\Aut{\operatorname{Aut}}
\def\End{\operatorname{End}}
\def\Mat{\operatorname{Mat}}


\def\a{\alpha}



\def\etale{\'etale~}
\def\tW{\tilde{W}}
\def\tH{\tilde{H}}
\def\tC{\tilde{C}}
\def\tS{\tilde{S}}
\def\tX{\tilde{X}}
\def\tZ{\tilde{Z}}
\def\HBM{H^{\text{BM}}}
\def\tHBM{\tilde{H}^{\text{BM}}}
\def\Hc{H_{\text{c}}}
\def\Hs{H_{\text{sing}}}
\def\cHs{{\mathcal H}_{\text{sing}}}
\def\sing{{\text{sing}}}
\def\Hms{H^{\text{sing}}}
\def\Hm{\Hms}
\def\tHms{\tilde{H}^{\text{sing}}}
\def\Grass{\operatorname{Grass}}
\def\image{\operatorname{im}}
\def\im{\image}
\def\ker{\operatorname{ker}}
\def\cone{\operatorname{cone}}
\newcommand{\Hom}{\mathrm{Hom}}


\def\ku{ku}
\def\bbu{\bf bu}
\def\KR{K{\mathbb R}}

\def\CW{\underline{CW}}
\def\cP{\mathcal P}
\def\cE{\mathcal E}
\def\cL{\mathcal L}
\def\cJ{\mathcal J}
\def\cJmor{\cJ^\mor}
\def\ctJ{\tilde{\mathcal J}}
\def\tPhi{\tilde{\Phi}}
\def\cA{\mathcal A}
\def\cB{\mathcal B}
\def\cC{\mathcal C}
\def\cZ{\mathcal Z}
\def\cD{\mathcal D}
\def\cF{\mathcal F}
\def\cG{\mathcal G}
\def\cO{\mathcal O}
\def\cI{\mathcal I}
\def\cS{\mathcal S}
\def\cT{\mathcal T}
\def\cM{\mathcal M}
\def\cN{\mathcal N}
\def\cMpc{{\mathcal M}_{pc}}
\def\cMpctf{{\mathcal M}_{pctf}}
\def\L{\Lambda}

\def\sA{\mathscr A}
\def\sB{\mathscr B}
\def\sC{\mathscr C}
\def\sZ{\mathscr  Z}
\def\sD{\mathscr  D}
\def\sF{\mathscr  F}
\def\sG{\mathscr G}
\def\sO{\mathscr  O}
\def\sI{\mathscr I}
\def\sS{\mathscr S}
\def\sT{\mathscr  T}
\def\sM{\mathscr M}
\def\sN{\mathscr N}



\def\Ext{\operatorname{Ext}}
 \def\ext{\operatorname{ext}}



\def\ov#1{{\overline{#1}}}

\def\vecthree#1#2#3{\begin{bmatrix} #1 \\ #2 \\ #3 \end{bmatrix}}

\def\tOmega{\tilde{\Omega}}
\def\tDelta{\tilde{\Delta}}
\def\tSigma{\tilde{\Sigma}}
\def\tsigma{\tilde{\sigma}}


\def\d{\delta}
\def\td{\tilde{\delta}}

\def\e{\epsilon}
\def\nsg{\unlhd}
\def\pnsg{\lhd}

\newcommand{\tensor}{\otimes}
\newcommand{\homotopic}{\simeq}
\newcommand{\homeq}{\cong}
\newcommand{\iso}{\approx}

\DeclareMathOperator{\ho}{Ho}
\DeclareMathOperator*{\colim}{colim}


\newcommand{\Q}{\mathbb{Q}}
\renewcommand{\H}{\mathbb{H}}

\newcommand{\bP}{\mathbb{P}}
\newcommand{\bM}{\mathbb{M}}
\newcommand{\A}{\mathbb{A}}
\newcommand{\bH}{{\mathbb{H}}}
\newcommand{\G}{\mathbb{G}}
\newcommand{\bR}{{\mathbb{R}}}
\newcommand{\bL}{{\mathbb{L}}}
\newcommand{\R}{{\mathbb{R}}}
\newcommand{\F}{\mathbb{F}}
\newcommand{\E}{\mathbb{E}}
\newcommand{\bF}{\mathbb{F}}
\newcommand{\bE}{\mathbb{E}}
\newcommand{\bK}{\mathbb{K}}


\newcommand{\bD}{\mathbb{D}}
\newcommand{\bS}{\mathbb{S}}

\newcommand{\bN}{\mathbb{N}}


\newcommand{\bG}{\mathbb{G}}

\newcommand{\C}{\mathbb{C}}
\newcommand{\Z}{\mathbb{Z}}
\newcommand{\N}{\mathbb{N}}

\newcommand{\M}{\mathcal{M}}
\newcommand{\W}{\mathcal{W}}



\newcommand{\itilde}{\tilde{\imath}}
\newcommand{\jtilde}{\tilde{\jmath}}
\newcommand{\ihat}{\hat{\imath}}
\newcommand{\jhat}{\hat{\jmath}}

\newcommand{\fc}{{\mathfrak c}}
\newcommand{\fp}{{\mathfrak p}}
\newcommand{\fm}{{\mathfrak m}}
\newcommand{\fn}{{\mathfrak n}}
\newcommand{\fq}{{\mathfrak q}}

\newcommand{\op}{\mathrm{op}}
\newcommand{\dual}{\vee}

\newcommand{\DEF}[1]{\emph{#1}\index{#1}}
\newcommand{\Def}[1]{#1 \index{#1}}


% The following causes equations to be numbered within sections
\numberwithin{equation}{section}


\theoremstyle{plain} %% This is the default, anyway
\newtheorem{thm}[equation]{Theorem}
\newtheorem{thmdef}[equation]{TheoremDefinition}
\newtheorem{introthm}{Theorem}
\newtheorem{introcor}[introthm]{Corollary}
\newtheorem*{introthm*}{Theorem}
\newtheorem{question}{Question}
\newtheorem{cor}[equation]{Corollary}
\newtheorem{por}[equation]{Porism}
\newtheorem{lem}[equation]{Lemma}
\newtheorem{lemminition}[equation]{Lemminition}
\newtheorem{prop}[equation]{Proposition}
\newtheorem{porism}[equation]{Porism}

\newtheorem{conj}[equation]{Conjecture}
\newtheorem{quest}[equation]{Question}

\theoremstyle{definition}
\newtheorem{defn}[equation]{Definition}
\newtheorem{chunk}[equation]{}
\newtheorem{ex}[equation]{Example}

\newtheorem{exer}[equation]{Optional Exercise}

\theoremstyle{remark}
\newtheorem{rem}[equation]{Remark}

\newtheorem{notation}[equation]{Notation}
\newtheorem{terminology}[equation]{Terminology}



\renewcommand{\sec}[1]{\section{#1}}
\newcommand{\ssec}[1]{\subsection{#1}}
\newcommand{\sssec}[1]{\subsubsection{#1}}

\newcommand{\br}[1]{\lbrace \, #1 \, \rbrace}
\newcommand{\li}{ < \infty}
\newcommand{\quis}{\simeq}
\newcommand{\xra}[1]{\xrightarrow{#1}}
\newcommand{\xla}[1]{\xleftarrow{#1}}
\newcommand{\xlra}[1]{\overset{#1}{\longleftrightarrow}}

\newcommand{\xroa}[1]{\overset{#1}{\twoheadrightarrow}}
\newcommand{\xria}[1]{\overset{#1}{\hookrightarrow}}
\newcommand{\ps}[1]{\mathbb{P}_{#1}^{\text{c}-1}}




\def\and{{ \text{ and } }}
\def\oor{{ \text{ or } }}

\def\Perm{\operatorname{Perm}}
\newcommand{\Ss}{\mathbb{S}}

\def\Op{\operatorname{Op}}
\def\res{\operatorname{res}}
\def\ind{\operatorname{ind}}

\def\sign{{\mathrm{sign}}}
\def\naive{{\mathrm{naive}}}
\def\l{\lambda}


\def\ov#1{\overline{#1}}
\def\cV{{\mathcal V}}
%%%-------------------------------------------------------------------
%%%-------------------------------------------------------------------

\newcommand{\chara}{\operatorname{char}}
\newcommand{\Kos}{\operatorname{Kos}}
\newcommand{\opp}{\operatorname{opp}}
\newcommand{\perf}{\operatorname{perf}}

\newcommand{\Fun}{\operatorname{Fun}}
\newcommand{\GL}{\operatorname{GL}}
\newcommand{\SL}{\operatorname{SL}}
\def\o{\omega}
\def\oo{\overline{\omega}}

\def\cont{\operatorname{cont}}
\def\te{\tilde{e}}
\def\gcd{\operatorname{gcd}}

\def\stab{\operatorname{stab}}

\def\va{\underline{a}}

\def\ua{\underline{a}}
\def\ub{\underline{b}}


\newcommand{\Ob}{\mathrm{Ob}}
\newcommand{\Set}{\mathbf{Set}}
\newcommand{\Grp}{\mathbf{Grp}}
\newcommand{\Ab}{\mathbf{Ab}}
\newcommand{\Sgrp}{\mathbf{Sgrp}}
\newcommand{\Ring}{\mathbf{Ring}}
\newcommand{\Fld}{\mathbf{Fld}}
\newcommand{\cRing}{\mathbf{cRing}}
\newcommand{\Mod}[1]{#1-\mathbf{Mod}}
\newcommand{\vs}[1]{#1-\mathbf{vect}}
\newcommand{\Vs}[1]{#1-\mathbf{Vect}}
\newcommand{\vsp}[1]{#1-\mathbf{vect}^+}
\newcommand{\Top}{\mathbf{Top}}
\newcommand{\Setp}{\mathbf{Set}_*}
\newcommand{\Alg}[1]{#1-\mathbf{Alg}}
\newcommand{\cAlg}[1]{#1-\mathbf{cAlg}}
\newcommand{\PO}{\mathbf{PO}}
\newcommand{\Cont}{\mathrm{Cont}}
\newcommand{\MaT}[1]{\mathbf{Mat}_{#1}}

%%%-------------------------------------------------------------------
%%%-------------------------------------------------------------------
%%%-------------------------------------------------------------------
%%%-------------------------------------------------------------------
%%%-------------------------------------------------------------------

\makeindex
\title{Final Exam}


\begin{document}
\onehalfspacing

\maketitle

\noindent Please turn in \emph{four} of the following problems. If you intend to take a written algebra comprehensive exam, I recommend attempting the problems in a timed setting with no notes at first, and then continuing with the problems later.

\


\begin{enumerate}




\item Let $K$ be a field and $S=K[x_1,\dots,x_n]$ be a polynomial ring over $K$. Let $R$ be any subring of $S$ that contains every polynomial $f\in S$ with the property that 
\[ f(x_1,x_2,\dots, x_{n-1}, x_n) = f(x_2, x_3,\dots,x_n, x_1).\]
Show that $R$ is a finitely generated $K$-algebra.

\

\item Let $K \subseteq L$ be fields, and $R$ be a finitely generated $K$-algebra. Determine if each of the following is true or false, and justify with a proof or counterexample.
\begin{enumerate}
\item If $R$ is a domain, then $L\otimes_K R$ is a domain.
\item If $L\otimes_K R$ is a domain, then $R$ is a domain.
\item If $R$ and $L\otimes_K R$ are domains, then $\dim(R) = \dim(L\otimes_K R)$.
\end{enumerate}

\

\item\label{Zar} Let $R$ be a Noetherian ring. Let $M$ be a nonzero $R$-module (not necessarily finitely generated!), and suppose that $\Ass_R(M)$ has finitely many minimal elements. Show that the support of $M$ is a Zariski closed subset of $\mathrm{Spec}(R)$.


\

\item Let $R$ be a domain and $F$ be its fraction field. For an $R$-module $M$, the \emph{rank} of $M$ is the dimension of the $F$-vector space $M_{(0)}\cong F\otimes_R M$. Assume that $M$ is finitely generated.
\begin{enumerate}
\item Show that the rank of $M$ is finite.
\item Show that there is a short exact sequence of the form
\[ 0 \to R^{\oplus r} \to M \to T \to 0\]
with $r=\mathrm{rank}(M)$ and $T$ a torsion module.
\end{enumerate}

\

\item Let $R$ be a Noetherian ring, and 
\[ I=\mathfrak{q}_1 \cap \mathfrak{q}_2 \cap \mathfrak{q}_3 = \mathfrak{r}_1 \cap \mathfrak{r}_2 \cap \mathfrak{r}_3\]
be two minimal primary decompositions of an ideal $I$.
Show that\footnote{Hint: Take $\mathfrak{q}_3, \mathfrak{r}_3$ whose radicals are equal and are not contained  in any other element of $\mathrm{Ass}(R/I)$ (and explain why you can). Then use the multiplicative set $W=R\smallsetminus (\sqrt{\mathfrak{q}_1} \cup \sqrt{\mathfrak{q}_2})$.} after possibly reordering the ideals above, there is a minimal primary decomposition of $I$ of the form
\[ I = \mathfrak{q}_1 \cap \mathfrak{q}_2 \cap \mathfrak{r}_3.\]

\

%\item Let $(R,\mathfrak{m},k)$ be a complete local ring, and $M$ be an $R$-module such that $\bigcap_{n\in \mathbb{N}} \mathfrak{m}^n M= 0$. Let $m_1,\dots,m_t \in M$. Show that $M = R m_1 + \cdots + R m_t$ if and only if the images of $m_1,\dots, m_t$ generate $M/\mathfrak{m} M$ as a $k$-vector space.


%\ 

\item Let $K$ be a field, and 
\[R=K\begin{bmatrix} x_{11} & x_{12} & x_{13} \\ x_{21} & x_{22} & x_{23}\end{bmatrix}.\] Let $I$ be the ideal generated by the $2\times 3$ minors of  the matrix of variables.
Let \[S=K\begin{bmatrix} u_1 v_1 & u_1 v_2 & u_1 v_3 \\ u_2 v_1 & u_2 v_2 & u_2 v_3\end{bmatrix}\subseteq K[u_1,u_2,v_1,v_2,v_3].\] 
Consider $R$ as a graded ring with $\deg(x_{ij}) = 1$ and $S$ as a graded ring with $\deg(u_i v_j) =1$.
\begin{enumerate}
\item Show that there is a surjective graded $K$-algebra homomorphism $\phi:R/I \to S$ given by sending $\phi(x_{ij}) = u_i v_j$ for all $i,j$.
\item Compute the Hilbert function $H_S(t)$ and show\footnote{Hint: Show that any monomial in $R$ is equivalent modulo $I$ to either 
\begin{itemize} 
\item a monomial in $K[x_{11},x_{12},x_{13},x_{23}]$
\item a monomial in $K[x_{11},x_{12},x_{22},x_{23}]$, or
\item a monomial in $K[x_{11},x_{21},x_{22},x_{23}]$.
\end{itemize}}
that $H_{R/I}(t)\leq H_S(t)$ for all $t$.
\item Conclude that $I$ is prime.
\end{enumerate}

\

\item Let $(R,\mathfrak{m},k)$ be a Noetherian local ring. Recall that a local ring is regular if $\dim(R) = \dim_k(\mathfrak{m}/\mathfrak{m}^2)$.
\begin{enumerate}
\item Show that if $R$ is regular then $\mathrm{gr}_{\mathfrak{m}}(R)$ is a polynomial ring.
\item Show that if $R$ is regular then $\widehat{R}$ is regular.
\end{enumerate}

\


\item[(Bonus)] Show that the conclusion of (\ref{Zar}) is false if $R$ is not Noetherian.


\end{enumerate}

\end{document}







  
 


