\documentclass{amsart}[12pt]
\usepackage{graphicx}
\usepackage{comment}
\usepackage{amscd,mathabx}
\usepackage{amssymb,setspace}
\usepackage{latexsym,amsfonts,amssymb,amsthm,amsmath,amscd,stmaryrd,mathrsfs}
\usepackage[all, knot]{xy}
\usepackage[top=1in, bottom=1in, left=1in, right=1in]{geometry}
\xyoption{all}
\xyoption{arc}
%\usepackage{hyperref}


%\usepackage[notcite,notref]{showkeys}
 
%\CompileMatricesx
\newcommand{\edit}[1]{\marginpar{\footnotesize{#1}}}
%\newcommand{\edit}[1]{}
\newcommand{\rperf}[2]{\operatorname{RPerf}(#1 \into #2)}



\newcommand{\vectwo}[2]{\begin{bmatrix} #1 \\ #2 \end{bmatrix}}

\newcommand{\vecfour}[4]{\begin{bmatrix} #1 \\ #2 \\ #3 \\ #4 \end{bmatrix}}

\newcommand{\Cat}[1]{\left<\left< \text{#1} \right>\right>}


\def\htpy{\simeq_{\mathrm{htpc}}}
\def\tor{\text{ or }}
\def\fg{finitely generated~}

\def\Ass{\operatorname{Ass}}
\def\ann{\operatorname{ann}}
\def\sign{\operatorname{sign}}

\def\ob{{\mathfrak{ob}} }
\def\BiAdd{\operatorname{BiAdd}}
\def\BiLin{\operatorname{BiLin}}

\def\Syl{\operatorname{Syl}}
\def\span{\operatorname{span}}

\def\sdp{\rtimes}
\def\cL{\mathcal L}
\def\cR{\mathcal R}



\def\ay{??}
\def\Aut{\operatorname{Aut}}
\def\End{\operatorname{End}}
\def\Mat{\operatorname{Mat}}


\def\a{\alpha}



\def\etale{\'etale~}
\def\tW{\tilde{W}}
\def\tH{\tilde{H}}
\def\tC{\tilde{C}}
\def\tS{\tilde{S}}
\def\tX{\tilde{X}}
\def\tZ{\tilde{Z}}
\def\HBM{H^{\text{BM}}}
\def\tHBM{\tilde{H}^{\text{BM}}}
\def\Hc{H_{\text{c}}}
\def\Hs{H_{\text{sing}}}
\def\cHs{{\mathcal H}_{\text{sing}}}
\def\sing{{\text{sing}}}
\def\Hms{H^{\text{sing}}}
\def\Hm{\Hms}
\def\tHms{\tilde{H}^{\text{sing}}}
\def\Grass{\operatorname{Grass}}
\def\image{\operatorname{im}}
\def\im{\image}
\def\ker{\operatorname{ker}}
\def\cone{\operatorname{cone}}
\newcommand{\Hom}{\mathrm{Hom}}


\def\ku{ku}
\def\bbu{\bf bu}
\def\KR{K{\mathbb R}}

\def\CW{\underline{CW}}
\def\cP{\mathcal P}
\def\cE{\mathcal E}
\def\cL{\mathcal L}
\def\cJ{\mathcal J}
\def\cJmor{\cJ^\mor}
\def\ctJ{\tilde{\mathcal J}}
\def\tPhi{\tilde{\Phi}}
\def\cA{\mathcal A}
\def\cB{\mathcal B}
\def\cC{\mathcal C}
\def\cZ{\mathcal Z}
\def\cD{\mathcal D}
\def\cF{\mathcal F}
\def\cG{\mathcal G}
\def\cO{\mathcal O}
\def\cI{\mathcal I}
\def\cS{\mathcal S}
\def\cT{\mathcal T}
\def\cM{\mathcal M}
\def\cN{\mathcal N}
\def\cMpc{{\mathcal M}_{pc}}
\def\cMpctf{{\mathcal M}_{pctf}}
\def\L{\Lambda}

\def\sA{\mathscr A}
\def\sB{\mathscr B}
\def\sC{\mathscr C}
\def\sZ{\mathscr  Z}
\def\sD{\mathscr  D}
\def\sF{\mathscr  F}
\def\sG{\mathscr G}
\def\sO{\mathscr  O}
\def\sI{\mathscr I}
\def\sS{\mathscr S}
\def\sT{\mathscr  T}
\def\sM{\mathscr M}
\def\sN{\mathscr N}



\def\Ext{\operatorname{Ext}}
 \def\ext{\operatorname{ext}}



\def\ov#1{{\overline{#1}}}

\def\vecthree#1#2#3{\begin{bmatrix} #1 \\ #2 \\ #3 \end{bmatrix}}

\def\tOmega{\tilde{\Omega}}
\def\tDelta{\tilde{\Delta}}
\def\tSigma{\tilde{\Sigma}}
\def\tsigma{\tilde{\sigma}}


\def\d{\delta}
\def\td{\tilde{\delta}}

\def\e{\epsilon}
\def\nsg{\unlhd}
\def\pnsg{\lhd}

\newcommand{\tensor}{\otimes}
\newcommand{\homotopic}{\simeq}
\newcommand{\homeq}{\cong}
\newcommand{\iso}{\approx}

\DeclareMathOperator{\ho}{Ho}
\DeclareMathOperator*{\colim}{colim}


\newcommand{\Q}{\mathbb{Q}}
\renewcommand{\H}{\mathbb{H}}

\newcommand{\bP}{\mathbb{P}}
\newcommand{\bM}{\mathbb{M}}
\newcommand{\A}{\mathbb{A}}
\newcommand{\bH}{{\mathbb{H}}}
\newcommand{\G}{\mathbb{G}}
\newcommand{\bR}{{\mathbb{R}}}
\newcommand{\bL}{{\mathbb{L}}}
\newcommand{\R}{{\mathbb{R}}}
\newcommand{\F}{\mathbb{F}}
\newcommand{\E}{\mathbb{E}}
\newcommand{\bF}{\mathbb{F}}
\newcommand{\bE}{\mathbb{E}}
\newcommand{\bK}{\mathbb{K}}


\newcommand{\bD}{\mathbb{D}}
\newcommand{\bS}{\mathbb{S}}

\newcommand{\bN}{\mathbb{N}}


\newcommand{\bG}{\mathbb{G}}

\newcommand{\C}{\mathbb{C}}
\newcommand{\Z}{\mathbb{Z}}
\newcommand{\N}{\mathbb{N}}

\newcommand{\M}{\mathcal{M}}
\newcommand{\W}{\mathcal{W}}



\newcommand{\itilde}{\tilde{\imath}}
\newcommand{\jtilde}{\tilde{\jmath}}
\newcommand{\ihat}{\hat{\imath}}
\newcommand{\jhat}{\hat{\jmath}}

\newcommand{\fc}{{\mathfrak c}}
\newcommand{\fp}{{\mathfrak p}}
\newcommand{\fm}{{\mathfrak m}}
\newcommand{\fn}{{\mathfrak n}}
\newcommand{\fq}{{\mathfrak q}}

\newcommand{\op}{\mathrm{op}}
\newcommand{\dual}{\vee}

\newcommand{\DEF}[1]{\emph{#1}\index{#1}}
\newcommand{\Def}[1]{#1 \index{#1}}


% The following causes equations to be numbered within sections
\numberwithin{equation}{section}


\theoremstyle{plain} %% This is the default, anyway
\newtheorem{thm}[equation]{Theorem}
\newtheorem{thmdef}[equation]{TheoremDefinition}
\newtheorem{introthm}{Theorem}
\newtheorem{introcor}[introthm]{Corollary}
\newtheorem*{introthm*}{Theorem}
\newtheorem{question}{Question}
\newtheorem{cor}[equation]{Corollary}
\newtheorem{por}[equation]{Porism}
\newtheorem{lem}[equation]{Lemma}
\newtheorem{lemminition}[equation]{Lemminition}
\newtheorem{prop}[equation]{Proposition}
\newtheorem{porism}[equation]{Porism}

\newtheorem{conj}[equation]{Conjecture}
\newtheorem{quest}[equation]{Question}

\theoremstyle{definition}
\newtheorem{defn}[equation]{Definition}
\newtheorem{chunk}[equation]{}
\newtheorem{ex}[equation]{Example}

\newtheorem{exer}[equation]{Optional Exercise}

\theoremstyle{remark}
\newtheorem{rem}[equation]{Remark}

\newtheorem{notation}[equation]{Notation}
\newtheorem{terminology}[equation]{Terminology}



\renewcommand{\sec}[1]{\section{#1}}
\newcommand{\ssec}[1]{\subsection{#1}}
\newcommand{\sssec}[1]{\subsubsection{#1}}

\newcommand{\br}[1]{\lbrace \, #1 \, \rbrace}
\newcommand{\li}{ < \infty}
\newcommand{\quis}{\simeq}
\newcommand{\xra}[1]{\xrightarrow{#1}}
\newcommand{\xla}[1]{\xleftarrow{#1}}
\newcommand{\xlra}[1]{\overset{#1}{\longleftrightarrow}}

\newcommand{\xroa}[1]{\overset{#1}{\twoheadrightarrow}}
\newcommand{\xria}[1]{\overset{#1}{\hookrightarrow}}
\newcommand{\ps}[1]{\mathbb{P}_{#1}^{\text{c}-1}}




\def\and{{ \text{ and } }}
\def\oor{{ \text{ or } }}

\def\Perm{\operatorname{Perm}}
\newcommand{\Ss}{\mathbb{S}}

\def\Op{\operatorname{Op}}
\def\res{\operatorname{res}}
\def\ind{\operatorname{ind}}

\def\sign{{\mathrm{sign}}}
\def\naive{{\mathrm{naive}}}
\def\l{\lambda}


\def\ov#1{\overline{#1}}
\def\cV{{\mathcal V}}
%%%-------------------------------------------------------------------
%%%-------------------------------------------------------------------

\newcommand{\chara}{\operatorname{char}}
\newcommand{\Kos}{\operatorname{Kos}}
\newcommand{\opp}{\operatorname{opp}}
\newcommand{\perf}{\operatorname{perf}}

\newcommand{\Fun}{\operatorname{Fun}}
\newcommand{\GL}{\operatorname{GL}}
\newcommand{\SL}{\operatorname{SL}}
\def\o{\omega}
\def\oo{\overline{\omega}}

\def\cont{\operatorname{cont}}
\def\te{\tilde{e}}
\def\gcd{\operatorname{gcd}}

\def\stab{\operatorname{stab}}

\def\va{\underline{a}}

\def\ua{\underline{a}}
\def\ub{\underline{b}}


\newcommand{\Ob}{\mathrm{Ob}}
\newcommand{\Set}{\mathbf{Set}}
\newcommand{\Grp}{\mathbf{Grp}}
\newcommand{\Ab}{\mathbf{Ab}}
\newcommand{\Sgrp}{\mathbf{Sgrp}}
\newcommand{\Ring}{\mathbf{Ring}}
\newcommand{\Fld}{\mathbf{Fld}}
\newcommand{\cRing}{\mathbf{cRing}}
\newcommand{\Mod}[1]{#1-\mathbf{Mod}}
\newcommand{\vs}[1]{#1-\mathbf{vect}}
\newcommand{\Vs}[1]{#1-\mathbf{Vect}}
\newcommand{\vsp}[1]{#1-\mathbf{vect}^+}
\newcommand{\Top}{\mathbf{Top}}
\newcommand{\Setp}{\mathbf{Set}_*}
\newcommand{\Alg}[1]{#1-\mathbf{Alg}}
\newcommand{\cAlg}[1]{#1-\mathbf{cAlg}}
\newcommand{\PO}{\mathbf{PO}}
\newcommand{\Cont}{\mathrm{Cont}}
\newcommand{\MaT}[1]{\mathbf{Mat}_{#1}}

%%%-------------------------------------------------------------------
%%%-------------------------------------------------------------------
%%%-------------------------------------------------------------------
%%%-------------------------------------------------------------------
%%%-------------------------------------------------------------------

\makeindex
\title{Assignment \#2}


\begin{document}
\onehalfspacing

\maketitle




\begin{enumerate}


\item Let $K$ be a field, and $x$ be an indeterminate. Let $R=K[x^2,x^3] \subseteq S= K[x]$. Find an ideal $I\subseteq R$ for which $IS \cap R \supsetneqq I$.

\


\item  Let $K$ be an infinite field, and $R = K[x_1,\dots, x_n]$ be a polynomial ring. Let $G= (K^\times)^m$ act on $R$ as follows: 
\[ \begin{aligned} (\lambda_1,\dots,\lambda_m) \cdot k &= k \qquad & k\in K,\\
 (\lambda_1,\dots,\lambda_m) \cdot x_i &= \lambda_1^{a_{1i}} \cdots  \lambda_m^{a_{mi}} x_i \qquad & {i=1,\dots,n}
 \end{aligned}\]
 for some $m\times n$ matrix of integers $A=[a_{ij}]$.

\begin{enumerate} 
\item Show that $R^G$ has a $K$-vector space basis given by the set of monomials $x_1^{b_1} \cdots x_n^{b_n}$ such that, for $b=(b_1,\dots,b_n)$, $Ab=0$.
\item Consider the polynomial ring $R$ with a (nonstandard) $\N^m$-grading given by setting \[ |x_i| = (a_{1i},\dots,a_{mi})\] for each $i$. Show that $R^G$ is the degree zero piece of $R$ under this grading.
\item Show that $R^G$ is a direct summand of $R$, and conclude that $R^G$ is a finitely generated $K$-algebra.
\end{enumerate}
(A combinatorial consequence of this: for any integer matrix $A$, there is a finite set of solution vectors $v_1,\dots,v_t$ such that every solution with nonnegative entries can be written as a nonnegative linear combination of $v_1,\dots,v_t$.)

\

\item Let $X\subseteq \A^m_K$ be an affine varieties over an infinite field $K$.
\begin{enumerate}
\item If $\phi:X \to \A^n_K$ is an algebraic map, show that $\mathcal{I}(\im \phi) = \ker(\phi^*)$ as ideals in $K[y_1,\dots,y_n]$, where $y_1,\dots,y_n$ are the coordinates of $\A^n_K$.
%\item If $\phi: X\to Y$ is an algebraic map, show that  $\mathcal{I}(\im \phi) K[Y] = \ker(\phi^*)$ as ideals in $K[Y] = K[y_1,\dots,y_n]/\mathcal{I}(Y)$.
\item Use (a) to compute $\mathcal{I}( \{ (t,t^2,t^3) \in \A^3_K \ | \ t\in K\} )$.
\item Use (a) to show\footnote{Suggestion: The homomorphism $K[x,y,z] \to K[t]$ sending $x\mapsto t^3, y\mapsto t^4, z\mapsto t^5$ is a graded homomorphism if we set $|x|=3$, $|y|=4$, $|z|=5$. Show that, if $J$ is the ideal on the right hand side, the $n$th graded piece of $K[x,y,z]/J$ is a $K$-vector space  of dimension at most $1$ for $n\geq 3$ and $n=0$, and is zero for $n=1,2$.} $\mathcal{I}( \{ (t^3,t^4,t^5) \in \A^3_K \ | \ t\in K\} ) = (x^3-yz,y^2-xz,z^2-x^2y)$.

\end{enumerate}

\


\hline 

\

\item Compute the irreducible decompositions of the following varieties over $\mathbb{C}$:
\begin{enumerate}
\item $\mathcal{Z}(y^3-x^2y^2)$.
\item $\mathcal{Z}(x_1x_2, x_1 x_3, x_2x_3x_4)$.
\item $\mathcal{Z}(x_1 x_3 + x_2 x_4, x_1 x_5 + x_2 x_6)$.
\end{enumerate}

\



\item Let $R$ be a finitely generated $\Z$-algebra and $\mathfrak{m}$ be a maximal ideal of $R$. Show that $R/\mathfrak{m}$ is finite.

\
\newpage

\item[(B)] In this problem we will prove the \textbf{Ax-Grothendieck Theorem:} Any injective algebraic morphism $\phi: \A^n_{\C} \to \A^n_{\C}$ is surjective.
\begin{enumerate} \item First, left $K$ be an arbitrary field, and $\phi(a_1,\dots,a_n) = (f_1(a_1,\dots,a_n),\dots, f_n(a_1,\dots,a_n))$ be an algebraic morphism, for polynomials $f_1,\dots,f_n$. Show that $\phi$ is \emph{not} surjective if and only if there is some $(a_1,\dots,a_n)\in \A^n_K$ such that
\[ \mathcal{Z}_K( f_1(\underline{x}) - a_1, \dots, f_n(\underline{x})-a_n ) = \varnothing.\]
\item Consider $\A^{2n}_{K}$ with variables $x_1,\dots,x_n,y_1,\dots,y_n$. Show that $\phi$ is injective if and only if
\[ \mathcal{Z}_K( f_1(\underline{x}) - f_1(\underline{y}), \dots, f_n(\underline{x})-f_n(\underline{y}) ) \subseteq \mathcal{Z}_K( x_1- y_1,\dots,x_n-y_n) \quad \text{in} \ \A^{2n}_K.\]
\item Now, let $K=\mathbb{C}$ and suppose that $\phi$ is injective but not surjective. Show that there exist $g_i(\underline{x}) , h_{i,j}(\underline{x},\underline{y})\in \mathbb{C}[\underline{x},\underline{y}]$, and integers $t_j$ such that
\[ \sum_i g_i(\underline{x}) (f_i(\underline{x}) - a_i) =1, \quad (x_j-y_j)^{t_j} = \sum_i h_{i,j}(\underline{x},\underline{y}) (f_i(\underline{x}) - f_i(\underline{y})) \ \text{in} \ \mathbb{C}[\underline{x},\underline{y}].\]
Setting $R=\mathbb{Z}[ \{\text{coefficients of} \ f_i's, g_i's, h_{i,j}'s\}, a_1,\dots,a_n]$, conclude that the same equations hold in a polynomial ring $R[\underline{x},\underline{y}]$ over a finitely generated $\mathbb{Z}$-subalgebra $R\subseteq \mathbb{C}$.
\item Go modulo a maximal ideal of $R$, and complete the proof of the theorem.
\end{enumerate}
\end{enumerate}

\end{document}







  
 


