 \documentclass{amsart}
\usepackage{pdfpages}
% \pagestyle{empty}
\usepackage{amscd}
\usepackage{amssymb}
%\usepackage[all, knot]{xy}



\usepackage[top=1in, bottom=1in, left=1.1in, right=1.1in]{geometry}

%\setlength{\oddsidemargin}{-.10in}
%\setlength{\evensidemargin}{-.10in}
%\setlength{\textwidth}{5.5in}
%\setlength{\topmargin}{-.250in}
%\setlength{\headheight}{0in}
%\setlength{\headsep}{0in}
%\setlength{\topskip}{0in}
%\setlength{\textheight}{9.5in}
%\parindent = 0in
\font\bigbf = cmbx10 scaled \magstep1
\font\medbf = cmbx10 scaled \magstephalf
\begin{document}
 
%\magnification \magstep1
%\parindent = 0pt
%\nopagenumbers
%\voffset = -.5truein
%\vsize = 10 truein
%\baselineskip = 1.5 \baselineskip
%\font\bigbf = cmbx10 scaled \magstep1
%\font\medbf = cmbx10 scaled \magstephalf








\centerline{\bigbf Topics in Algebra, Spring Semester 2023}
\centerline{\bigbf TR 9:30pm -- 10:50pm, Burnett 121}



\bigskip

\noindent
{\bf Instructor:}  Jack Jeffries

\noindent
{\bf Office:} 325 Avery Hall

\noindent
{\bf email:} jack.jeffries@unl.edu

\

\noindent
{\bf Office Hours:} Regular hours TBA, plus by appointment, or just drop in.

\

\noindent
{\bf Course Description:} 
What are K\"ahler differentials? The content of this class is based on the observation that, given a polynomial, if you want to take its derivative, you don’t actually need to deal with limits at all: you just need to know the product rule. This means that we can do calculus entirely algebraically\dots and that we can do calculus over any ring, including not-so-calculus-y ones like $\mathbb{Z}[i]$. K\"ahler differentials are fundamental modules that arise from, and govern, this notion of calculus.

In this class, we will study these modules, and apply them to understand crinkles and wrinkles in high-dimensional spaces. The surprising thing is that this is all algebra! Along the way, we will find many connections with multivariable calculus and topology, and take the excuse to remember what we have learned elsewhere.

Likely topics include: derivations, basic properties of K\"ahler differentials, more field theory (transcendence bases, p-bases, separably generated extensions), complete local rings, coefficient fields and Cohen’s structure theorem, regular rings, Jacobian criterion, smooth / \'etale / unramified extensions, Zariski’s main theorem, structure theory of smooth / \'etale / unramified extensions, reduction to positive characteristic, regular sequences and complete intersections, Ferrand-Vasconcelos theorem, differential graded algebras, and Andr\'e-Quillen cohomology.

\

\noindent
\textbf{Textbook:} There will be no assigned textbook for the course. I will provide lecture notes instead. Some recommended complementary sources include:
\begin{itemize}
\item \emph{Commutative Ring Theory} by Hideyuki Matsumura
\item \emph{K\"ahler Differentials} by Ernst Kunz
\item \emph{Commutative Algebra with a View Towards Algebraic Geometry} by David Eisenbud 
 \item \emph{Math 615 Lecture Notes, Winter 2017} by Melvin Hochster
\end{itemize}
\

\noindent
{\bf Assessments:} We will have four or five problem sets throughout the semester, as well as a few short quizzes, to help us make sure we stay up-to-date on our definitions and key theorems.

\

\noindent
{\bf Prerequisites:} This class will assume knowledge of basic Commutative Algebra. Anyone who has taken Math 905 or Math 902 should be prepared for the course. On occasion, we will use basic tools from Homological Algebra covered in Math 901; in these cases, we will either give a quick overview in class and/or provide supplementary resources for those who haven’t encountered the material. Students attending this course may also benefit from taking Math 935 concurrently to obtain additional geometric motivation and intuition for the material.


\



\noindent
\textbf{UNL Course Policies and Resources}
Students are responsible for knowing the university policies and resources found on this page (https://go.unl.edu/coursepolicies):
 
 \begin{itemize}
\item University-wide Attendance Policy
\item Academic Honesty Policy
\item Services for Students with Disabilities
\item Mental Health and Well-Being Resources
\item Final Exam Schedule 
\item Fifteenth Week Policy 
\item Emergency Procedures 
\item Diversity \& Inclusiveness 
\item Title IX Policy 
\item Other Relevant University-Wide Policies
\end{itemize} 


\vfill
\pagebreak
\smallskip
%\includepdf{UNLFace.pdf}



\end{document}


