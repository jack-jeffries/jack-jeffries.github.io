
%\documentclass{amsart}
\documentclass[12pt]{amsart}



\usepackage{times, framed,graphicx,stmaryrd}
\usepackage[all, knot]{xy}
\usepackage{amsmath}
\usepackage{color}
\usepackage{xcolor}
\newcommand{\Q}{\mathbb{Q}}
\newcommand{\N}{\mathbb{N}}
\newcommand{\Z}{\mathbb{Z}}
\newcommand{\R}{\mathbb{R}}
\newcommand{\inv}{^{-1}}
\newcommand{\dabs}[1]{\left| #1 \right|}
\newcommand{\blue}{\color {blue}}                   
\newcommand{\red}{\color {red}}                   
\newcommand{\olive}{\color {olive}} 
\newcommand{\violet}{\color {violet}} 
\newcommand{\orange}{\color{orange}} 

\DeclareMathOperator{\res}{res}

%\usepackage{times}
\textheight 24.cm
\textwidth      16.000cm
\topmargin -1cm
\oddsidemargin .5cm
\evensidemargin -1cm
%\addtolength{\textwidth}{100pt}
%\addtolength{\evensidemargin}{-45pt}
%\addtolength{\oddsidemargin}{-60pt}

\pagestyle{empty}
%\begin{document}\begin{itemize}

%\thispagestyle{empty}




\begin{document}
	
	\thispagestyle{empty}
	
	\begin{center}
		\Large{Math 918. Quiz \#3 }\\

	\end{center}
	
	

	
	\
	
	\begin{enumerate}
	
	
	
	\item Explain why there is a power series $f\in \R \llbracket x, y \rrbracket$ such that $f^2 + (x+y -3) f + xy-2 =0$.
		
	\vfill
	
	\item Let $R=\Z[x]$ and $I=(x^2-2)$. Describe an element of $\widehat{R}^I$ that is not an element of (the image of) $R$. You can use any way of describing elements of completions that we have discussed, and you don't have to prove that your element is not in $R$.
	
		\vfill	
	
	\end{enumerate}
	
	
	\end{document}