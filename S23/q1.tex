
%\documentclass{amsart}
\documentclass[12pt]{amsart}



\usepackage{times, framed,graphicx}
\usepackage{amsmath}
\usepackage{color}
\usepackage{xcolor}
\newcommand{\Q}{\mathbb{Q}}
\newcommand{\N}{\mathbb{N}}
\newcommand{\Z}{\mathbb{Z}}
\newcommand{\R}{\mathbb{R}}
\newcommand{\inv}{^{-1}}
\newcommand{\dabs}[1]{\left| #1 \right|}
\newcommand{\blue}{\color {blue}}                   
\newcommand{\red}{\color {red}}                   
\newcommand{\olive}{\color {olive}} 
\newcommand{\violet}{\color {violet}} 
\newcommand{\orange}{\color{orange}} 

\DeclareMathOperator{\res}{res}

%\usepackage{times}
\textheight 24.cm
\textwidth      16.000cm
\topmargin -1cm
\oddsidemargin .5cm
\evensidemargin -1cm
%\addtolength{\textwidth}{100pt}
%\addtolength{\evensidemargin}{-45pt}
%\addtolength{\oddsidemargin}{-60pt}

\pagestyle{empty}
%\begin{document}\begin{itemize}

%\thispagestyle{empty}




\begin{document}
	
	\thispagestyle{empty}
	
	\begin{center}
		\Large{Math 918. Quiz \#1 }\\

	\end{center}
	
	

	
	\
	
	\begin{enumerate}
	
	
	
	\item State the definition of \emph{derivation}.
	
	\vfill
	
	\item Describe concretely\footnote{i.e., what they actually are, not just up to isomorphism}  all of the $\Q$-linear derivations from the ring $\Q[x,y]$ to itself. 
	
	\vfill
	
	\end{enumerate}
	
	
	\end{document}