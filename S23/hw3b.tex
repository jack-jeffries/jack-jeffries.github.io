\documentclass{amsart}[12pt]
\usepackage{graphicx}
\usepackage{comment}
\usepackage{amscd,mathabx}
\usepackage{amssymb,setspace}
\usepackage{latexsym,amsfonts,amssymb,amsthm,amsmath,amscd,stmaryrd,mathrsfs}
\usepackage[all, knot]{xy}
\usepackage[top=.7in, bottom=.8in, left=1in, right=1in]{geometry}
\xyoption{all}
\xyoption{arc}
%\usepackage{hyperref}

%\usepackage[notcite,notref]{showkeys}
 
%\CompileMatricesx
\newcommand{\edit}[1]{\marginpar{\footnotesize{#1}}}
%\newcommand{\edit}[1]{}
\newcommand{\rperf}[2]{\operatorname{RPerf}(#1 \into #2)}



\newcommand{\vectwo}[2]{\begin{bmatrix} #1 \\ #2 \end{bmatrix}}

\newcommand{\vecfour}[4]{\begin{bmatrix} #1 \\ #2 \\ #3 \\ #4 \end{bmatrix}}

\newcommand{\Cat}[1]{\left<\left< \text{#1} \right>\right>}


\def\htpy{\simeq_{\mathrm{htpc}}}
\def\tor{\text{ or }}
\def\fg{finitely generated~}

\def\Ass{\operatorname{Ass}}
\def\ann{\operatorname{ann}}
\def\sign{\operatorname{sign}}

\def\ob{{\mathfrak{ob}} }
\def\BiAdd{\operatorname{BiAdd}}
\def\BiLin{\operatorname{BiLin}}

\def\Syl{\operatorname{Syl}}
\def\span{\operatorname{span}}

\def\sdp{\rtimes}
\def\cL{\mathcal L}
\def\cR{\mathcal R}



\def\ay{??}
\def\Aut{\operatorname{Aut}}
\def\End{\operatorname{End}}
\def\Mat{\operatorname{Mat}}


\def\a{\alpha}



\def\etale{\'etale~}
\def\tW{\tilde{W}}
\def\tH{\tilde{H}}
\def\tC{\tilde{C}}
\def\tS{\tilde{S}}
\def\tX{\tilde{X}}
\def\tZ{\tilde{Z}}
\def\HBM{H^{\text{BM}}}
\def\tHBM{\tilde{H}^{\text{BM}}}
\def\Hc{H_{\text{c}}}
\def\Hs{H_{\text{sing}}}
\def\cHs{{\mathcal H}_{\text{sing}}}
\def\sing{{\text{sing}}}
\def\Hms{H^{\text{sing}}}
\def\Hm{\Hms}
\def\tHms{\tilde{H}^{\text{sing}}}
\def\Grass{\operatorname{Grass}}
\def\image{\operatorname{im}}
\def\im{\image}
\def\ker{\operatorname{ker}}
\def\cone{\operatorname{cone}}
\newcommand{\Hom}{\mathrm{Hom}}


\def\ku{ku}
\def\bbu{\bf bu}
\def\KR{K{\mathbb R}}

\def\CW{\underline{CW}}
\def\cP{\mathcal P}
\def\cE{\mathcal E}
\def\cL{\mathcal L}
\def\cJ{\mathcal J}
\def\cJmor{\cJ^\mor}
\def\ctJ{\tilde{\mathcal J}}
\def\tPhi{\tilde{\Phi}}
\def\cA{\mathcal A}
\def\cB{\mathcal B}
\def\cC{\mathcal C}
\def\cZ{\mathcal Z}
\def\cD{\mathcal D}
\def\cF{\mathcal F}
\def\cG{\mathcal G}
\def\cO{\mathcal O}
\def\cI{\mathcal I}
\def\cS{\mathcal S}
\def\cT{\mathcal T}
\def\cM{\mathcal M}
\def\cN{\mathcal N}
\def\cMpc{{\mathcal M}_{pc}}
\def\cMpctf{{\mathcal M}_{pctf}}
\def\L{\Lambda}

\def\sA{\mathscr A}
\def\sB{\mathscr B}
\def\sC{\mathscr C}
\def\sZ{\mathscr  Z}
\def\sD{\mathscr  D}
\def\sF{\mathscr  F}
\def\sG{\mathscr G}
\def\sO{\mathscr  O}
\def\sI{\mathscr I}
\def\sS{\mathscr S}
\def\sT{\mathscr  T}
\def\sM{\mathscr M}
\def\sN{\mathscr N}



\def\Ext{\operatorname{Ext}}
 \def\ext{\operatorname{ext}}



\def\ov#1{{\overline{#1}}}

\def\vecthree#1#2#3{\begin{bmatrix} #1 \\ #2 \\ #3 \end{bmatrix}}

\def\tOmega{\tilde{\Omega}}
\def\tDelta{\tilde{\Delta}}
\def\tSigma{\tilde{\Sigma}}
\def\tsigma{\tilde{\sigma}}


\def\d{\delta}
\def\td{\tilde{\delta}}

\def\e{\epsilon}
\def\nsg{\unlhd}
\def\pnsg{\lhd}

\newcommand{\tensor}{\otimes}
\newcommand{\homotopic}{\simeq}
\newcommand{\homeq}{\cong}
\newcommand{\iso}{\approx}

\DeclareMathOperator{\ho}{Ho}
\DeclareMathOperator*{\colim}{colim}


\newcommand{\Q}{\mathbb{Q}}
\renewcommand{\H}{\mathbb{H}}

\newcommand{\bP}{\mathbb{P}}
\newcommand{\bM}{\mathbb{M}}
\newcommand{\A}{\mathbb{A}}
\newcommand{\bH}{{\mathbb{H}}}
\newcommand{\G}{\mathbb{G}}
\newcommand{\bR}{{\mathbb{R}}}
\newcommand{\bL}{{\mathbb{L}}}
\newcommand{\R}{{\mathbb{R}}}
\newcommand{\F}{\mathbb{F}}
\newcommand{\E}{\mathbb{E}}
\newcommand{\bF}{\mathbb{F}}
\newcommand{\bE}{\mathbb{E}}
\newcommand{\bK}{\mathbb{K}}


\newcommand{\bD}{\mathbb{D}}
\newcommand{\bS}{\mathbb{S}}

\newcommand{\bN}{\mathbb{N}}


\newcommand{\bG}{\mathbb{G}}

\newcommand{\C}{\mathbb{C}}
\newcommand{\Z}{\mathbb{Z}}
\newcommand{\N}{\mathbb{N}}

\newcommand{\M}{\mathcal{M}}
\newcommand{\W}{\mathcal{W}}



\newcommand{\itilde}{\tilde{\imath}}
\newcommand{\jtilde}{\tilde{\jmath}}
\newcommand{\ihat}{\hat{\imath}}
\newcommand{\jhat}{\hat{\jmath}}

\newcommand{\fc}{{\mathfrak c}}
\newcommand{\fp}{{\mathfrak p}}
\newcommand{\fm}{{\mathfrak m}}
\newcommand{\fn}{{\mathfrak n}}
\newcommand{\fq}{{\mathfrak q}}

\newcommand{\op}{\mathrm{op}}
\newcommand{\dual}{\vee}

\newcommand{\DEF}[1]{\emph{#1}\index{#1}}
\newcommand{\Def}[1]{#1 \index{#1}}


% The following causes equations to be numbered within sections
\numberwithin{equation}{section}


\theoremstyle{plain} %% This is the default, anyway
\newtheorem{thm}[equation]{Theorem}
\newtheorem{thmdef}[equation]{TheoremDefinition}
\newtheorem{introthm}{Theorem}
\newtheorem{introcor}[introthm]{Corollary}
\newtheorem*{introthm*}{Theorem}
\newtheorem{question}{Question}
\newtheorem{cor}[equation]{Corollary}
\newtheorem{por}[equation]{Porism}
\newtheorem{lem}[equation]{Lemma}
\newtheorem{lemminition}[equation]{Lemminition}
\newtheorem{prop}[equation]{Proposition}
\newtheorem{porism}[equation]{Porism}

\newtheorem{conj}[equation]{Conjecture}
\newtheorem{quest}[equation]{Question}

\theoremstyle{definition}
\newtheorem{defn}[equation]{Definition}
\newtheorem{chunk}[equation]{}
\newtheorem{ex}[equation]{Example}

\newtheorem{exer}[equation]{Optional Exercise}

\theoremstyle{remark}
\newtheorem{rem}[equation]{Remark}

\newtheorem{notation}[equation]{Notation}
\newtheorem{terminology}[equation]{Terminology}



\renewcommand{\sec}[1]{\section{#1}}
\newcommand{\ssec}[1]{\subsection{#1}}
\newcommand{\sssec}[1]{\subsubsection{#1}}

\newcommand{\br}[1]{\lbrace \, #1 \, \rbrace}
\newcommand{\li}{ < \infty}
\newcommand{\quis}{\simeq}
\newcommand{\xra}[1]{\xrightarrow{#1}}
\newcommand{\xla}[1]{\xleftarrow{#1}}
\newcommand{\xlra}[1]{\overset{#1}{\longleftrightarrow}}

\newcommand{\xroa}[1]{\overset{#1}{\twoheadrightarrow}}
\newcommand{\xria}[1]{\overset{#1}{\hookrightarrow}}
\newcommand{\ps}[1]{\mathbb{P}_{#1}^{\text{c}-1}}




\def\and{{ \text{ and } }}
\def\oor{{ \text{ or } }}

\def\Perm{\operatorname{Perm}}
\newcommand{\Ss}{\mathbb{S}}

\def\Op{\operatorname{Op}}
\def\res{\operatorname{res}}
\def\ind{\operatorname{ind}}

\def\sign{{\mathrm{sign}}}
\def\naive{{\mathrm{naive}}}
\def\l{\lambda}


\def\ov#1{\overline{#1}}
\def\cV{{\mathcal V}}
%%%-------------------------------------------------------------------
%%%-------------------------------------------------------------------

\newcommand{\chara}{\operatorname{char}}
\newcommand{\Kos}{\operatorname{Kos}}
\newcommand{\opp}{\operatorname{opp}}
\newcommand{\perf}{\operatorname{perf}}

\newcommand{\Fun}{\operatorname{Fun}}
\newcommand{\GL}{\operatorname{GL}}
\newcommand{\SL}{\operatorname{SL}}
\def\o{\omega}
\def\oo{\overline{\omega}}

\def\cont{\operatorname{cont}}
\def\te{\tilde{e}}
\def\gcd{\operatorname{gcd}}

\def\stab{\operatorname{stab}}

\def\va{\underline{a}}

\def\ua{\underline{a}}
\def\ub{\underline{b}}


\newcommand{\Ob}{\mathrm{Ob}}
\newcommand{\Set}{\mathbf{Set}}
\newcommand{\Grp}{\mathbf{Grp}}
\newcommand{\Ab}{\mathbf{Ab}}
\newcommand{\Sgrp}{\mathbf{Sgrp}}
\newcommand{\Ring}{\mathbf{Ring}}
\newcommand{\Fld}{\mathbf{Fld}}
\newcommand{\cRing}{\mathbf{cRing}}
\newcommand{\Mod}[1]{#1-\mathbf{Mod}}
\newcommand{\vs}[1]{#1-\mathbf{vect}}
\newcommand{\Vs}[1]{#1-\mathbf{Vect}}
\newcommand{\vsp}[1]{#1-\mathbf{vect}^+}
\newcommand{\Top}{\mathbf{Top}}
\newcommand{\Setp}{\mathbf{Set}_*}
\newcommand{\Alg}[1]{#1-\mathbf{Alg}}
\newcommand{\cAlg}[1]{#1-\mathbf{cAlg}}
\newcommand{\PO}{\mathbf{PO}}
\newcommand{\Cont}{\mathrm{Cont}}
\newcommand{\MaT}[1]{\mathbf{Mat}_{#1}}
\newcommand{\Der}{\mathrm{Der}}

%%%-------------------------------------------------------------------
%%%-------------------------------------------------------------------
%%%-------------------------------------------------------------------
%%%-------------------------------------------------------------------
%%%-------------------------------------------------------------------

\makeindex
\title{Problem set \#3}


\begin{document}
\onehalfspacing

\maketitle


\vspace{-1mm}

\begin{enumerate}


\item  Is $\Z[\sqrt[3]{37}]$ a regular ring? What about $\Z[\sqrt[3]{43}]$?

\

\item Let $R$ be an $A$-algebra, $f(x_1,\dots,x_n) \in A[x_1,\dots,x_n]$ a polynomial with coefficients in $A$, and $r_1,\dots,r_n, s_1,\dots,s_n\in R$.
\begin{enumerate}
\item Prove the \emph{chain rule} for the universal derivation: $d_{R|A}(f(r_1,\dots,r_n)) = \sum_i \frac{d f}{d x_i}(r_1,\dots,r_n) dr_i$.
\item Prove the \emph{Taylor expansion} formula: $f(r_1+s_1,\dots,r_n+s_n) = \sum_{\alpha \in \N^n} \frac{1}{|\alpha|!} \frac{d^{|\alpha|}f}{dx_1^{\alpha_1} \cdots dx_n^{\alpha_n}}(r_1,\dots,r_n) s_1^{\alpha_1} \cdots s_n^{\alpha_n}$.
\end{enumerate}

\

\item Facts about $p$-bases/ $p$-degree:
 \begin{enumerate}
\item Let $L$ be a field of positive characteristic. 
Let $T$ be a $p$-basis for $L$. Show that for any $e$, the set $T^{[<p^e]}$ is a basis for $L$.
\item Let $K \subseteq L$ be a finite extension of fields of positive characteristic. Show that $p\deg(K)=p\deg(L)$.
\item Let $L=K(x_1,\dots,x_m)$ be a field of rational functions in $m$ variables over $K$. Show that $p\deg(L) = p\deg(K) +m$.
\end{enumerate}


\begin{proof} 
\begin{enumerate}
\item By induction on $e$, with $e=1$ as the definition. If the claim is true for $e$, so $T^{[<p^e]}$ is a basis for $L/L^{p^e}$, taking $p$th powers we have that $(T^p)^{[<p^e]}$ is a basis for $L^p / L^{p^{e+1}}$. But $T^{[<p^{e+1}]} = (T^p)^{[<p^e]} T^{[<p]}$ (i.e., the first set is the set of products of the two sets on the right-hand side), so from field theory, the left hand side is a basis for $L/ L^{p^{e+1}}$.
\item Consider the diagram
\[ \xymatrix{  & L & \\ L^p \ar[ur] & & K \ar[ul] \\ & K^p  \ar[ur]  \ar[ul] &  }\]
From field theory $[L:K][K:K^p] = [L^p : K^p][L:L^p]$, and $[L^p : K^p] = [L:K]$, so $[K:K^p] = [L:L^p]$. Then $p\deg(K)=\log_p([K:K^p]) = \log_p([L:L^p]) = p\deg(L)$.
\item Take a $p$-basis $T$ for $K$. One checks that $T \cup \{x_1,\dots,x_m\}$ is a $p$-basis for $L$.
\end{enumerate}
\end{proof}


\item Let $k$ be a field of positive characteristic with a finite $p$-basis, $R$ be a finitely generated $k$-algebra, and $\fp \subseteq \fq$ be prime ideals of $R$.
Show that 
\[ \mathrm{dim} R_\fq / \fp R_\fq = p\deg(\kappa(\fp)) - p\deg(\kappa(\fq)).\]


\begin{proof}
We will show that $\dim(R/\fp) = p\deg(\kappa(\fp)) - p\deg(k)$; the formula above then follows.
We can replace $R$ by $R/\fp$ and assume $R$ is a domain and $\fp=0$,  Take a Noether normalization $A$ for $R$. By part (2) of the previous problem, $p\deg(\kappa(\fp)) = p\deg(\mathrm{frac}(R)) = p\deg(\mathrm{frac}(A))$. By part (3) of the previous problem, $p\deg(\mathrm{frac}(A)) = \dim(A) + p\deg(k) =  \dim(R) + p\deg(k)$. The conclusion then follows.
\end{proof}

\item Let $K$ be a field.
\begin{enumerate}
\item Let $R=K[x]$ be a polynomial ring in one variable and $M= R^{\oplus \N}$ be a free $R$-module on a countable basis. Compute the $(x)$-adic completion of $M$.
\item Let $R=K[x_1,x_2,\dots]$ be a polynomial ring in countably many variables and $\fm=(x_1,x_2,\dots)$. Describe the elements of $\hat{R}^{\fm}$. Find an element in the maximal ideal of $\hat{R}^{\fm}$ that is \emph{not} an element of $\fm \hat{R}^\fm$.
\end{enumerate}

\





\item Let $K\subseteq L$ be an extension of fields.
\begin{enumerate}
\item Suppose that $L$ is a finitely generated over $K$ as fields. Show that $L$ is formally unramified over $K$ if and only if the extension is separable algebraic.
\item Show that the finite generation hypothesis is strictly necessary in part (1).
\end{enumerate}

\begin{proof} 
\begin{enumerate}
\item We just need to show that unramified implies separable algebraic. Any transcendental element is a $p$-independent set, which contradicts unramified, so unramified implies algebraic. Write $K\subseteq F \subseteq L$, with $F\subseteq L$ purely inseparable. By finite generation plus algebraic, this is finite. We can then choose some $f$ with $f\in L \smallsetminus F$ and $f^p\in F$ using finiteness. Then $f$ is $p$-independent in $L$ over $F$, contradicting unramified.
\item Take $K=\F_p(t)$ and $L=\bigcup_{e\in \N} \F_p(t^{1/p^{e}})$.
\end{enumerate}
\end{proof}


\end{enumerate}

\end{document}







  
 


