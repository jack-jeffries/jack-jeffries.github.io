\documentclass{amsart}[12pt]
\usepackage{graphicx}
\usepackage{comment}
\usepackage{amscd,mathabx}
\usepackage{amssymb,setspace,color,xcolor}
\usepackage{latexsym,amsfonts,amssymb,amsthm,amsmath,amscd,stmaryrd,mathrsfs}
\usepackage[all, knot]{xy}
\usepackage[top=.8in, bottom=.8in, left=1in, right=1in]{geometry}
\xyoption{all}
\xyoption{arc}


%\usepackage{hyperref}


%\usepackage[notcite,notref]{showkeys}
 
%\CompileMatricesx
\newcommand{\edit}[1]{\marginpar{\footnotesize{#1}}}
%\newcommand{\edit}[1]{}
\newcommand{\rperf}[2]{\operatorname{RPerf}(#1 \into #2)}



\newcommand{\vectwo}[2]{\begin{bmatrix} #1 \\ #2 \end{bmatrix}}

\newcommand{\vecfour}[4]{\begin{bmatrix} #1 \\ #2 \\ #3 \\ #4 \end{bmatrix}}

\newcommand{\Cat}[1]{\left<\left< \text{#1} \right>\right>}


\def\htpy{\simeq_{\mathrm{htpc}}}
\def\tor{\text{ or }}
\def\fg{finitely generated~}

\def\Ass{\operatorname{Ass}}
\def\ann{\operatorname{ann}}
\def\sign{\operatorname{sign}}

\def\ob{{\mathfrak{ob}} }
\def\BiAdd{\operatorname{BiAdd}}
\def\BiLin{\operatorname{BiLin}}

\def\Syl{\operatorname{Syl}}
\def\span{\operatorname{span}}

\def\sdp{\rtimes}
\def\cL{\mathcal L}
\def\cR{\mathcal R}



\def\ay{??}
\def\Aut{\operatorname{Aut}}
\def\End{\operatorname{End}}
\def\Mat{\operatorname{Mat}}


\def\a{\alpha}



\def\etale{\'etale~}
\def\tW{\tilde{W}}
\def\tH{\tilde{H}}
\def\tC{\tilde{C}}
\def\tS{\tilde{S}}
\def\tX{\tilde{X}}
\def\tZ{\tilde{Z}}
\def\HBM{H^{\text{BM}}}
\def\tHBM{\tilde{H}^{\text{BM}}}
\def\Hc{H_{\text{c}}}
\def\Hs{H_{\text{sing}}}
\def\cHs{{\mathcal H}_{\text{sing}}}
\def\sing{{\text{sing}}}
\def\Hms{H^{\text{sing}}}
\def\Hm{\Hms}
\def\tHms{\tilde{H}^{\text{sing}}}
\def\Grass{\operatorname{Grass}}
\def\image{\operatorname{im}}
\def\im{\image}
\def\ker{\operatorname{ker}}
\def\cone{\operatorname{cone}}
\newcommand{\Hom}{\mathrm{Hom}}


\def\ku{ku}
\def\bbu{\bf bu}
\def\KR{K{\mathbb R}}

\def\CW{\underline{CW}}
\def\cP{\mathcal P}
\def\cE{\mathcal E}
\def\cL{\mathcal L}
\def\cJ{\mathcal J}
\def\cJmor{\cJ^\mor}
\def\ctJ{\tilde{\mathcal J}}
\def\tPhi{\tilde{\Phi}}
\def\cA{\mathcal A}
\def\cB{\mathcal B}
\def\cC{\mathcal C}
\def\cZ{\mathcal Z}
\def\cD{\mathcal D}
\def\cF{\mathcal F}
\def\cG{\mathcal G}
\def\cO{\mathcal O}
\def\cI{\mathcal I}
\def\cS{\mathcal S}
\def\cT{\mathcal T}
\def\cM{\mathcal M}
\def\cN{\mathcal N}
\def\cMpc{{\mathcal M}_{pc}}
\def\cMpctf{{\mathcal M}_{pctf}}
\def\L{\Lambda}

\def\sA{\mathscr A}
\def\sB{\mathscr B}
\def\sC{\mathscr C}
\def\sZ{\mathscr  Z}
\def\sD{\mathscr  D}
\def\sF{\mathscr  F}
\def\sG{\mathscr G}
\def\sO{\mathscr  O}
\def\sI{\mathscr I}
\def\sS{\mathscr S}
\def\sT{\mathscr  T}
\def\sM{\mathscr M}
\def\sN{\mathscr N}



\def\Ext{\operatorname{Ext}}
 \def\ext{\operatorname{ext}}



\def\ov#1{{\overline{#1}}}

\def\vecthree#1#2#3{\begin{bmatrix} #1 \\ #2 \\ #3 \end{bmatrix}}

\def\tOmega{\tilde{\Omega}}
\def\tDelta{\tilde{\Delta}}
\def\tSigma{\tilde{\Sigma}}
\def\tsigma{\tilde{\sigma}}


\def\d{\delta}
\def\td{\tilde{\delta}}

\def\e{\epsilon}
\def\nsg{\unlhd}
\def\pnsg{\lhd}

\newcommand{\tensor}{\otimes}
\newcommand{\homotopic}{\simeq}
\newcommand{\homeq}{\cong}
\newcommand{\iso}{\approx}

\DeclareMathOperator{\ho}{Ho}
\DeclareMathOperator*{\colim}{colim}


\newcommand{\Q}{\mathbb{Q}}
\renewcommand{\H}{\mathbb{H}}

\newcommand{\bP}{\mathbb{P}}
\newcommand{\bM}{\mathbb{M}}
\newcommand{\A}{\mathbb{A}}
\newcommand{\bH}{{\mathbb{H}}}
\newcommand{\G}{\mathbb{G}}
\newcommand{\bR}{{\mathbb{R}}}
\newcommand{\bL}{{\mathbb{L}}}
\newcommand{\R}{{\mathbb{R}}}
\newcommand{\F}{\mathbb{F}}
\newcommand{\E}{\mathbb{E}}
\newcommand{\bF}{\mathbb{F}}
\newcommand{\bE}{\mathbb{E}}
\newcommand{\bK}{\mathbb{K}}


\newcommand{\bD}{\mathbb{D}}
\newcommand{\bS}{\mathbb{S}}

\newcommand{\bN}{\mathbb{N}}


\newcommand{\bG}{\mathbb{G}}

\newcommand{\C}{\mathbb{C}}
\newcommand{\Z}{\mathbb{Z}}
\newcommand{\N}{\mathbb{N}}

\newcommand{\M}{\mathcal{M}}
\newcommand{\W}{\mathcal{W}}



\newcommand{\itilde}{\tilde{\imath}}
\newcommand{\jtilde}{\tilde{\jmath}}
\newcommand{\ihat}{\hat{\imath}}
\newcommand{\jhat}{\hat{\jmath}}

\newcommand{\fc}{{\mathfrak c}}
\newcommand{\fp}{{\mathfrak p}}
\newcommand{\fm}{{\mathfrak m}}
\newcommand{\fn}{{\mathfrak n}}
\newcommand{\fq}{{\mathfrak q}}

\newcommand{\op}{\mathrm{op}}
\newcommand{\dual}{\vee}

\newcommand{\DEF}[1]{\emph{#1}\index{#1}}
\newcommand{\Def}[1]{#1 \index{#1}}


% The following causes equations to be numbered within sections
\numberwithin{equation}{section}


\theoremstyle{plain} %% This is the default, anyway
\newtheorem{thm}[equation]{Theorem}
\newtheorem{thmdef}[equation]{TheoremDefinition}
\newtheorem{introthm}{Theorem}
\newtheorem{introcor}[introthm]{Corollary}
\newtheorem*{introthm*}{Theorem}
\newtheorem{question}{Question}
\newtheorem{cor}[equation]{Corollary}
\newtheorem{por}[equation]{Porism}
\newtheorem{lem}[equation]{Lemma}
\newtheorem{lemminition}[equation]{Lemminition}
\newtheorem{prop}[equation]{Proposition}
\newtheorem{porism}[equation]{Porism}

\newtheorem{conj}[equation]{Conjecture}
\newtheorem{quest}[equation]{Question}

\theoremstyle{definition}
\newtheorem{defn}[equation]{Definition}
\newtheorem{chunk}[equation]{}
\newtheorem{ex}[equation]{Example}

\newtheorem{exer}[equation]{Optional Exercise}

\theoremstyle{remark}
\newtheorem{rem}[equation]{Remark}

\newtheorem{notation}[equation]{Notation}
\newtheorem{terminology}[equation]{Terminology}



\renewcommand{\sec}[1]{\section{#1}}
\newcommand{\ssec}[1]{\subsection{#1}}
\newcommand{\sssec}[1]{\subsubsection{#1}}

\newcommand{\br}[1]{\lbrace \, #1 \, \rbrace}
\newcommand{\li}{ < \infty}
\newcommand{\quis}{\simeq}
\newcommand{\xra}[1]{\xrightarrow{#1}}
\newcommand{\xla}[1]{\xleftarrow{#1}}
\newcommand{\xlra}[1]{\overset{#1}{\longleftrightarrow}}

\newcommand{\xroa}[1]{\overset{#1}{\twoheadrightarrow}}
\newcommand{\xria}[1]{\overset{#1}{\hookrightarrow}}
\newcommand{\ps}[1]{\mathbb{P}_{#1}^{\text{c}-1}}




\def\and{{ \text{ and } }}
\def\oor{{ \text{ or } }}

\def\Perm{\operatorname{Perm}}
\newcommand{\Ss}{\mathbb{S}}

\def\Op{\operatorname{Op}}
\def\res{\operatorname{res}}
\def\ind{\operatorname{ind}}

\def\sign{{\mathrm{sign}}}
\def\naive{{\mathrm{naive}}}
\def\l{\lambda}


\def\ov#1{\overline{#1}}
\def\cV{{\mathcal V}}
%%%-------------------------------------------------------------------
%%%-------------------------------------------------------------------

\newcommand{\chara}{\operatorname{char}}
\newcommand{\Kos}{\operatorname{Kos}}
\newcommand{\opp}{\operatorname{opp}}
\newcommand{\perf}{\operatorname{perf}}

\newcommand{\Fun}{\operatorname{Fun}}
\newcommand{\GL}{\operatorname{GL}}
\newcommand{\SL}{\operatorname{SL}}
\def\o{\omega}
\def\oo{\overline{\omega}}

\def\cont{\operatorname{cont}}
\def\te{\tilde{e}}
\def\gcd{\operatorname{gcd}}

\def\stab{\operatorname{stab}}

\def\va{\underline{a}}

\def\ua{\underline{a}}
\def\ub{\underline{b}}


\newcommand{\Ob}{\mathrm{Ob}}
\newcommand{\Set}{\mathbf{Set}}
\newcommand{\Grp}{\mathbf{Grp}}
\newcommand{\Ab}{\mathbf{Ab}}
\newcommand{\Sgrp}{\mathbf{Sgrp}}
\newcommand{\Ring}{\mathbf{Ring}}
\newcommand{\Fld}{\mathbf{Fld}}
\newcommand{\cRing}{\mathbf{cRing}}
\newcommand{\Mod}[1]{#1-\mathbf{Mod}}
\newcommand{\vs}[1]{#1-\mathbf{vect}}
\newcommand{\Vs}[1]{#1-\mathbf{Vect}}
\newcommand{\vsp}[1]{#1-\mathbf{vect}^+}
\newcommand{\Top}{\mathbf{Top}}
\newcommand{\Setp}{\mathbf{Set}_*}
\newcommand{\Alg}[1]{#1-\mathbf{Alg}}
\newcommand{\cAlg}[1]{#1-\mathbf{cAlg}}
\newcommand{\PO}{\mathbf{PO}}
\newcommand{\Cont}{\mathrm{Cont}}
\newcommand{\MaT}[1]{\mathbf{Mat}_{#1}}
\newcommand{\Der}{\mathrm{Der}}


\newcommand{\red}[1]{{\color{red}#1}}

%%%-------------------------------------------------------------------
%%%-------------------------------------------------------------------
%%%-------------------------------------------------------------------
%%%-------------------------------------------------------------------
%%%-------------------------------------------------------------------

\makeindex
\title{Problem set \#1}


\begin{document}
\onehalfspacing

\maketitle


\vspace{-1mm}

\begin{enumerate}


\item* Basic rules with derivations:
\begin{enumerate}
\item Prove the generalized product rule for derivations: if $\partial:R\to M$ is a derivation, then $\partial( a_1\cdots a_n) = \sum_{j=1}^n (\prod_{j\neq i} a_i)  \partial(a_j)$.
\item Prove the power rule for derivations: if $\partial:R\to M$ is a derivation, then $\partial(r^n) = n r^{n-1} \partial(r)$.
\item Show that if $R$ is a ring of characteristic $p$, then the subring $R^p:=\{ r^p \ | \ r\in R\}$ is in the kernel of every derivation.
\end{enumerate}

\
 



\item* Let $A$ be a ring and $S=A[x_1,\dots,x_n]$ be a polynomial ring.
\begin{enumerate}
\item Let $R$ be an $\N$-graded $A$-algebra such that $A$ lives in degree zero. Show that there is a derivation on $R$ such that for every homogeneous element $f$ of degree $d$, $\partial(f) = d\cdot  f$. This derivation is called the \emph{Euler operator} associated to the grading. 
\begin{proof} The rule above describes a well-defined function on $R$. We need to check that it is $A$-linear and satisfies the product rule.
Let $r=\sum_i r_i$ and $s=\sum_i s_i$ be elements of $R$ expressed as (finite) sums of homogeneous pieces with degree $r_i=i$ and $a\in A$. Then\begin{itemize}
\item $\partial(r+s) = \partial(\sum_i r_i + \sum_i s_i) =  \partial(\sum_i (r_i + s_i) ) = \sum_i i(r_i + s_i) = \sum_i i r_i + \sum_i i s_i = \partial(r) + \partial(s)$.
\item $\partial(ar) = \partial( a \sum_i r_i) = \partial(\sum_i ar_i ) = \sum_i i ar_i = a \sum_i i r_i = a\partial(r)$.
\item $\partial(rs) = \partial(\sum_k \sum_{i+j=k} r_i s_j) = \sum_k k (\sum_{i+j=k} r_i s_j) = \sum_{i,j} i r_i s_j + r_i j s_j = s\partial(r) + r\partial(s)$.\qedhere
\end{itemize}
\end{proof}
\item Let $S$ \red{be, as above,}\footnote{For infinitely many variables, we will get the same formula with a formal sum, but this is not an $S$-linear combination of partial derivatives. Oops!} a polynomial ring over $A$ endowed with the $\N$-grading by the rule $\deg(x_i) = n_i$. Express the Euler operator of the grading as an $S$-linear combination of the partial derivatives.
\begin{proof}
Take $\partial = \sum_i n_i x_i \frac{d}{dx_i}$. To check that this agrees with the Euler operator, by  $A$-linearity it suffices to check on any monomial $x_1^{a_1}\cdots x_n^{a_n}$: we get
\[ \partial(x_1^{a_1}\cdots x_n^{a_n}) = \sum_i n_i a_i x_1^{a_1}\cdots x_n^{a_n}\]
and $\sum_i n_i a_i$ is just the degree of $x_1^{a_1}\cdots x_n^{a_n}$.
\end{proof}
\end{enumerate}

\

\item Let $A$ be a ring and $R=A[x_1,\dots,x_n]$ be a polynomial ring.
\begin{enumerate}
\item Give an explicit formula for the Lie algebra bracket on $\Der_{R|A}(R)$.
\item Does $\Der_{R|A}(R)$ have any nontrivial proper Lie ideals (i.e., $A$-submodules $B$ such that $[d,b]\in B$ for all $b\in B$ and $d\in \Der_{R|A}(R)$)?
\begin{proof}
It is possible in general. For a fun example, over $A=\mathbb{F}_2$, we can take $\mathbb{F}_2[x^2] \frac{d}{dx}$ as a Lie ideal of $\Der_{\mathbb{F}_2[x]|\mathbb{F}_2}(\mathbb{F}_2[x])$. Indeed, note that for any $f\in \mathbb{F}_2[x]$, $\frac{d}{dx}(f) \in \mathbb{F}_2[x^2]$, since any even power of $x$ picks up a coefficient of two in the derivative. Then given $f\in \mathbb{F}_2[x^2]$ and $g\in \mathbb{F}_2[x^2]$ we have \[ [f\frac{d}{dx}, g \frac{d}{dx}] = (f \frac{d}{dx}(g) - g \frac{d}{dx}(f)) \frac{d}{dx} =g \frac{d}{dx}(f) \frac{d}{dx} \in \mathbb{F}_2[x^2] \frac{d}{dx}.\]
However, over a field of characteristic zero, this is false.
\end{proof}
\end{enumerate}


\


\item Let $R$ be a ring of characteristic $p>0$ and $\partial:R\to R$ be a derivation. Show that $\partial^p$, i.e., the $p$-fold self composition of $\partial$, is a derivation on $R$.

\

\item Let $R=\mathcal{C}^\infty(\R^n)$ be the ring of smooth functions on $\R^n$, and $\fm$ be the maximal ideal consisting of functions that vanish at some point $x_0\in \R^n$.
\begin{enumerate}
\item * Show that $\fm^t$ consists of the functions $f\in R$ such that $\frac{d^{a_1}}{dx_1^{a_1}} \cdots \frac{d^{a_n}}{dx_n^{a_n}}(f)|_{x=x_0} = 0$ for all $a_1,\dots,a_n$ with $0
\leq a_1+\cdots+a_n<t$.
\begin{proof}
Let $J_n= \{ f\in R \ | \ \frac{d^{a_1}}{dx_1^{a_1}} \cdots \frac{d^{a_n}}{dx_n^{a_n}}(f)|_{x=x_0} = 0\ \forall a_1,\dots,a_n:0
\leq a_1+\cdots+a_n<t\}$. We'll write $d^a$ for an $n$-tuple $a$ as shorthand for the iterated derivative above. 

First we show that $\fm^t\subseteq J_n$. We proceed by induction on $t$ with $t=1$ immediate from the definitions. Supposing the inclusion for a given $t$, take $f\in \fm^{t+1}$ and write $f=\sum g_i h_i$ with $g_i\in \fm^t$ and $h_i\in \fm$. Then each $g_i\in J_t$ by the induction hypothesis. Since $f\in \fm^t \in J_t$, we have $d^a(f)|_{x_0} = 0$ for all $|a|<t$. Given some $a$ with $|a|=t+1$, we can write $d^a = d^b \frac{d}{d_{x_j}}$ for some $j$ and some $b$ with $|b|=t$. Then 
\[ d^a(f) = \sum_i d^a(g_i h_i) = \sum_i d^b \frac{d}{d_{x_j}}(g_i h_i) = \sum_i d^b \big(h_i \frac{d}{d_{x_j}}(g_i)\big) + 
\sum_i d^b\big( g_i \frac{d}{d_{x_j}}(h_i) \big).\]
We have $g_i \frac{d}{d_{x_j}}(h_i) \in \fm^t \subseteq J_t$ so the second sum evaluates to zero at $x_0$.
Since $\frac{d}{d_{x_j}}(\fm^t) \subseteq \fm^{t-1}$, we have
$h_i \frac{d}{d_{x_j}}(g_i) \in \fm^t$, so the first sum evaluates to $0$ at $x_0$ as well. Thus, $f\in J_{t+1}$, as required.

For the other containment, we will apply Taylor's Theorem for multivariate functions\footnote{cf., Folland's \emph{Advanced Calculus}, Theorem 2.68}. Recall that this this says that $f$ agrees with a polynomial (in $x_i - (x_0)_i$) whose coefficients are determined by the iterated partial derivatives of $f$ at $x_0$, plus some error term. Beware that in general a smooth function is not equal to its Taylor series, so we will need to consider the polynomial plus remainder version. Applying this, if $f\in J_t$, we can write
\[ f =  \sum_{|a| = t} \frac{t}{a_1!\cdots a_n!} \widetilde{x_1}^{a_1}\cdots \widetilde{x_n}^{a_n} \int_0^1 (1-s)^t d^a(f)|_{x_0+ s(x-x_0)} \mathrm{ds},\]
where $\widetilde{x_i}:=x_i - (x_0)_i$.
What is important to observe about this expression is that each \[ j_a(x):= \frac{t}{a_1!\cdots a_n!} \int_0^1 (1-s)^t d^a(f)|_{x_0+ s(x-x_0)} \mathrm{ds}\]
is a $\mathcal{C}^\infty$ function on $\R^n$: we omit the details, but the point is essentially that smoothness lets us differentiate under the integral sign. Thus, we have
\[ f = \sum_{|a| = t} j_a \widetilde{x_1}^{a_1}\cdots \widetilde{x_n}^{a_n} \] with $j_a\in R$ and $\widetilde{x_i}\in \fm$ for each $i$, so $f\in \fm^t$.
\end{proof}
\item Show that $\Der_{R|\R}(R/\fm) \cong (\fm/\fm^2)^* \cong \R^n$ as vector spaces.
\end{enumerate}
As a moral, we conclude that $\Der_{R|\R}(R/\fm)$ serves as a model for the tangent space of $\R^n$ at $x_0$ constructed from the ring of smooth functions.


\

\item* Let $R$ be an $A$-algebra and $I$ an ideal. Show that if the identity map on $I/I^2$ is in the image of $\Der_{R|A}(I/I^2) \xra{\mathrm{res}} \Hom_{\red{R}}(I/I^2,I/I^2)$, then there is an $A$-algebra right inverse to the quotient map $\pi:R/I^2 \to R/I$. Conclude that the following are equivalent:
\begin{itemize}
\item $\Der_{R|A}(M) \xra{\mathrm{res}} \Hom_{\red{R}}(I/I^2,M)$ is surjective for all $R/I$-modules $M$;
\item $\Der_{R|A}(I/I^2) \xra{\mathrm{res}} \Hom_{\red{R}}(I/I^2,I/I^2)$ is surjective;
\item The quotient map $R/I^2 \to R/I$ has an $A$-algebra right inverse.
\end{itemize}
\begin{proof}
Suppose that $\partial:R\to I/I^2$ is a derivation whose restriction to $I/I^2$ (after factoring through $R/I^2$ as usual) is the identity map. Viewing $\partial$ as a derivation on $R/I^2$ by abuse of notation, note that $K:=\ker(\partial)$ is a subring of $R/I^2$ containing $A$. Let $i:K\to R/I^2$ be the inclusion map. We claim that $K\cong R/I$ as $A$-algebras. 

Since $-\partial$ is a derivation, the map $1-\partial:R/I^2\to R/I^2$ is a ring homomorphism, and $(1-\partial)\circ i$ is the identity on $K$ (because $K$ is the kernel of $\partial$). In particular, $1-\partial$ is surjective. We just need to see that the kernel of $1-\partial$ is $I/I^2$. We have $I//I^2$ is contained in the kernel, since for $a\in I/I^2$, $(1-\partial)(a) =a-\partial(a) = 0$; on the other hand if $r\in \ker(1-\partial)$, then $r\in \im(\partial)$, so $r\in I/I^2$. This completes the proof.

For the equivalences, the first implies the second since $I/I^2$ is an $R/I$-module, the second implies the third by what we just showed, and the third implies the first by a theorem from class.
\end{proof}


\

\item Let $R$ be a ring and $M$ an $R$-module. Recall that $R\rtimes M$ denotes the Nagata idealization of $M$: the ring with additive structure $R\oplus M$ and multiplication $(r,m)(s,n) = (rs,rn+sm)$. Show that $\alpha:R\to M$ is a derivation if and only if $(1,\alpha):R\to R\rtimes M$ ($r\mapsto (r,\alpha(r))$) is a ring homomorphism.

\end{enumerate}


\end{document}







  
 


