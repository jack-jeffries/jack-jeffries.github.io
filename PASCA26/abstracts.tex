\documentclass[12pt]{amsart}


\usepackage{times}
\usepackage[margin=0.8in]{geometry}
\usepackage{amsmath,amssymb,multicol,graphicx,framed,ifthen,color,xcolor,stmaryrd,enumitem,colonequals}
\usepackage[outline]{contour}
\contourlength{.4pt}
\contournumber{10}
\newcommand{\Bold}[1]{\contour{black}{#1}}

\definecolor{chianti}{rgb}{0.6,0,0}
\definecolor{meretale}{rgb}{0,0,.6}
\definecolor{leaf}{rgb}{0,.35,0}
\newcommand{\Q}{\mathbb{Q}}
\newcommand{\N}{\mathbb{N}}
\newcommand{\Z}{\mathbb{Z}}
\newcommand{\R}{\mathbb{R}}
\newcommand{\C}{\mathbb{C}}
\newcommand{\e}{\varepsilon}
\newcommand{\inv}{^{-1}}
\newcommand{\dabs}[1]{\left| #1 \right|}
\newcommand{\ds}{\displaystyle}
\newcommand{\solution}[1]{\ifthenelse {\equal{\displaysol}{1}} {\begin{framed}{\color{meretale}\noindent #1}\end{framed}} { \ }}
\newcommand{\solutione}[1]{\ifthenelse {\equal{\displaysol}{1}} {\begin{framed}{\color{leaf}This solution is embargoed.}\end{framed}} { \ }}
\newcommand{\showsol}[1]{\def\displaysol{#1}}

\newcommand{\rsa}{\rightsquigarrow}


\newcommand\itemA{\stepcounter{enumi}\item[{\Bold{(\theenumi)}}]}
\newcommand\itemB{\stepcounter{enumi}\item[(\theenumi)]}
\newcommand\itemC{\stepcounter{enumi}\item[{\it{(\theenumi)}}]}
\newcommand\itema{\stepcounter{enumii}\item[{\Bold{(\theenumii)}}]}
\newcommand\itemb{\stepcounter{enumii}\item[(\theenumii)]}
\newcommand\itemc{\stepcounter{enumii}\item[{\it{(\theenumii)}}]}
\newcommand\itemai{\stepcounter{enumiii}\item[{\Bold{(\theenumiii)}}]}
\newcommand\itembi{\stepcounter{enumiii}\item[(\theenumiii)]}
\newcommand\itemci{\stepcounter{enumiii}\item[{\it{(\theenumiii)}}]}
\newcommand\ceq{\colonequals}


\DeclareMathOperator{\ord}{ord}

\DeclareMathOperator{\res}{res}
\setlength\parindent{0pt}
%\usepackage{times}

%\addtolength{\textwidth}{100pt}
%\addtolength{\evensidemargin}{-45pt}
%\addtolength{\oddsidemargin}{-60pt}

\pagestyle{empty}
%\begin{document}\begin{itemize}

%\thispagestyle{empty}






%%%%%%%%%%%%%%%%%%%%%%%%%%%%%%%%%%%%%%%%%%%%%%%%%%%%%%%%%%%%%%%%%%%%%%%%
\begin{document}
\title{Tentative titles and abstracts for PASCA 2026}
\maketitle

\section*{Lecture series}

\vfill


\begin{itemize}

\item \textsc{Federico Castillo:} \textbf{The commutative algebra of lattice polytopes}

\

\noindent Semigroup algebras form a bridge between ring theory and discrete geometry. The plan of these lectures is to explore this connection in order to understand results about the number of lattice points in polytope by using tools from commutative algebra. The main goal is to explore the known inequalities among the delta (also know as $h*$) vector of a lattice polytope and to propose new conjectures about further inequalities.

\vfill

\item  \textsc{Daniel Duarte:} \textbf{The module of derivations of graded Tjurina algebras}

\

\noindent In the classification theory of isolated hypersurface singularities, the celebrated Mather-Yau theorem states that the Tjurina algebra is a complete invariant of the hypersurface. This result motivated the study of the module of derivations of the Tjurina algebra, which is known as the Yau algebra. This series of lectures will be an introduction to several questions on the structure of the Yau algebra in the case of homogeneous hypersurfaces. We will also discuss higher-order versions of these algebras and some recent conjectures about them.

\vfill

\item \textsc{Elo\'isa Grifo:} \textbf{Symbolic powers}

\

\noindent Symbolic powers arise naturally in commutative algebra from the theory of primary decomposition, but they also contain geometric information, thanks to a classical result of Zariski and Nagata. Computing primary decompositions is a difficult computational problem, and as a result, many natural questions about symbolic powers remain wide open. We will introduce symbolic powers and describe some of the main open problems on the subject, and point to some recent research advances. We will build the theory starting from primary decomposition and associated primes, so a first course in commutative algebra is the only prerequisite necessary.

\vfill


\item  \textsc{Karl Schwede:} \textbf{Singularities in mixed characteristic}

\

\noindent We will introduce perfectoid algebras, and how they can be used to
produce big Cohen-Macaulay algebras.  As a consequence, we will discuss the
resulting theory of singularities in mixed characteristic which replaces
Frobenius in characteristic p > 0 and resolution of singularities in mixed
characteristic.

\vfill

\end{itemize}

\vfill

\newpage

\section*{Complementary lectures}

\begin{itemize}

\vfill

\item \textsc{Lara Bossinger:} \textbf{A cluster of algebra, geometry and combinatorics}

\

\noindent  In this talk I will introduce cluster algebras: certain commutative algebras defined recursively by sets of algebraically independent elements called seeds, and a combinatorial rule creating a new seed from a given one, called mutation. The intricate combinatorial structure underlying this construction has revealed its importance in many different areas of maths over the last twenty years such as polyhedral geometry, tropical geometry, combinatorics, birational geometry, knot theory and particle physics. In this talk I will give the basics of the theory, due to Fomin and Zelevinsky, and also summarize the state of the art regarding applications in physics.

\vfill

\item  \textsc{Javier Carvajal Rojas:}  \textbf{Singularities of foliations in positive characteristic}

\

\noindent  I'll explain to what extent the theory of F-singularities extends to foliations in positive characteristic. This is done through the so-called purely inseparable Galois theory equating foliations to purely inseparable covers and a generalized theory of tame ramification. 

\vfill

\item  \textsc{Alessandro De Stefani:} \textbf{The defect of the F-pure threshold}

\

\noindent  For a local ring of prime characteristic, the F-pure threshold is a numerical invariant that detects and measures its singularities, and can be regarded as a “Frobenius analogue” of the log canonical threshold from birational geometry in equal characteristic zero. This study of F-pure thresholds was carried out by several authors, bust for most part within the framework of local (or standard graded) rings. I will report on a joint work with Luís Núñez-Betancourt and Ilya Smirnov, which instead takes the theory into a new direction by studying the global properties of F-pure thresholds, and relates them with the local singularities of the ring. The key tools are a formula for this invariant in terms of differential operators, as well as a new perspective on the F-pure threshold which we now view as a "defect".

\vfill


\item \textsc{Lisa Seccia:} \textbf{F-singularities of geometrically vertex decomposable ideals}

\

\noindent In this talk we prove that in many cases geometrically vertex decomposable ideals are F-pure. After a brief overview of some basic concepts in F-singularity theory, I will present a way to iteratively construct Frobenius splittings using geometric vertex decomposition. This technique will be illustrated with several examples.
 This is joint work with De Negri, Gorla, Klein, Mayada, Rajchgot, Shahada.

\vfill

\end{itemize}

\vfill

\end{document}
